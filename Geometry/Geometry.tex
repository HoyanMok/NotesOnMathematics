% TeXplates/Mathematics.tex
% v0.1.1
% https://github.com/HoyanMok/TeXplates
\documentclass[openany]{book} 
% \documentclass{ctexbook} 如果用中文
% \documentclass[10pt,a4paper]{ctexart}  字体大小和纸张大小,默认分别为10pt和letterpaper
% 五号 = 10.5pt,小四=12pt,四号=14pt
% 其他可选参量如twocolumn, 两行排版

\usepackage{biblatex} %[style=gb7714-2015]{biblatex} 可以选择样式
\addbibresource{Geometry.bib} % 把这里改成实际的文件名

% 令参考资料能够加入目录中:
\defbibheading{bibliography}[\bibname]{% 
	% \addcontentsline{toc}{chapter}{参考文献}
	\chapter{#1}% 
	\markboth{#1}{#1}}

\usepackage{imakeidx} %索引
	\makeindex[intoc]
	\newcommand*{\indexbf}[1]{\emph{\textbf{#1}}\index{#1}} % Index for definition
	\newcommand*{\indexfm}[2]{#1\index{  Symbols@Symbols used!#2@$#1$}} % Used Symbol

% 将PATH换成绝对路径 (Windows) 或相对路径 (Mac OS或Linux)
% 使用「/」而不是「\」
\newcommand{\PATH}{PATH/}

% 对目录项等的修改
\usepackage{chngcntr}
	\counterwithout{section}{chapter} % So that the section won't reset when newing a chapter
\renewcommand{\thesection}{\textmd{\S}\arabic{section}}
\renewcommand{\thesubsection}{\arabic{section}.\arabic{subsection}}

% 引用的宏包:
% 宏包的使用, 可以在命令行运行texdoc <宏包名>获得文档
\usepackage{multicol} % 分栏 (全局分栏建议在文档类处设置)
\usepackage{amsmath} % AMS数学标准
	\makeatletter % '@' now normal "letter"
	\@addtoreset{equation}{section} % 每次换section就把equation清零
	\makeatother  % '@' is restored as "non-letter"
	\renewcommand\theequation{\oldstylenums{\arabic{section}}%
					-\oldstylenums{\arabic{equation}}} % 显示为section数-equation数
\usepackage{amssymb} % 数学符号
\usepackage{mathrsfs} % 花体
\usepackage{amsthm} %定义、证明、定理等
	\theoremstyle{plain}
		\newtheorem{axion}{Axion} %公理
		\newtheorem{theorem}{Theorem}[section] %定理
		\newtheorem{corollary}{Corollary} %推论
		\newtheorem{lemma}{Lemma} %引理
	\theoremstyle{definition}
		\newtheorem{definition}{Definition}[section] %定义
		\newtheorem{proposition}{Proposition} %命题
	\renewcommand{\proofname}{\textbf{Proof}}

\renewcommand{\thetheorem}{%
	\arabic{section}.\arabic{theorem}%
} % 公式编号不显示`\S`
\renewcommand{\thedefinition}{%
	\arabic{section}.\arabic{definition}%
} % 公式编号不显示`\S`

\usepackage{esint} % 积分
\usepackage{siunitx} % 标准SI数值和单位处理

\usepackage{tikz} % 绘图
\usepackage{float} % 浮动体 (供图片, 表格等) 扩展, 主要用于提供h模式
\usepackage{graphicx} % 插入图片
\usepackage{titlepic}
\usepackage[font=small, skip=5pt]{caption} % 缩小题注字体和题注与图片距离
\usepackage{subcaption} % 子图和子图的题注
\usepackage{svg} % svg位图
\usepackage{wrapfig} % 简单的图文绕排
\usepackage[inline]{enumitem} % 编号
\usepackage{geometry} % 调整页边距
% \geometry{left=1.6cm,right=1.6cm}
\usepackage{xcolor} % 颜色
\usepackage[colorlinks=true,bookmarks=true]{hyperref} % 引用, 交叉引用, 图表等的链接; 生成书签
\hypersetup{linkcolor=[rgb]{1,0.27,0},bookmarksopen = true}% 更多设置请查阅: texdoc hyperref


% 定义一些笔者常用的指令:
\newcommand{\me}{\mathrm{e}} % 自然对数的底
\newcommand{\mi}{\mathrm{i}} % 虚数单位
\newcommand{\dif}{\mathop{}\!\mathrm{d}} % 微分算子d
\newcommand*{\basis}[1]{\hat{\boldsymbol{#1}}} % 基底
\newcommand*{\bv}{\boldsymbol} % 向量加粗

\newcommand*{\diff}[3][1]
{\if#11%
	\frac{\mathrm{d} #2}{\mathrm{d} #3}% 导数\diff{y}{x}
\else%
	\frac{\mathrm{d}^{#1} #2}{\mathrm{d} #3^{#1}}% n阶导数\diff[n]{y}{x}
\fi}
\newcommand*{\pdiff}[3][1]
{\if#11%
	\frac{\partial #2}{\partial #3}% 偏导数\pdiff{y}{x}
\else%
	\frac{\partial^{#1} #2}{\partial #3^{#1}}% n阶偏导数\pdiff[n]{y}{x}
\fi}
\newcommand{\emphbf}[1]{\emph{\textbf{#1}}}
% \indexbf 的定义见前imakeidx的引用下

% 笔者习惯的运算符:
\DeclareMathOperator{\tg}{tg}
\DeclareMathOperator{\ctg}{ctg}
\DeclareMathOperator{\arctg}{arctg}
\DeclareMathOperator{\sh}{sh}
\DeclareMathOperator{\ch}{ch}
\DeclareMathOperator{\dom}{dom}
\DeclareMathOperator{\ran}{ran}
\DeclareMathOperator{\interior}{int}
% \DeclareMathOperator*{\指令}{显示} 
% 带星号的版本会像\lim一样
\begin{document}
\frontmatter
\tableofcontents

\mainmatter

\chapter{Geometry in Regions of a Space}
\section{Co-odinate and its transformation}
\begin{definition}
A transformation of co-odinate from $\boldsymbol{x}$ to $\boldsymbol{y}$
\begin{align*}
	\bv y(\bv x ) = y^i(x^j)\basis e_i=y(x^j).
\end{align*}
Its \indexbf{Jacobian}:
\begin{align}\label{Jacobian of transformation}
	\bv J = 
	\begin{pmatrix}
		\cfrac{\partial y^i}{\partial x^j}
	\end{pmatrix}
\end{align}
\end{definition}
A vector $\bv u$ at point $\bv x_0$ under such transformation would follow:
\begin{align}\label{vector}
	v^i = \left.\frac{\partial y^i}{\partial x^j}\right|_{\bv x_0} u^i
\end{align}
\begin{align*}
	\text{i.e.}&&
	\bv v = \bv J_0\bv u
\end{align*}

A linear form $\ell: \bv x\mapsto\ell(\bv x) =l_ix^i $ under such transformation would follow:
\begin{align}\label{linear form}
	l_i'\dif y^i = l_j\dif x^j 
	&&\Rightarrow&&
	l'_i = \left.\frac{\partial x^j}{\partial y_i}\right|_{\bv x_0}l_j
\end{align}
\begin{align*}
	\text{i.e.}&&
	\bv l' = \bv l \bv J^{-1}_0
\end{align*}

A linear transformation $\mathscr{L}:\bv x \mapsto \bv L\bv x$ where $\bv L = \left( L^i_{\;j}\right)$ under such transformation would follow:
\begin{align*}
	\dif y\left(\mathscr L^i(\bv x)\right) 
	&=(L')^i_{\;j} \dif y^j 
	\\
	&=  \left.\frac{\partial y^i}{\partial x^k}\right|_{\bv x_0} \dif \mathscr L^k (\bv x)
	=\left.\frac{\partial y^i}{\partial x^k}\right|_{\bv x_0} L^k_{\;h} \dif x^h
	=\left.\frac{\partial y^i}{\partial x^k}\right|_{\bv x_0} 
	L^k_{\;h} 
	\left.\frac{\partial x^h}{\partial y_j}\right|_{\bv x_0}\dif y^j
\end{align*} 
\begin{align}\label{linear transformation} 
	(L')^i_{\;j} = 
	\left.\frac{\partial y^i}{\partial x^k}\right|_{\bv x_0} 
	L^k_{\;h} 
	\left.\frac{\partial x^h}{\partial y_j}\right|_{\bv x_0} 
	&&\text{or}&&
	\bv L'  = \bv J_0\bv L\bv J^{-1}_0 
\end{align} 

A bilinear form $\mathscr{B}: \bv x \mapsto \bv x^{\mathrm{T}} \bv b \bv x = x^{\;i}b_{ij}x^j$: 
\begin{align*}
	b'_{ij}\dif y^{\;i} \dif y^j 
	= b'_{ij} \left.\frac{\partial y^{\,i}}{\partial x^{\;k}}\right|_{\bv x_0}
	\left.\frac{\partial y^j}{\partial x^h}\right|_{\bv x_0}\dif x^{\;k}\dif x^h 
	=b_{kh} \dif x^{\;k}\dif x^h 
	&&\Rightarrow&&
	b'_{ij} =\left. \frac{\partial x^{\;k}}{\partial y^{\;h}}\right|_{\bv x_0}
	b_{kh} 
	\left.\frac{\partial x^h}{\partial y^j}\right|_{\bv x_0}
\end{align*}
\begin{align}\label{bilinear form}
	\text{i.e.} &&
	\bv b' = (\bv J^{-1}_0)^{\mathrm{T}} \bv b \bv J^{-1}_0
\end{align}
\backmatter
\nocite{*} % 这个表示列出所有没有在文中被引用的参考文献
\printbibliography[heading=bibliography, title={Bibliography}]
\printindex
\end{document}