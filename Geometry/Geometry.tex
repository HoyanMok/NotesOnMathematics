\documentclass{article}
\usepackage{amsmath} %数学
	\makeatletter % '@' now normal "letter"
	\@addtoreset{equation}{section}
	\makeatother  % '@' is restored as "non-letter"
	\renewcommand\theequation{\oldstylenums{\thesection}%
					-\oldstylenums{\arabic{equation}}}
\usepackage{amsthm} %定义、证明、定理等
	\theoremstyle{plain}
	\theoremstyle{definition}
		\newtheorem{dfn}{Definition}[section] %定义
		\newtheorem{thrm}{Theorem}[section] %定理
		\newtheorem{crll}{Corollary} %推论
		\newtheorem{lmm}{Lemma} %引理
		\newtheorem{prp}{Proposition} %命题
	\renewcommand{\proofname}{\textbf{Proof}}
\usepackage{amssymb} %数学格式
\usepackage{mathrsfs} %花体
\usepackage{enumerate} %编号
\usepackage{makeidx} %索引
	\makeindex
\usepackage[colorlinks=true,bookmarks=true]{hyperref}%hyperlinks for reference, table of contents and indexes
\hypersetup{linkcolor=[rgb]{1,0.27,0},bookmarksopen = true}%More options read texdoc hyperref
\newcommand{\me}{\mathrm{e}}
\newcommand*{\imagine}{\mathrm{i}}
\newcommand\dif{\mathop{}\!\mathrm{d}}
\DeclareMathOperator{\ctg}{ctg}
\newcommand*{\basis}[1]{\hat{\boldsymbol{#1}}}
\newcommand*{\bv}{\boldsymbol}
\newcommand*{\bm}{\boldsymbol}
\newcommand*{\indexbf}[1]{\textbf{#1}\index{#1}}
\begin{document}
\tableofcontents
\newpage
\section{Geometry in Regions of a Space}
\subsection{Co-odinate and its transformation}
\begin{dfn}
A transformation of co-odinate from $\boldsymbol{x}$ to $\boldsymbol{y}$
\begin{align*}
	\bv y(\bv x ) = y^i(x^j)\basis e_i=y(x^j).
\end{align*}
Its \indexbf{Jacobian}:
\begin{align}\label{Jacobian of transformation}
	\bm J = 
	\begin{pmatrix}
		\cfrac{\partial y^i}{\partial x^j}
	\end{pmatrix}
\end{align}
\end{dfn}
A vector $\bv u$ at point $\bv x_0$ under such transformation would follow:
\begin{align}\label{vector}
	v^i = \left.\frac{\partial y^i}{\partial x^j}\right|_{\bv x_0} u^i
\end{align}
\begin{align*}
	\text{i.e.}&&
	\bv v = \bm J_0\bv u
\end{align*}

A linear form $\ell: \bv x\mapsto\ell(\bv x) =l_ix^i $ under such transformation would follow:
\begin{align}\label{linear form}
	l_i'\dif y^i = l_j\dif x^j 
	&&\Rightarrow&&
	l'_i = \left.\frac{\partial x^j}{\partial y_i}\right|_{\bv x_0}l_j
\end{align}
\begin{align*}
	\text{i.e.}&&
	\bv l' = \bv l \bm J^{-1}_0
\end{align*}

A linear transformation $\mathscr{L}:\bv x \mapsto \bm L\bv x$ where $\bm L = \left( L^i_{\;j}\right)$ under such transformation would follow:
\begin{align*}
	\dif y\left(\mathscr L^i(\bv x)\right) 
	&=(L')^i_{\;j} \dif y^j 
	\\
	&=  \left.\frac{\partial y^i}{\partial x^k}\right|_{\bv x_0} \dif \mathscr L^k (\bv x)
	=\left.\frac{\partial y^i}{\partial x^k}\right|_{\bv x_0} L^k_{\;h} \dif x^h
	=\left.\frac{\partial y^i}{\partial x^k}\right|_{\bv x_0} 
	L^k_{\;h} 
	\left.\frac{\partial x^h}{\partial y_j}\right|_{\bv x_0}\dif y^j
\end{align*} 
\begin{align}\label{linear transformation} 
	(L')^i_{\;j} = 
	\left.\frac{\partial y^i}{\partial x^k}\right|_{\bv x_0} 
	L^k_{\;h} 
	\left.\frac{\partial x^h}{\partial y_j}\right|_{\bv x_0} 
	&&\text{or}&&
	\bm L'  = \bm J_0\bm L\bm J^{-1}_0 
\end{align} 

A bilinear form $\mathscr{B}: \bv x \mapsto \bv x^{\mathrm{T}} \bm b \bv x = x^{\;i}b_{ij}x^j$: 
\begin{align*}
	b'_{ij}\dif y^{\;i} \dif y^j 
	= b'_{ij} \left.\frac{\partial y^{\,i}}{\partial x^{\;k}}\right|_{\bv x_0}
	\left.\frac{\partial y^j}{\partial x^h}\right|_{\bv x_0}\dif x^{\;k}\dif x^h 
	=b_{kh} \dif x^{\;k}\dif x^h 
	&&\Rightarrow&&
	b'_{ij} =\left. \frac{\partial x^{\;k}}{\partial y^{\;h}}\right|_{\bv x_0}
	b_{kh} 
	\left.\frac{\partial x^h}{\partial y^j}\right|_{\bv x_0}
\end{align*}
\begin{align}\label{bilinear form}
	\text{i.e.} &&
	\bm b' = (\bm J^{-1}_0)^{\mathrm{T}} \bm b \bm J^{-1}_0
\end{align}
\printindex
\end{document}