\documentclass[openany, oneside, a5paper]{book} 
\usepackage{../TeXplatesMathematics}
\addbibresource{DiffGeometry.bib} % 把这里改成实际的文件名

\title{Differential Geometry}
\author{Hoyan Mok}

\DeclareMathOperator{\Vect}{Vect}

\begin{document}
\pagenumbering{Alph}
\maketitle
\frontmatter

\tableofcontents
\mainmatter{}

\chapter{Manifolds}

\chapter{Scalar and Vector Fields}

\section{Scalar Fields}

\begin{definition}[Scalar Field]
    Let $M$ be a smooth manifold, $f \in C^{(\infty)}(M)$ is called a \indexbf{scalar field}.
\end{definition}

The scalar field over a manifold, form an algebra.

\section{Vector Fields}

\begin{definition}[vector field]
    A \indexbf{vector field} $v$ over manifold $M$ is a $C^{(\infty)}(M) \to C^{(\infty)}(M)$ map that satisfies
    \begin{enumerate}[label=(\alph*)]
        \item $\forall f, g \in C^{(\infty)}(M)$, $\forall \lambda, \mu \in \mathbb R$, $v(\lambda f + \mu g) = \lambda v(f) + \mu v(g)$ (\emph{linearity}).
        \item $\forall f, g \in C^{(\infty)}(M)$, $v(fg) = v(f) g + f v(g)$ 
    \end{enumerate}
\end{definition}

\begin{definition}[tangent vector]
    Let $v$ be a vector field over $M$, $p$ be a point on $M$.
    The tangent vector $v_p$ at $p$ is defined as a $C^{(\infty)}(M) \to C^{(\infty)}(M)$ map that satisfies
    \begin{equation}
        v_p(f) = v(f)(p).
    \end{equation}
\end{definition}

The collection of tangent vectors at $p$ is called the \indexbf{tangent space} at $p$, denoted by $\indexmath[TpM]{T_p M}$.

The derivative of a path $\gamma \colon [0, 1] \to M$ (or $\mathbb R \to M$) in a smooth manifold is defined as:
\begin{equation}
    \begin{aligned}
        \gamma'(t) \colon& C^{(\infty)}(M) \to \mathbb R;
        \\
        & \gamma'(t)(f) = \diff{}{t} f \circ \gamma(t)
    \end{aligned}
\end{equation}

We can see that $\gamma'(t) \in T_{\gamma(t)} M$.

\section{Covariant and Contravariant}

\begin{definition}[pullback]
    Let $f$ be a scalar field over $M$, $\varphi \in C^{(\infty)}(M, N)$. Then the \indexbf{pullback} of $f$ by $\varphi$ is defined as
    \begin{equation}
        \varphi^* f = f \circ \varphi \in C^{(\infty)}(N).
    \end{equation}
\end{definition}

Fields that are pullbacked are \indexbf{covariant} fields.

\begin{definition}[pushforward]
    Let $v_p$ be a tangent vector of $M$ at $p$, $\varphi \in C^{(\infty)}(M, N)$, $q = \varphi(p)$. Then the \indexbf{pushforward} of $v_p$ by $\varphi$ is defined as
    \begin{equation}
        (\varphi_* v)_q(f) = v_p(\varphi^* f).
    \end{equation}
\end{definition}

Note that the pushforward of a vector field can only be obtained when $\varphi$ is a diffeomorphism.

Fields that are pushforwarded are \indexbf{contravariant} fields.

Mathematicians and physicists might have disagreement on whether a tangent vector is covariant or contravariant.
This is because of that physicists might consider the coordinates ($v^\mu$) of a tangent vector as a vector field, instead of linear combination of bases $\partial_\mu$.

\section{Flows}

Let a path $\gamma \colon \mathbb R$ follows a vector field (a velocity field), that is
\begin{equation}
    \gamma'(t) = v_{\gamma(t)},
\end{equation}
then we call $\gamma$ the \indexbf{integral curve} through $p := gamma(0)$ of the vector field $v$.

\begin{definition}
    Suppose $v$ is an integrable vector field.
    Let $\varphi_t(p)$ be the point at time $t$ on the integral curve through $p$.
    \begin{equation}
        \varphi_t \colon M \to M
    \end{equation}
    is then called a \indexbf{flow} generated by $v$.
\end{definition}

\begin{equation}
    \diff{}{t}\varphi_t(p) = v_{\varphi_t(p)}.
\end{equation}

\chapter{Differential Forms}

\section{1-forms}

\begin{definition}[1-form]
    A \indexbf{1-form} $\dif \omega$ on $M$ is a $\Vect(M) \to C^{(\infty)}(M)$ which satisfies that
    \begin{enumerate}[label=(\alph*)]
        \item $\forall v, w \in \Vect(M)$, $\forall f, g \in C^{(\infty)}(M)$, 
        \begin{equation}
            \dif \omega (fv + gw) = f \dif \omega (v) + g \dif \omega (w)
        \end{equation}
    \end{enumerate}
\end{definition}

\backmatter{}
\nocite{*} % 这个表示列出所有没有在文中被引用的参考文献
\printbibliography[heading=bibliography, title={bibliography}]

\indexprologue{Here listed the important symbols used in these notes}
\printindex[symbol]

\printindex
\end{document}