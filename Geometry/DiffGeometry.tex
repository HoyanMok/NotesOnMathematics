\documentclass[openany, oneside, a5paper]{book} 
\usepackage{../TeXplatesMathematics}
\addbibresource{DiffGeometry.bib} % 把这里改成实际的文件名

\title{Differential Geometry}
\author{Hoyan Mok}

\DeclareMathOperator{\Vect}{Vect}

\begin{document}
\pagenumbering{Alph}
\maketitle
\frontmatter

\tableofcontents
\mainmatter{}

\part{Domestic Differential Geometry}

\chapter{Manifolds}

\chapter{Scalar and Vector Fields}

\section{Scalar Fields}

\begin{definition}[Scalar Field]
    Let $M$ be a smooth manifold, $f \in C^{(\infty)}(M)$ is called a \indexbf{scalar field}.
\end{definition}

The scalar field over a manifold, form an algebra.

\section{Vector Fields}

\begin{definition}[vector field]
    A \indexbf{vector field} $v$ over manifold $M$ is a $C^{(\infty)}(M) \to C^{(\infty)}(M)$ map that satisfies
    \begin{enumerate}[label=(\alph*)]
        \item $\forall f, g \in C^{(\infty)}(M)$, $\forall \lambda, \mu \in \mathbb R$, $v(\lambda f + \mu g) = \lambda v(f) + \mu v(g)$ (\emph{linearity}).
        \item $\forall f, g \in C^{(\infty)}(M)$, $v(fg) = v(f) g + f v(g)$ 
    \end{enumerate}
\end{definition}

The space of all vector fields on $M$ is denoted by $\indexmath[Vect(M)]{\Vect(M)}$

\begin{definition}[tangent vector]
    Let $v$ be a vector field over $M$, $p$ be a point on $M$.
    The tangent vector $v_p$ at $p$ is defined as a $C^{(\infty)}(M) \to C^{(\infty)}(M)$ map that satisfies
    \begin{equation}
        v_p(f) = v(f)(p).
    \end{equation}
\end{definition}

The collection of tangent vectors at $p$ is called the \indexbf{tangent space} at $p$, denoted by $\indexmath[TpM]{T_p M}$.

The derivative of a path $\gamma \colon [0, 1] \to M$ (or $\mathbb R \to M$) in a smooth manifold is defined as:
\begin{equation}
    \begin{aligned}
        \gamma'(t) \colon& C^{(\infty)}(M) \to \mathbb R;
        \\
        & \gamma'(t)(f) = \diff{}{t} f \circ \gamma(t)
    \end{aligned}
\end{equation}

We can see that $\gamma'(t) \in T_{\gamma(t)} M$.

Let a path $\gamma \colon \mathbb R$ follows a vector field (a velocity field), that is
\begin{equation}
    \gamma'(t) = v_{\gamma(t)},
\end{equation}
then we call $\gamma$ the \indexbf{integral curve} through $p := \gamma(0)$ of the vector field $v$.

\begin{definition}
    Suppose $v$ is an integrable vector field.
    Let $\varphi_t(p)$ be the point at time $t$ on the integral curve through $p$.
    \begin{equation}
        \varphi_t \colon M \to M
    \end{equation}
    is then called a \indexbf{flow} generated by $v$.
\end{definition}

\begin{equation}
    \diff{}{t}\varphi_t(p) = v_{\varphi_t(p)}.
\end{equation}

\section{Covariant and Contravariant}

\begin{definition}[pullback]
    Let $f$ be a scalar field over $N$, $\varphi \in C^{(\infty)}(M, N)$. Then the \indexbf{pullback} of $f$ by $\varphi$ 
    \begin{equation}
        \varphi^* \colon C^{(\infty)}(N) \to C^{(\infty)}(M),
    \end{equation}
    is defined as
    \begin{equation}
        \varphi^* f = f \circ \varphi \in C^{(\infty)}(M).
    \end{equation}
\end{definition}

Fields that are pullbacked are \indexbf{covariant} fields.

\begin{definition}[pushforward]
    Let $v_p$ be a tangent vector of $M$ at $p$, $\varphi \in C^{(\infty)}(M, N)$, $q = \varphi(p)$. 
    Then the \indexbf{pushforward} of $v_p$ by $\varphi$ 
    \begin{equation}
        \varphi_* \colon T_p M \to T_q N,
    \end{equation}
    is defined as
    \begin{equation}
        (\varphi_* v)_q(f) = v_p(\varphi^* f).
    \end{equation}
\end{definition}

Note that the pushforward of a vector field can only be obtained when $\varphi$ is a diffeomorphism.

Fields that are pushforwarded are \indexbf{contravariant} fields.

Mathematicians and physicists might have disagreement on whether a tangent vector is covariant or contravariant.
This is because of that physicists might consider the coordinates ($v^\mu$) of a tangent vector as a vector field, instead of linear combination of bases $\partial_\mu$.

\section{Components of Vector Fields}

Let $\varphi \colon U \to \mathbb R^n$ be a chart of $M$ ($U \subset M$).

Let $p \in U$, $\varphi(p) = x = (x^\mu)$ ($\mu = 0, \ldots, n-1$).
Locally, a function $f \in C^{(\infty)}(M)$ can be written as
\begin{equation}
    (\varphi^{-1})^* f = f \circ \varphi^{-1} \colon \mathbb R^n \to \mathbb R,
\end{equation}
and a vector field $v \in \Vect(M)$ can be written as
\begin{equation}
    (\varphi_* v)_{x} = \varphi_* v_p \colon C^{(\infty)}(\mathbb R^n) \to \mathbb R,
\end{equation}
or
\begin{equation}
    \varphi_* v  \in \Vect(\mathbb R^n)
\end{equation}



Since $T_x\mathbb R^n \cong \mathbb R^n$ is a linear space, one can find a basis for $T_x\mathbb R^n$ as
\begin{equation}
    \partial_\mu \colon C^{(\infty)} (\mathbb R^n) \to C^{(\infty)} (\mathbb R^n),
\end{equation}
and $(\varphi_* v)_x = v^\mu (x) \partial_\mu$.

Pushing forward $v^\mu (x) \partial_\mu$ by $\varphi^{-1}$ one obtains $v$.

In an abuse of symbols, one may just omit the pullback and pushforward, and refer to the $f$ and $v$ by $(\varphi^{-1})^* f$ and $\varphi_*v$.

Consider another chart $\psi \colon U \to \mathbb R^n$ of $M$,
and
\begin{equation}
    y = \psi(p),
    \quad
    (\psi_* v)_x = u^\mu \partial_\mu,
\end{equation}
where we have chosen the same basis in $T_y \mathbb R^n$ as in $T_x\mathbb R^n$.

We would like to know how to relate $v^\mu$ and $u^\mu$ i.e.\ we want to know how the components of $v$ transforms under a coordinate transformation $\tau = \psi \circ \varphi^{-1}$.

Consider any $f \in C^{(\infty)}(M)$, 
\begin{equation}
    v(f) = \varphi_* v((\varphi^{-1})_* f)
    = \psi_* v((\psi^{-1})_* f)
\end{equation}
\hence
\begin{equation}
    u^\mu \partial_\mu (f \circ \psi^{-1})
    = v^\mu \partial_\mu (f \circ \varphi^{-1})
    = v^\mu \partial_\mu (f \circ \psi^{-1} \circ \tau)
    = v^\mu {\tau'}^\nu_\mu \partial_\nu (f \circ \psi^{-1})
\end{equation}
\hence
\begin{equation}
    u^\mu = v^\nu {\tau'}^\mu_\nu,
\end{equation}
where
\begin{equation}
    {\tau'}^\mu_\nu = \pdiff{y^\mu}{x^\nu}.
\end{equation}

\section{Lie Bracket}
\begin{definition}[Lie bracket]
    Let $v, w \in \Vect(M)$, then the \indexbf{Lie bracket} of $v$ and $w$ is defined as
    \begin{equation}
        [v, w] \colon C^{(\infty)}(M) \to C^{(\infty)}(M);
        \;
        f \mapsto v \circ w (f) - w \circ v (f).
    \end{equation}
\end{definition}

The Lie bracket is an antisymmetric bilinear map\footnote{Note that it is not $C^{(\infty)}$-linear}, and an important property of the Lie bracket is the Leibniz rule:
\begin{equation}
    [v, w] (fg) = {[v, w] (f)} g + f {[v, w] (g)}.
\end{equation}

Another important property of the Lie bracket is the Jacobi identity:
\begin{equation}
    [v, [w, u]] + [w, [u, v]] + [u, [v, w]] = 0.
\end{equation}

\chapter{Differential Forms}

\section{1-forms}

\begin{definition}[1-form]
    A \indexbf{1-form} $\omega$ on $M$ is a $\Vect(M) \to C^{(\infty)}(M)$ which satisfies that
    \begin{enumerate}[label=(\alph*)]
        \item $\forall v, w \in \Vect(M)$, $\forall f, g \in C^{(\infty)}(M)$, 
        \begin{equation}
            \omega (fv + gw) = f \omega (v) + g \omega (w).
        \end{equation}
    \end{enumerate}
\end{definition}

The space of all 1-forms on $M$ is denoted as $\indexmath[Omega 1 (M)]{\Omega^1(M)}$, which is a module over $C^{(\infty)}(M)$.

The operator $\dif$, when given a $C^{(\infty)}(M)$ function (which is called a \indexbf{0-form}), would give a 1-form:
\begin{equation}
    (\dif f) (v) = v(f).
\end{equation}

This is called the \indexbf{exterior derivative} or \indexbf{differential} of $f$.

The \indexbf{cotangent vector} or \indexbf{covector} is similar as the tangent vector:
\begin{equation}
    \omega_p(v_p) = \omega(v)(p).
\end{equation}

The space of cotangent vectors at $p$ on $M$ is denoted by $\indexmath[Tp ast M]{T_p^*M}$.

1-forms are covariant, that is, if $\varphi \colon M \to N$, then the pushforward of a 1-form $\omega$ by $\varphi$ is
\begin{equation}
    (\varphi^* \omega)_p (v_p) = \omega_q(\varphi_* v_p),
\end{equation}
where $\varphi(p) = q$.

\begin{theorem}
    $f \in C^{(\infty)}(N)$, $\varphi \colon M \to N$ is differential, then
    \begin{equation}
        \varphi^* (\dif f) = \dif (\varphi^* f).
    \end{equation}
\end{theorem}

\section{Components of 1-Forms}

Let $\varphi \colon U \to \mathbb R^n$ be a chart of $M$ ($U \subset M$).

Let $p \in U$, $\varphi(p) = x = (x^\mu)$ ($\mu = 0, \ldots, n-1$).
Locally a 1-form $\omega \in \Omega^1(M)$ can be written as 
\begin{equation}
    (\varphi^{-1})^* \omega \in T^*_x \mathbb R^n.
\end{equation}

A natural way to impose a basis $\dif x^\mu$ in $T^*_x \mathbb R^n$ is 
\begin{equation}
    \dif x^\mu (\partial_\nu) = \delta^\mu_\nu,
\end{equation}
and $(\varphi^{-1})^*\omega = \omega_\mu(x) \dif x^\mu$.

Now by the definition of 1-form:
\begin{equation}
    \omega_\mu \dif x^{\mu} (v^\nu \partial_\nu)
    = v^\nu \omega_\mu \delta^\mu_\nu = v^\mu \omega_\mu.
\end{equation}

By the transformation rule of components of a vector, one have
\begin{equation}
    {\tau'}^\nu_\mu \alpha_\nu = \omega_\mu,
\end{equation}
where $\psi \colon U \to \mathbb R^n$, $(\psi^{-1})_*\omega = \alpha_\mu \dif x^\mu$, $\tau = \psi \circ \varphi^{-1}$.

\section{\texorpdfstring{$k$-Forms}{k-Forms}}

\begin{definition}
    If we assign an antisymmetric multilinear $k$-form $\omega_p \in \bigotimes_{i \in k} T^*_p M$ to each point $p \in M$, we say we have a \emphbf{$k$-form} on $M$.%
    \index{k-form@{$k$-form}}

    The collection of all $k$-forms is denoted by $\indexmath[Omega k (M)]{\Omega^k(M)}$, and $\indexmath[Omega(M)]{\Omega(M)} := \bigcup_{k \in \mathbb N} \Omega^k(M)$.
\end{definition}

\begin{theorem}[Dimension of forms]
    If $M$ is an $n$D manifold, then the dimension of $\Omega^k(M)$ is $\frac{n!}{k! (n - k)!}$ ($k \leq n$), and $0$ for $k > n$; The dimension of $\Omega(M)$ is $2^n$.
\end{theorem}

\begin{definition}[Wedge product]
    The \indexbf{wedge product} $\indexmath[wedge]{\wedge}$ is defined as a binary operater that takes a $k$-form and $\ell$-form and gives a $(k + \ell)$-forms, satisfying $\forall \alpha \in \Omega^k(M)$, $\forall \beta \in \Omega^\ell(M)$:
    \begin{enumerate}[label=(\alph*)]
        \item (Associativity) $\forall \gamma \in \Omega^m(M)$,
        \begin{equation}
            (\alpha \wedge \beta) \wedge \gamma = \alpha \wedge (\beta \wedge \gamma).
        \end{equation}
        \item (Supercommutativity) 
        \begin{equation}
            \alpha \wedge \beta = (-1)^{k \ell} \beta \wedge \alpha.
        \end{equation}
        \item (Distributiveness) $\forall \gamma \in \Omega^\ell(M)$,
        \begin{equation}
            \alpha \wedge (\beta + \gamma) = \alpha \wedge \beta + \alpha \wedge \gamma.
        \end{equation}
        \item (Bilinearity over $C^{(\infty)}(M)$) $\forall f \in C^{(\infty)}(M)$,
        \begin{equation}
            (f \alpha) \wedge \beta = f (\alpha \wedge \beta).
        \end{equation}
        \item (Naturality) If $\varphi \colon M \to N$ is a smooth map, then the pullback of a form by $\varphi$ can be given by repeatingly applying ($\forall \gamma \in \Omega^\ell(M)$)
        \begin{equation}
            \begin{aligned}
                \varphi^* (\beta + \gamma) &= \varphi^* \alpha + \varphi^* \beta
                \\ 
                \varphi^* (\alpha \wedge \beta) &= \varphi^* \alpha \wedge \varphi^* \beta,
            \end{aligned}
        \end{equation}
        while the pullback of a 0-form and a 1-form agree with what we have already defined before.
    \end{enumerate}
\end{definition}

By convention if $f \in C^{(\infty)}(M)$ then
\begin{equation}
    f \wedge \omega =: f\omega.
\end{equation}

It can be shown that any $k$-form $\omega$ can be written as
\begin{equation}
    (\varphi^{-1})^* \omega = \frac{\omega_{\mu_1 \cdots \mu_k}}{n!} \bigwedge_{i = 1}^k \dif x^{\mu_i},
\end{equation}
where $\varphi \colon M \to \mathbb R^n$ is a chart.

\begin{definition}[Interior product]\label{def: interior product}
    Let $v \in \Gamma(TM)$, we can define the \indexbf{interior product} $\indexmath[iv]{i_v} \colon \Omega^k(M) \to \Omega^{k - 1}(M)$ by:
    $\forall \omega \in \Omega^k(M)$, $\forall v_i$ ($i \in k - 1$):
    \begin{equation}
        i_v(\omega) (v_0, \ldots,v_{k-2}) = \omega (v, v_0, \ldots, v_{k-2}).
    \end{equation}

    Specially, if $k = 0$, then $i_v(\omega) = 0$.
\end{definition}

\begin{theorem}\label{theorem: properties of interior product}
    $\forall v \in \Gamma(TM)$, 
    \begin{enumerate}
        \item $i_v$ is a $C^{(\infty)}(M)$-linear function;
        \item $\forall \alpha \in \Omega^k(M)$, $\forall \beta \in \Omega(M)$,
        \begin{equation}
            i_v (\alpha \wedge \beta) = i_v (\alpha) \wedge \beta + (-1)^k \alpha \wedge i_v (\beta).
        \end{equation}
    \end{enumerate}
\end{theorem}

\section{Exterior Derivative}
\begin{definition}[Exterior derivative]
    The \indexbf{exterior derivative} $\indexmath[exterior derivative]{\dif}$ is defined as a linear operator that takes a $k$-form and gives a $(k + 1)$-form, satisfying $\forall \alpha \in \Omega^k(M)$, $\forall \beta \in \Omega^\ell(M)$:
    \begin{enumerate}[label=(\alph*)]
        \item (Linearity) $\forall \lambda, \mu \in \mathbb R$, $\forall \gamma \in \Omega^\ell(M)$,
        \begin{equation}
            \dif (\lambda\beta + \mu\gamma) = \lambda \dif \alpha + \mu \dif \beta.
        \end{equation}
        \item (Leibniz rule)
        \begin{equation}
            \dif (\alpha \wedge \beta) = \dif \alpha \wedge \beta + (-1)^k \alpha \wedge \dif \beta.
        \end{equation}
        \item
        \begin{equation}
            \dif[2] \omega = 0.
        \end{equation}
        \item (Naturality) If $\varphi \colon M \to N$ is a smooth map, then 
        \begin{equation}
            \varphi^* \dif \omega = \dif \varphi^* \omega.
        \end{equation}
    \end{enumerate}
    
\end{definition}

\section{Derivation ana Antiderivation}
\begin{definition}[Derivation]\label{def: derivation}\index{derivation}
    A map $\theta \colon \Omega(M) \to \Omega(M)$ is called a \indexbf{derivation of degree $k$} ($k \in \mathbb Z$) if
    $\forall p \in \mathbb N$, $\theta[\Omega^p(M)] \subset \Omega^{p+k}(M)$
    and, $\theta$ is a homomorphism of the $\mathbb R$-exterior algebra.
    Or, explicitly, $\theta$ is $\mathbb R$-linear, and $\forall \alpha, \beta \in \Omega(M)$,
    \begin{equation}
        \theta (\alpha \wedge \beta) 
        = \theta (\alpha) \wedge \beta
            + \alpha \wedge \theta (\beta).
    \end{equation}
\end{definition}

\begin{definition}[Antiderivation]\label{def: antiderivation}\index{antiderivation}
    A map $\theta \colon \Omega(M) \to \Omega(M)$ is called an \indexbf{antiderivation of degree $k$} ($k \in \mathbb Z$) if 
    \begin{enumerate*}[label=\roman*)]
        \item $\forall p \in \mathbb N$, $\theta[\Omega^p(M)] \subset \Omega^{p+k}(M)$, 
        \item $\theta$ is $\mathbb R$-linear, 
        \item $\forall \alpha \in \Omega^p(M)$,
        $\beta \in \Omega(M)$,
    \end{enumerate*}
    \begin{equation}
        \theta (\alpha \wedge \beta) 
        = \theta (\alpha) \wedge \beta
            + (-1)^p \alpha \wedge \theta (\beta).
    \end{equation}
\end{definition}

\chapter{Metric}
\section{Pseudo-Riemannian Metric}
\begin{definition}[Psedudo-Riemannian metric]
    Let $M$ be a manifold.
    A \indexbf{pseudo-Riemannian metric} or simply \indexbf{metric} $g$ on a manifold $M$ is a field ($g \in \Gamma(T^*M \otimes T^*M)$) that $\forall p \in M$, 
    \begin{equation}
        g_p \colon T^*_p M \times T^*_p M \to \mathbb R, 
    \end{equation}
    is a bilinear form satisfying the following properties:
    \begin{enumerate}[label=(\alph*)]
        \item (Symmetry) $\forall u, v \in T_p M$,
        \begin{equation}
            g_p(u, v) = g_p(v, u).
        \end{equation}
        \item (Non-degenerate)
        \begin{equation}
            u \mapsto g_p(u, \dummy) \colon T_p M \to T^*_p M 
        \end{equation}
        is an isomorphism.
        \item (Bilinearity) $\forall p \in M$, $\forall u, v \in T_p M$, $\forall \lambda, \mu \in \mathbb R$,
        \begin{equation}
            g_p(\lambda u + \mu v, w) = \lambda g_p(u, w) + \mu g_p(v, w).
        \end{equation}
        \item (Smoothness) If $v, u \in \Vect(M)$, then
        \begin{equation}
            p \mapsto g_p(v_p, u_p) \in C^{(\infty)}(M).
        \end{equation}
    \end{enumerate}
\end{definition}

Given a metric, $\forall p \in M$, we can always find an orthonormal basis $\{e_\mu\}$ of $T_p M$ such that
\begin{equation}
    g_p(e_\mu, e_\nu) = \sign(\mu) \delta_{\mu\nu},
\end{equation}
where $\sign(\mu) = \pm 1$.
Conventionally we order the basis such that $\sign(\mu) = 1$ for $\mu \in s$ and $\sign(\mu) = -1$ for $\mu - s \in n - s$, and say that the metric has \indexbf{signature} $(s, n - s)$.

If $\gamma \colon [0, 1] \to M$ is a smooth path and $\forall t, s \in [0, 1]$,
\begin{equation}
    g(\gamma'(t), \gamma'(t)) g(\gamma'(s), \gamma'(s)) \geq 0,
\end{equation}
then we can define the arclength of $\gamma$ as
\begin{equation}
    \int_0^1 \sqrt{|g(\gamma'(t), \gamma'(t))|} \dif t
\end{equation}
if the integral converges.

The metric gives an \indexbf{inner product} on $\Vect(M)$:
\begin{equation}
    \langle u, v \rangle := g(u, v).
\end{equation}

The metric also gives a way to relate a vector field $v$ to a 1-form $\omega$.
If $v$ and $\omega$ satisfies:
$\forall u \in \Vect(M)$, 
\begin{equation}
    g(v, u) = \omega(u),
\end{equation}
then we say that $v$ is the corresponding vector field of $\omega$, denoted by $v = \indexmath[omega sharp]{\omega^\sharp}$, and $\omega$ is the corresponding 1-form of $v$, denoted by $\omega = \indexmath[v flat]{v^\flat}$.

We can also define the \indexbf{inner product} on $\Omega^1(M)$ by 
\begin{equation}
    \langle \alpha, \beta\rangle = \langle a, b\rangle,
\end{equation}
where $a$ and $b$ is the corresponding vector fields of $\alpha$ and $\beta$.

The \indexbf{inner product}%
    \footnote{This inner product makes $\Omega_p(M)$ for each $p \in M$, yet not for $\Omega(M)$. The full inner product requires integration over $M$.}
on $\Omega^k(M)$ is defined by induction with
\begin{equation}
    \langle \bigwedge_{i \in k} \alpha_i, \bigwedge_{i \in k} \beta_i \rangle 
    = \det(\langle \alpha_i, \beta_j \rangle)_{i, j \in k}.
\end{equation}

Hence, if $\{e_\mu\}$ is an orthonormal basis (field) of $T_p M$, while the corresponding covectors are $\{f^\mu\}$ ($f^\mu(e_\nu) = \delta^\mu_\nu$) then
\begin{equation}
    \langle \bigwedge_{i \in k} f^{\mu_i}, \bigwedge_{i \in k} f^{\mu_i}\rangle
    = \prod_{i \in k} \sign(\mu_i).
\end{equation}

Specially, when $f, g \in \Omega^0(M) = C^{(\infty)}(M)$,
\begin{equation}
    \langle f, g \rangle = fg.
\end{equation}

\section{Volume Form}

Notice that if $M$ is an $n$D manifold, $\dim \Omega^n(M) = 1$, meaning at $p \in M$, $\{\omega_p \mid \omega \in \Omega^n(M)\}$ can be labelled by a parametre $\lambda_p \in \mathbb R$. 
If we have a basis $\{f^\mu\}$ of $T^*_p M$ (or corresponding vectors $\{e_\mu\}$), then
\begin{equation}
    \{\omega_p \mid \omega \in \Omega^n(M)\} = \lambda_p \bigwedge_{\mu \in n} f^\mu.
\end{equation}

If there were another basis $\{g^\mu\}$ of $T^*_p M$ (or corresponding vectors $\{h_\mu\}$), and the transformation between the two bases is given by
\begin{equation}
    P e^\mu = f^\mu,
\end{equation}
where $P \in \Aut(T^*_p M)$.
When $\det P > 0$, we say that $\{f^\mu\}$ and $\{g^\mu\}$ have the same \indexbf{orientation}.

%TODO: how to give a oriented basis field without a chart?


\begin{definition}[Volume form]
    Let $M$ be an orientable manifold.
    If $\forall p \in M$, we find an oriented orthonormal basis $\{f_\mu\}$ of $T_p^* M$ at point $p$, 
    then the \indexbf{volume form} $\indexmath[vol]{\vol}$ is defined by
    \begin{equation}
        \bigwedge_{\mu \in n} f_\mu = \vol_p.
    \end{equation}
\end{definition}

\section{Hodge Star Operator}

\begin{definition}[Hodge Star Operator]
    Let $M$ be an orientable manifold.
    The \indexbf{Hodge star operator} $\indexmath[star]{\star}$ is defined by the linear map
    \begin{equation}
        \star \colon \Omega^k(M) \to \Omega^{n - k}(M),
    \end{equation}
    $\forall \alpha, \beta \in \Omega^k(M)$,
    \begin{equation}
        \alpha \wedge \star \beta = \langle \alpha, \beta \rangle \vol.
    \end{equation}
    
    We call $\star \omega$ the \indexbf{dual} of $\omega$.
\end{definition}

The special case is when $k = 0$,
\begin{equation}
    \star f = f \vol,
\end{equation}
and $k = n$,
\begin{equation}
    \star (f \vol) = f \prod_{\mu \in n} \sign(\mu) = (-1)^{n - s} f
\end{equation}
if the signature of the metric is $(s, n - s)$ ($s$ positives and $n - s$ negatives).

The Hodge star operator is also called the \indexbf{Hodge duality} because:
\begin{theorem}\label{theorem: hodge duality}
    $\forall \alpha \in \Omega^p(M)$,
    \begin{equation}
        \star \star \alpha = (-1)^{p(n - p)} \alpha\sign(g),
    \end{equation}
    where $\sign(g) := \det g / |\det g|$.
\end{theorem}

In local coordinates, 
\begin{equation}
    \star \alpha 
        = \frac {\varepsilon_{i_0\cdots i_{n - 1}}} {p!}
            \alpha_{j_0 \cdots j_{p - 1}}\prod_{k \in p} g^{i_k j_k}
            \sqrt{- \det g} \bigwedge_{\ell \in n \backslash p} \dif x^{i_\ell}.
\end{equation}

\begin{definition}[Codifferential]
    Let $M$ be an orientable manifold.
    The \indexbf{codifferential} $\indexmath[delta]{\delta}$ is defined by
    \begin{equation}
        \delta \colon \Omega^k(M) \to \Omega^{k - 1}(M),
    \end{equation}
    $\forall \alpha \in \Omega^k(M)$,
    \begin{equation}
        \delta \alpha = \star \dif \star \alpha.
    \end{equation}
    
\end{definition}

\begin{definition}[Laplacian]
    The \indexbf{Laplacian} $\indexmath[square]{\square}$ is defined by
    \begin{equation}
        \square := \dif \circ \delta + \delta \circ \dif.
    \end{equation}
\end{definition}

\begin{theorem}
    \begin{equation}
        \square \circ \star = \star \circ \square,
    \end{equation}
    \begin{equation}
        \square \circ \delta = \delta \circ \square,
    \end{equation}
    \begin{equation}
        \square \circ \dif = \dif \circ \square.
    \end{equation}
\end{theorem}

% self-adjointness of the Laplacian, and d and delta are adjoint to each other (page 660 of GR)

\section{Metric and Coordinates}

\chapter{DeRham Theory}
\section{Closed and Exact 1-Forms}

\begin{definition}[Closed and exact forms]
    Consider $\dif \colon \Omega(M) \to \Omega(M)$.
    The differential forms in $\ker \dif$ is said to be \indexbf{closed}, and the differential forms in $\dif(\Omega(M))$ is said to be \indexbf{exact}.
\end{definition}

For closed form:
\begin{equation}
    \dif \omega = 0,
\end{equation}

For exact form:
\begin{equation}
    \exists \alpha \in \Omega(M),
    \quad
    \omega = \dif \alpha,
\end{equation}
where $\alpha$ is often called \emph{potential}.



We want to study, 
given two points $p$, $q$ that are located in the same arcwise connected component of $M$, 
and a smooth path $\gamma \colon [0, 1] \to M$ s.t.\ $\gamma(0) = p$, $\gamma(1) = q$,
for a closed 1-form $E$,
\begin{equation}
    \phi(p, q) := - \int_\gamma E := - \int_0^1 E_{\gamma(t)} (\gamma'(t)) \,\dif t.
\end{equation}

We want to know that how $\phi$ depends on the choice of $\gamma$.


Assumes that there are two smooth paths $\gamma_1$ and $\gamma_2$ connecting $p$ and $q$,
and a fix-ends smooth homotopy $H \colon [0, 1] \times [0, 1] \to M$ s.t.\ 
\begin{equation}
    H(0, t) = \gamma_1(t),
    \quad
    H(1, t) = \gamma_2(t),
    \quad
    H(s, 0) = p,
    \quad
    H(s, 1) = q.
\end{equation}

By choosing proper charts (if there is no chart that can cover the whole path, we break the path into pieces),
\begin{align}
    I_s 
    &= \int_{H(s, \dummy)} E 
    = \int_0^1 E_{H(s, t)}(H'(s, t)) \dif t
    \\
    &= \int_0^1 E_\mu [H(s, t)] \partial_t H^\mu(s, t) \dif t,
\end{align}
where $H'(s, t)$ is the tangent vector of $H(s, \dummy)$ at $t$.

\begin{align}
    \diff{I_s}{s}
    &= \diff{} s \int_0^1 E_\mu [H(s, t)] \partial_t H^\mu(s, t) \dif t
    \\
    &= \int_0^1 \big(
        \partial_s E_\mu [H(s, t)] \partial_t H^\mu
        +
        E_\mu [H(s, t)] \partial_s\partial_t H^\mu
    \big) 
    \dif t
    \\
    &= \partial_s \left.\big(
      E_\mu(H(s, t)) H^\mu(s, t)  
    \big)\right|^{t = 1}_{t=0}
    \\
    &\phantom{= {}}
    + \int_0^1 \big(
        \partial_s E_\mu [H(s, t)] \partial_t H^\mu
        -
        \partial_t E_\mu [H(s, t)] \partial_s H^\mu
    \big) 
    \dif t
    \\
    &=
    \partial_s (E_\mu(q) q^\mu - E_\mu(p) p^\mu)
    \\
    &\phantom{= {}} 
    + \int_0^1 \partial_\nu E_\mu \big(
        \partial_s H^\nu  \partial_t H^\mu
        -
        \partial_t H_\nu \partial_s H^\mu
    \big) \,\dif t
    \\
    &= \int (\dif E)_{\mu\nu} \partial_s H^\mu \partial_t H
    = 0.
\end{align}

Now we have proven that if $\gamma_1$ and $\gamma_2$ are homotopic, then the integral for $\phi(p, q)$ is the same.

Then, if $M$ is simply connected, then a closed form $E$ is also exact, and
\begin{equation}
    E = - \dif \phi(p, \dummy).
\end{equation}


\section{Stokes' Theorem}
\section{DeRham Cohomology}

We have shown that, if the manifold is simply connected, then a closed 1-form must also be exact.
The study of whether a closed form is exact is called the \indexbf{deRham cohomology}.

Since $\dif \circ \dif = 0$, we know that
\begin{equation}
    \dif(\Omega(M)) \subset \dif(\ker \dif).
\end{equation} 

% One has an short exact sequence

% % https://tikzcd.yichuanshen.de/#N4Igdg9gJgpgziAXAbVABwnAlgFyxMJZABgBpiBdUkANwEMAbAVxiRGJAF9T1Nd9CKAIzkqtRizYAdKQGsYAJwAEULjxAZseAkQBMo6vWatEIGQHkAtjADmdABQBZAJRreWgUQDMB8celSUFgAZvYW1nZOzq7c7vw6KAAsvkaSphycYjBQNvBEoMEKEJZIZCA4EEhCsSCFxVXUFUj6fmlmUvg4dG61RSWILU2IPq0m7UHBPXX9I0OJmZxAA
% \begin{tikzcd}
%     0 \arrow[r] & \ker \dif \arrow[r, "\iota"] & \Omega(M) \arrow[r, "\dif"] & \dif(\Omega(M)) \arrow[r] & 0
% \end{tikzcd},

The space of exact $p$-forms is denoted by $\indexmath[Bp(M)]{B^p(M)}$ and the space of closed $p$-forms is denoted by $\indexmath[Zp(M)]{Z^p(M)}$.

\begin{definition}[DeRham cohomology]
    The $p$-th \indexbf{deRham cohomology} of $M$ is defined as
    \begin{equation}
        \indexmath[Hp(M)]{H^p(M)} = Z^p(M) /B^p(M).
    \end{equation}
\end{definition}

Every element of $H^p(M)$ is a \indexbf{cohomologous class}:
\begin{equation}
    [\omega] = \{
        \omega' \in Z^p(M) \mid \omega - \omega' \in B^p(M)
    \}.
\end{equation}

For $p = 0$, $B^0(M) = \{0\}$ (there is no ($-1$)-form), and $H^0(M) = Z^0(M)$, where $Z^0(M)$ is made of $f$ that is constant in every connected components of $M$.
Let $\chi_i$ be the characteristic function of $M$'s $i$th connected components $M_i$ (we assume that $\{M_i\}$ is finite)
\begin{equation}
    H^0(M) = Z^0(M) = \left\{
        f
    \middle\vert
        f = x^i \chi_i
    \right\}
    \cong \mathbb R^n,
\end{equation}
where $n$ is the number of connected components of $M$.

\chapter{Bundles and Connections}
\section{Fibre Bundles}
\begin{definition}[Bundle]
    A \indexbf{bundle} is a triple $(E, \pi, B)$, where $\pi \colon E \to B$ is a surjective map.
    $E$ is called the \indexbf{total space}, $\pi$ is called the \indexbf{projection map}, and $B$ is called the \indexbf{base space}.

    A bundle $(E, \pi, B)$ can be denoted as $\pi \colon E \to B$ or $\begin{tikzcd}
        E \arrow[r, "\pi"] & B
    \end{tikzcd}$.
\end{definition}

\begin{definition}[Fibre]
    For $p \in B$, $\pi^{-1}(\{p\})$ is the \indexbf{fibre} over $b$.
\end{definition}

\begin{definition}[Subbundle]
    Let $\pi \colon E \to B$ be a bundle. $F \subset E$, $C \subset B$, $\rho \colon F \to C$.
    If $\pi|_{C} = \rho$, then $\rho \colon F \to C$ is called a \indexbf{subbundle} of $\pi \colon E \to B$.
\end{definition}

\begin{definition}[Section]
    A \indexbf{section} is a map $s \colon B \to E$ such that
    \begin{equation}
        p \circ s = \id_B.
    \end{equation}

    All sections of a bundle $\pi \colon E \to B$ is denoted as $\indexmath[Gamma(E)]{\Gamma(E)}$.
\end{definition}


\begin{definition}[Fibre bundle]
    A \indexbf{fibre bundle} $(E, \pi, B, F)$ is a bundle $\pi \colon E \to B$, where $E$, $B$, $F$ are topology spaces, and $\pi$ is a continuous map, and $\forall p \in B$, $\exists U \in \mathscr U(p)$ s.t.\ 
    \begin{equation}
        \varphi \colon \pi^{-1}(U) \to U \times F,
    \end{equation}
    is a homeomorphism and $\pi_1 \circ \varphi = \pi$.
    $\pi_1$ is defined as $\pi_1(p, q) = p$.

    A fibre bundle can be denoted as the exact sequence
    \begin{equation}
        \begin{tikzcd}
            F \arrow[r] & E \arrow[r, "\pi"] & B
        \end{tikzcd}
    \end{equation}

    The last condition is called the \indexbf{local triviality condition}.
    $F$ is called the \indexbf{standard fibre}
\end{definition}

If $E = B \times F$, then $(E, \pi, B, F)$ is called a \indexbf{trivial fibre bundle}.

\begin{definition}[Morphism]
    Let $\pi \colon E \to B$, $\rho \colon F \to C$ be two fibre bundles.
    A \indexbf{morphism} $(\varphi, \psi)$ is a pair of two continuous maps such that
    \begin{equation}
        \begin{tikzcd}
            E \arrow[d, "\pi"] \arrow[r, "\psi"] & F \arrow[d, "\rho"] \\
            B \arrow[r, "\varphi"]               & C                  
        \end{tikzcd}
    \end{equation}
    commutes.
\end{definition}

\section{Vector Bundles}
\begin{definition}[Vector bundle]
    A \indexbf{vector bundle} is a fibre bundle $(E, \pi, B, F)$, where $F$ is a vector space, and the local trivialisation $\varphi \colon \pi^{-1}(U) \to U \times F$ ($U$ is a neibourhood of $p \in B$) satisfies that $\forall x \in U$, $\forall v \in F$, 
    \begin{equation}
        \begin{aligned}
            F &\to \pi^{-1}(\{x\}) \\ 
            v &\mapsto \varphi^{-1}(x, v)
        \end{aligned}
    \end{equation}
    is a linear isomorphism (\indexbf{fibrewise linear}).
\end{definition}

\begin{definition}[Morphism (vector bundle)]
    A morphism from a vector bundle $(E, \pi, B, F)$ to $(E', \pi', B', F')$ is a morphism (of fibre bundles) $(\varphi, \psi)$ such that $\forall x \in B$, 
    \begin{equation}
        \psi_* \colon \pi^{-1}(\{x\}) \to (\pi')^{-1}(\{\varphi(x)\})
    \end{equation}
    is a linear homomorphism.
\end{definition}

\begin{definition}[Smooth vector bundle]
    A \indexbf{smooth vector bundle} is a vector bundle $(E, \pi, B, F)$, where the projection $\pi \colon E \to B$ and the local trivialisation $\varphi \colon \pi^{-1}(U) \to U \times F$ are smooth.
\end{definition}

\begin{definition}[Tangent bundle]
    The \indexbf{tangent bundle} $\indexmath[TM]{{TM}}$ is the smooth vector bundle over an $n$D smooth manifold $M$ with the standard fibre $T_p M = \mathbb R^n$.
\end{definition}

A vector field $v \in \Vect(M)$ is the smooth section of the tangent bundle $\Gamma(TM)$.

\begin{definition}[Cotangent bundle]
    The \indexbf{cotangent bundle} of an $n$D manifold $M$, 
    denoted by $\indexmath[T ast M]{{T^*M}}$, 
    is the smooth vector bundle over with the standard fibre $T^*_p M = (\mathbb R^n)^*$.
\end{definition}

A 1-form $\omega \in \Omega^1(M)$ is the smooth section of the cotangent bundle $\Gamma(T^*M)$.

\section{Constructions of Vector Bundles}
% \begin{definition}[Duality]
%     %TODO
% \end{definition}

We use local trivialisation to destruct a vector bundle into trivial bundles.
We can also construct a vector bundle by ``gluing'' trivial bundles.
We must make sure that in the intersections of bases, we must make sure that they are compactible by introducing \emph{transition functions} to relate points on the fibres.
Naturally, transition functions make a group structure.

\begin{definition}[$G$-bundle]\label{definition: G-bundle}
    Consider an open cover $\mathcal U = \{U_i \mid i \in I\}$ of the manifold $M$. For each $i \in I$, there is a trivial vector bundle $\begin{tikzcd}
        U_i \times V \arrow[r, "\pi_i"] & U_i
    \end{tikzcd}$
    with vector fibre $V$.
    $\rho \colon G \to \GL(V)$ is a representation of $G$ on $V$.    

    For any $p \in M$, if $p \in \bigcap_{j \in J} U_j$ ($J \subset I$),
    then $\pi^{-1}(\{p\})$ is identified 
    by a equivalence class in $\bigsqcup_{j \in J} \pi_j^{-1} (p)$ 
    where two points are equivalent if they are related by the transformation
    \begin{align}
        \rho_*(g_{jj'}(p)) \colon & U_j \times V \to U_{j'} \times V;
        \\
                                & (p, v) \mapsto (p, \rho(g_{jj'}(p)v)),
    \end{align}
    where the \indexbf{transition functions} $g_{ij} \in G$ satisfy that:
    \begin{enumerate}
        \item $g_{ii} = 1$;
        \item $g_{ij} g_{jk} g_{ki} = 1$.
    \end{enumerate}

    The bundle $\begin{tikzcd}
        E \arrow[r, "\pi"] & M
    \end{tikzcd}$ is then called the \emphbf{$G$-bundle}\index[G-bundle]{$G$-bundle},
    the element of which is denoted as $\indexmath[p, vp]{[p, v_p]}$ for some $v_p \in U_i$
    , where $G$ is the \indexbf{gauge group} or \textbf{structure group}
\end{definition}

One can show that the $G$-bundles are also vector bundles.

Consider transformations of the sections of the $G$-bundle.
If $T \colon E_p \to E_p$ can be expressed by \emph{some} $g \in G$ s.t.\ 
\begin{equation}
    T([p, v_p]) = [p, \rho(g) v_p],
\end{equation}
then we say $T$ \indexbf{lives in} $G$. 
Similarly we can define when $T$ lives in $\mathfrak{g}$.

Notice that we do not specify which $g \in G$ corresponds to $T$, because we have the freedom to choose the $v_p$ as the representative of the equivalence class, and for different $v_p$, we have different $g \in G$.

If, $\forall p \in M$, we have $T_p \colon E_p \to E_p$ that $T_p$ lives in $G$, we call $T$ a \indexbf{gauge transformation}.
The set of all gauge transformations is denoted as $\indexmath[G]{\mathcal G}$\footnote{physicists call it the \indexbf{gauge group}, as opposite to $G$.}.

\section{Principal Bundles}

\begin{definition}[Principal bundle]
    Let $G$ be a topological group, $\begin{tikzcd}
        P \arrow[r, "\pi"] & M
    \end{tikzcd}$ be a fibre bundle.
    If there is a continuous right action $P \times G \to P$ s.t.\ $\forall x \in M$, $\forall p, q \in P_x$
    \begin{equation}
        P_x g \subseteq P_x,
        \quad
        \exists g \in G 
        \;\text{s.t.\ }
        p g = q,
        \quad
        \forall g \in G\;
        {p g = p} \to {g = 1},
    \end{equation}
    that is, $P_x$ is a $G$-torsor, and, $\forall x \in M$, the right action $G \to P_x$ is a homeomorphism, then we say $P$ is a \indexbf{principal $G$-bundle}.
\end{definition}

The prototypical example of a principal bundle is the \indexbf{frame bundle}, that is a principal $\GL(\mathbb R, n)$-bundle on a $n$-D manifold $M$.

\begin{theorem}
    Let $P$ be a (smooth) manifold, $G$ a topological group with (smooth) left action on $P$:
    \begin{equation}
        G \times P \to P,
    \end{equation}
    which is free and transitive, that is $\forall p, q \in P$, 
    \begin{equation}
        \exists g \in G \;\text{s.t.\ } g p = q,
        \quad
        \forall g \in G\; {g p = p} \to {g = 1},
    \end{equation}
    then $P$ is a principal $G$-bundle, with base space $M = P / G$.
\end{theorem}

We are more interested in the structure group than in the fibre, hence we can compare two bundles with the same structure group but different fibres:

\begin{definition}[Associated bundle]
    If the structure group $G$ of a bundle $E \to M$ has an (smooth) action on $F'$, one can construct a new bundle $E' \to M$ with fibre $F'$ and with structure group $G$, called the \indexbf{associated bundle} of $E$ with group $G$ with fibre $F'$.
\end{definition}

\begin{theorem}
    All fibre bundles can be constructed by the associated bundle of some principal bundles.
\end{theorem}

The theorem reduce the classification of fibre bundle to the classification of principal bundles.

\section{Connections}

\begin{definition}[Connection]
    A \indexbf{connection} on a smooth vector bundle $(E, \pi, M, F)$ is map
    \begin{equation}
        D \colon \Gamma(TM) \times \Gamma(E) \to \Gamma(E),
    \end{equation}
    that satisfies the following conditions:
    $\forall v, w \in \Gamma(TM)$, $\forall s, t \in \Gamma(E)$, $\forall f \in C^{(\infty)}(M)$,
    \begin{enumerate}[label=(\alph*)]
        \item $D_v (s + t) = D_v s + D_v t$;
        \item $D_v (fs) = v(f) s +  f D_v s$;
        \item $D_{v + w} s = D_v s + D_w s$;
        \item $D_{fv} s = f D_v s$.
    \end{enumerate}
\end{definition}

When a vector field $v \in \Gamma(TM)$ is given to the connection $D$, the map $D_v \colon \Gamma(E) \to \Gamma(E)$ is called the \indexbf{covariant derivative} with respect to $v$.

If $E = TM$, the connection is called a \indexbf{affine connection}.

\begin{definition}[Vector potential]
    A \indexbf{vector potential} $A$ is an $\End(E)$-valued 1-form, that is
    \begin{equation}
        A \in \Gamma(\End(E) \otimes T^*M),
    \end{equation}
    where $\End(E) \cong E \otimes E^*$ can be considered as a vector bundle over $M$ with the standard fibre $\End(E_p) \cong E_p \otimes E_p^*$ ($p \in E$).
\end{definition}

Locally if $s \in \Gamma(E)$ we can have a trivialisation $\varphi \colon E|_U \to U \times F$ ($U \subset M$).
If we assign a basis $\{f_i\}_{i \in m}$ for the $m$D standard fibre $F$, then 
\begin{equation}
    s = s^i e_i := s^i\varphi^{-1}(f_i),
    \quad s^i \in C^{(\infty)}(U),
\end{equation} 
where we can call $\{s^i\}_{i \in m}$ the \indexbf{components of the section} $s$.
With this specific normalisation, one can define that
\begin{equation}
    D^0_v s = v(s^i) e_i
\end{equation}
where $\indexmath[D0]{D^0}$ is called the \indexbf{standard flat connection} (which depends on trivialisation).


\begin{theorem}
    Let $(E, \pi, M, F)$ be a smooth vector bundle.
    If $D$ is a connection on $E$, $A \in \Gamma(\End(E)) \otimes T^*M$, then
    the $D + A$, which defined as
    \begin{equation}
        D + A \colon (v, s)  \mapsto D_v s + A(v) s,
    \end{equation}
    is also a connection.
\end{theorem}

\begin{theorem}
    Let $(E, \pi, M, F)$ be a smooth vector bundle, and $D^0$ is the standard flat connection on $U \subset E$ with the trivialisation $\varphi \colon E|_U \to U \times F$.
    If $D$ is a connection on a $(E, \pi, M, F)$,
    then $\exists A \in \Gamma(\End(E|_U)) \otimes T^*U$ s.t.\ 
    \begin{equation}
        D|_U = D^0 + A.
    \end{equation}
\end{theorem}

\begin{definition}[$G$-connection]
    Let $E$ be a $G$-bundle, we define a \emphbf{$G$-connection}\index{G-connection@$G$-connection} as a connection $D$ on $E$ that
    \begin{equation}
        D = D^0 + A,
    \end{equation}
    where in any local coordinates $A = A_\mu \dif x^\mu$, $A_\mu \in \End(E)$ lives in $\mathfrak g$.
\end{definition}

\begin{definition}[Gauge transformation of $G$-connection]
    Let $E$ be a $G$-bundle with $G$-connection $D$. 
    If $g \in \mathcal G$ is a gauge transformation, then
    \begin{equation}
        D'_v (s) = g D_v(g^{-1}s) 
    \end{equation}
    is also a $G$-connection, and we say that $D$ and $D'$ are \indexbf{gauge equivalent}.
\end{definition}

In local coordinates, 
\begin{equation}
    A'_\mu = g A_\mu g^{-1} + g \partial_\mu g^{-1}.
\end{equation}

Let $\mathcal A$ be the space of all $G$-connections on $E$, then we say $\mathcal A / \mathcal G$ is the space of connections modulo gauge transformation.

Given a connection $D$ on $E$, we can construct conncections for different structures built upon $E$.

The \indexbf{dual connection} $D^*$ on $E^*$ is defined as
\begin{equation}
    (D^*_v \sigma) (s) = v[\sigma(s)] - \sigma (D_v s),
\end{equation}
where $v \in \Gamma(TM)$, $s \in \Gamma(E)$, $\sigma \in \Gamma(E^*)$.

The \indexbf{direct sum of connections} $D \oplus D'$ on $E \oplus E'$ is defined as
\begin{equation}
    (D \oplus D')_v (s \oplus s') = D_v s \oplus D'_v s',
\end{equation}
where $v \in \Gamma(TM)$, $s \in \Gamma(E)$, $s' \in \Gamma(E')$.

The \indexbf{tensor product of connections} $D \otimes D'$ on $E \otimes E'$ is defined as
\begin{equation}
    (D \otimes D')_v (s \otimes s') = D_v s \otimes s' + s \otimes D'_v s',
\end{equation}
where $v \in \Gamma(TM)$, $s \in \Gamma(E)$, $s' \in \Gamma(E')$.

Since $\End(E) \cong E \otimes E^*$, the connection $D$ (we use the same symbol for $D$ on $E$) on $\End(E)$ can be shown to 
\begin{equation}
    D_v T (s) = D_v(Ts) - T(D_v s),
\end{equation}
where $v \in \Gamma(TM)$, $s \in \Gamma(E)$, $T \in \Gamma(\End(E))$.

Let $S \subset M$ be a submanifold, 
and $E$ a vector bundle over $M$, 
with connection $D$, while the restriction of $E$ to $S$ is $E|_S$.
Note that even if $v \in \Gamma(TS)$, $s \in \Gamma(E|_S)$, 
$D_v s $ is not necessarily in $\Gamma(E|_S)$.

\begin{definition}[Projection of connection]
    Let $N$ be a submanifold of $M$, and $\nabla$ be the affine connection of $M$.
    If we have a projection $P \in \Gamma(T^*M \otimes TN)$ that linear maps a vector field on $M$ to a vector field on $N$, and $\forall v \in \Vect(N)$, $Pv = v$,
    then the \indexbf{projection} of $\nabla$ onto $N$ is defined as
    \begin{equation}
        \forall v \in \Gamma(TN), \quad
        \nabla^{(P)}_v = P \circ \nabla_v
    \end{equation}
\end{definition}

\begin{definition}[Principal connection]
    Let $P \to M$ be a smooth principal $G$-bundle, where the right-action $p \mapsto p g$ will be written as $R_g$.
    If a $\mathfrak g$-valued 1-form $\omega$ on $P$, that is, $\omega \in \Omega^1_{\mathfrak g}(P) \cong \Gamma(\mathfrak g \otimes T^*P)$ that can be written as
    \begin{equation}
        \omega_p = X_p \dif f_p,
        \quad 
        p \in P, \,
        X_p \in \mathfrak g,\,
        f \in C^{(\infty)}(P) \cong \Omega^0(P),
    \end{equation}
    or
    \begin{equation}
    \begin{aligned}
        \omega \colon & \Vect(P) \to C^{(\infty)}(P),
        \\
        & v \mapsto (p \mapsto v(f)(p) X_p),
    \end{aligned}
    \end{equation}
    s.t.\ 
    $\forall g \in G$, 
    \begin{equation}
        \Ad_g(R^*_g \omega) = \omega,
    \end{equation}
    and, $\forall p \in P$, if $\gamma \colon \mathbb R \to P$ is a smooth path s.t.\ $\gamma(0) = p$, and $\forall t \in \mathbb R$, $\exists g_t \in G$, $\gamma(t) = p g_t = R_{g_t}(p)$ and such $t \mapsto g_t$ is smooth, then
    \begin{equation}
        \omega_p(\gamma'(0)) = \left.
            \diff {g_t} t
        \right|_{t = 0} \in \mathfrak g, 
    \end{equation}
    then we say $\omega$ is a \indexbf{principal $G$-connection} on $P$.
\end{definition}

\section{Parallel Transport}

\begin{definition}[Parallel transport]
    Let $(E, \pi, M, F)$ be a smooth vector bundle, and $D$ is a connection on $E$.
    A \indexbf{paralell transport} of $s_0 \in \pi^{-1}(\{p\})$ ($p \in M$) along a curve $\gamma \colon [0, 1] \to M$ is a section $s \in \Gamma(E|_{\gamma([0, 1])})$ such that
    \begin{equation}\label{eq: paralell transport}
        \forall t \in [0, 1], 
        \quad 
        D_{\gamma'(t)} s(t) = 0,
        \quad
        s(0) = s_0,
    \end{equation}
    where $s(t) := s_{\gamma(t)}$.
\end{definition}

If $s =:v$ is a vector field, the Eq.~\eqref{eq: paralell transport} can be rewritten as
\begin{equation}
    \diff {u \circ \gamma} {t} (t) + A[\gamma'(t)] u \circ \gamma(t) = 0,
\end{equation}
which is a 1st order ODE.
Given $\gamma_x(0) = x \in M$, there is a unique curve $\gamma_x$ associated to the vector field $u$.

We can extend the domain of $\gamma_x$ to $\mathbb R$ (note that $\mathbb R$ is diffeomorphic to $(0, 1)$), and define:
\begin{equation}
    \phi \colon \mathbb R \times M \to M; (t, x) \mapsto \gamma_x(t),
\end{equation}
which is called the \indexbf{flow} of $u$.  % TODO: the domain shall be restricted to J_x ...

\begin{definition}[Holonomy]
    Let $D$ be a connection on a smooth vector bundle $(E, \pi, M, F)$, $\gamma$ is a (piecewise) smooth curve in $M$, with ends $\gamma(0) = p$ and $\gamma(1) = q$.
    $u_0 \in E_p$.
    The \indexbf{holonomy} of $u$ along $\gamma$ is the map
    \begin{equation}
        \indexmath[H(gamma, D)]{H(\gamma, D)} \colon E_p \to E_q,
    \end{equation}
    such that $H(\gamma, D) u_0$ is the end of the parallel transport of $u_0$ along $\gamma$.
\end{definition}

It can be shown that $H(\gamma, D)$ is a linear transformation, and it transforms as
\begin{equation}
    H(\gamma, D') = g(q) H(\gamma, D) g(p)^{-1}
\end{equation}
under gauge transformation $g \in \mathcal G$.

Specially, if $\gamma$ is a loop ($p = q$), then $H(\gamma, D) \in \End(E)_p$ (and it can be proved to be living in $G$), and $H(\gamma, D') = g(p) H(\gamma, D) g(p)^{-1}$.
Therefore, $\tr H(\gamma, D)$ is a \indexbf{gauge invariant}.

\begin{definition}[Wilson loop]
    Let $D$ be a connection on a smooth vector bundle $(E, \pi, M, F)$, $\gamma$ is a (piecewise) smooth loop in $M$.
    The \indexbf{Wilson loop} is defined as 
    \begin{equation}
        \indexmath[W(gamma, D)]{W(\gamma, D)} := \tr H(\gamma, D).
    \end{equation}
\end{definition}

\chapter{Curvature}

\begin{definition}[Curvature]
    The \indexbf{curvature} of a connection $D$ on a smooth vector bundle $(E, \pi, M, F)$ is a section $F \in \Gamma(\End(E) \otimes \Omega^2(M))$ (a \emph{$\End(E)$-valued $2$-form}) defined as $\forall v, w \in \Gamma(TM)$, $s \in \Gamma(E)$:
    \begin{equation}
        F(v, w) s = D_v D_w s - D_w D_v s - D_{[v, w]} s.
    \end{equation} 
\end{definition}

If $\forall v, w \in \Gamma(TM)$, $\forall s \in \Gamma(E)$, $F(v, w) s = 0$, then $D$ is called a \indexbf{flat connection}.

Consider a local trivialisation $\varphi \colon E|_U \to U \times F$ ($U \subset M$) s.t.\ 
\begin{equation}
    s = s^i e_i := s^i\varphi^{-1}(f_i),
\end{equation}
where $s \in \Gamma(E|_U)$, $s^i \in C^{(\infty)}(U)$ and $\{f_i\}_{i \in m}$ is a set of bases of $F$,
and $\sigma \colon U \to \mathbb R^n$ is a chart of $M$, $\sigma_* d_\mu := \partial_\mu$.
Notice that $[\partial_\mu, \partial_\nu] = 0$,
\begin{equation}
    \begin{aligned}
        &F(v, u) (s^i e_i) 
        = v^\mu u^\nu F(d_\mu, d_\nu) (s^i e_i)
        \\
        &= v^\mu u^\nu [
            D_\mu (d_\nu(s^i)e_i + s^i A_{\nu i}^j e_j)
            - D_\nu (d_\mu(s^i)e_i + s^i A_{\mu i}^j e_j)
        ]
        \\
        &= v^\mu u^\nu \big[
            d_\nu d_\mu (s^i) e_i + d_\nu (s^i) A_{\mu i}^j e_j
            + d_\mu(s^i A_{\nu i}^j) e_j + s^i A_{\nu i}^j A_{\mu j}^k e_k
        \\
        &\quad
        - d_\mu d_\nu(s^i) e_i - d_\mu(s^i) A_{\nu i}^j e_j
        - d_\nu(s^i A_{\mu i}^j) e_j - s^i A_{\mu i}^j A_{\nu j}^k e_k
        \big]
        \\
        &= v^\mu u^\nu s^i\big[ 
            d_\mu (A_{\nu i}^k) + A_{\nu i}^j A_{\mu j}^k
            - d_\nu (A_{\mu i}^k) - A_{\mu i}^j A_{\nu j}^k 
        \big] e_k
    \end{aligned}
\end{equation}

If we write $F(d_\mu, d_\nu) = {F^i}_{j\mu\nu} e_i \otimes e^j$,
then
\begin{equation}
    {F^i}_{j\mu\nu} = d_\mu (A_{\nu j}^i) - d_\nu (A_{\mu j}^i) + A_{\mu k}^i A_{\nu j}^k - A_{\nu k}^i A_{\mu j}^k.
\end{equation}

By definition, we have
\begin{equation}
    F(u, v) = - F(v, u).
\end{equation}

\begin{theorem}
    Let $D$ be a $G$-connection on a $G$-bundle $E$, $F$ is the curvature of $D$.
    If $g \in \mathcal G$, $F'$ is the corresponding curvature of $D' = g D g^{-1}$, 
    then $\forall u, v \in \Gamma(TM)$
    \begin{equation}
        F'(u, v) = g F(u, v) g^{-1}.
    \end{equation}
\end{theorem}

\section{\texorpdfstring{$E$-Valued $p$-Form}{E-Valued p-Form}}

\begin{definition}
    We define the \emphbf{$E$-valued $p$-form} as a section of $E \otimes \bigwedge\nolimits^p T^*M$, denoted as $\indexmath[Omega p E M]{\Omega^p_E(M)}$.\index{E-valued p-form@$E$-valued $p$-form}
\end{definition}

Also, we define the wedge product of a $E$-valued $p$-form and a $q$-form as
\begin{equation}
    \left(
        \sum_i s_i \otimes \omega_i 
    \right) \wedge \mu := \sum_i s_i \otimes (\omega_i \wedge \mu). 
\end{equation}

\begin{definition}[Exterior covariant derivative]
    We define the \indexbf{exterior covariant derivative} as a map
    $\indexmath[dD]{\dif_D} \colon \Omega_E^p(M) \to \Omega_E^{p+1}(M)$ s.t.
    \begin{equation}
        i_v (\dif_D s) = D_v s,
    \end{equation}
    where $i_v$ is the interior product, $D$ is a connection on $E$ and $s \in \Omega^0_E(M) = \Gamma(E)$, and
    \begin{equation}
        \dif_D \sum_i s_i \otimes \omega_i := 
        \sum_i (
            \dif_D s_i \wedge \omega_i
            + s \otimes \dif \omega_i
        ),
    \end{equation}
    for $\omega_i \in \Omega^p(M)$.
\end{definition}

In local coordinates ($\varphi \colon M \to \mathbb R^n$, $\varphi^* \dif x^\mu = e^\mu$)
\begin{equation}
    \frac 1 {p!} \dif_D (s_{i_1 \cdots i_p} \otimes e^{i_1} \wedge \cdots \wedge e^{i_p})
    = D_\mu s_{i_1 \cdots i_p} \otimes e^\mu \wedge e^{i_1} \wedge \cdots \wedge e^{i_p}.
\end{equation}

\begin{theorem}
    Let $\eta \in \Omega^p_E(M)$ be a $E$-valued $p$-form, and $F \in \Omega^2_{\End(E)}(M)$ is the curvature form of connection $D$ on $E$,
    \begin{equation}
        \dif[2]_D \eta (u, v, w_0, \ldots, w_{p-1}) = F(u, v) \eta(w_0, \ldots, w_{p-1}).
    \end{equation}
\end{theorem}

We can denote $(u, v, w_0, \ldots, w_{p-1}) \mapsto F(u, v) \eta(w_0, \ldots, w_{p-1})$ as $\indexmath[F wedge eta]{F \wedge \eta}$, therefore
\begin{equation}
    \dif[2]_D \eta = F \wedge \eta.
\end{equation}

\begin{theorem}[Gauge transformation of exterior covariant derivative]
    Let $E$ be a $G$-bundle and $D$ is the $G$-connection on $E$.
    $g \in \mathcal G$ is a gauge transformation of $E$, then
    \begin{equation}
        \dif_{gDg^{-1}} = g \dif_D g^{-1}
    \end{equation}
\end{theorem}
\begin{proof}
    \begin{align}
        \dif_{gDg^{-1}} (gs) (v, u_0, \ldots, u_{p-1}) 
            &= g D_v g^{-1} gs (u_0, \ldots, u_{p-1})
            \\
            &= g \dif_D s (v, u_0, \ldots, u_{p-1})
    \end{align}
\end{proof}

\section{Bianchi Identity}

\begin{theorem}[Bianchi identity]
    Given any connection $D$ on $E$, for the curvature $F$ we have
    \begin{equation}
        \dif_D F = 0,
    \end{equation}
    where $\dif_D$ should be understood as the exterior covariant derivative of $D$ on $\End(E)$.
\end{theorem}
\begin{proof}
    It can be proved (by calculating in local coordinates) that $\forall \omega \in \Omega^p_{\End(E)}(M)$, $\forall \eta \in \Omega_E(M)$,
    \begin{equation}
        \dif_D (\omega \wedge \eta) = \dif_D \omega \wedge \eta + (-1)^p \omega \wedge \dif_D \eta.
    \end{equation}

    Hence, 
    \begin{align}
        \dif[3]_D \eta &= \dif_D (F \wedge \eta)
        = \dif_D F \wedge \eta + (-1)^2 F \wedge \dif_D \eta
        \\
        &= \dif_D F \wedge \eta + F \wedge \dif_D \eta.
    \end{align}
    
    On the other hand,
    \begin{equation}
        \dif[3]_D \eta = F \wedge \dif_D \eta.
    \end{equation}

    \hence
    \begin{equation}
        \dif_D F \wedge \eta = 0,
    \end{equation}
    for any $\eta \in \Omega_E(M)$.
\end{proof}

In local coordinates, we have
\begin{equation}
    D_\mu F_{\nu \lambda} + D_\nu F_{\lambda \mu} + D_\lambda F_{\mu \nu} = 0.
\end{equation}

Or, using the definition of $D$ on $\End(E)$, if written in $D$ on $E$:
\begin{equation}\label{eq: Bianchi identity (D and F, local coordinates)}
    [D_\mu, F_{\nu \lambda}] + [D_\nu, F_{\lambda \mu}] + [D_\lambda, F_{\mu \nu}] = 0,
\end{equation}
or
\begin{equation}\label{eq: Bianchi identity (D, local coordinates)}
    [D_u, [D_v, D_w]] + [D_v, [D_w, D_u]] + [D_w, [D_u, D_v]] = 0.
\end{equation}

We can have a different approach.
We need several algebraic constructions first.

We define the wedge product of two $\End(E)$-valued forms as
\begin{equation}
    \sum_i (S_i \otimes \omega_i) \wedge \sum_j(T_j \otimes \mu_j) 
    := \sum_{i, j} (S_i T_j) \otimes (\omega_i \wedge \mu_j).
\end{equation}


It can be proved that
\begin{equation}
    \dif_D (\omega \wedge \mu) = \dif_D \omega \wedge \mu + (-1)^p \omega \wedge \dif_D \mu,
\end{equation}
if $\omega$ is a $\End(E)$-valued $p$-form.

\begin{definition}[Graded commutator]
    For $\End(E)$-valued forms $\omega$ and $\mu$, we define the \indexbf{graded commutator} as
    \begin{equation}
        [\omega, \mu] := \omega \wedge \mu - (-1)^{pq} \mu \wedge \omega.
    \end{equation}
\end{definition}

The graded commutator gives a graded Lie algebra structure on $\Omega^*_{\End(E)}(M)$, with graded antisymmetric:
\begin{equation}
    [\omega, \mu] = - (-1)^{pq} [\mu, \omega],
\end{equation}
and \indexbf{graded Jacobi identity}:
\begin{equation}
    (-1)^{pr} [\omega, [\mu, \nu]] + (-1)^{pq} [\mu, [\nu, \omega]] + (-1)^{qr} [\nu, [\omega, \mu]] = 0.
\end{equation}

\begin{proof}[Alternative proof for Bianchi identity]
    Let $\dif := \dif_{D^0}$ in some local trivialisation of $E$ ($E|_U \cong U \times V$).
    Since $D^0$ is flat, we have $\dif[2] = 0$.

    If $D = D^0 + A$, then
    for $\omega \in \Omega_E(U)$, 
    \begin{equation}\label{eq: exterior covariant derivative of E-valued form in local trivialisation}
        \dif_D \omega = \dif \omega + A \wedge \omega;
    \end{equation}
    while $T \in \Omega_{\End(E)}(U)$,
    \begin{equation}
        \dif_D T = \dif T + [A, T].
    \end{equation}

    \hence
    \begin{equation}
        \dif[2]_D \omega = 
        \dif_D \eqref{eq: exterior covariant derivative of E-valued form in local trivialisation} = (\dif A + A \wedge A) \wedge \omega.
    \end{equation}
    \hence
    \begin{equation}
        F = \dif A + A \wedge A.
    \end{equation}

    \begin{equation}
        \dif_D F = \dif F + [A, F] 
        = \dif(A \wedge A) + [A, \dif A + A \wedge A] = 0.
    \end{equation}
\end{proof}

\chapter{Chern Classes}
\section{Chern Form}
\begin{definition}[Chern form]
    Let $M$ be a $2n$-D manifold.
    $E \to M$ is a vecor bundle with connection~$D$, and $F$ the curvature of $D$.
    The $2k$-form
    \begin{equation}
        \indexmath[tr (F k)]{\tr(F^k)} := \tr \bigwedge_{\dummy \in n} F 
    \end{equation}
    is called the $k$-th \indexbf{Chern form} of $E$.
\end{definition}

\begin{theorem}[Chern forms are closed]
    $\tr(F^k) \in Z^{2k}(M)$, i.e.\ 
    \begin{equation}
        \dif \tr(F^k) = 0.
    \end{equation}
\end{theorem}
\begin{proof}
    \begin{align*}
        &\dif \tr (F^k) = \tr \dif_D (F^k)
        \\
        &= \tr \left( 
            \dif_D F \wedge F^{k-1}
            + F \wedge \dif_D F \wedge F^{k-2}
            + \cdots
            + F^{k-1} \wedge \dif_D F
         \right) 
        \\
        &= 0
    \end{align*}
\end{proof}

\begin{theorem}\label{theorem: difference of chern forms is exact}
    Given two connection, $D$, $D'$, the difference of the corresponding Chern forms is an exact form, i.e.\
    \begin{equation}
        \exists \omega \in \Omega^{2k-1}(M) 
        \quad
        \tr(F^k) - \tr (F'^k) = \dif \omega.
    \end{equation}
\end{theorem}


\begin{theorem}\label{theorem: Chern class is topological invariant}
    The integral of the $n$-th Chern form over the $2n$-D manifold is a topological invariant of the vector bundle, i.e.\ 
    the integral
    \begin{equation}
        c(E) = \int_M \tr(F^n)
    \end{equation} 
    is independent of the choice of connection $D$.
\end{theorem}

\section{Chern Classes}

Due to Theorem~\ref{theorem: difference of chern forms is exact}, one can define the $n$th Chern class of $E$ as
\begin{definition}[Chern classes]
    The $k$th \indexbf{Chern class} of $E$ is defined as
    \begin{equation}
        \indexmath[ck(E)]{c_k(E)} := \{
            \tr(F^k) + \dif \omega
            \mid \omega \in \Omega^{2k-1}(M)
        \}  = [\tr (F^k)] \in H^{2k}(M).
    \end{equation}
    
\end{definition}

\begin{theorem}[Integral Chern class]
    If $E \to M$ is a complex vector bundle (the standard bundle $F$ is a complex vector space) over $M$, which is a $2n$-D compact and oriented manifold, then the integral
    \begin{equation}
        \frac 1 {k!} \left( 
            \frac{\mi}{2\pi}
         \right)^k c_k(E) = \int_M \tr(F^k)
    \end{equation}
\end{theorem}


\chapter{Pseudo-Riemannian Geometry}
% TODO: since most of the topics are on Pseudo-Riemannian Geometry
%       rearrange?
\section{Tensors}

\begin{definition}[Tensor]\label{def: tensor}
    Let $M$ be a smooth manifold.
    A $(r, s)$-\indexbf{tensor} is a smooth section of the tensor product of $r$th tensor power of $TM$ and $s$th tensor power of $T^*M$:
    \begin{equation}
        t \in \Gamma(TM^{\otimes r} \otimes T^*M^{\otimes s})
        =: \indexmath[TMrs]{TM^r_s}.
    \end{equation}
\end{definition}

In local coordinates:
\begin{equation}
    t^{\mu_1 \cdots \mu_r}_{\nu_1 \cdots \nu_s} \bigotimes_{k = 1}^r \partial_{\mu_k} \otimes \bigotimes_{k = 1}^s \dif x^{\nu_k}.
\end{equation}

It is conventional to use the local coordinates form in pseudo-Riemannian geometry, 
and do not distinguish between a tensor and its components, 
written in forms of \indexbf{abstract indices}, 
where indices are written just to indicates types and operations on tensors.

And since we can raise and lower indices of a tensor, it is sometimes important to distinguish the orders between covariant and contravariant indices.
e.g.\ ${T^\mu}_\nu \neq {T^\nu}_\mu$.

\paragraph{Raising and Lowering of Indices}
We have defined $\omega^\sharp$ for a $1$-form and $v^{\flat}$ for a vector field.
Now we can generalise the definition for any $(p, q)$ tensor $T$ that:
\begin{align}
    &T^\sharp \in \Gamma(TM^{\otimes (p + q)}), 
    \\
    &T^\sharp(\omega_0, \ldots, \omega_{p+q-1}) 
    = T(\omega_0, \ldots, \omega_{p-1}, \omega_p^\sharp, \ldots, \omega_{p+q-1}^\sharp); 
    \\
    &T^{\flat} \in \Gamma(T^*M^{\otimes (p + q)}),
    \\
    &T^{\flat}(v_0, \ldots, v_{p+q-1})
    = T(v_0^{\flat}, \ldots, v_{p-1}^{\flat}, v_p, \ldots, v_{p+q-1}).
\end{align}

We can even raise or lower some instead of all indices in $T$, by writing $T^{\sharp i\flat j}$ or $T^{\sharp\{i_0, \ldots\}\flat\{j_0, \ldots\}}$.

In abstract indices, it is conventional to keep the order of the indices including the raised and lowered ones, and abuse the original symbol of the tensor, for example if $T$ is a $(3, 4)$ tensor:
\begin{equation}
    {(T^{\sharp4})^{\alpha_0 \alpha_1 \alpha_2 \mu}}_{\beta_0 \beta_2\beta_3}
    =: {{{T^{\alpha_0 \alpha_1 \alpha_2}}_{\beta_0}}^{\mu}}_{\beta_2\beta_3} 
\end{equation}

Strictly speaking, the tensors after raising or lowering indices might not be the tensors as we defined in Def.~\ref{def: tensor}, since it might belong to e.g.\ 
\begin{equation}
    TM^{\otimes r} \otimes T^*M \otimes TM^{\otimes s} \otimes T^*M^{\otimes t}
\end{equation}
where the order of the tensor product is not canonical.
One way to avoid this is to reorder the indices, but this approach is not conventional to those who use abstract indices.
Since we can still consider the ``tensors'' as multilinear maps, we can include these non-canonical tensors, while, in order to avoid confusion in the order of indices, we will prefer to use the abstract indices form if there is any ambiguity.


\paragraph{Tensor Product}
Let $T_1$, $T_2$ be $(p_1, q_1)$ and $(p_2, q_2)$ tensors, we can have their tensor product:
\begin{equation}
    T_1 \otimes T_2 \in TM^{p_1 + p_2}_{q_1 + q_2},
\end{equation}
where at each point $p \in M$, the tensor product is but the tensor product of the corresponding multilinear functions.

In abstract indices, we have:
\begin{align}
    &{(T_1 \otimes T_2)^{\mu_1 \cdots \mu_{p_1 + p_2}}}_{\nu_1 \cdots \nu_{q_1 + q_2}}
    \\
    &= {T_1^{\mu_0 \cdots \mu_{p_1 - 1}}}_{\nu_0 \cdots \nu_{q_0 - 1}} 
    {T_2^{\mu_{p_1} \cdots \mu_{p_1 + p_2 - 1}}}_{\nu_{q_1} \cdots \nu_{q_1 + q_2 - 1}}.
\end{align}

\paragraph{Contractions}

The contraction is a generalisation of the inner product of vectors.
Let $T$ is a $(p + 1, q + 1)$ tensor, we can define the $(i, p + j)$ contraction of $T$ as
{\small
\begin{align}
    &\tr_{(i, p+j)} T \colon  T^*M^p \times TM^q \to \mathbb{R}
    \\
    &\; (\omega_0, \ldots, \omega_{i-1}, \omega_{i+1}, \ldots, \omega_p,  v_0, \ldots, v_{j-1},  v_{j+1}, \ldots, v_q) \mapsto 
    \\
    &
    \sum_{\mu \in N} T(\omega_0, \ldots, \omega_{i-1}, \dif x^\mu,\omega_{i+1}, \ldots, \omega_p,  v_0, \ldots, v_{j-1},  \partial_\mu,v_{j+1}, \ldots, v_q).
\end{align}
}

The index-free notation can be found at~\cite{Yuri-Vyatkin-StackExchange}.

\section{Diffeomorphism and Invariance}

Let $\phi \colon M \to N$ be a diffeomorphism, we have already known that we have pushforward $\phi_*$ for $(p, 0)$-tensors, and pullback $\phi^*$ for $(0, q)$-tensors.
Since $\phi$ is a diffeomorphism, both $\phi^*$ and $\phi_*$ are isomorphisms, and we can generalise the definitions to obtain a pair of isomorphisms:
\begin{equation}
    \phi_* \colon TM^p_q \to TN^p_q, \quad \phi^* \colon T^*N^p_q \to T^*M^p_q,
\end{equation}
such that $\phi_* \circ \phi^* = \phi_* \circ \phi^*$, and
\begin{align}
    &\phi_* T(\omega_0, \ldots, \omega_p, v_0, \ldots, v_q)
    \\
    &\quad =
    T(\phi^* \omega_0, \ldots, \phi^* \omega_p, \phi^{-1}_* v_0, \ldots, \phi^{-1}_* v_q),
\end{align}
\begin{align}
    &\phi^* T(\omega_0, \ldots, \omega_p, v_0, \ldots, v_q)
    \\
    &\quad =
    T((\phi^{-1})^* \omega_0, \ldots, (\phi^{-1})^* \omega_p, \phi_* v_0, \ldots, \phi_* v_q).
\end{align}

The special case when $M = N$ ($\phi$ is an endomorphism), if $\phi_*T = T$, then we say that $T$ is \indexbf{invariant} under $\phi$.

\begin{theorem}
    Let $\phi \colon M \to N$ be a diffeomorphism, $T \in TM^p_q$, and $S \in TN^r_s$.
    \begin{enumerate}
        \item $\phi^*$ and $\phi_*$ are isomorphisms of $\mathbb R$-algebras.
        \item $\phi_* (T \otimes S) = \phi_* T \otimes \phi_*S$, and $\phi^* (T \otimes S) = \phi^* T \otimes \phi^*S$.
    \end{enumerate}
\end{theorem}

\begin{theorem}
    Let $\phi \colon M \to N$ be a \emph{homeomorphism} (differentiable map), then $\forall \alpha, \beta \in \Omega(M)$, we have
    \begin{equation}
        \phi^* (\alpha \wedge \beta) = \phi^* \alpha \wedge \phi^* \beta,
    \end{equation}
    that is, $\phi^*$ is the induced homeomorphism of the exterior algebra $\Omega(M)$.
\end{theorem}

\section{Lie Derivative}

Let $u$ be a vector field on $M$, and $\phi$ be the corresponding flow.

\begin{definition}[Lie derivative]
    Let $T$ be a $(p, q)$ tensor, then the \indexbf{Lie derivative} of $T$ along $u$ is defined as
    \begin{equation}\label{eq: lie derivative}
        \mathcal L_u T = \lim_{t \to 0} \frac{\phi^*_t T - T}{t}.
    \end{equation}
\end{definition}

\begin{theorem}
    $u, v \in \Vect(M)$.
    \begin{enumerate}
        \item $\mathcal L_{u + v} = \mathcal L_u + \mathcal L_v$.
        \item $\mathcal L_{[u, v]} = \mathcal L_u \mathcal L_v - \mathcal L_v \mathcal L_u =: [\mathcal L_u, \mathcal L_v]$.
    \end{enumerate}
\end{theorem}

\begin{theorem}
    $u \in \Vect(M)$, $T \in TM^p_q$, $S \in TM^r_s$.
    \begin{enumerate}
        \item $\mathcal L_u $ is $\mathbb R$-linear.
        \item $\mathcal L_u (T \otimes S) = \mathcal L_u T \otimes S + T \otimes \mathcal L_u S$ (\indexbf{Leibniz law}).
        \item $\tr_{(i, j)} \mathcal L_u T = \mathcal L_u \tr_{(i, j)} T$.
        \item $\forall f \in C^{(\infty)}(M)$, $\mathcal L_u f = u(f)$.
        \item $\forall v \in \Vect(M)$, $\mathcal L_u v = [u, v]$.
    \end{enumerate}
\end{theorem}

Applying the laws, we can calculate
\begin{equation}
    \mathcal L_u \omega(v) = u[\omega(v)] - \omega([u, v]),
    \quad
    \omega \in \Omega^1(M)
\end{equation}
by $u [\omega(v)] = T_u [\omega(v)] = T_u \tr_{(0, 0)} (\omega \otimes v) = T_u \omega(v) + \omega([u, v])$.
Similarly:
\begin{align}
    &\mathcal L_u \omega(v_0, \ldots, v_{q - 1}) 
    = u[\omega(v_0, \ldots, v_{q - 1})] 
    \\
    &\qquad - \omega([u, v_0], \ldots, v_{q - 1})
    - \cdots - \omega(v_0, \ldots, [u, v_{q - 1}]).
\end{align}

And, in local coordinates, we have

\begin{align}
    &{\mathcal L_u T^{\alpha_0 \cdots \alpha_{p-1}}}_{\beta_0 \cdots \beta_{q-1}}
    = u^\mu {T^{\alpha_0 \cdots \alpha_{p-1}}}_{\beta_0 \cdots \beta_{q-1} ,\mu}
    \\
    &\quad
    - {T^{\mu \alpha_1 \cdots \alpha_{p-1}}}_{\beta_0 \cdots \beta_{q-1}} {u^{\alpha_0}}_{,\mu}
    \cdots - {T^{\alpha_0 \cdots \alpha_{p-2} \mu}}_{\beta_0 \cdots \beta_{q-1}} {u^{\alpha_{p-1}}}_{,\mu} 
    \\
    &\quad
    + {T^{\alpha_0 \cdots \alpha_{p-1}}}_{\mu\beta_1 \cdots \beta_{q-1}} {u^{\mu}}_{,\beta_0}
    \cdots + {T^{\alpha_0 \cdots \alpha_{p-1}}}_{\beta_0 \cdots \beta_{q-2} \mu} {u^{\mu}}_{,\beta_{q-1}}.
\end{align}

\begin{definition}[Divergence]
    Let $u \in \Vect(M)$, then the \indexbf{divergence} of $u$ is defined as
    \begin{equation}
        \operatorname{div} u = (-1)^{\sign(g)} \star (\mathcal L_u \vol)
    \end{equation}
\end{definition}

% TODO: propeties of divergence

\begin{definition}[Killing field]
    If $u \in \Vect(M)$ is such that $\mathcal L_u g = 0$, then $u$ is called a \indexbf{Killing field}. 
    The equation
    \begin{equation}
        \mathcal L_u g = 0,
        \quad \text{or} \quad
        u_{(\alpha;\beta)} = 0
    \end{equation}
    is called the \indexbf{Killing equation}.
\end{definition}

\begin{theorem}[\emphbf{Cartan's formula}]\label{theorem: Cartan formula}\index{Cartan's formula}
    \begin{equation}
        \mathcal L_u |_{\Omega(M)} = \dif \circ i_u + i_u \circ \dif.
    \end{equation}
\end{theorem}

\begin{corollary}
    \begin{equation}
        \mathcal L_u |_{\Omega(M)} \circ \dif = \dif \circ \mathcal L_u |_{\Omega(M)}.
    \end{equation}
\end{corollary}

\begin{corollary}
    \begin{equation}
        i_{[u, v]} = [\mathcal L_u |_{\Omega(M)}, i_v].
    \end{equation}
\end{corollary}

As an application of Cartan's formula, we can prove the following theorem by induction.
\begin{theorem}\label{theorem: exterior differential formula}
    {\small
        \begin{align}
        &\dif \omega(u_0, \ldots, u_p) 
            = \sum_{i \in p + 1} (-1)^i u_i \omega(u_0, \ldots, u_{i-1}, u_{i+1}, \ldots, u_p)
            \\
            &
            + \sum_{\substack{
                (i, j) \in (p + 1)^2
                \\
                i < j
                }} (-1)^ {i + j} \omega([u_i, u_j], u_1, \ldots, u_{i-1}, u_{i+1}, \ldots, u_{j-1}, u_{j+1}, \ldots, u_p).
    \end{align}
    }
\end{theorem}

\section{Levi-Civita Connection}
\begin{definition}[Levi-Civita connection]
    Let $E \to M$ be a smooth vector bundle, where $M$ is a Riemannian manifold with metric $g \in T^*M \otimes T^*M$.
    Let $\nabla \in \Gamma(\End(E) \otimes T^*M)$ be a connection on $E$.
    Then $\indexmath[nabla]{\nabla}$ is called a \indexbf{Levi-Civita connection} if
    \begin{equation}\label{eq: metric compatibility}
        u g(v, w) = g(\nabla_u v, w) + g(v, \nabla_u w),
    \end{equation} 
    and
    \begin{equation}\label{eq: torsion free}
        [v, w] = \nabla_v w - \nabla_w v,
    \end{equation}
    where $u, v, w \in \Gamma(TM)$.
\end{definition}

Since $T(u, v) = \nabla_u v - \nabla_v u - [v, u]$ is called the \indexbf{torsion} of $u$ and $v$, 
Eq.~\eqref{eq: torsion free} is called the \indexbf{torsion free} condition. 
The Eq.~\eqref{eq: metric compatibility} is called the \indexbf{metric} condition.

In local coordinates:
\begin{equation}
    \nabla_\alpha \partial_\beta = \Gamma^\gamma_{\alpha\beta} \partial_\gamma,
\end{equation}
where $\indexmath[Gamma]{\Gamma^\gamma_{\alpha\beta}}$ is the \indexbf{Christoffel symbol}.

The torsion free condition is equivalent to
\begin{equation}\label{eq: torsion free (coordinate)}
    \Gamma^\gamma_{\alpha\beta} = \Gamma^\gamma_{\beta\alpha}.
\end{equation}

For any $T \in \Gamma(TM^{\otimes p} \otimes T^*M^{\otimes q})$, we have
\begin{align}
    \nabla T 
    &= {T^{\alpha_0 \cdots \alpha_{p - 1}}}_{\beta_0 \cdots \beta_{q - 1}; \mu} \bigotimes_{k \in p} \partial_{\alpha_k} \otimes \bigotimes_{\ell \in q} \dif x^{\beta_\ell} \otimes \dif x^\mu
\end{align}
\begin{align}
    {T^{\alpha_0 \cdots \alpha_{p - 1}}}_{\beta_0 \cdots \beta_{q - 1}; \mu} &= {T^{\alpha_0 \cdots \alpha_{p - 1}}}_{\beta_0 \cdots \beta_{q - 1}, \mu} 
    \\
    &
    + \sum_{i \in p} \Gamma^{\alpha_i}_{\lambda \mu} {T^{\alpha_0 \cdots \alpha_{i-1} \lambda \alpha_{i + 1} \cdots \alpha_{p - 1}}}_{\beta_0 \cdots \beta_{q - 1}}
    \\
    &
    - \sum_{i \in q} \Gamma^{\lambda}_{\beta_i \mu} {T^{\alpha_0 \cdots \alpha_{p - 1}}}_{\beta_0 \cdots \beta_{i-1} \lambda \beta_{i + 1} \cdots \beta_{q - 1}}.
\end{align}

It is useful to define the generalisation of divergence:
\begin{equation}
    \nabla \cdot T = \tr_{(0, q)} (\nabla T)
\end{equation}
if $T$ is a $(p, q)$ tensor.

It can be shown that
\begin{equation}
    \nabla \cdot u = \operatorname{div} u = \delta u^\flat,
    \quad
    u \in \Vect(M).
\end{equation}

\begin{theorem}
    $\forall u \in \Gamma(TM)$,
    \begin{equation}
        \nabla_u \tr_{(i, j)} = \tr_{(i, j)} \nabla_u.
    \end{equation}
\end{theorem}

\begin{theorem}\label{theorem: covariant derivative and exterior differential (1-form)}
    $\forall \omega \in \Gamma(T^*M)$,
    \begin{equation}
        - \dif \omega(u, v) = \nabla \omega(u, v) - \nabla\omega(v, u).
    \end{equation}
    (Notice that $\nabla \omega(u, v) = (\nabla_v \omega) (u)$)
\end{theorem}
\begin{proof}
    \begin{equation}
        u[\omega(v)] = u g(\omega^\sharp, v) = \nabla \omega(v, u) + \omega(\nabla_u v)
    \end{equation}
    \hence (Theorem~\ref{theorem: exterior differential formula})
    \begin{align}
        \nabla \omega(v, u) - \nabla \omega (u, v) 
        &= u[\omega(v)] - v[\omega(u)] - \omega([u, v])
        \\
        &= \dif \omega(u, v)
    \end{align}
\end{proof}

In fact, the Theorem~\ref{theorem: covariant derivative and exterior differential (1-form)} is but a special case of:
\begin{theorem}\label{theorem: covariant derivative and exterior differential}
    $\forall \omega \in \Omega^p(M)$,
    \begin{equation}
        (-1)^p \dif \omega(u_0, \ldots, u_p) = 
        (p + 1) \sum_{\pi \in S_{p + 1}} \nabla \omega (u_{\pi(0)}, \ldots, u_{\pi(p)}).
    \end{equation}
    
\end{theorem}

Note that the parallel transport of a tangent vector along some submanifold might be no longer tangent to that submanifold, and hence we have:

\begin{definition}[Autoparallel submanifold]
    Let $N$ be a submanifold of $M$, and $\nabla$ be the Levi-Civita connection of $M$.
    Then $N$ is called \indexbf{autoparallel} if
    \begin{equation}
        \forall u, v \in \Gamma(TN), \quad
        \nabla_u v \in \Gamma(TN).
    \end{equation}
\end{definition}


Autoparallel curves in $M$ are called \indexbf{geodesics}.

\begin{definition}[Projection of connection with respect to the metric]
    Let $N$ be a submanifold of $M$, and $\nabla$ be the Levi-Civita connection of $M$.
    Given a linear projection $P \in \Gamma(\End(TN) \otimes T^*M)$,
    the \indexbf{projection} of $\nabla$ onto $N$ with respect to $g$ is a projection in the sense of affine connection if $P$ satisfies that
    \begin{equation}
        \forall u \in \Gamma(TM), \quad
        \forall v \in \Gamma(TN), \quad
        \langle P(v), u \rangle = \langle v, u \rangle.
    \end{equation}
\end{definition}


\section{Curvatues}

The \indexbf{Rieman tensor} $\Riem$ is defined as the curvature of the affine connection $\nabla$:
\begin{equation}
    \Riem(u, v) w = \nabla_u \nabla_v w - \nabla_v \nabla_u w - \nabla_{[u, v]} w.
\end{equation}

It is conventional to write the components of $\Riem$ as
\begin{equation}
    {R^\mu}_{\nu\alpha\beta} = \dif x^\mu (\Riem(\partial_\alpha, \partial_\beta) \partial_\nu), 
\end{equation}
and consider it as a $(1, 3)$-tensor.

The trace of the linear map $u \mapsto \Riem(u, v) w$ is defined as $\Ric(v, w)$ where $\Ric$ is the \indexbf{Ricci tensor}, with components
\begin{equation}
    R_{\mu\nu} = {R^\alpha}_{\mu\alpha\nu}.
\end{equation}

Finnally, the trace of $u \mapsto \Ric(u, \dummy)^\sharp$ is called the \indexbf{scalar curvature} or \indexbf{Ricci scalar}:
\begin{equation}
    R = g^{\mu\nu} R_{\mu\nu} =: \tr_g \Ric.
\end{equation}

\backmatter{}
\nocite{*} % 这个表示列出所有没有在文中被引用的参考文献
\printbibliography[heading=bibliography, title={bibliography}]

\indexprologue{Here listed the important symbols used in these notes}
\printindex[symbol]

\printindex
\end{document}