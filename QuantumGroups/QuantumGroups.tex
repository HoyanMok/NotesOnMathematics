% TeXplates/Mathematics.tex
% v0.2.2
% https://github.com/HoyanMok/TeXplates
\documentclass[openany, a5paper]{book} 
% \documentclass{ctexbook} 如果用中文
% \documentclass[10pt,a4paper]{ctexart}  字体大小和纸张大小,默认分别为10pt和letterpaper
% 五号 = 10.5pt,小四=12pt,四号=14pt
% 其他可选参量如twocolumn, 两行排版

\usepackage{../TeXplatesMathematics}
\addbibresource{QuantumGroups.bib} % 把这里改成实际的文件名

% 文章标题页信息:
\title{Quantum Groups}
\author{Hoyan Mok\thanks{hoyanmok@outlook.com}}
\date{\today} % 自动生成日期
% \titlepic{\includegraphics{latex-project-logo.pdf}}

\DeclareMathOperator{\Op}{Op}

\begin{document}
\pagenumbering{Alph}
\maketitle % 打印标题
\frontmatter
\chapter{Preface}

% \paragraph{Classical and Quantum Mechanics}
% In classical mechanics, the phase space $M$ is a \indexbf{Poisson manifold}.
% A Lie bracket $\{,\} \colon \mathcal F(M) \times \mathcal F(M) \to \mathcal F(M) $ is defined, which is called the \indexbf{Poisson bracket}.
% The dynamical equations:
% \begin{equation}\label{equation: classical dynamics}
% 	\diff{}{t} f(m(t)) = \{\mathcal H, f\} (m(t)),
% \end{equation}
% where $f \in \mathcal F(M)$, $\mathcal H \in \mathcal F(M)$ is the Hamiltonian, $m \colon \mathbb R \to M$ is a path in the phase space. 

% In quantum mechanics, the space $M$ is replaced by a complex Hilbert space $V$ (devided by norm), $\mathcal F(M)$ is replaced by the algebra $\Op(V)$.
% The dynamical equations now become
% \begin{equation}\label{equation: Heisenberg}
% 	\diff{A}{t} = [H, A],
% \end{equation}
% for $A, H \in \Op(V)$.

% The quantisation is to find a $\mathcal Q \colon \mathcal F(M) \to \Op(V)$ s.t.\ 
% \begin{equation}
% 	\mathcal Q(\{f, g\}) = \frac{[\mathcal Q(f_1), \mathcal Q(f_2)]}{\mi \hbar}
% \end{equation}

% Yet, such $\mathcal Q$ just does not always exist, even for simple $M$.

% \paragraph{Space and Algebra}

% A space is determined by the algebra of functions defined on it.

% {
% 	\small
% 	\centering
% 	\begin{tabular}{@{}p{.28\textwidth}p{.6\textwidth}@{}}
% 		\toprule
% 		Space & Algebra \\ \midrule
% 		Affine algebraic variety over $\mathbb C$ & Commutative algebra of regular functions on $\mathbb C$ \\
% 		Compact topological space & Commutative $\mathrm C^*$-algebra of complex-valued continuous functions \\ \bottomrule
% 		\end{tabular}
% }

% Given an algebra $\mathcal F(G)$ of functions on group $G$, the multiplication $\mu \colon G \times G \to G$ gives a pullback $\mu^* \colon \mathcal F(G) \to \mathcal F(G \times G)$.
% If $\mathcal F(G)$ and $\otimes$ between algebras are defined appropriately, $\mathcal F(G \times G) \cong \mathcal F(G) \otimes \mathcal F(G)$, and
% \begin{equation}
% 	\varDelta = \mu^* \colon \mathcal F(G) \to \mathcal F(G) \otimes \mathcal F(G)
% \end{equation}
% is defined, and it is called the \indexbf{comultiplication}.

% The inverse map $i \colon G \to G$ induces
% \begin{equation}
% 	S = i^* \colon \mathcal F(G) \to \mathcal F(G),
% \end{equation}
% called the \indexbf{antipode}, and evaluation at the identity $e \in G$ ($f \mapsto f(e)$) gives an
% \begin{equation}
% 	\varepsilon \colon \mathcal F(G) \to \mathbb C,
% \end{equation} 
% called the \indexbf{counit}.

% $\varDelta$, $S$ and $\varepsilon$ give $\mathcal F(G)$ the structure of a \indexbf{Hopf algebra}.
% The category of spaces dual to the category of Hopf algebras (with some restriction) can be defined as the categories of \indexbf{quantum groups}.



\tableofcontents

\mainmatter

\chapter{Poisson-Lie Groups and Lie Bialgebras}
\section{Poisson Manifolds}

\begin{definition}[Poisson bracket]
	Let $M$ be a smooth manifold of finite dimension $m$, 
	$C^{(\infty)}(M)$ be the algebra of smooth real-valued functions on $M$.

	A \indexbf{Poisson bracket} on $M$ is an $\mathbb R$-bilinear map
	\begin{equation}
		\{, \} \colon C^{(\infty)}(M) \times C^{(\infty)}(M) \to C^{(\infty)}(M),
	\end{equation}
	which satisfies the following conditions:
	\begin{enumerate}
		\item Anti-symmetric:
		\begin{equation}
			\{f, g\} = -\{g, f\},
		\end{equation}
		\item Jacobi identity:
		\begin{equation}
			\{f, \{g, h\}\} + \{g, \{h, f\}\} + \{h, \{f, g\}\} = 0,
		\end{equation}
		\item Leibniz identity:
		\begin{equation}
			\{f, \{gh\}\} = \{f, g\} h + \{f, h\} g,
		\end{equation}
	\end{enumerate}
	for any $f, g, h \in C^{(\infty)}(M)$.
\end{definition}

$X_f = \{f, \} \colon g \mapsto \{f, g\}$ defines a vector field on $M$, called the \indexbf{Hamiltonian vector field}.
\begin{equation}
	B_M \colon T^*M \to TM; \dif f \mapsto X_f.
\end{equation}

$\exists w_M \in TM^{\otimes 2}$ (\indexbf{Poisson bivector}), s.t.\ $\{f, g\} = (\dif f \otimes \dif g)(w_M)$.

In coordinates:
\begin{equation}
	\{f, g\} = w^{ij}(x) \partial_i f(x) \partial_j g(x).
\end{equation}

\begin{definition}[Poisson map]
	Let $N$, $M$ be two Poisson manifolds.
	If $F \colon N \to M$ is smooth and $\forall f, g \in C^{(\infty)}(M)$,
	\begin{equation}
		\{f, g\}_M \circ F = \{f \circ F, g \circ F\}_N,
	\end{equation}
	then $F$ is called a \indexbf{Poisson map}.
\end{definition}

If $y = F(x)$, $x \in N$, in coordinates:
\begin{align*}
	w_M^{ij}(y) \pdiff{f}{y^i}(y) \pdiff{g}{y^i}(y)
	&= w_N^{k\ell}(x) \pdiff{f \circ F}{x^k}(x) \pdiff{g \circ F}{x^\ell}(x)
	\\
	&= w_N^{k\ell}(x) \pdiff{f}{y^i}(y) \pdiff{g}{y^j}(y) \pdiff{F^i}{x^k} \pdiff{F^j}{x^\ell}
\end{align*}
\hence
\begin{equation}
	(F'(x) \otimes F'(x)) (w_N(x)) = w_M(F(x)).
\end{equation}

\begin{definition}[Poisson submanifold]
	Let $S$ be a submanifold of $M$.
	The inclusion map $S \hookrightarrow M$ is a Poisson map \emph{iff} $\forall x \in S$,
	$w_M(x) \in (T_x S)^{\otimes 2}$. 
	In this case, $S$ is called a \indexbf{Poisson submanifold} of $M$.
\end{definition}

\begin{definition}[Product of Poisson manifolds]
	Let $M$ and $N$ be Poisson manifolds, their product $M \times N$,
	with $\{, \}_{M \times N}$ defined as
	\begin{equation}
		\begin{aligned}
		\{f, g\}_{M \times N} (x, y) =& \{
			x \mapsto f(x, y),
			x \mapsto g(x, y)
			\}_M (x)
		\\
		&
		+ \{
			y \mapsto f(x, y),
			y \mapsto g(x, y)
		\}_N (y),
		\end{aligned}
	\end{equation}
	is also a Poisson manifold.
\end{definition}

$X_f = \{f, \} \colon g \mapsto \{f, g\}$ defines a vector field on $M$, called the \indexbf{Hamiltonian vector field}.
\begin{equation}
	B_M \colon T^*M \to TM;\; 
	\dif f \mapsto X_f.
\end{equation}

If $B$ is an isomorphism, $M$ is said to be \indexbf{symplectic}.
Equivalently this means that $\forall x \in M$, $w(x)$ is a non-degenerate bilinear form on $T^*_x M$.
In this case, $(B \otimes B)^{-1} (w) =: \omega \in T^* M \otimes T^* M$ is a non-degenerate 2-form on $M$.

\begin{definition}[Symplectic leaves]\label{definition: symplectic leaves}
	Let $M$ be a Poisson manifold.
	Define an equivalence relation $\sim$ as $x \sim y$ \emph{iff} $x$ and $y$ can be joined by a piecewise smooth curve on $M$, each smooth segment of which is part of an integral curve of a Hamiltonian vector field on $M$.
	Then each class in $M / \sim$ is called a \indexbf{symplectic leaf}. 
\end{definition}

\begin{theorem}[Symplectic leaves are Poisson submanifolds] 
	$\forall L \in M / \sim$ (defined in Def.~\ref{definition: symplectic leaves}), $L$ is a Poisson submanifold of $M$.
\end{theorem}

\paragraph{Example: Lie-Poisson structure}
\begin{definition}[Lie-Poisson structure]
	Let $\mathfrak g$ be a $m$D Lie algebra, whose bases are $x_i$ ($i \in m$), and Lie bracket is defined as
	\begin{equation}\label{equation: Lie bracket}
		[x_i, x_j] = c^k_{ij} x_k.
	\end{equation}
	We can define the Poisson bracket on $\mathfrak g^*$ as
	\begin{equation}
		\{f_1,f_2\}(\xi) = \langle[(\dif f_1)_\xi, (\dif f_2)_\xi], \xi\rangle,
	\end{equation}
	or in coordinates form:
	\begin{equation}
		\{f_1, f_2\}(\xi) = c^k_{ij} \pdiff{f_1}{x^i} \pdiff{f_2}{x^j} \langle\xi, x_k\rangle.
	\end{equation}
\end{definition}

Note that the bases of $T^* \mathfrak g^* \cong (\mathfrak g^*)^* \cong \mathfrak g$ can be considered as $x_i$ i.e.\ bases of $\mathfrak g$,
\begin{equation}
	\dif f = \pdiff{f}{x_i} x_i,
\end{equation}
and where the $x_i$ which partial derivative with respect to are coordinates on $\mathfrak g^*$.


\appendix
\chapter{Appendix}

\backmatter
\nocite{*} % 这个表示列出所有没有在文中被引用的参考文献
\printbibliography[heading=bibliography, title={Bibliography}]

\indexprologue{Here listed the important symbols used in these notes.}
\printindex[symbol]

\printindex

\end{document}