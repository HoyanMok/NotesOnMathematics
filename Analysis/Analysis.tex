\documentclass{article}
\usepackage{amsmath} %数学
	\makeatletter % '@' now normal "letter"
	\@addtoreset{equation}{section}
	\makeatother  % '@' is restored as "non-letter"
	\renewcommand\theequation{\oldstylenums{\thesection}%
					-\oldstylenums{\arabic{equation}}}
\usepackage{amsthm} %定义、证明、定理等
	\theoremstyle{plain}
	\theoremstyle{definition}
		\newtheorem{dfn}{Definition}[section] %定义
		\newtheorem{thrm}{Theorem}[section] %定理
		\newtheorem{crll}{Corollary} %推论
		\newtheorem{lmm}{Lemma} %引理
		\newtheorem{prp}{Proposition} %命题
	\renewcommand{\proofname}{\textbf{Proof}}
\usepackage{amssymb} %数学格式
\usepackage{mathrsfs} %花体
\usepackage{enumerate} %编号
\usepackage{makeidx} %索引
	\makeindex
\usepackage[colorlinks=true,bookmarks=true]{hyperref}%hyperlinks for reference, table of contents and indexes
\hypersetup{linkcolor=[rgb]{1,0.27,0},bookmarksopen = true}%More options read texdoc hyperref
\newcommand{\me}{\mathrm{e}}
\newcommand{\mi}{\mathrm{i}}
\newcommand{\md}{\mathrm{d}}
\DeclareMathOperator{\ctg}{ctg}
\newcommand*{\basis}[1]{\hat{\boldsymbol{#1}}}
\newcommand*{\bv}{\boldsymbol}
\newcommand*{\bm}{\boldsymbol}
\newcommand*{\indexbf}[1]{\textbf{#1}\index{#1}}
\begin{document}
\tableofcontents
\newpage
\section{Metric Space and Continuous Map}
\subsection{Metric Space}
\begin{dfn}\label{metric_def} function
\begin{align}\label{metric}
	d:X^2\to\mathbb{R}
\end{align}
	$\forall x_1,x_2,x_2\in X$ satisfied: 
	\begin{enumerate}[a)]
	\item	$d(x_1,x_2)=0\Leftrightarrow x_1=x_2$;
	\item	$d(x_1,x_2)=d(x_2,x_1)$ (symmetry);
	\item	$d(x_1,x_3)\leqslant d(x_1,x_2)+d(x_2,x_3)$(Triangle inequality),
	\end{enumerate}
	is called a \textbf{metric}\index{metric} or \textbf{distance}\index{distance} in $X$. Such $X$ is said to be equiped with metric $d$, $(X;d)$ is called a \textbf{metric space}\index{metric space}.
\end{dfn}

Some examples:
\begin{itemize}
\item $(\mathbb{R}^n;d_p)$, where $d_p(x_1,x_2)=\left(\sum^n_{i=1}\left|x^i_1-x^i_2\right|^p\right)^{1/p}$, while $d_\infty(x_1,x_2)=
\max\limits_{1\leqslant i\leqslant n}
\left|x^i_1-x^i_2\right|$.
\item Similarly we can define metric spaces as $(C[a,b];d_p)$ or $C_p[a,b]$. $d_p(f,g)=\left(
	\int^b_a\left|f-g\right|^p\,\mathrm{d}x
\right)^{\frac{1}{p}}$. $C_\infty$ is called a \textbf{Chebyshev metric}\index{Chebyshev metric}.
\item On class $\mathfrak{\tilde{R}}[a,b]$ over $\mathfrak{R}[a,b]$ similar metric can be defined. Functions are considered of one same class if they are equivalent expect on a set not larger than null set. 
\end{itemize}
\begin{lmm}\label{quadruple_inequality}
If $(X;d)$ is a metric space, then $\forall a,b,u,v $, $\left| d(a,b) - d(u,v) \right| \leq d(a,u)+d(b,v) $.
\end{lmm}
\begin{proof}
	Without loss of generality, we assume that $d(a,b) > d(u,v)$. According to the triangle inequality (see def. \ref{metric}), $d(a,b) \leq d(a,u)+d(u,v)+d(v,b)$, which is to proof.  
\end{proof}
\begin{dfn}\label{delta_ball}
$\delta\in\mathbb{R}_+$, $a\in X$. Set
\[
	B(a;\delta)=\{x\in X|d(a,x)<\delta\}
\]
is then called a \textbf{ball}\index{ball} with centre $a\in X$, and a radius of $\delta$, or a \textbf{$\delta$-ball}\index{$\delta$-ball} of point $a$.
\end{dfn}
\begin{dfn}\label{open_set}
A \textbf{open set}\index{open set} $G\subset X$ in metric space $(X;d)$ satisfies: $\forall x\in G$, $\exists B(x;\delta)$, s.t. $B(x;\delta)\subset G$.
\end{dfn}
\begin{dfn}\label{closed_set}
A \textbf{closed set}\index{closed set} $F$ in metric space $(X;d)$ satisfies: $X-F$ is a open set in $(X;d)$.
\end{dfn}
$\tilde{B}(x;\delta)=\{x\in X|d(a,x)\leq r\}$ is an example of closed sets in $(X;d)$.
\begin{prp}\label{open_sets_metric}
\begin{enumerate}[a)]
\item An infinite union of open sets is an open set.
\item A definite intersection of open sets is an open set.
\item A definite union of closed sets is a closed set.
\item An infinite intersection of closed sets is a closed set.
\end{enumerate}
\end{prp}
\begin{proof}
\begin{enumerate}[a)]
	\item If open sets $G_\alpha\subset X,\forall\alpha\in A$, $\forall a\in\bigcap\limits_{\alpha\in A}G_\alpha$, $\exists\alpha_0\in A$, $a\in G_{\alpha_0}$, $\exists B(a;\delta)\subset G_{\alpha_0}\subset \bigcap\limits_{\alpha\in A}G_\alpha$.
	\item Open sets $G_1\cup G_2\subset X$, $a\in G_1\cap G_2$, therefore $\exists\delta_1,\delta_2\in\mathbb{R}_+$, $B(a;\delta_1)\subset G_1,B(a;\delta_2)\subset G_2$, without loss of generality, let $\delta_1\geq\delta_2$, 那么$a\in B(a;\delta_1)\cap B(a;\delta_2)=B(a;\delta_2)\subset G_1\cap G_2$.
	\item Just consider $\complement_X
		\left(\bigcap_{\alpha\in A}F_\alpha\right)
		=\bigcup_{\alpha\in A}\complement_X(F_\alpha)$ and a).
	\item Similarly, $\complement_X\left(F_1\cup F_2\right)=\complement_X(F_1)\cap\complement_X(F_2)$.
	
\end{enumerate}
\end{proof}
\begin{dfn}\label{neighbourhood_metric}
If $x\in X$ is an element of an open set, then such open set is called a \textbf{neighbourhood}\index{neighbourhood} of point $x$ in $X$, denoted by $U(x)$.
\end{dfn}
\begin{dfn}\label{interior_metric}
$x\in X$, $E\subset X$.
\begin{enumerate}[a)]
\item If $\exists U(x)\subset E$, $x$ is called an \textbf{interior point}\index{interior point} of $E$.
\item If $\exists U(x)\subset X-E$, $x$ is called an \textbf{exterior point}\index{exterior point} of $E$.
\item If $x$ isn't an interior point nor exterior point of $E$, it is called a \textbf{boundary point}\index{boundary point} of $E$. The set of boundary points is called \textbf{boundary}\index{boundary}, denoted by $\partial E$.
\end{enumerate}
\end{dfn}
\begin{dfn}\label{limit_point_metric}
$a\in X$, $E\subset X$. If $\forall U(a)$, $\left|E\cap U(a)\right|=\infty$, $a$ is called a \textbf{limit point}\index{limit point} of $E$.
\end{dfn}
\begin{dfn}\label{closure_metric}
The intersections of $E\subset X$ and set of all its limit points is called the \textbf{closure}\index{closure} of $E$, denoted by $\overline{E}$.
\end{dfn}
\begin{thrm}\label{closed_closure}
$F\subset X$ is a closed set in $X\,\Leftrightarrow \overline{F}=F$.
\end{thrm}
\begin{proof}
$\Rightarrow$: $\complement_X(F)$ is open, hence its elements are all its interior points. Therefore $\overline{F}-F=\overline{F}\cup\complement_X(F)=\varnothing$, 又$F\subset\overline{F}\Rightarrow F=\overline{F}$.

$\Leftarrow$: $F=\overline{F}$ means that $\forall x\in \complement_X(F)$, $x$ is not a boundary of $F$, which indicates that $x$ is an interior point of $X-F$. Therefore $F-X$ is open while $F$ is closed.
\end{proof}
\begin{thrm}\label{closure_closed}
$\overline{E} $ is always closed.
\end{thrm}
\begin{proof}
$\forall x \in X - \overline{E}$, since it is not a element of  the set $E$ or its limit points, $\exists U( x) $ s.t. $U( x) \cap \overline{E} \varnothing$, which implies that $x$ is an extorior point of $E$, therefore $\overline{E}$ is closed.
\end{proof}
\begin{thrm}\label{closure_closure}
$\overline{E}=\overline{ \overline{E}}$.
\end{thrm}
\begin{proof}
Since $\overline{E}$ is closed, its complement is open, which implies that its elements are all exterior point of $\overline{E}$, therefore $\overline{E}$ has contained all of its limit points.
\end{proof}
\begin{dfn}\label{subspace_metric}
We called $(X';d')$ a \textbf{subspace}\index{subspace} of $(X;d)$ when $X'\subset X$ and $\forall x,y\in X',d'(x,y)=d(x,y)$.
\end{dfn}
\subsection{Topological Space}
\begin{dfn}\label{topological_space}
We say $X$ is equiped with a \textbf{topological space}\index{topological space} or epuiped with \textbf{topology}\index{topology} if we assigned a $\mathscr{T}\subset 2^X$, which has got the following propoties:
\begin{enumerate}[a)]
	\item $\varnothing\in\mathscr{T};X\in\mathscr{T}$.
	\item $\left(\forall\alpha\in A,\mathscr{T}_\alpha\in\mathscr{T}\right)
	\Rightarrow\bigcup\limits_{\alpha\in A}\mathscr{T}_\alpha\in\mathscr{T}$.
	\item $\left(\mathscr{T}_1,\mathscr{T}_2\in\mathscr{T}\right)\Rightarrow \mathscr{T}_1\cap\mathscr{T}_2$.
\end{enumerate}
Then we call $(X;\mathscr{T})$ a \textbf{topological space}\index{topological space}.
\end{dfn}

These are correspondence of propoties of open sets (See proposition \ref{open_sets_metric}). Topology made of all open sets defined in metric space $(\mathbb{R};d_2)$ is called the \textbf{standard topology}\index{standard topology} of $n$-dimension Euclidean space.

\begin{dfn}\label{open_set_topology}
Topology $\mathscr{T}$'s elements are called \textbf{open sets}\index{open set}, and their complements are called \textbf{closed sets}\index{closed set}.
\end{dfn}
\begin{dfn}\label{topological_base}
$(X;\mathscr{T})$ is a topological space, $\mathfrak{B}\subset 2^X$. If $\forall G \in\mathscr{T}$, $\exists B_\alpha\in\mathfrak{B}$ ($\alpha\in A$) s.t. $\bigcup\limits_{\alpha\in A}B_\alpha = G$, it is called a (topological or open) \textbf{base}\index{base}\index{topological base}\index{open base}.
\end{dfn}
\begin{dfn}\label{weight}
The smallest possible cardinity of base is called the \textbf{weight}\index{weight} of the topological space.
\end{dfn}
\begin{dfn}\label{neighbourhood_topology}
If $x\in G$ and $G\in\mathscr{T}$, then $G$ is a \textbf{neighbourhood}\index{neighbourhood} of $x$ in topological space $(X;\mathscr{T})$.
\end{dfn}
For example, we define an equivalence relation $\sim$ in $C(\mathbb{R};\mathbb{R})$. If $f,g\in C(\mathbb{R};\mathbb{R})$, at point $a\in\mathbb{R}$:
\begin{equation}\label{germ}
	f\sim_ag\Leftrightarrow
		\left(
		\exists U(a)\left(\forall x\in U(a),f(x)=g(x)\right)
		\right).
\end{equation}
Then we call $f$ and $g$ define a \textbf{germ}\index{germ} at point $a$, denoted by $f_a$.
If $f\in C(\mathbb{R};\mathbb{R})$ is defined in $U(a)$, then we can call ${f_x}:=\{f_x|x\in U(a)\}$ a neighbourhood of germ $f_a$. Class of neighbourhoods of each $f_x$ constructs a base of topological space $(C(\mathbb{R};\mathbb{R});\mathscr{T})$, where $\mathscr{T}$ is made of the sets of germs of continuous function in $C(\mathbb{R};\mathbb{R})$. 
\begin{dfn}\label{T_2_space}
We call a topological space $(X;\mathscr{T})$ a \textbf{Hausdorff space}\index{Hausdorff space}, \textbf{separated space}\index{separated space} or \textbf{$\mathrm{T}_2$ space}\index{$\mathrm{T}_2$ space}, if $\forall x,y\in X$, $\exists U(x),U(y)$ s.t. $U(x)\cap U(y)=\varnothing$ (\textbf{Hausdorff axiom}\index{Hausdorff axiom} or \textbf{separation axiom}\index{separation axiom}).
\end{dfn}
\begin{dfn}\label{dense_set}
$E\subset X$ is a \textbf{dense set}\index{dense set} in topological space $(X;\mathscr{T})$, if $\forall x\in X$, $\forall U(x)$, $U(x)\cap E\neq \varnothing$.
\end{dfn}
\begin{dfn}\label{separable}
If there is a countable dense set in topological space $(X;\mathscr{T})$, then $(X;\mathscr{T})$ is \textbf{separable}\index{separable}.
\end{dfn}
We can also define interior points, exterior points, boundary points, limit points in topological space as in metric space.
\begin{dfn}\label{subspace_topology}
Each subset $Y$ of $X$ equiped with topology $\mathscr{T}$ can be given a \textbf{subspace topology}\index{subspace topology} $\mathscr{T}_Y$ whose elements $G_Y$ are intersections of the subset with an open set $G$ in $(X;\mathscr{T})$ i.e. $\forall G_Y\in \mathscr{T}_Y$, $\exists G\in \mathscr{T}$ s.t. $G_Y=G\cap Y$. Subset equiped with such topology construct a \textbf{topological subspace}\index{subspace} $(Y;\mathscr{T}_Y)$. 
\end{dfn}
If two topology $\mathscr{T}_1,\mathscr{T}_2$ are defined on the same $X$, $\mathscr{T}_1$ is said to be \textbf{stronger}\index{stronger} than $\mathscr{T}_2$ if $\mathscr{T}_1\subsetneqq \mathscr{T}_2$.
\subsection{Compact Set}
\begin{dfn}\label{compact}
Set $K$ in topological space $(X;\mathscr{T})$ is called a \textbf{compact set}\index{compact set} if each of its \textbf{open covers}\index{open cover} has a finite \textbf{subcover}\index{subcover}. Class $\Omega$ is called a open cover of $K$ if $K\subset \cup{\Omega}$ and for all sets in $\Omega$ are open sets.
\end{dfn}
Specially, $\varnothing$ is compact.
\begin{thrm}\label{selfcompact}
	Set $K\subset X$ is compact in $(X;\mathscr{T})$ \textbf{iff} $K$ is compact in $(K;\mathscr{T}_K)$ itself. 
\end{thrm}
This theorem tells a truth that whether $K$ is compact or not isn't dependent on the topological space it's in, it can be easily proofed: just need to notice that every open set $G_K$ in $(K;\mathscr{T}_K)$ is an intersection of an open set $G$ in $(X;\mathscr{T})$ and $K$. 
\begin{thrm}\label{compact_Hausdorff_closed}
	If $K$ is compact in a Hausdorff space $(X;\mathscr{T})$ (See definition~\ref{T_2_space}), then $K$ is a closed set in $(X;\mathscr{T})$.
\end{thrm}
\begin{proof}
	If $x_0$ is a limit point of $K$, which means $\forall U(x_0)$, 
\[
	\left|U(x_0)\cap K\right|\notin \mathbb{N}.
\]
Assume that $x_0\notin K$. In a Hausdorff space, $\forall x\in K$, $\exists U(x)$ s.t. $U(x)\cap U(x_0)=\varnothing$. Such $U(x)$ construct a open cover $\Omega=\left\{U(x)|x\in K\right\}\subset 2^K$. Since $K$ is compact, $\exists \Omega'\subset\Omega$ s,t. $\left|\Omega\right|\in\mathbb{N}$. 
\[
	\left(\cup\Omega'\right)
	\cap U(x_0)
	=
	\left(\bigcup_{k=1}^n{U_k}\right)
	\cap U(x_0)
	=
	\bigcup_{k=1}^n\left(
		U_k\cap U(x_0)
	\right)
	=
	\varnothing
\]
Since $K\subset \cup\Omega'$, $x_0$ is an exterior point of $K$, which leads to a contradiction. Hence $x_0\in K$. $\overline{K}=K$.
\end{proof}
\begin{thrm}\label{nested_compact}
Each decreasing \textbf{nested sequences}\index{nested sequence} of non-empty compact sets has a non-empty limit, i.e. $\forall\{K_n\}$ s.t. $\forall n\in\mathbb{N}_+$, $K_n\supset K_{n+1}\wedge K_n\neq\varnothing\wedge (K_n$ is compact), $K_n\downarrow K\neq \varnothing$.
\end{thrm}
\begin{proof}
	Assume that $K=\varnothing$. Compact subsets of $K_1$ are all colsed, while their complements are all open. An open cover $\Omega$ can be constructed as $\{K_1-K_n|n\in\mathbb{N}_+\}$. Since $K_1$ is compact, there would be a finite subcover $\Omega'\subset\Omega$, notice that $\{X-K_n\}$ is also a nested sequence, there must be oone single $X-K_{n_0}\in\Omega'$ that covers $K_1$, which means $K_{n_0}=\varnothing$ contradicting that $\forall n\in\mathbb{N}_+$, $K_n$ is non-empty.
\end{proof}
\begin{thrm}\label{compact_closed_subset}
Closed subsets $F$ of a compact set $K$ are also compact.
\end{thrm}
\begin{proof}
	If $\Omega_F\subset 2^K$ is an open cover of $F$. Notice that $K - F$ is open, $\Omega=\left(\cup\Omega_F\right)\cap\{K-F\}$ constructs an open cover over over $K$. Since $K$ is compact there must be a finite cover $\Omega'\subset\Omega$ which obviously also covers over $F$.
\end{proof}
The following propoties of compact sets are on the topological space induced from a metric space.
\begin{dfn}\label{e-net}
	$(X;d)$ is a metric space, $E\subset X$. $E$ is called an \textbf{$\varepsilon$-net}\index{$\varepsilon$-net} if $\forall x\in X$,$\exists e\in E$, $d(e,x)<\varepsilon$.
\end{dfn}
\begin{thrm}\label{finite_e-net}
If $(K,d)$ is a compact metric space, then $\forall \varepsilon\in\mathbb{R}_+$, $\exists$ finite $\varepsilon$-net in $(K;d)$. 
\end{thrm}
\begin{proof}
	For each point $x\in K$, find it a $B(x,\varepsilon)$, of which an infinite cover $\Omega$ over $K$ is made. Since $K$ is compact, there exists a finite cover $\Omega'=\{B(x_1,\varepsilon),\cdots,B(x_n,\varepsilon)\}$ ($n\in\mathbb{N}_+$). Therefore $\{x_1,\cdots,x_n\}$ is a finite $\varepsilon$-net in $K$.
\end{proof}
\begin{thrm}\label{sequentially_compact_metric}
$(K;d)$ is compact \textbf{iff} it is \textbf{sequentially compact}\index{sequentially compact}, that is, $\forall \{x_n\}$ ($x_n\in K$, $n\in\mathbb{N}_+$), it has convergent subsequence $\{x_{k_n}\}$ whose limit $a\in K$.
\end{thrm}
To proof it, we need to proof two lemmata first.
\begin{lmm}\label{sequentially_compact_finite_e-net}
	If $(K;d)$ is sequentially compact, then $\forall \varepsilon\in\mathbb{R}_+$, $\exists$ finite $\varepsilon$-net in $(K;d)$. 
\end{lmm}
\begin{proof}
	Assume that there were no finite  $\varepsilon_0$-net in $(K;d)$. Define such sequence : $ \{x_n\}$ s.t. $\forall k,n\in\mathbb{N}_+$ ($1\leq k< n$), $d(x_n,x_k)\geq\varepsilon_0$ (There would always be the next one since there exists no $\varepsilon_0$-net). It has no convergent subsequence for it there were a $\{x_{k_n}\}$ convergent to $a\in K$, $\exists N,M\in\mathbb{N}_+$, $d(x_N,x_M)\leq d(x_N,a)+d(x_M,a)\leq \varepsilon_0$, which lead to a contradictary. 
\end{proof}
\begin{lmm}\label{sequentially_compact_nested_closed}
If $(K;d)$ is sequentially compact then every nested sequence of closed non-empty sets $\{F_n\}$ in $K$ have a non-empty intersection.
\end{lmm}
\begin{proof}
	Let $\{x_{k_n}\}$ be a convergent subsequence of $\{x_n\}$, Let $a$ be the limit of $\{x_{k_n}\}$ ($\forall n\in\mathbb{N}_+$,$x_n\in F_n$). Assume that $a\notin \bigcap_{n\in\mathbb{N}_+}F_n$, in metric space, $\exists U(a)\cap \left(\bigcap_{n\in\mathbb{N}_+}F_n\right)=\varnothing\Rightarrow $ $U(a)\cap \left(\bigcap_{n\in\mathbb{N}_+}F_{k_n}\right)=\varnothing$. But this conflict the fact that $\exists N\in\mathbb{N}_+$, s.t. $n>N$, $x_{k_n}\in U(a)$ while $x_{k_n}\in F_{k_n}$.
\end{proof}
Then get back to theorem \ref{sequentially_compact_metric}. 
\begin{proof}

	$\Rightarrow$: If $\left|\{x_n\}\right|\in\mathbb{N}$, it is obvious; if $\left|\{x_n\}\right|=\infty$, make finite $\frac{1}{n}$-net (Theorem \ref{finite_e-net}), $n\in\mathbb{N}_+$. For each $n$, there must be at least one $B(x_n;\frac{1}{n})$ that includes infinite elements in $\{x_n\}$. Select $x_{k_n}\in B(x_n;\frac{1}{n})$, and $\{\tilde{B}(x_n;\frac{1}{n})\}$ is a nested sequence of a closed non-empty sets in sequentially compact $K$, (Lemma \ref{sequentially_compact_nested_closed}) $\lim\limits_{n\to \infty} x_{k_n} \in K$.
	
	$\Leftarrow$: Assume that there were a open cover $\Omega$ over $K$ having no finite subcover, $\forall n\in\mathbb{N}_+$, $\exists$ finite $\frac{1}{n}$-net (Lemma \ref{sequentially_compact_nested_closed}), in which there would be at least one $x_n$ whose $\tilde{B}(x_n;\frac{1}{n})$ can't be covered finitely. Then $\tilde{B}(x_n;\frac{1}{n})\downarrow B=\{a\}$ (Theorem \ref{nested_compact}) can't be finitely covered by any subcover of $\Omega$ which means $\Omega$ can't cover the whole $K$, leading to the contradiction.
\end{proof}
\subsection{Connected Set}
\begin{dfn}\label{connected_space}
Topological space $(X;\mathscr{T})$ is called \textbf{connected}\index{connected}\index{connected space} if there is no \textbf{open-closed set}\index{open-closed set} (i.e. both open and closed) besides $\varnothing$ and $X$ itself. 
\end{dfn}
Notice that if $A\subset X$ is open-closed, its complement $X - A$ is also open-closed, which means a topological space is connected \textbf{iff} it is not a union of its two open subsets. 
\begin{dfn}\label{connected_set}
$(X;\mathscr{T})$ is a topological space. Subset $C$ is said to be \textbf{connected}\index{connected}\index{connected set} if subspace $(C;\mathscr{T}_C)$ is connected. 
\end{dfn}
\begin{thrm}\label{union_connected}
$( X; \mathscr{T} )$ is a topological space. $\forall\alpha \in A$, $C_\alpha$ are connected subsets of $X$. If $\bigcap\limits_{ \alpha \in A} C_\alpha \neq \varnothing$, then $\bigcup\limits_{\alpha \in A} C_\alpha$ is also connected. 
\end{thrm}
\begin{proof}
If $C = \bigcup\limits_{\alpha \in A} C_\alpha$ were not connected, $\exists E \subset C$ s.t. $E\neq \varnothing \wedge E \neq C \wedge E, C - E \in \mathscr{T}_C$. For $E$ is not empty there exists a $\beta \in A$ s.t. $E \cap C_\beta \neq \varnothing$. It can be proofed that $C_\beta \subset  E$.

Suppose that $C_\beta \nsubseteq  E$, which implies that $( C - E) \cap C_\beta \neq \varnothing$. 
$E, C - E, C_\beta \in \mathscr{T}_C 
	\Rightarrow 
	E\cap C_\beta, ( C - E) \cap C_\beta \in \mathscr{T}_C$. 
This conflicts to the fact that $C_\beta$ is connected. Therefore $C_\beta \subset  E$. 

Hence there exists a $B \subsetneqq A$, $\bigcup\limits_{ \beta \in B} C_\beta = A$. Since $C_\gamma$, $\gamma \in A - B$ would have a empty intersection with $E$, which contradicts $\bigcap\limits_{ \alpha \in A} C_\alpha \neq \varnothing$.
\end{proof}
\begin{thrm}\label{closure_connected}
Connected sets have connected closure.
\end{thrm}
\begin{proof}

\end{proof}
\begin{thrm}\label{R_connected}
$E\subset\mathbb{R}$ is connected \textbf{iff} that if $\forall x,z\in E$, $y\in\mathbb{R}$ s.t. $x<y<z$, then $y\in C$.
\end{thrm}
\begin{proof}
	$\Rightarrow$: Assume that there were such $y\in\mathbb{R}$ that $\exists x,z\in C$, $x<y<z$ but $y\notin C$. $\{x\in C|x<y\}$ and $\{x\in C|x>y\}$ are open in $C$for they are intersection of open sets in $\mathbb{R}$ and $C$. Since they're each other's complement, they are both open-closed, which conflict to the definition of connected set.
	
	$\Leftarrow$: It can be proofed that $(\inf C,\sup C)\subset C$. Assume that there were an open-closed proper subset $E\neq\varnothing$ contained in $C$. Find two points $x\in E$, $z\in C - E$. Without loss of generality, let $x<z$. Since $E$ and $C-E$ are closed, $c_1=\inf \{E\cap \left[a,b\right]\}\in E$ while $c_2=\inf \{(C-E)\cap [a,b]\}\in C-E$. However $E\cap(C-E)=\varnothing\Rightarrow c_1<c_2$, which means $(c_1,c_2)\cap E=\varnothing$. Here's the contradiction.
\end{proof}
\begin{dfn}\label{locally_connected}
A topological space $( X; \mathscr{T})$ is said to be \textbf{locally connected}\index{locally connected} if $\forall x \in X$, $\exists U( x) $ s.t. $U( x)$ is connected.
\end{dfn}
\subsection{Complete Metric Spaces}
We now take a closer look at one of the most important sorts of metric spaces: complete spaces.
\begin{dfn}\label{Cauchy_sequence}
A sequence $\{x_n\mid n\in \mathbb{N}\}$ of points of a metric space $(X;d)$ is called a \textbf{fundamental}\index{fundamental}\index{fundamental sequence} or \textbf{Cauchy sequence}\index{Cauchy sequence} if $\forall \varepsilon \in\mathbb{R}_+$, $\exists N\in\mathbb{N}$ s.t. as long as $m,n> N$, $d(x_n,x_m) < \varepsilon$.
\end{dfn}
\begin{dfn}\label{complete_space}
A metric space $(X;d)$ is \textbf{complete}\index{complete} if every Cauchy sequence of its points is convergent.
\end{dfn}
For example, metric space $C_\infty[a,b]$ is complete while $C_1[a,b]$ isn't. Proof see p22, Zorich.
Consider incomplete space $\mathbb{Q}_1$, which is a subspace of the complete space $\mathbb{R}_1$. If $\mathbb{R}_1$ is the smallest complete space containing $\mathbb{Q}_1$, we can say that we have achieved a \textbf{completion} of $\mathbb{Q}_1$. However, the definition of ``completion'' hasn't been defined yet. 
\begin{dfn}\label{completion}
If a metric space $(X;d)$ is a subspace of a complete metric space $(Y;d)$ and everywhere dense in it, we call the latter one the \textbf{completion}\index{completion} of $(X;d)$. 
\end{dfn}
We need to confirm that such completion is the smallest and unique. So we introduce:
\begin{dfn}\label{isometric}
If there exists a \textbf{isometry}\index{isometry} $f:X_1\to X_2$ when $(X_1;d_1)$ and $(X_2;d_2)$ are both metric space, i.e. $f$ is a bijective and for each $a,b\in X_1$, $d_2\left(f(a),f(b)\right)=d_1\left(a,b\right)$, then these two metric space is \textbf{isometric}\index{isometric}.
\end{dfn}
This relation is reflexive ($e$), symmetric ($f ^{-1}$), and transitive ($f\circ g$), so it is a equivalence relation, noted by $\sim$. We shall consider isometric spaces are identical.
\begin{thrm}\label{completion_unique}
If metirc spaces $(Y_1;d_1)$ and $(Y_2;d_2)$ are both completions of $(X;d)$, then they are isometric.
\end{thrm}
\begin{proof}
Such isometry $f:Y_1\to Y_2$ can be defined: if $x_1,x_2\in X$, 
\[
	d_2( f(x_1) , f(x_2) )=d(f(x_1), f(x_2)) = d(x_1 , x_2)=d_1(x_1, x_2).
\]

For each $y_1\in Y_1 - X_1$, a Cauchy sequence $\{x_n\}$ can be found in the nested sequence of balls centered in $y_1$. It is obvious that $\{x_n\}$ is also fundamental in $Y_2$, limitting to $y_2\in Y_2$. Different sequences of points $\{x'_n\}$ selected won't result in a diffrent $y'_2$, or $d(x_n,x'_n)$ wouldn't converge to $0$, which violate the fact that the radii of balls converge to $0$. Let $f(y_1)=y_2$. 

a) For each $y_2\in Y_2 - X$, there always exists a Cauchy sequence converging to it, which implies that $f$ is a surjection.

b) Also notice that $\forall y'_1,y''_1\in Y_1 - X$,
\[
	d_1(y'_1,y''_1)= \lim_{n\to\infty} d(x'_n,x''_n)=d_2(y'_2,y''_2)
\]
while $\{x'_n\}$ and $\{x''_n\}$ are both Cauchy sequence. This equality also proofed that $f$ is a injection.
\end{proof}
\begin{thrm}\label{completion_exists}
There always exists a completion for every metric space.
\end{thrm}
\begin{proof}
	A isometric space $(S_X;d)$ to the metric space $(X;d_X)$ can be constructed, which consists of constant sequence of points in $X$. Its completion $(S;d)$ can be defined as Cauchy sequences whose mutual distances' limits are not $0$.
\end{proof}
\subsection{Continuous Mapping}
Let's recall the definition of the limitation.
\begin{dfn}\label{filter_base}
A set $\mathscr{B}\subset 2^X$ is called a \textbf{(filter) base}\index{filter base}\index{base} in $X$ if the following conditions hold:
\begin{enumerate}[a)]
\item $\varnothing \notin \mathscr{B}$.
\item $\forall B_1,B_2\in \mathscr{B}$, $\exists B\in \mathscr{B}$ s.t. $B\subset B_1\cap B_2\subset B_2$. 
\end{enumerate}
\end{dfn}
Introduction of the limits in a topological space is as follows.
\begin{dfn}\label{limit}
Let $a\in Y$ be the \textbf{limit}\index{limit} over the base $\mathscr{B}\subset 2^{\mathscr{D}(f)}$ of a mapping $f:\mathscr{D}( f )\to Y$, in which $Y$ is epuiped with a topology $\mathscr{T}$. 
\[
	\lim_\mathscr{B} f = a 
	\quad:=\quad
	\forall U(a)\subset Y\;
	\exists B\in \mathscr{B}(f(B)\subset U(a)).
\]
\end{dfn}

Such definition is parallel to the definition we have introduced on the limits of real number, hence it basically holds the same propoties. 
\begin{dfn}\label{continuous}
A mapping $f:X\to Y$, where $X$,$Y$ is respectively equiped with topology $\mathscr{T}_X$,$\mathscr{T}_Y$, is said to be \textbf{continuous}\index{continuous} at $x_0\in X$ (let $y_0 = f( x_0 ) \in Y$), if $\forall U( y_0 )$, $\exists U( x_0 )$ s.t. $f( U(x_0) )\subset U( y_0 )$. It is \textbf{continuous}\index{continuous} in $X$ if it is continuous at each point $x\in X$. 
\end{dfn}
The set of continuous mappings from $X$ into $Y$ can be denoted by $C(X,Y)$ or $C(X)$ when $Y$ is clear. 
\begin{thrm}\label{criterion_continuity}
\textbf{(Criterion for continuity)}

\noindent 
$(X;\mathscr{T}_X)$ and $(Y;\mathscr{T}_Y)$ are both topological spaces. A mapping $f: X\to Y$ is continuous \textbf{iff} $\forall G_Y\in \mathscr{T}_Y$, $f ^{-1} ( G_Y ) \in \mathscr{T}_X$.
\end{thrm}
\begin{proof}
$\Rightarrow$: It is obvious if $f  ^{-1}(G_Y) = \varnothing$. If $f  ^{-1}(G_Y) \neq \varnothing$ and $x_0 \in X$, since $f \in C(X,Y)$, for $G_Y$, $\exists U( x_0) $ s.t $f ( U( x_0)) \subset G_Y$. Also notice that $f ( U( x_0)) \subset G_Y \Rightarrow U( x_0) \subset f  ^{-1}(G_Y)$, therefore $f  ^{-1}(G_Y) $ is open.

$\Leftarrow$: $\forall x_0 \in X$, let $y_0 = f ( x_0)$, $f ^{-1} ( U( y_0)) \in \mathscr{T}_X$. Notice that $x_0 \in f ^{-1} ( U( y_0))$, therefore $f \in C( X, Y) $.
\end{proof}
\begin{dfn}\label{homeomorphism}
$(X;\mathscr{T}_X)$ and $(Y;\mathscr{T}_Y)$ are both topological spaces. A bijective mapping $f: X\to Y$ is a \textbf{homeomorphism}\index{homeomorphism} if $f \in C( X, Y) \wedge f ^{ -1} \in C( Y, X)$. 
\end{dfn}
\begin{dfn}\label{homeomorphic}
Two topological spaces $(X;\mathscr{T}_X)$ and $(Y;\mathscr{T}_Y)$ are said to be \textbf{homeomorphic}\index{homeomorphic} if there exists a homeomorphism $f: X\to Y$.
\end{dfn}
Homeomorphic topological spaces are identical with respect to their topological propoties since the theorem~\ref{criterion_continuity} has shown that their open sets correspond to each other.
\begin{thrm}\label{compact_continuous}
$(X;\mathscr{T}_X)$ and $(Y;\mathscr{T}_Y)$ are both topological spaces. $K\subset X$ is a compact set. If $f: X\to Y \in C( X, Y)$, then $f( K)$ is compact.
\end{thrm}
\begin{proof}
For each open cover $\Omega_Y = \{ G_Y \in \mathscr{T}_Y\} \subset \mathscr{T}_Y$ over $f( K)$, $f ^{-1} ( G_Y) \in \mathscr{T}_X$ (Therem~\ref{criterion_continuity}). $f( K) \subset \cup\,\Omega_Y \Rightarrow K \subset f ^{-1} \left(  \cup\,\Omega_Y \right) = \cup\,\Omega_X $, where $\Omega_X = \{ f ^{-1} ( G_Y) \mid G_Y \in \Omega_Y\} $ is an open cover over $K$. Since $K$ is compact,
$\exists \Omega'_X \subset \Omega_X\left( 
\lvert \Omega'_X \rvert \in \mathbb{N}_+ 
\;\wedge\;K\subset \cup\,\Omega'_X
\right)$, $f( K) \subset f ( \cup\,\Omega'_X) $. $f ( G'_X) \in \Omega_Y$, hence $\Omega'_Y = \{ f ( G'_X) \mid G'_X \in \Omega'_X\}$ is a finite subcover over $f( K)$.
\end{proof}
\begin{thrm}\label{compact_Hausdorff_continuous_homeomorphism}
$(K;\mathscr{T}_K)$ is a compact space and $(Y;\mathscr{T}_Y)$ is a Hausdorff space. If a bijective $f: K\to Y \in C( K, Y)$, then it is a homeomorphism.
\end{thrm}
\begin{proof}
$\forall F = K - G$ s.t. $G \in \mathscr{T}_K$ is compact (Theorem~\ref{compact_closed_subset}). Hence $f ( F) $ is compact (Theorem~\ref{compact_continuous}), then it is also closed
(Theorem~\ref{compact_Hausdorff_closed}). This fact shows that $f ^{-1}$ is continuous (Theorem~\ref{criterion_continuity}).
\end{proof}
\begin{thrm}\label{connected_continuous}
$(X;\mathscr{T}_X)$ and $(Y;\mathscr{T}_Y)$ are both topological spaces. $E\subset X$ is a connected set. If $f: X\to Y \in C( X, Y)$, then $f( E)$ is also connected.
\end{thrm}
\begin{proof}
Only to notice that the open-closed sets in $( f( E) ; \mathscr{T}_{ f( E)})$ have concurrently open-closed pre-images in $( E; \mathscr{T}_E)$.
\end{proof}
\subsection{Contraction}
\begin{dfn}\label{fixed_point}
A point $a\in X$ is a \textbf{fixed point}\index{fixed point} of a mapping $f: X\to X$ if $f( a) = a$.
\end{dfn}
\begin{dfn}\label{contraction}
Let~$(X;d)$ be a metric space. A mapping~$f: X\to X$ is called a \textbf{contraction}\index{contraction} if~$\exists q \in ( 0, 1) \subset \mathbb{R}$ s.t. $\forall x_1,x_2\in X$, 
\begin{align}\label{inequality_contraction}
	d ( f ( x_1), f (x_2)) \leq q d( x_1, x_2).
\end{align}
\end{dfn}
\begin{lmm}\label{contraction_continuous}
A contraction $f: X\to X$ is always continuous.
\end{lmm}
\begin{proof}
$\forall x\in X$, $\forall \varepsilon \in \mathbb{R}_+ $, $\exists \delta < \varepsilon / q$, according to inequality~\ref{inequality_contraction}:
\[
f \left( B( x;\delta )\right) 
\subset 
B\left( f( x); \varepsilon \right).
\]
\end{proof}
\begin{thrm}\label{contraction_mapping_principle}
(\textbf{Picard-Banach fixed-point principle}\index{Picard-Banach fixed-point principle} or \textbf{contraction mapping principle}\index{contraction mapping principle})
Let $( X; d)$ be a complete metric space. Each contraction $f: X\to X$ has a unique fixed point $a$. 
Also, $\forall \{x_n\} \subset X$ s.t.
$\forall n \in \mathbb{N}
\left(f ( x_n) = x_{n+1} \right)$ then $\lim\limits_{n\to \infty} x_n = a$, and
\begin{align}\label{inequality_fixed-point_principle}
	d( x_n, a) \leq \frac{q^n}{1-q} d( x_1, x_0).
\end{align}
\end{thrm}
\begin{proof}
By the inequality~\ref{inequality_contraction}:
\[
d ( x_{n+1}, x_n) \leq q d( x_n, x_{n-1})
\leq \cdots 
\leq q^n d( x_1, x_0)
\]
Therefore, $\forall n, k \in \mathbb{N}$, 
\begin{align}\label{inequality_Cauchy_contraction}
	d ( x_{n+k}, x_n) \leq 
	\sum^{k-1}_{i=0} d( x_{n+i+1}, x_{n+i}) \leq
	\sum^{k-1}_{i=0} q^{n+i} d( x_1, x_0) \leq
	\frac{q^n}{1-q} d( x_1, x_0),
\end{align}
which implies that ${ x_n}$ is a Cauchy sequence in a complete space $(X;d)$, hence it converges to a point $a \in X$.

To proof that $a$ is a fixed point of $f$, since $f$ is continuous (Lemma~\ref{contraction_continuous}), just notice that 
\[
	a = \lim_{n\to \infty} f(x_n) = 
	f( \lim_{n\to \infty} x_n) = f( x_n).
\]
If there were a second fixed point $a'\in X$ of $f$, then:
\[
	0 \leq d( a, a')  = d( f(a), f(a') ) \leq q d( a, a')
\]
which can't be true unless $a = a'$. 

By passing to the limit as $k \to \infty$ in the inequality~\ref{inequality_Cauchy_contraction}, we have the inequality~\ref{inequality_fixed-point_principle}.
\end{proof}

\section{Normed Linear Space and Differential Calculus}
\subsection{Normed Linear Space}
\begin{dfn}
Let $V$ be a linear space over $\mathbb{R}$ or $\mathbb{C}$. A function $\|\;\|: X\to \mathbb{R}$ assigning to each vector $\boldsymbol{x}\in X$ a real number $\|x\|$ is called a \textbf{norm}\index{norm} in the linear space $X$ if:
\begin{enumerate}[a)]
	\item 
	$\|\boldsymbol{x}\|=0\Leftrightarrow \boldsymbol{x}=\boldsymbol{0}$ (nondegeneracy);
	\item
	$\|\lambda \boldsymbol{x}\| = |\lambda|\|\boldsymbol{x}\|$ (homogeneity);
	\item
	$\|\boldsymbol{x}_1+\boldsymbol{x}_2\|\leq 
	\|\boldsymbol{x}_1\|+\|\boldsymbol{x}_2\|$ (the triangle inequality).
\end{enumerate}	
A linear space with a norm defined on it is called \textbf{normed}\index{normed}.
\end{dfn}
\printindex
\end{document}