% Used TeXplate:
% TeXplates/Mathematics.tex
% v0.1.7
% https://github.com/HoyanMok/TeXplates
\documentclass[openany]{book} 
% \documentclass{ctexbook} 如果用中文
% \documentclass[10pt,a4paper]{ctexart}  字体大小和纸张大小,默认分别为10pt和letterpaper
% 五号 = 10.5pt,小四=12pt,四号=14pt
% 其他可选参量如twocolumn, 两行排版
\newcommand{\PATH}{./}

\usepackage{biblatex} %[style=gb7714-2015]{biblatex} 可以选择样式
\addbibresource{Analysis.bib} % 把这里改成实际的文件名

% 令参考资料能够加入目录中:
\defbibheading{bibliography}[\bibname]{% 
	% \addcontentsline{toc}{chapter}{参考文献}
	\chapter{#1}% 
	\markboth{#1}{#1}}

\usepackage[notbib, notindex]{tocbibind} % 解决TOC在TOC中的问题

\usepackage{imakeidx} %索引
	\makeindex[intoc, title={Index}]
	\makeindex[intoc, name=symbol, title={Symbol List}]
	\newcommand*{\indexbf}[1]{\emph{\textbf{#1}}\index{#1}} % Index for definition
	\newcommand*{\indexfm}[2][\ ]{#2\index[symbol]{#1@$#2$}} % Used Symbol
	% \indexfm[name for sort]{display} 

% 将PATH换成绝对路径 (Windows) 或相对路径 (Mac OS或Linux)
% 使用「/」而不是「\」

% 对目录项等的修改
\usepackage{chngcntr}
	\counterwithout{section}{chapter} % So that the section won't reset when newing a chapter
\renewcommand{\thesection}{\textmd{\S}\arabic{section}}
\renewcommand{\thesubsection}{\arabic{section}.\arabic{subsection}}


% 引用的宏包:
% 宏包的使用, 可以在命令行运行texdoc <宏包名>获得文档
\usepackage{multicol} % 分栏 (全局分栏建议在文档类处设置)
\usepackage{amsmath} % AMS数学标准
	\makeatletter % '@' now normal "letter"
	\@addtoreset{equation}{section} % 每次换section就把equation清零
	\makeatother  % '@' is restored as "non-letter"
	\renewcommand\theequation{\oldstylenums{\arabic{section}}%
					-\oldstylenums{\arabic{equation}}} % 显示为section数-equation数
\usepackage{amsthm} %定义、证明、定理等
	\theoremstyle{plain}
		\newtheorem{axiom}{Axiom} %公理
		\newtheorem{theorem}{Theorem}[section] %定理
		\newtheorem{corollary}{Corollary} %推论
		\newtheorem{lemma}{Lemma} %引理
	\theoremstyle{definition}
		\newtheorem{definition}{Definition}[section] %定义
		\newtheorem{proposition}{Proposition} %命题
	\renewcommand{\proofname}{\textbf{Proof}}

\renewcommand{\thetheorem}{%
	\arabic{section}.\arabic{theorem}%
} % 公式编号不显示`\S`
\renewcommand{\thedefinition}{%
	\arabic{section}.\arabic{definition}%
} % 定义编号不显示`\S`
\usepackage{amssymb} % 数学符号
\usepackage{mathrsfs} % 花体
\usepackage{esint} % 积分
\usepackage{siunitx} % 标准SI数值和单位处理

\usepackage{tikz} % 绘图
\usepackage{float} % 浮动体 (供图片, 表格等) 扩展, 主要用于提供h模式
\usepackage{graphicx} % 插入图片
\usepackage{titlepic}
\usepackage[font=small, skip=5pt]{caption} % 缩小题注字体和题注与图片距离
\usepackage{subcaption} % 子图和子图的题注
\usepackage{svg} % svg位图
\usepackage{wrapfig} % 简单的图文绕排
\usepackage[inline]{enumitem} % 编号
	% 新列表:
	\newlist{conditionlist}{enumerate}{2}
	\setlist[conditionlist,1]{topsep = 0pt, itemsep = 0pt, parsep = 0pt,%
		label=\arabic*), leftmargin=2\parindent}
	\setlist[conditionlist,2]{topsep = 0pt, itemsep = 0pt, parsep = 0pt,%
		label=\alph*), leftmargin=3\parindent}
\usepackage{geometry} % 调整页边距
% \geometry{left=1.6cm,right=1.6cm}
\usepackage{xcolor} % 颜色
\usepackage[colorlinks=true,bookmarks=true]{hyperref} % 引用, 交叉引用, 图表等的链接; 生成书签
\hypersetup{linkcolor=[rgb]{1,0.27,0},bookmarksopen = true}% 更多设置请查阅: texdoc hyperref


% 定义一些笔者常用的指令:
\newcommand{\me}{\mathrm{e}} % 自然对数的底
\newcommand{\mi}{\mathrm{i}} % 虚数单位
\newcommand{\dif}{\mathop{}\!\mathrm{d}} % 微分算子d
\newcommand*{\basis}[1]{\hat{\boldsymbol{#1}}} % 基底
\newcommand*{\bv}{\boldsymbol} % 向量加粗
\newcommand*{\id}{\mathrm{id}} % 单位映射
\newcommand*{\IFF}{\;\leftrightarrow\;} % 充要条件

\newcommand*{\diff}[3][1]
{\if#11%
	\frac{\mathrm{d} #2}{\mathrm{d} #3}% 导数\diff{y}{x}
\else%
	\frac{\mathrm{d}^{#1} #2}{\mathrm{d} #3^{#1}}% n阶导数\diff[n]{y}{x}
\fi}
\newcommand*{\pdiff}[3][1]
{\if#11%
	\frac{\partial #2}{\partial #3}% 偏导数\pdiff{y}{x}
\else%
	\frac{\partial^{#1} #2}{\partial #3^{#1}}% n阶偏导数\pdiff[n]{y}{x}
\fi}
\newcommand{\emphbf}[1]{\emph{\textbf{#1}}}
% \indexbf 的定义见前imakeidx的引用下

% 笔者习惯的运算符:
\DeclareMathOperator{\tg}{tg}
\DeclareMathOperator{\ctg}{ctg}
\DeclareMathOperator{\arctg}{arctg}
\DeclareMathOperator{\sh}{sh}
\DeclareMathOperator{\ch}{ch}
\DeclareMathOperator{\dom}{dom}
\DeclareMathOperator{\ran}{ran}
\DeclareMathOperator{\interior}{int}
\DeclareMathOperator{\card}{card}
% \DeclareMathOperator*{\指令}{显示} 
% 带星号的版本会像\lim一样

% 文章标题页信息:
\title{Analysis}
\author{ Hoyan Mok\thanks{E-mail: victoriesmo@hotmail.com}
}
\date{\today} % 自动生成日期
\begin{document}
\pagenumbering{Alph}
\maketitle % 打印标题
\frontmatter
\chapter{preface}
The latest version: \url{https://github.com/HoyanMok/NotesOnMathematics/tree/master/Analysis} .
\tableofcontents
\mainmatter
\part{Mathematical Analysis}

\chapter{Metric Space and Continuous Map}

\section{Metric Space}
\begin{definition}[Metric]\label{definition: metric} A function
	\begin{equation*}
		d \colon X^2 \to \mathbb R
	\end{equation*}
	$\forall x, y, z\in X$ satisfying: 
	\begin{conditionlist}[label=\alph*)]
	\item	$d(x, y)=0 \IFF x = y$;
	\item	$d(x, y)=d(y, x)$ (symmetry);
	\item	$d(x, z) \leq d(x,y)+d(y,z)$ (triangle inequality),
	\end{conditionlist}
	is called a \indexbf{metric} or \indexbf{distance} in $X$. 
	Such $X$ is said to be equiped with a metric $d$, $\indexfm[X d]{(X; d)}$ is called a \indexbf{metric space}. 
	If the metric defined over $X$ is definite, we just simply call the $X$ the metric space.
\end{definition}

Some examples:
\begin{itemize}
	\item 
	We can define $(\mathbb R^n; d_p)$, where
	\begin{equation}\label{equation:d_p}
		d_p (x, y) := \left(
			\sum_{i \in n}\big|x^i - y^i \big|^p \right)^{1/p}\,,
	\end{equation}
	while
	\begin{equation}\label{equation:d_infty}
		d_\infty (x, y) :=
		\max_{i \in n} \big|x^i - y^i\big|\,.
	\end{equation}
	\item 
	Similarly we can define metric spaces as $(C[a, b]; d_p)$ or simplified $C_p[a, b]$. 
	\begin{equation}
		d_p(f, g) =
		\left(
			\int^b_a \big| f - g \big|^p \dif x
		\right) ^{1/p}\,.
	\end{equation}
	while $C_\infty[a, b]$ is called a \indexbf{Chebyshev metric}, 
	where the metric is defined as $d_\infty(f, g) := \max_{x \in [a, b]} |f(x) - g(x)|$.
	\item 
	On equivalence class $\tilde {\mathfrak R}[a,b]$ over $\mathfrak R[a,b]$ similar metric can be defined. 
	Functions are considered equicalent if they are equal up to a null set. 
\end{itemize}

\begin{lemma}[Quadruple inequality]\label{lemma: quadruple inequality}
	Let $(X;d)$ be a metric space. 
	\begin{equation}\label{equation: quadruple inequality}
		\forall a, b, u, v \in X,\; \big| d(a, b) - d(u, v) \big| \leq d(a, u) + d(b, v) 
	\end{equation}
\end{lemma}
\begin{proof}
	Without loss of generality, we assume that $d(a, b) > d(u, v)$. 
	According to the triangle inequality (see def.~\ref{definition: metric}), $d(a, b) \leq d(a, u) + d(u, v) + d(v,b)$, which is to prove.  
\end{proof}

\begin{definition}[$\delta$-ball]\label{definition: delta ball}
	Let $(X; d)$ be a metric space, and $\delta \in \mathbb R_+$, $a \in X$. 
	A set
	\begin{equation*}
		\indexbf[B a delta]{B(a; \delta)} = \{x \in X \mid d(a, x) < \delta\}
	\end{equation*}
	is then called a \indexbf{ball} with a centre at $a \in X$ and a radius of $\delta$, or a \emphbf{ball} of point $a$.
\end{definition}

\begin{definition}[Open set]\label{definition: open set (metric)}
	An \indexbf{open set}~$G \in 2^X$ in a metric space~$(X; d)$ is a set that satisfies: 
	$\forall x\in G$, $\exists \delta \in \mathbb R_+$, s.t.\ $B(x; \delta) \in 2^G$.
\end{definition}

\begin{definition}[Closed set]\label{closed_set}
	A \indexbf{closed set}~$F \in 2^X$ in a metric space~$(X; d)$ is a set that satisfies: 
	$X - F$ is an open set in $(X; d)$.
\end{definition}

A \indexbf{closed ball}~$\indexbf[overline B x delta]{\overline B(x; \delta)} := \{x \in X \mid d(a, x) \leq r \}$ is an example of closed sets in $(X; d)$.

\begin{proposition}\label{proposition: open sets (metric)}
	\begin{conditionlist}[label=\alph*)]
	\item An infinite union of open sets is an open set.
	\item A definite intersection of open sets is an open set.
	\item A definite union of closed sets is a closed set.
	\item An infinite intersection of closed sets is a closed set.
	\end{conditionlist}
\end{proposition}
\begin{proof}
	Let $\forall \alpha \in A$, $G_\alpha$ be open sets.
	\begin{enumerate}[label=\alph*)]
		\item
			$\forall x \in \bigcup_{\alpha \in A} G_\alpha$, $\exists \alpha \in A$ s.t.\ $x \in G_\alpha$. 
			Since $G_\alpha$ is open, $\exists \delta \in \mathbb R_+$ s.t.\ $B(x; \delta) \subset G_\alpha \subset \bigcup_{\alpha \in A} G_\alpha$.
		\item 
			Let $G_1$, $G_2$ be open sets in $(X; d)$. 
			$\forall a \in G_1 \cap G_2$, $\exists \delta_1, \delta_2 \in \mathbb R_+$ s.t.\ $B(a; \delta_1) \subset G_1$, $B(a;\delta_2)\subset G_2$. 
			Without loss of generality, let $\delta_1 \geq \delta_2$, therefore $a \in B(a; \delta_1)\cap B(a; \delta_2) = B(a; \delta_2) \subset G_1 \cap G_2$.
		\item 
			Just consider $\complement_X
			\left(\bigcap_{\alpha\in A}F_\alpha\right)
			=\bigcup_{\alpha\in A}\complement_X(F_\alpha)$ and a).
		\item 
		Similarly, $\complement_X\left(F_1\cup F_2\right)=\complement_X(F_1)\cap\complement_X(F_2)$.
	\end{enumerate}
\end{proof}

\begin{definition}[Neighbourhood]\label{definition: neighbourhood (metric)}
	If $x \in X$ is an element of an open set, then such open set is called a \indexbf{neighbourhood} of point $x$ in $X$, denoted by $\indexbf[U x]{U(x)}$.
\end{definition}

\begin{definition}[Interior point]\label{definition: interior point (metric)}
	Let $x \in X$, $E \subset X$.
	\begin{conditionlist}[label=\alph*)]
	\item If $\exists U(x) \subset E$, $x$ is called an \indexbf{interior point} of $E$.
	\item If $\exists U(x) \subset X - E$, $x$ is called an \indexbf{exterior point} of $E$.
	\item If $x$ isn't an interior point nor exterior point of $E$, it is called a \indexbf{boundary point} of $E$. The set of boundary points is called \indexbf{boundary}, denoted by $\indexbf[partial E]{\partial E}$.
\end{conditionlist}
\end{definition}

\begin{definition}[Limit point]\label{definition: limit point (metric)}
	$a \in X$, $E \subset X$. If $\forall U(a)$, $\card\big(E \cap U(a) \big) = \infty$, $a$ is called a \indexbf{limit point} of $E$.
\end{definition}

\begin{definition}[Closure]\label{definition: closure (metric)}
	The intersections of $E \subset X$ and set of all its limit points is called the \indexbf{closure} of $E$, denoted by $\indexbf[overline]{\overline E}$.
\end{definition}

\begin{theorem}\label{theorem: closed sets' closure (metric)}
	Let $F\in 2^X$.
	$F$ is a closed set in $X$ $\IFF$ $\overline F = F$.
\end{theorem}
\begin{proof}
	$\to$: 
	$\complement_X(F)$ is open, hence its elements are all its interior points. 
	Therefore $\overline F - F = \overline F \cup \complement_X(F) = \varnothing$, also we know that $F \subset \overline F$, hence $F = \overline F$.

	$\gets$: 
	$F = \overline F$ means that $\forall x \in \complement_X(F)$, $x$ is not a boundary of $F$, which implies that $x$ is an interior point of $X - F$. Therefore $X - F$ is open while $F$ is closed.
\end{proof}

\begin{theorem}\label{theorem: closure is closed (metric)}
	$\overline E$ is always closed.
\end{theorem}
\begin{proof}
	$\forall x \in X - \overline E$, since it is not an element of the set $E$ nor its limit points, $\exists U(x) $ s.t.\ $U(x) \cap \overline E =\varnothing$, which implies that $x$ is an extorior point of $E$, therefore $\overline E$ is closed.
\end{proof}

\begin{theorem}\label{theorem: closure's closure}
	$\overline E = \overline{ \overline E}$.
\end{theorem}
\begin{proof}
	Since $\overline E$ is closed, its complement is open, which implies that its elements are all exterior points of $\overline E$, therefore $\overline E$ has contained all of its limit points.
\end{proof}

\begin{definition}(Metric subspace)\label{definition: metric subspace}
	We called $(X'; d')$ a \indexbf{subspace} of $(X;d)$ when $X' \subset X$ and $\forall x, y \in X'$, $d'(x, y)=d(x, y)$.
\end{definition}

\section{Topological Space}

\begin{definition}[Topology]\label{definition: topology}
	We say $X$ is epuiped with a \indexbf{topology} if we assigned a $\mathscr T \subset 2^X$, with the following propoties:
	\begin{conditionlist}[label=\alph*)]
		\item 
		$\varnothing \in \mathscr T$; 
		$X \in \mathscr T$.
		\item 
		$\big(\forall \alpha \in A, G_\alpha \in \mathscr T\big)
			\to \bigcup_{\alpha \in A} G_\alpha \in \mathscr T$.
		\item $\forall G_1, G_2 \in \mathscr T$, $G_1 \cap G_2 \in \mathscr T$.
	\end{conditionlist}

	We call $(X; \mathscr{T})$ a \indexbf{topological space}, and sometimes we might simply call $X$ the topological space.
\end{definition}

These conditions is the intrinsic propoties of the open sets we have defined in the metric space%
	\footnote{See proposition~\ref{proposition: open sets (metric)}%
		}. 
The topology consisting of all the open sets defined in the metric space $(\mathbb{R}; d_2)$ is called the \indexbf{standard topology} of the $n$-dimension Euclidean space.

\begin{definition}[Open set]\label{definition: open set (topology)}
	Topology $\mathscr{T}$'s elements are called \emphbf{open sets}%
		\index{open set}%
	, and their complements are called \emphbf{closed sets}%
		\index{closed set}%
	.
\end{definition}

\begin{definition}[Base]\label{definition: base}
	Let $(X; \mathscr T)$ be a topological space, and $\mathfrak B \subset 2^X$. 
	If $\forall G \in \mathscr T$, $\exists \{B_\alpha\}_{\alpha \in A} \in 2^\mathfrak B$ s.t.\ $\bigcup_{\alpha \in A} B_\alpha = G$, we called $\mathfrak B$ a (topological or open) \indexbf{base}%
		\index{topological base}\index{open base}%
		of the topology $\mathscr T$.
\end{definition}

\begin{definition}[Weight]\label{definition: weight}
	The smallest possible cardinity of a base of a topology is called the \indexbf{weight} of the topological space.
\end{definition}

\begin{definition}[Neighbourhood]\label{definition: neighbourhood (topology)}
	If $x \in G$ and $G \in \mathscr T$, then $G$ is a \indexbf{neighbourhood} of $x$ in topological space $(X; \mathscr T)$.
\end{definition}

For example, we define an equivalence relation $\sim$ in $C(\mathbb R;\mathbb R)$. If $f, g\in C(\mathbb R; \mathbb R)$, at point $a \in \mathbb R$:
\begin{equation}\label{equation: germ}
	f \sim_a g \IFF
			\exists U(a) \big(\forall x \in U(a),\; f(x) = g(x)\big)\,.
\end{equation}

By collecting all of the continuous functions that are euivalent to $f$, 
we call $f$ define a \indexbf{germ} at point $a$, denoted by $f_a$.
If $f \in C(\mathbb R; \mathbb R)$ is defined in $U(a)$, then we can call $\{f_x \mid x \in U(a)\}$ a neighbourhood of germ $f_a$. 
Class of neighbourhoods of each $f_x$ constructs a base of topological space $(C(\mathbb R; \mathbb R); \mathscr T)$, where $\mathscr T$ is made of the sets of germs of continuous function in $C(\mathbb{R}; \mathbb{R})$.

\begin{definition}[Hausdorff space]\label{definition: Hausdorff space}
	We call a topological space $(X; \mathscr{T})$ a \indexbf{Hausdorff space}, \indexbf{separated space} or \indexbf{$\mathrm T_2$ space}, if $\forall x,y \in X$, $\exists U(x), U(y)$ s.t.\ $U(x) \cap U(y) = \varnothing$%
		\footnote{This definition is also called \indexbf{Hausdorff axiom} or \indexbf{separation axiom}. }%
	.
\end{definition}

\begin{definition}[Dense set]\label{definition: dense set}
	$E \subset X$ is a \indexbf{dense set} in the topological space $(X; \mathscr{T})$, if $\forall x \in X$, $\forall U(x)$, $U(x) \cap E \neq \varnothing$.
\end{definition}

\begin{definition}[Separable]\label{definition: separable}
	If there is a \emph{countable} dense set in topological space $(X; \mathscr{T})$, then $(X;\mathscr{T})$ is \indexbf{separable}.
\end{definition}

We can also define interior points, exterior points, boundary points, limit points in topological space as in metric space.

\begin{definition}[Topological subspace]\label{subspace (topology)}
	Each subset $Y$ of $X$ equiped with topology $\mathscr T$ can be given a \indexbf{subspace topology} $\mathscr{T}_Y$ whose elements $G_Y$ are intersections of the subset with an open set $G$ in $(X; \mathscr{T})$ i.e.\ $\forall G_Y \in \mathscr T_Y$, $\exists G \in \mathscr{T}$ s.t.\ $G_Y = G \cap Y$. 
	Subsets equiped with such topology construct a \emphbf{topological subspace}%
		\index{subspace}%
		~$(Y; \mathscr{T}_Y)$. 
\end{definition}

If two topology~$\mathscr{T}_1, \mathscr{T}_2$ are defined on the same $X$, $\mathscr{T}_1$ is said to be \indexbf{stronger} than $\mathscr{T}_2$ if $\mathscr{T}_1 \subsetneqq \mathscr{T}_2$.

\section{Compact Set}
\begin{definition}\label{compact}
Set $K$ in topological space $(X;\mathscr{T})$ is called a \indexbf{compact set} if each of its \textbf{open covers}\index{open cover} has a finite \indexbf{subcover}. Class $\Omega$ is called a open cover of $K$ if $K\subset \cup{\Omega}$ and for all sets in $\Omega$ are open sets.
\end{definition}
Specially, $\varnothing$ is compact.
\begin{theorem}\label{selfcompact}
	Set $K\subset X$ is compact in $(X;\mathscr{T})$ \emphbf{iff} $K$ is compact in $(K;\mathscr{T}_K)$ itself. 
\end{theorem}
This theorem tells a truth that whether $K$ is compact or not isn't dependent on the topological space it's in, it can be easily proofed: just need to notice that every open set $G_K$ in $(K;\mathscr{T}_K)$ is an intersection of an open set $G$ in $(X;\mathscr{T})$ and $K$. 
\begin{theorem}\label{compact_Hausdorff_closed}
	If $K$ is compact in a Hausdorff space $(X;\mathscr{T})$ (See definition~\ref{T_2_space}), then $K$ is a closed set in $(X;\mathscr{T})$.
\end{theorem}
\begin{proof}
	If $x_0$ is a limit point of $K$, which means $\forall U(x_0)$, 
\[
	\left|U(x_0)\cap K\right|\notin \mathbb{N}.
\]
Assume that $x_0\notin K$. In a Hausdorff space, $\forall x\in K$, $\exists U(x)$ s.t. $U(x)\cap U(x_0)=\varnothing$. Such $U(x)$ construct a open cover $\Omega=\left\{U(x)|x\in K\right\}\subset 2^K$. Since $K$ is compact, $\exists \Omega'\subset\Omega$ s,t. $\left|\Omega\right|\in\mathbb{N}$. 
\[
	\left(\cup\Omega'\right)
	\cap U(x_0)
	=
	\left(\bigcup_{k=1}^n{U_k}\right)
	\cap U(x_0)
	=
	\bigcup_{k=1}^n\left(
		U_k\cap U(x_0)
	\right)
	=
	\varnothing
\]
Since $K\subset \cup\Omega'$, $x_0$ is an exterior point of $K$, which leads to a contradiction. Hence $x_0\in K$. $\overline{K}=K$.
\end{proof}
\begin{theorem}\label{nested_compact}
Each decreasing \emph{\textbf{nested sequences}}\index{nested sequence} of non-empty compact sets has a non-empty limit, i.e. $\forall\{K_n\}$ s.t. $\forall n\in\mathbb{N}_+$, $K_n\supset K_{n+1}\wedge K_n\neq\varnothing\wedge (K_n$ is compact), $K_n\downarrow K\neq \varnothing$.
\end{theorem}
\begin{proof}
	Assume that $K=\varnothing$. Compact subsets of $K_1$ are all colsed, while their complements are all open. An open cover $\Omega$ can be constructed as $\{K_1-K_n|n\in\mathbb{N}_+\}$. Since $K_1$ is compact, there would be a finite subcover $\Omega'\subset\Omega$, notice that $\{X-K_n\}$ is also a nested sequence, there must be oone single $X-K_{n_0}\in\Omega'$ that covers $K_1$, which means $K_{n_0}=\varnothing$ contradicting that $\forall n\in\mathbb{N}_+$, $K_n$ is non-empty.
\end{proof}
\begin{theorem}\label{compact_closed_subset}
Closed subsets $F$ of a compact set $K$ are also compact.
\end{theorem}
\begin{proof}
	If $\Omega_F\subset 2^K$ is an open cover of $F$. Notice that $K - F$ is open, $\Omega=\left(\cup\Omega_F\right)\cap\{K-F\}$ constructs an open cover over over $K$. Since $K$ is compact there must be a finite cover $\Omega'\subset\Omega$ which obviously also covers over $F$.
\end{proof}
The following propoties of compact sets are on the topological space induced from a metric space.
\begin{definition}\label{e-net}
	$(X;d)$ is a metric space, $E\subset X$. $E$ is called an \indexbf{$\varepsilon$-net} if $\forall x\in X$,$\exists e\in E$, $d(e,x)<\varepsilon$.
\end{definition}
\begin{theorem}\label{finite_e-net}
If $(K,d)$ is a compact metric space, then $\forall \varepsilon\in\mathbb{R}_+$, $\exists$ finite $\varepsilon$-net in $(K;d)$. 
\end{theorem}
\begin{proof}
	For each point $x\in K$, find it a $B(x,\varepsilon)$, of which an infinite cover $\Omega$ over $K$ is made. Since $K$ is compact, there exists a finite cover $\Omega'=\{B(x_1,\varepsilon),\cdots,B(x_n,\varepsilon)\}$ ($n\in\mathbb{N}_+$). Therefore $\{x_1,\cdots,x_n\}$ is a finite $\varepsilon$-net in $K$.
\end{proof}
\begin{theorem}\label{sequentially_compact_metric}
$(K;d)$ is compact \emphbf{iff} it is \indexbf{sequentially compact}, that is, $\forall \{x_n\}$ ($x_n\in K$, $n\in\mathbb{N}_+$), it has convergent subsequence $\{x_{k_n}\}$ whose limit $a\in K$.
\end{theorem}
To proof it, we need to proof two lemmata first.
\begin{lemma}\label{sequentially_compact_finite_e-net}
	If $(K;d)$ is sequentially compact, then $\forall \varepsilon\in\mathbb{R}_+$, $\exists$ finite $\varepsilon$-net in $(K;d)$. 
\end{lemma}
\begin{proof}
	Assume that there were no finite  $\varepsilon_0$-net in $(K;d)$. Define such sequence : $ \{x_n\}$ s.t. $\forall k,n\in\mathbb{N}_+$ ($1\leq k< n$), $d(x_n,x_k)\geq\varepsilon_0$ (There would always be the next one since there exists no $\varepsilon_0$-net). It has no convergent subsequence for it there were a $\{x_{k_n}\}$ convergent to $a\in K$, $\exists N,M\in\mathbb{N}_+$, $d(x_N,x_M)\leq d(x_N,a)+d(x_M,a)\leq \varepsilon_0$, which lead to a contradictary. 
\end{proof}
\begin{lemma}\label{sequentially_compact_nested_closed}
If $(K;d)$ is sequentially compact then every nested sequence of closed non-empty sets $\{F_n\}$ in $K$ have a non-empty intersection.
\end{lemma}
\begin{proof}
	Let $\{x_{k_n}\}$ be a convergent subsequence of $\{x_n\}$, Let $a$ be the limit of $\{x_{k_n}\}$ ($\forall n\in\mathbb{N}_+$,$x_n\in F_n$). Assume that $a\notin \bigcap_{n\in\mathbb{N}_+}F_n$, in metric space, $\exists U(a)\cap \left(\bigcap_{n\in\mathbb{N}_+}F_n\right)=\varnothing\Rightarrow $ $U(a)\cap \left(\bigcap_{n\in\mathbb{N}_+}F_{k_n}\right)=\varnothing$. But this conflict the fact that $\exists N\in\mathbb{N}_+$, s.t. $n>N$, $x_{k_n}\in U(a)$ while $x_{k_n}\in F_{k_n}$.
\end{proof}
Then get back to theorem \ref{sequentially_compact_metric}. 
\begin{proof}

	$\Rightarrow$: If $\left|\{x_n\}\right|\in\mathbb{N}$, it is obvious; if $\left|\{x_n\}\right|=\infty$, make finite $\frac{1}{n}$-net (Theorem \ref{finite_e-net}), $n\in\mathbb{N}_+$. For each $n$, there must be at least one $B(x_n;\frac{1}{n})$ that includes infinite elements in $\{x_n\}$. Select $x_{k_n}\in B(x_n;\frac{1}{n})$, and $\{\tilde{B}(x_n;\frac{1}{n})\}$ is a nested sequence of a closed non-empty sets in sequentially compact $K$, (Lemma \ref{sequentially_compact_nested_closed}) $\lim\limits_{n\to \infty} x_{k_n} \in K$.
	
	$\Leftarrow$: Assume that there were a open cover $\Omega$ over $K$ having no finite subcover, $\forall n\in\mathbb{N}_+$, $\exists$ finite $\frac{1}{n}$-net (Lemma \ref{sequentially_compact_nested_closed}), in which there would be at least one $x_n$ whose $\tilde{B}(x_n;\frac{1}{n})$ can't be covered finitely. Then $\tilde{B}(x_n;\frac{1}{n})\downarrow B=\{a\}$ (Theorem \ref{nested_compact}) can't be finitely covered by any subcover of $\Omega$ which means $\Omega$ can't cover the whole $K$, leading to the contradiction.
\end{proof}
\section{Connected Set}
\begin{definition}\label{connected_space}
Topological space $(X;\mathscr{T})$ is called \indexbf{connected}\index{connected space} if there is no \indexbf{open-closed set} (i.e. both open and closed) besides $\varnothing$ and $X$ itself. 
\end{definition}
Notice that if $A\subset X$ is open-closed, its complement $X - A$ is also open-closed, which means a topological space is connected \emphbf{iff} it is not a union of its two open subsets. 
\begin{definition}\label{connected_set}
$(X;\mathscr{T})$ is a topological space. Subset $C$ is said to be \indexbf{connected}\index{connected set} if subspace $(C;\mathscr{T}_C)$ is connected. 
\end{definition}
\begin{theorem}\label{union_connected}
$( X; \mathscr{T} )$ is a topological space. $\forall\alpha \in A$, $C_\alpha$ are connected subsets of $X$. If $\bigcap\limits_{ \alpha \in A} C_\alpha \neq \varnothing$, then $\bigcup\limits_{\alpha \in A} C_\alpha$ is also connected. 
\end{theorem}
\begin{proof}
If $C = \bigcup\limits_{\alpha \in A} C_\alpha$ were not connected, $\exists E \subset C$ s.t. $E\neq \varnothing \wedge E \neq C \wedge E, C - E \in \mathscr{T}_C$. For $E$ is not empty there exists a $\beta \in A$ s.t. $E \cap C_\beta \neq \varnothing$. It can be proofed that $C_\beta \subset  E$.

Suppose that $C_\beta \nsubseteq  E$, which implies that $( C - E) \cap C_\beta \neq \varnothing$. 
$E, C - E, C_\beta \in \mathscr{T}_C
	 \Rightarrow
	  	E\cap C_\beta, ( C - E) \cap C_\beta \in \mathscr{T}_C$. 
This conflicts to the fact that $C_\beta$ is connected. Therefore $C_\beta \subset  E$. 

Hence there exists a $B \subsetneqq A$, $\bigcup\limits_{ \beta \in B} C_\beta = A$. Since $C_\gamma$, $\gamma \in A - B$ would have a empty intersection with $E$, which contradicts $\bigcap\limits_{ \alpha \in A} C_\alpha \neq \varnothing$.
\end{proof}
\begin{theorem}\label{closure_connected}
Connected sets have connected closure.
\end{theorem}
\begin{proof}

\end{proof}
\begin{theorem}\label{R_connected}
$E\subset\mathbb{R}$ is connected \emphbf{iff} that if $\forall x,z\in E$, $y\in\mathbb{R}$ s.t. $x<y<z$, then $y\in C$.
\end{theorem}
\begin{proof}
	$\Rightarrow$: Assume that there were such $y\in\mathbb{R}$ that $\exists x,z\in C$, $x<y<z$ but $y\notin C$. $\{x\in C|x<y\}$ and $\{x\in C|x>y\}$ are open in $C$for they are intersection of open sets in $\mathbb{R}$ and $C$. Since they're each other's complement, they are both open-closed, which conflict to the definition of connected set.
	
	$\Leftarrow$: It can be proofed that $(\inf C,\sup C)\subset C$. Assume that there were an open-closed proper subset $E\neq\varnothing$ contained in $C$. Find two points $x\in E$, $z\in C - E$. Without loss of generality, let $x<z$. Since $E$ and $C-E$ are closed, $c_1=\inf \{E\cap \left[a,b\right]\}\in E$ while $c_2=\inf \{(C-E)\cap [a,b]\}\in C-E$. However $E\cap(C-E)=\varnothing\Rightarrow c_1<c_2$, which means $(c_1,c_2)\cap E=\varnothing$. Here's the contradiction.
\end{proof}
\begin{definition}\label{locally_connected}
A topological space $( X; \mathscr{T})$ is said to be \indexbf{locally connected} if $\forall x \in X$, $\exists U( x) $ s.t. $U( x)$ is connected.
\end{definition}
\section{Complete Metric Spaces}
We now take a closer look at one of the most important sorts of metric spaces: complete spaces.
\begin{definition}\label{Cauchy_sequence}
A sequence $\{x_n\mid n\in \mathbb{N}\}$ of points of a metric space $(X;d)$ is called a \indexbf{fundamental}\index{fundamental sequence} or \indexbf{Cauchy sequence} if $\forall \varepsilon \in\mathbb{R}_+$, $\exists N\in\mathbb{N}$ s.t. as long as $m,n> N$, $d(x_n,x_m) < \varepsilon$.
\end{definition}
\begin{definition}\label{complete_space}
A metric space $(X;d)$ is \indexbf{complete} if every Cauchy sequence of its points is convergent.
\end{definition}
For example, metric space $C_\infty[a,b]$ is complete while $C_1[a,b]$ isn't. Proof see p22, Zorich.
Consider incomplete space $\mathbb{Q}_1$, which is a subspace of the complete space $\mathbb{R}_1$. If $\mathbb{R}_1$ is the smallest complete space containing $\mathbb{Q}_1$, we can say that we have achieved a \textbf{completion} of $\mathbb{Q}_1$. However, the definition of ``completion'' hasn't been defined yet. 
\begin{definition}\label{completion}
If a metric space $(X;d)$ is a subspace of a complete metric space $(Y;d)$ and everywhere dense in it, we call the latter one the \indexbf{completion} of $(X;d)$. 
\end{definition}
We need to confirm that such completion is the smallest and unique. So we introduce:
\begin{definition}\label{isometric}
If there exists a \indexbf{isometry} $f:X_1\to X_2$ when $(X_1;d_1)$ and $(X_2;d_2)$ are both metric space, i.e. $f$ is a bijective and for each $a,b\in X_1$, $d_2\left(f(a),f(b)\right)=d_1\left(a,b\right)$, then these two metric space is \indexbf{isometric}.
\end{definition}
This relation is reflexive ($e$), symmetric ($f ^{-1}$), and transitive ($f\circ g$), so it is a equivalence relation, noted by $\sim$. We shall consider isometric spaces are identical.
\begin{theorem}\label{completion_unique}
If metirc spaces $(Y_1;d_1)$ and $(Y_2;d_2)$ are both completions of $(X;d)$, then they are isometric.
\end{theorem}
\begin{proof}
Such isometry $f:Y_1\to Y_2$ can be defined: if $x_1,x_2\in X$, 
\[
	d_2( f(x_1) , f(x_2) )=d(f(x_1), f(x_2)) = d(x_1 , x_2)=d_1(x_1, x_2).
\]

For each $y_1\in Y_1 - X_1$, a Cauchy sequence $\{x_n\}$ can be found in the nested sequence of balls centered in $y_1$. It is obvious that $\{x_n\}$ is also fundamental in $Y_2$, limitting to $y_2\in Y_2$. Different sequences of points $\{x'_n\}$ selected won't result in a diffrent $y'_2$, or $d(x_n,x'_n)$ wouldn't converge to $0$, which violate the fact that the radii of balls converge to $0$. Let $f(y_1)=y_2$. 

a) For each $y_2\in Y_2 - X$, there always exists a Cauchy sequence converging to it, which implies that $f$ is a surjection.

b) Also notice that $\forall y'_1,y''_1\in Y_1 - X$,
\[
	d_1(y'_1,y''_1)= \lim_{n\to\infty} d(x'_n,x''_n)=d_2(y'_2,y''_2)
\]
while $\{x'_n\}$ and $\{x''_n\}$ are both Cauchy sequence. This equality also proofed that $f$ is a injection.
\end{proof}
\begin{theorem}\label{completion_exists}
There always exists a completion for every metric space.
\end{theorem}
\begin{proof}
	A isometric space $(S_X;d)$ to the metric space $(X;d_X)$ can be constructed, which consists of constant sequence of points in $X$. Its completion $(S;d)$ can be defined as Cauchy sequences whose mutual distances' limits are not $0$.
\end{proof}
\section{Continuous Mapping}
Let's recall the definition of the limitation.
\begin{definition}\label{filter_base}
A set $\mathscr{B}\subset 2^X$ is called a \textbf{(filter) base}\index{filter base}\index{base} in $X$ if the following conditions hold:
\begin{conditionlist}[label=\alph*)]
\item $\varnothing \notin \mathscr{B}$.
\item $\forall B_1,B_2\in \mathscr{B}$, $\exists B\in \mathscr{B}$ s.t. $B\subset B_1\cap B_2\subset B_2$. 
\end{conditionlist}
\end{definition}
Introduction of the limits in a topological space is as follows.
\begin{definition}\label{limit}
Let $a\in Y$ be the \indexbf{limit} over the base $\mathscr{B}\subset 2^{\mathscr{D}(f)}$ of a mapping $f:\mathscr{D}( f )\to Y$, in which $Y$ is epuiped with a topology $\mathscr{T}$. 
\[
	\lim_\mathscr{B} f = a 
	\quad:=\quad
	\forall U(a)\subset Y\;
	\exists B\in \mathscr{B}(f(B)\subset U(a)).
\]
\end{definition}

Such definition is parallel to the definition we have introduced on the limits of real number, hence it basically holds the same propoties. 
\begin{definition}\label{continuous}
A mapping $f:X\to Y$, where $X$,$Y$ is respectively equiped with topology $\mathscr{T}_X$,$\mathscr{T}_Y$, is said to be \indexbf{continuous} at $x_0\in X$ (let $y_0 = f( x_0 ) \in Y$), if $\forall U( y_0 )$, $\exists U( x_0 )$ s.t. $f( U(x_0) )\subset U( y_0 )$. It is \indexbf{continuous} in $X$ if it is continuous at each point $x\in X$. 
\end{definition}
The set of continuous mappings from $X$ into $Y$ can be denoted by $C(X,Y)$ or $C(X)$ when $Y$ is clear. 
\begin{theorem}[\textbf{Criterion for continuity}]
	\index{criterion for continuity}\label{criterion_continuity}

\noindent 
$(X;\mathscr{T}_X)$ and $(Y;\mathscr{T}_Y)$ are both topological spaces. A mapping $f: X\to Y$ is continuous \emphbf{iff} $\forall G_Y\in \mathscr{T}_Y$, $f ^{-1} ( G_Y ) \in \mathscr{T}_X$.
\end{theorem}
\begin{proof}
$\Rightarrow$: It is obvious if $f  ^{-1}(G_Y) = \varnothing$. If $f  ^{-1}(G_Y) \neq \varnothing$ and $x_0 \in X$, since $f \in C(X,Y)$, for $G_Y$, $\exists U( x_0) $ s.t $f ( U( x_0)) \subset G_Y$. Also notice that $f ( U( x_0)) \subset G_Y \Rightarrow U( x_0) \subset f  ^{-1}(G_Y)$, therefore $f  ^{-1}(G_Y) $ is open.

$\Leftarrow$: $\forall x_0 \in X$, let $y_0 = f ( x_0)$, $f ^{-1} ( U( y_0)) \in \mathscr{T}_X$. Notice that $x_0 \in f ^{-1} ( U( y_0))$, therefore $f \in C( X, Y) $.
\end{proof}
\begin{definition}\label{homeomorphism}
$(X;\mathscr{T}_X)$ and $(Y;\mathscr{T}_Y)$ are both topological spaces. A bijective mapping $f: X\to Y$ is a \indexbf{homeomorphism} if $f \in C( X, Y) \wedge f ^{ -1} \in C( Y, X)$. 
\end{definition}
\begin{definition}\label{homeomorphic}
Two topological spaces $(X;\mathscr{T}_X)$ and $(Y;\mathscr{T}_Y)$ are said to be \indexbf{homeomorphic} if there exists a homeomorphism $f: X\to Y$.
\end{definition}
Homeomorphic topological spaces are identical with respect to their topological propoties since the theorem~\ref{criterion_continuity} has shown that their open sets correspond to each other.
\begin{theorem}\label{compact_continuous}
$(X;\mathscr{T}_X)$ and $(Y;\mathscr{T}_Y)$ are both topological spaces. $K\subset X$ is a compact set. If $f: X\to Y \in C( X, Y)$, then $f( K)$ is compact.
\end{theorem}
\begin{proof}
For each open cover $\Omega_Y = \{ G_Y \in \mathscr{T}_Y\} \subset \mathscr{T}_Y$ over $f( K)$, $f ^{-1} ( G_Y) \in \mathscr{T}_X$ (Therem~\ref{criterion_continuity}). $f( K) \subset \cup\,\Omega_Y \Rightarrow K \subset f ^{-1} \left(  \cup\,\Omega_Y \right) = \cup\,\Omega_X $, where $\Omega_X = \{ f ^{-1} ( G_Y) \mid G_Y \in \Omega_Y\} $ is an open cover over $K$. Since $K$ is compact,
$\exists \Omega'_X \subset \Omega_X\left( 
\lvert \Omega'_X \rvert \in \mathbb{N}_+ 
\;\wedge\;K\subset \cup\,\Omega'_X
\right)$, $f( K) \subset f ( \cup\,\Omega'_X) $. $f ( G'_X) \in \Omega_Y$, hence $\Omega'_Y = \{ f ( G'_X) \mid G'_X \in \Omega'_X\}$ is a finite subcover over $f( K)$.
\end{proof}
\begin{theorem}\label{compact_Hausdorff_continuous_homeomorphism}
$(K;\mathscr{T}_K)$ is a compact space and $(Y;\mathscr{T}_Y)$ is a Hausdorff space. If a bijective $f: K\to Y \in C( K, Y)$, then it is a homeomorphism.
\end{theorem}
\begin{proof}
$\forall F = K - G$ s.t. $G \in \mathscr{T}_K$ is compact (Theorem~\ref{compact_closed_subset}). Hence $f ( F) $ is compact (Theorem~\ref{compact_continuous}), then it is also closed
(Theorem~\ref{compact_Hausdorff_closed}). This fact shows that $f ^{-1}$ is continuous (Theorem~\ref{criterion_continuity}).
\end{proof}
\begin{theorem}\label{connected_continuous}
$(X;\mathscr{T}_X)$ and $(Y;\mathscr{T}_Y)$ are both topological spaces. $E\subset X$ is a connected set. If $f: X\to Y \in C( X, Y)$, then $f( E)$ is also connected.
\end{theorem}
\begin{proof}
Only to notice that the open-closed sets in $( f( E) ; \mathscr{T}_{ f( E)})$ have concurrently open-closed pre-images in $( E; \mathscr{T}_E)$.
\end{proof}
\section{Contraction}
\begin{definition}\label{fixed_point}
A point $a\in X$ is a \indexbf{fixed point} of a mapping $f: X\to X$ if $f( a) = a$.
\end{definition}
\begin{definition}\label{contraction}
Let~$(X;d)$ be a metric space. A mapping~$f: X\to X$ is called a \indexbf{contraction} if~$\exists q \in ( 0, 1) \subset \mathbb{R}$ s.t. $\forall x_1,x_2\in X$, 
\begin{align}\label{inequality_contraction}
	d ( f ( x_1), f (x_2)) \leq q d( x_1, x_2).
\end{align}
\end{definition}
\begin{lemma}\label{contraction_continuous}
A contraction $f: X\to X$ is always continuous.
\end{lemma}
\begin{proof}
$\forall x\in X$, $\forall \varepsilon \in \mathbb{R}_+ $, $\exists \delta < \varepsilon / q$, according to inequality~\ref{inequality_contraction}:
\[
f \left( B( x;\delta )\right) 
\subset 
B\left( f( x); \varepsilon \right).
\]
\end{proof}
\begin{theorem}[%
	\textbf{Picard-Banach fixed-point principle} or \textbf{contraction mapping principle}%
]\label{contraction_mapping_principle}
\index{Picard-Banach fixed-point principle}\index{contraction mapping principle}

Let $( X; d)$ be a complete metric space. Each contraction $f: X\to X$ has a unique fixed point $a$. 
Also, $\forall \{x_n\} \subset X$ s.t.
$\forall n \in \mathbb{N}
\left(f ( x_n) = x_{n+1} \right)$ then $\lim\limits_{n\to \infty} x_n = a$, and
\begin{align}\label{inequality_fixed-point_principle}
	d( x_n, a) \leq \frac{q^n}{1-q} d( x_1, x_0).
\end{align}
\end{theorem}
\begin{proof}
By the inequality~\ref{inequality_contraction}:
\[
d ( x_{n+1}, x_n) \leq q d( x_n, x_{n-1})
\leq \cdots 
\leq q^n d( x_1, x_0)
\]
Therefore, $\forall n, k \in \mathbb{N}$, 
\begin{align}\label{inequality_Cauchy_contraction}
	d ( x_{n+k}, x_n) \leq 
	\sum^{k-1}_{i=0} d( x_{n+i+1}, x_{n+i}) \leq
	\sum^{k-1}_{i=0} q^{n+i} d( x_1, x_0) \leq
	\frac{q^n}{1-q} d( x_1, x_0),
\end{align}
which implies that ${ x_n}$ is a Cauchy sequence in a complete space $(X;d)$, hence it converges to a point $a \in X$.

To proof that $a$ is a fixed point of $f$, since $f$ is continuous (Lemma~\ref{contraction_continuous}), just notice that 
\[
	a = \lim_{n\to \infty} f(x_n) = 
	f( \lim_{n\to \infty} x_n) = f( x_n).
\]
If there were a second fixed point $a'\in X$ of $f$, then:
\[
	0 \leq d( a, a')  = d( f(a), f(a') ) \leq q d( a, a')
\]
which can't be true unless $a = a'$. 

By passing to the limit as $k \to \infty$ in the inequality~\ref{inequality_Cauchy_contraction}, we have the inequality~\ref{inequality_fixed-point_principle}.
\end{proof}

\chapter{Normed Linear Space and Differential Calculus}
\section{Normed Linear Space}
\begin{definition}
Let $V$ be a linear space over $\mathbb{R}$ or $\mathbb{C}$. A function $\|\;\|: X\to \mathbb{R}$ assigning to each vector $\boldsymbol{x}\in X$ a real number $\|x\|$ is called a \indexbf{norm} in the linear space $X$ if:
\begin{conditionlist}[label=\alph*)]
	\item 
	$\|\boldsymbol{x}\|=0\Leftrightarrow \boldsymbol{x}=\boldsymbol{0}$ (nondegeneracy);
	\item
	$\|\lambda \boldsymbol{x}\| = |\lambda|\|\boldsymbol{x}\|$ (homogeneity);
	\item
	$\|\boldsymbol{x}_1+\boldsymbol{x}_2\|\leq 
	\|\boldsymbol{x}_1\|+\|\boldsymbol{x}_2\|$ (the triangle inequality).
\end{conditionlist}	
A linear space with a norm defined on it is called \indexbf{normed}.
\end{definition}
%\appendix

\backmatter
\nocite{*} % 这个表示列出所有没有在文中被引用的参考文献
\printbibliography[heading=bibliography, title={Bibliography}]

\indexprologue{Here listed the important symbols used in this notes.}
\printindex[symbol]

\printindex
\end{document}