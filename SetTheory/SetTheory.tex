% TeXplates/Mathematics.tex
% v0.1.7
% https://github.com/HoyanMok/TeXplates
\documentclass[openany]{ctexbook} 
% \documentclass[10pt,a4paper]{ctexart}  字体大小和纸张大小,默认分别为10pt和letterpaper
% 五号 = 10.5pt,小四=12pt,四号=14pt
% 其他可选参量如twocolumn, 两行排版

% 将PATH换成绝对路径 (Windows) 或相对路径 (Mac OS或Linux)
% 使用「/」而不是「\」
\newcommand{\PATH}{./}

\usepackage{biblatex} %[style=gb7714-2015]{biblatex} 可以选择样式
\addbibresource{SetTheory.bib} % 把这里改成实际的文件名

% 令参考资料能够加入目录中:
\defbibheading{bibliography}[\bibname]{% 
	% \addcontentsline{toc}{chapter}{参考文献}
	\chapter{#1}% 
	\markboth{#1}{#1}}

\usepackage[notbib, notindex]{tocbibind} % 解决TOC在TOC中的问题

\usepackage{imakeidx} %索引
	\makeindex[intoc, title={索引}]
	\makeindex[intoc, name=symbol, title={符号列表}]
	\newcommand*{\indexbf}[1]{\emph{\textbf{#1}}\index{#1}} % Index for definition
	\newcommand*{\indexfm}[2][\ ]{#2\index[symbol]{#1@$#2$}} % Used Symbol
	% \indexfm[name for sort]{display} 



% 对目录项等的修改
\usepackage{chngcntr}
	\counterwithout{section}{chapter} % So that the section won't reset when newing a chapter
\renewcommand{\thesection}{\textmd{\S}\arabic{section}}
\renewcommand{\thesubsection}{\arabic{section}.\arabic{subsection}}


% 引用的宏包:
% 宏包的使用, 可以在命令行运行texdoc <宏包名>获得文档
\usepackage{multicol} % 分栏 (全局分栏建议在文档类处设置)
\usepackage{amsmath} % AMS数学标准
	\makeatletter % '@' now normal "letter"
	\@addtoreset{equation}{section} % 每次换section就把equation清零
	\makeatother  % '@' is restored as "non-letter"
	\renewcommand\theequation{\oldstylenums{\arabic{section}}%
					-\oldstylenums{\arabic{equation}}} % 显示为section数-equation数
\usepackage{amssymb} % 数学符号
\usepackage{mathrsfs} % 花体
\usepackage{amsthm} %定义、证明、定理等
	\theoremstyle{plain}
		\newtheorem{axion}{Axion} %公理
		\newtheorem{theorem}{Theorem}[section] %定理
		\newtheorem{corollary}{Corollary} %推论
		\newtheorem{lemma}{Lemma} %引理
	\theoremstyle{definition}
		\newtheorem{definition}{Definition}[section] %定义
		\newtheorem{proposition}{Proposition} %命题
	\renewcommand{\proofname}{\textbf{Proof}}

\renewcommand{\thetheorem}{%
	\arabic{section}.\arabic{theorem}%
} % 公式编号不显示`\S`
\renewcommand{\thedefinition}{%
	\arabic{section}.\arabic{definition}%
} % 定义编号不显示`\S`

\usepackage{esint} % 积分
\usepackage{siunitx} % 标准SI数值和单位处理

\usepackage{tikz} % 绘图
\usepackage{float} % 浮动体 (供图片, 表格等) 扩展, 主要用于提供h模式
\usepackage{graphicx} % 插入图片
\usepackage{titlepic}
\usepackage[font=small, skip=5pt]{caption} % 缩小题注字体和题注与图片距离
\usepackage{subcaption} % 子图和子图的题注
\usepackage{svg} % svg位图
\usepackage{wrapfig} % 简单的图文绕排
\usepackage[inline]{enumitem} % 编号
	% 新列表:
	\newlist{conditionlist}{enumerate}{2}
	\setlist[conditionlist,1]{topsep = 0pt, itemsep = 0pt, parsep = 0pt,%
		label=\arabic*), leftmargin=2\parindent}
	\setlist[conditionlist,2]{topsep = 0pt, itemsep = 0pt, parsep = 0pt,%
		label=\alph*), leftmargin=3\parindent}
\usepackage{geometry} % 调整页边距
% \geometry{left=1.6cm,right=1.6cm}
\usepackage{xcolor} % 颜色
\usepackage[colorlinks=true,bookmarks=true]{hyperref} % 引用, 交叉引用, 图表等的链接; 生成书签
\hypersetup{linkcolor=[rgb]{1,0.27,0},bookmarksopen = true}% 更多设置请查阅: texdoc hyperref


% 定义一些笔者常用的指令:
\newcommand{\me}{\mathrm{e}} % 自然对数的底
\newcommand{\mi}{\mathrm{i}} % 虚数单位
\newcommand{\dif}{\mathop{}\!\mathrm{d}} % 微分算子d
\newcommand*{\basis}[1]{\hat{\boldsymbol{#1}}} % 基底
\newcommand*{\bv}{\boldsymbol} % 向量加粗
\newcommand*{\properclass}[1]{\mathbf{#1}}
\newcommand*{\id}{\mathrm{id}}
\newcommand*{\IFF}{\;\leftrightarrow\;} % 充要条件

\newcommand*{\diff}[3][1]
{\if#11%
	\frac{\mathrm{d} #2}{\mathrm{d} #3}% 导数\diff{y}{x}
\else%
	\frac{\mathrm{d}^{#1} #2}{\mathrm{d} #3^{#1}}% n阶导数\diff[n]{y}{x}
\fi}
\newcommand*{\pdiff}[3][1]
{\if#11%
	\frac{\partial #2}{\partial #3}% 偏导数\pdiff{y}{x}
\else%
	\frac{\partial^{#1} #2}{\partial #3^{#1}}% n阶偏导数\pdiff[n]{y}{x}
\fi}
\newcommand{\emphbf}[1]{\emph{\textbf{#1}}}
% \indexbf 的定义见前imakeidx的引用下

% 笔者习惯的运算符:
\DeclareMathOperator{\tg}{tg}
\DeclareMathOperator{\ctg}{ctg}
\DeclareMathOperator{\arctg}{arctg}
\DeclareMathOperator{\sh}{sh}
\DeclareMathOperator{\ch}{ch}
\DeclareMathOperator{\dom}{dom}
\DeclareMathOperator{\ran}{ran}
\DeclareMathOperator{\interior}{int}

% \DeclareMathOperator*{\指令}{显示} 
% 带星号的版本会像\lim一样

\title{Set Theory}
\author{Hoyan Mok
	\thanks{E-mail: victoriesmo@hotmail.com}}
\date{\today}
\begin{document}
\pagenumbering{Alph}
\maketitle

%\thispagestyle{empty}
\frontmatter
\chapter{笔记说明}
该笔记是笔者学习集合论的笔记, 主要是为了数学服务 (而非哲学). 主要的参考资料是\cite{2014集合论}.

笔者曾疑惑于逻辑公理和集合论公理的先后关系, 查阅资料后, 初步得出这样的结论: 我们必须选择朴素的集合论或者简单的一阶逻辑作为前置, 而它们是在我们的元语言中得到保证的. 

本笔记尽量做到自足.

\tableofcontents
\mainmatter

\chapter{集合与公理}
在介绍集合论的
$\properclass{ZFC}$公理之前, 需要先介绍一些数理逻辑的概念. 


\section{数理逻辑准备}

句法概念如\indexbf{形式语言}, \indexbf{逻辑符号}, \indexbf{非逻辑符号}, \indexbf{项}, \indexbf{公式}, \indexbf{自由变元}, \indexbf{约束变元}, \indexbf{语句}等主要见\cite{2014数理逻辑}.

设$\Sigma$是一个公式集, $\varphi$是一个公式.

\begin{definition}
有穷公式序列$\varphi_1,\,\cdots,\,\varphi_n$表示从$\Sigma$到$\varphi$的一个\indexbf{推演}, 如果其中的任意$\varphi_i$要么是属于$\Sigma$的, 要么可从之前的公式$\varphi_j$和$\varphi_k=\varphi_j\to\varphi_i$得到, 而且$\varphi_n=\varphi$. 记作$\indexfm[Sigma vdash varphi]{\Sigma\vdash\varphi}$.

特别地, 如果$T$是语句集, 而$\sigma$是语句, 如果$\indexfm[T vdash sigma]{T\vdash\sigma}$, 就称存在从$T$到$\sigma$的一个\indexbf{证明}.
\end{definition}

如果语句集$T$满足: 对任意语句$\sigma$, $T\vdash\sigma$当且仅当$\sigma\in T$, 即$T$是一个对证明封闭的语句集, 就称$T$为\indexbf{理论}. 假设$T$是理论, 如果存在一个语句集$A\subseteq T$使得对任意的$\sigma\in T$都有$A\vdash \sigma$, 就称$A$为$T$的一集\indexbf{公理}.

如果理论$T$的公理$A$是\indexbf{递归的} (\indexbf{可判定的}, \indexbf{可计算的}) i.e.\ 任给一语句, 总可以在有穷步骤内完全机械地判定它是否属于$A$, 就称$T$是\indexbf{可公理化的}. 理论$T$往往不是递归的, 但如果任给$\sigma\in T$, 我们可在有穷的步骤内得出结论, 但如果$\sigma\notin T$, 我们可能不能在有穷步骤内得出结论, 则称其为\indexbf{递归可枚举的}.

一个理论是\indexbf{一致的}当且仅当没有语句$\sigma$ s.t.\ $T\vdash \sigma\wedge \neg\sigma$.

\begin{definition}
若$\psi$是性质.
\begin{align}
	\exists ! x \psi(x) :=
		\exists x\psi(x)\wedge \forall x\forall y (
			\psi(x)\wedge\psi(y)\to x = y)
\end{align}
\end{definition}

\section{\texorpdfstring{$\properclass{ZFC}$公理}{ZFC公理}}

\setcounter{axion}{-1}
\begin{axion}\label{Exi} \indexbf{存在公理} (\indexbf{Exi})
存在一个集合, i.e.
\begin{align}
	\exists x (x = x).
\end{align}
\end{axion}
\begin{axion}\label{Ext} \indexbf{外延公理} (\indexbf{Ext})
两个有相同元素的集合相等, i.e.
\begin{align}
	\forall X\forall Y\forall u (u\in X\leftrightarrow u\in Y) \rightarrow X=Y.
\end{align}
\end{axion}

而逻辑上有$X=Y\rightarrow \forall X\forall Y\forall u (u\in X\leftrightarrow u\in Y)$, 所以:
\begin{align}
	\forall X\forall Y\forall u (u\in X\leftrightarrow u\in Y) \leftrightarrow X=Y
\end{align}

记$\neg(X=Y) =: X\neq Y$.

\begin{axion}\label{Sep} \indexbf{分离公理模式} (\indexbf{Sep})

令$\varphi(u)$为公式. 对任意集合$X$, 存在一个集合$Y=\{u\in X\mid \varphi(u)\}$, i.e.
\begin{align}
	\forall X\exists Y\forall u (u\in Y\leftrightarrow u\in X \wedge \varphi(u)).
\end{align}	
\end{axion}

\begin{corollary}\label{Russell set}
\begin{align}
	\forall X\exists R_X (R_X\notin R_X).
\end{align}
\end{corollary}
\begin{proof}
令$R_X = \{x\in X\mid x\notin x\}$即可.
\end{proof}

令$\varphi(u)$为一个性质. 倘若$\exists X\forall u(\varphi(u)\to u\in X)$, 则${u\mid \varphi(u)} = {u\mid \varphi(u)}$, 根据Sep (Axion \ref{Sep}), $\exists \varnothing = {u\mid \varphi(u)}$. 分离于不同的集合$X$和$X'$的$\varnothing$是相同的. 考虑到$x\neq x \to x \in X$是重言式, 再根据Exi (Axion \ref{Exi}), 可以得出:

\begin{definition}
	$\varnothing=\{x\mid x\neq x\}$是集合, 称为\indexbf{空集}.
\end{definition}

\begin{definition}
$\varphi(u)$是一个性质. 称$\{u\mid u(u)\}$为一个\indexbf{类} (class). 若一个类不是集合, 则称其为\indexbf{真类} (proper class).
\end{definition}

如所有集合的类$\indexfm[V]{\properclass{V}}$就是一个真类 (根据Corollary \ref{Russell set}). 

\begin{definition}
由Sep, 两个集合的\indexbf{交}和\indexbf{差}也是集合:
\begin{align}
	X\cap Y = \{u\in X\mid u\in Y\}
	&&
	X - Y = \{u\in X\mid u\notin Y\}
\end{align}
\end{definition}

\begin{corollary}
而非空集$X\neq \varnothing$的\indexbf{任意交}
\begin{align}
	\bigcap X = \{u\mid \forall Y\in X( u\in Y)\}
\end{align}
也是集合.
\end{corollary}
\begin{proof}
因$X\neq\varnothing$, $\exists x_0\in X$. 由Sep, 
\begin{align*}
	Y = \{ y\in x_0\mid \forall x\in X(y\in x) \}
\end{align*}
是集合. 
\end{proof}

\begin{axion} \indexbf{对集公理} (\indexbf{Pai})
\begin{align}
	\forall a\forall b\exists c \forall x ( x\in c\leftrightarrow x=a\vee x=b).
\end{align}
这样的$c$可记为$\{a,b\}$. 
\end{axion}

\begin{axion} \indexbf{并集公理} (\indexbf{Uni})
\begin{align}
	\forall X\exists Y \forall u (u\in Y\leftrightarrow \exists z\in X(u\in z).
\end{align}
\end{axion}

\begin{definition} \indexbf{子集}和\indexbf{真子集}关系定义如下:
\begin{align}
	X\subseteq Y &:= \forall x\in X(x\in Y) ,\\
	X \supseteq Y &:= Y\subseteq X, \\
	X \subset Y &:= X\subseteq Y \wedge X\neq Y,\\
	X \supset Y&:= X.
\end{align}
\end{definition}

\begin{corollary}
$\forall X(\varnothing\subseteq X).$
\end{corollary}

\begin{axion} \indexbf{幂集公理} (\indexbf{Pow})
\begin{align}
	\forall X\exists Y\forall u (u\in Y\leftrightarrow u\subseteq X).
\end{align}
这样的$Y$称为$X$的\indexbf{幂集}, 记为$\indexfm[P X]{\mathscr{P}(X)}$或$\indexfm[2 X]{2^X}$.
\end{axion}

\begin{definition}
对任意集合$x$, $x\cup \{ x\} $称为其\indexbf{后继}, 记为$\indexfm[S x]{S(x)}$或$\indexfm[x plus]{x^+}$.
\end{definition}

\begin{axion}\label{Inf} \indexbf{无穷公理} (\indexbf{Inf})
\begin{align}
	\exists X \big( 
		\varnothing \in X\wedge \forall x(
			x\in X\to S(x)\in X)\big).
\end{align}
\end{axion}

\begin{axion}\label{Fnd} \indexbf{基础公理} (\indexbf{Fnd})
\begin{align}
	\forall x\big( 
		x\neq \varnothing\rightarrow \exists y\in x (
			x\cap y=\varnothing )\big).
\end{align}
\end{axion}

\begin{theorem}\label{Russell}
\begin{align}
	\forall x (x\notin x).
\end{align}
\begin{proof}
考虑$X=\{x\}$. 与Fnd矛盾.
\end{proof}
\end{theorem}

\begin{theorem}
\begin{align}
	\nexists X\big( 
		X\neq\varnothing\wedge\forall x\in X(
			\exists y\in X(
				y\in x))\big).
\end{align}
\end{theorem}
\begin{proof}
\begin{align*}
	\text{Fnd} \wedge \forall x\in X\big(
		\exists y\in X (
			y\in x\cap X)\big) \rightarrow\bot.
\end{align*}
\end{proof}

\begin{axion}\label{Rep} \indexbf{替换公理模式} (\indexbf{Rep})
对公式$\psi(x,y)$, $\forall x$都存在唯一的$y$ s.t.\ $\psi(x,y)$成立. 那么$\forall A\in \properclass V$, 存在集合:
\begin{align}\label{range(Rep)}
	B = \{ 
		y \mid \exists x\in A\,\psi(x,y)\}
\end{align}
i.e.\ 
\begin{align}
	\forall A\forall x\in A\,\exists! y\,\psi(x,y) \to
		\exists B\forall x\in A\,\exists y\in B\,\psi(x,y)
\end{align}
\end{axion}

\begin{axion}\label{AC} \indexbf{选择公理} ($\indexfm[AC]{\properclass{AC}}$)
对任意非空集合$X\neq \varnothing$, 若
\begin{conditionlist}[label = (\arabic*)]
\item $\varnothing \notin X$,
\item $X$中两两不交, 即$\forall x\in X\,\forall y\in X$且$x\neq y$, 那么$x\cap y=\varnothing$,
\end{conditionlist}
则存在集合$S$, 对$\forall x\in S$, $S\cap x$是单点集. i.e. 
\begin{align}
\begin{aligned}
	\forall X\big( 
		\varnothing\in X\wedge \forall x\in X\,&\forall y\in X (
			x= y\vee x\cap y =\varnothing)\\
				&\to  \exists S\forall x\in X\,\exists!y(
					S\cap x = \{y\})\big).
\end{aligned}
\end{align}
\end{axion}

Axion \ref{Exi} 到 \ref{Rep} 构成的公理系统称为\indexbf{Zermelo-Fraenkel系统}, 记为$\properclass{ZF}$, 加上选择公理则记为$\properclass{ZFC}$.


\chapter{关系与函数}

\section{关系}
\begin{definition}集合$a,b$的\indexbf{有序对}$\indexfm[a b]{( a,b)} := \big\{\{a\},\{a,b\}\big\}$.
\end{definition}

\begin{theorem}\label{equality_ordered_pair}
\begin{align*}
	(a,b)=(a',b') \leftrightarrow a=a'\wedge b=b'.
\end{align*}
\end{theorem}
\begin{proof}
只证明 "$\rightarrow$":
\begin{enumerate}[label=(\arabic*)]
\item $a=b$. $(a,b) = \big\{\{a\}\big\}=(a',b')$, 故$(a',b')=\{\{a\}\}=\big\{\{a'\},\{a',b'\}\big\}$, 由Ext (axion \ref{Ext}), $\{a'\} = \{a',b'\}=\{a\}$, 即$a=b=a'=b'$.
\item $a\neq b$. 假设$\{a,b\}=\{a'\}$, 得$\forall x\in \{a,b\} (x=a')$即$a=b=a'$与$a\neq b$矛盾. 从而只有$\{a,b\}=\{a',b'\}\wedge\{a\}=\{a'\}$, 仍然由Ext易证.
\end{enumerate}
\end{proof}

\begin{definition}\label{Cartesian_product}
令$X$和$Y$是集合, 其\indexbf{直积}或\indexbf{Cartesian积}定义为:
\begin{align}
	\indexfm[X times Y]{X\times Y} := \{ ( x,y) \mid x\in X\wedge y\in Y\}.
\end{align}
简记$X\times X =: \indexfm[X 2]{ X^2}$.
\end{definition}

\begin{theorem}对于$\forall X\forall Y$, $X\times Y$是集合.
\end{theorem}
\begin{proof}
令$\varphi(z) = \exists x\in X\exists y\in Y( (x,y)=z)$, 取$Z=\{z\in \mathscr P(X\cup Y)\mid \varphi(z)\}$, 由Ext和Sep (Axion \ref{Sep}) 即可知$X\times Y = Z$.
\end{proof}

\begin{definition}
如果存在集合$X,Y$ s.t.\ $R\subseteq X\times Y$, 则称集合$R$是\indexbf{二元关系}. 通常记$(x,y)\in R =: \indexfm[R x y]{R(x,y)}X$, 或$\indexfm[x R y]{xRy}{xRy}$. $\dom R := \{ x \mid \exists y R(x,y)\}$称为其\indexbf{定义域}, $\ran R=\{ y\mid \exists x R(x,y)\}$称为其\indexbf{值域}. 

特别地, 如果$R\subseteq X^2$, 则称其为$X$上的二元关系.
\end{definition}

\begin{definition}
集合$X$在关系$R$的\indexbf{像}$R[X]$定义为$\{ y\in\ran R \mid \exists x\in X\,(R(x,y) )\}$. 
集合$Y$的逆像$R^{-1} [Y]$则定义为$\{x\in\dom R \mid \exists y\in Y\,( R(x,y))\}$. 
二元关系$R$的\indexbf{逆}$R^{-1}$是$\{ (x,y) \mid R(y,x)\}$. 
两个二元关系$R$, $S$的\indexbf{复合}$S\circ R$则定义为$\{ (x,z) \mid \exists y ( R(x,y) \wedge S(y,z)) \}$.
\end{definition}

\begin{theorem}
令$R$是二元关系, $A,B$是集合. $R[ A\cup B] = R[A] \cup R[B]$, $R[ A\cap B] \subseteq R[A]\cap R[B]$, $R[A -B] \supseteq R[A] - R[B]$.
\end{theorem}

Cartesian积可递归地推广到$n$元: 
\begin{align}
	(x_1,\cdots,x_{n+1}) = ( (x_1,\cdots,x_n),x_{n+1});\\
	X_1\times\cdots\times X_n = \{ (x_1,\cdots, x_n)\mid x_1\in X_1\wedge\cdots \wedge x_n\in X_n\} 
\end{align}

$n$元Cartesian积的子集可类似地定义\indexbf{$n$元关系}.

\begin{theorem}
$n$元Cartesian积$X_1\times X_2\times \cdots \times X_n$是空集, 则存在$X_i = \varnothing$.
\end{theorem}

\section{函数}

\begin{definition}\label{function}
二元关系$f$倘满足: 
\begin{align*}
	\forall x\big(
		(x,y)\in f\wedge (x,z)\in f \to y=z\big),
\end{align*}
则称$f$是\indexbf{函数}, $y$是$f$在$x$处的\indexbf{值}, 记为$f(x) = y$, 或$f\colon x\mapsto y$. 倘若$\dom f = X$, $\ran f \subseteq Y$, 则称$f$是$X$到$Y$的函数, 记为$f\colon X\to Y$. 
\end{definition}

对任意集合$X$定义$\id _X \colon X\to X$为$\forall x\in X(\id _X(x)=x)$, 称为\indexbf{等同函数}.

\begin{theorem}\label{equality_function}
令$f$, $g$都是函数.
\begin{align*}
	f=g\leftrightarrow 
		\dom f=\dom g \wedge \forall x\in \dom f \big(
			f(x)=g(x)\big). 
\end{align*}
\end{theorem}
\begin{proof}
只证明 "$\leftarrow$":
\begin{align*}
	\forall (x,y)\in f  \big(
		x\in \dom f\wedge y=f(x)\big) \wedge
		\dom f=\dom g \wedge \forall x\in \dom f \big(
			f(x)=g(x)\big) &\\\rightarrow
		\forall (x,y)\in f\big(
			x\in \dom g\wedge y=g(x)\big).
	\qquad\qquad&
\end{align*}
同理, $
\forall (x,y)\in g\big(
	x\in \dom f\wedge y=f(x)\big)$, 
即$\forall (x,y) \big(
	y=f(x)\leftrightarrow y=g(x)\big)$. 由Ext, $f=g$.
\end{proof}

通常以集合为值的函数$i\mapsto X_i$, 其中$i\in I$, 可视为\indexbf{指标系统}, $I$是\indexbf{指标集}. 记为$X=\{X_i\mid i\in I\}$或$\{X_i\}_{i\in I}$. 

\begin{theorem}
$\psi (i,x)$是公式. $\forall I\forall X$, 
\begin{align*}
	\bigcup_{i\in I}\{x\in X\mid \psi(i,x)\} &=
		\{x\in X\mid \exists i\in I\big(
			\psi(i,x)\big)\},\\
	\bigcap_{i\in I}\{x\in X\mid \psi(i,x)\} &=
		\{x\in X\mid \forall i\in I\big(
			\psi(i,x)\big)\}.
\end{align*}
\end{theorem}

\begin{definition}
令$X=\{X_i\mid i\in I\}$是一个指标系统. $X$的\indexbf{一般Cartesian积}为:
\begin{align*}
	\prod_{i\in I}X_i := \{ f\mid f\colon I\to X_i\}.
\end{align*}
另, $p_i\colon \prod_{i\in I}X_i\to X_i$称为\indexbf{指标函数}.
\end{definition}

\textbf{注}: 虽然这样的定义和Cartesian积不同, 但接下来的概念确保了, 两者之间可以一一对应, 从而是等同的.

\begin{definition}
令$f\colon X\to Y$是函数. 若$f(x_1) = f(x_2) \leftrightarrow x_1=x_2$则称$f$为\indexbf{单射} (injection). 若$\ran f = Y$则称其为\indexbf{满射} (surjection). 既单又满的函数称为\indexbf{双射} (bijection). 如果函数的逆$f^{-1}$也是函数, 则函数$f$称为\indexbf{可逆}的.
\end{definition}

作为例子, 若空映射$\ran f=\dom f=\varnothing$, $f=\varnothing$总是单的, $f^{-1} = \{(y,x)\mid y=f(x)\} = \varnothing$也是空映射. 

\textbf{注}: 这里函数的逆的定义与通常不同, 因$\dom f^{-1} = \ran f$而非$Y$. 因而下面的定理在这样的定义下是成立的 (否则还要加上满射的条件):

记$\indexfm[Y X]{Y^X} := \{ f \mid f\colon X\to Y\}$. $\forall Y\big(Y^\varnothing = \{\varnothing_Y\}\big)$, 而若$X\neq \varnothing$, $\varnothing^X=\varnothing$.
\begin{theorem}
函数$f$可逆 \emph{iff} $f$是单射. 
\end{theorem}
\begin{proof}
可逆意味着$(y,x_1)\in f^{-1}\wedge (y, x_2) \in f^{-1} \leftrightarrow x_1=x_2$, 又由逆的定义, $y=f(x_1)\wedge y=f(x_2)\leftrightarrow x_1=x_2$, 这即是单射的定义.
\end{proof}

\begin{theorem}
函数$f$若可逆, 则$f^{-1}$可逆, 且$(f^{-1})^{-1} = f$.
\end{theorem}
证明从略.


\begin{theorem}\label{composition_functions}
如果$f$和$g$是函数, 它们的复合$h=g\circ f$也是函数. 
而且$
\dom h=  f^{-1}\big(
	\dom g\big)$.
\end{theorem}
\textbf{注}: 这里的复合和通常的定义有细微不同, 但保持了与二元关系的统一. 
\begin{proof}
复合的定义: $
h=g\circ f \leftrightarrow 
	\forall (x,z)\in h\Big(
		\exists y\big(
			y=f(x)\wedge z=g(y)\big)\Big)$.
倘若$(x,u)\in h\wedge (x,u)\in h$, 有$\exists! y$ s.t. $y=f(x)$, 且$u=v=g(y)$. 因而$h$也是函数. 

其定义域$
\dom h = \{
	x\mid \exists z\big(
		z=h(x)\big)\}$, 
又因$
\exists z(z=h(x)) \leftrightarrow \exists z\exists y\big(
	y=f(x)\wedge z=g(y)\big)$, 后者又等价于$
\exists y\Big(
	y=f(x)\wedge \exists z\big(
		z=g(y)\big)\Big)$, i.e.\ $
\exists y\in \dom g \big(
	y=f(x)\big)$, 
\begin{align*}
	\dom h 
		= \{x\mid \exists y\in \dom g \big(
			y=f(x)\big)\}
		= f^{-1}(\dom g).
\end{align*}
\end{proof}


\begin{definition}
令$f$是任意函数, $A$是任意集合. 函数$f\upharpoonright A = \{ (x,y) \in f\mid x\in A\}$是$f$在$A$上的\indexbf{限制}. 若$g=f\upharpoonright A$, 则称$f$是$g$在$\dom f$的\indexbf{扩张}.
\end{definition}

\begin{definition}
函数$f$, $g$被认为是\indexbf{相容}的, 如果:
\begin{align*}
	\forall x\in \dom f\cap \dom g\big(
		f(x)=g(x)\big)
\end{align*}
指标系统$\mathcal F = \{f_i\mid i\in I\}$被称为\indexbf{相容系统}, 如果
\begin{align*}
\forall f_i\in \mathcal F\forall f_j\in \mathcal F\big(
	\text{ $f_i$ 和$f_j$相容.} \big)
\end{align*}
\end{definition}

\begin{theorem}\label{compatibility}
$f$和$g$是函数.以下的命题是等价的:
\begin{enumerate}[label=(\arabic*)]
\item $f$与$g$相容;
\item $f\cup g$是函数;
\item $f\upharpoonright \big(
	\dom f\cap \dom g\big) =
	g\upharpoonright \big(
	\dom f\cap \dom g\big)$.
\end{enumerate}

\end{theorem}
\begin{proof}
(1)$\leftrightarrow$(3): 注意到$\dom \Big(
	{f\upharpoonright \big(
		\dom f\cap \dom g\big)}\Big) = \dom f\cap \dom g$. 由相容的定义和定理\ref{equality_function}可得证.

(2)$\leftrightarrow$(1): 假设$f$与$g$不相容, 即$\exists x\in \dom f\cap \dom g\subseteq \dom{f\cup g}$ s.t.\ $f(x)\neq g(x)$, 这与函数$f\cup g$的定义不相符. 若$f\cup g$不是函数, 它至少是$\dom f\cup \dom g \times \ran f\cup \ran g$上的二元关系, 由函数的定义, $\exists x\in \dom{f\cup g}$ s.t.\ $\exists y_1\exists y_2\big(
	(x,y_1)\in f\cup g\wedge (x,y_2)\in f\cup g\wedge y_1\neq y_2\big)$. 通过对$x$在$\dom f - \dom g$, $\dom g - \dom f$, $\dom g\cap \dom g$讨论, 可以得出要么$f$或$g$不是函数 (从而与题设矛盾), 要么$f$, $g$不相容.
\end{proof}

\setcounter{axion}{8} %这样下面的axion才会计数为9
\begin{axion}\label{AC2} \indexbf{选择公理} (第二形式) ($\indexfm[AC 2]{\properclass{AC\;II} }$)

\begin{align*}
	\forall \mathcal F\Big(
		\varnothing \notin \mathcal F\wedge 
			\mathcal F\neq \varnothing \wedge
				\exists f\colon \mathcal F\to \cup \mathcal F \big(
					\forall F\in \mathcal F(
						f(F)\in F)\big)\Big)
\end{align*}

其中这样的$f$通常被称为\indexbf{选择函数}.
\end{axion}

对于任意非空$\omega$ (尤其是当$\omega$是无穷集时), $\prod_{i\in \omega X_i} = \varnothing \to \exists i\in \omega ( X_i = \varnothing)$与选择公理等价.\footnote{\href{https://zh.wikipedia.org/wiki/\%E7\%AC\%9B\%E5\%8D\%A1\%E5\%84\%BF\%E7\%A7\%AF\#\%E6\%97\%A0\%E7\%A9\%B7\%E4\%B9\%98\%E7\%A7\%AF
	}{维基百科页面}}

\section{等价和划分}
\begin{definition}
令$R\subseteq X^2$是$X$上的二元关系. $R$是:
\begin{conditionlist}[label=(\arabic*)]
\item \indexbf{自反}的, 若$\forall x\in X(xRx)$;
\item \indexbf{对称}的, 若$\forall x\in X\forall y\in X( xRy \to yRx)$;
\item \indexbf{传递}的, 若$\forall  x\in X\forall y\in X\forall z\in X (xRy\wedge yRz \to xRz)$;
\item \indexbf{等价关系}或\indexbf{等价}的, 若$R$自反, 对称, 传递. 记为$\sim$.
\end{conditionlist}
\end{definition}

\begin{definition}
令$\sim$是$X$上的等价关系, $x\in X$. $x$关于$\sim$的\indexbf{等价类}定义为:
\begin{align*}
	[x]_\sim := \{ t\in X\mid t\sim x\}.
\end{align*}
\end{definition}

\textbf{注}: 由Sep, 等价类是集合而不是真类. 

\begin{theorem}
令$\sim$是$X$上的等价关系.
\begin{align*}
	\forall x\in X\forall y\in Y\big(
		[x]_\sim=[y]_\sim \wedge{}[x]_\sim \cap{} [y]_\sim =\varnothing\big)
\end{align*}
\end{theorem}

\begin{definition}
令$X$是一集合, $S\subseteq \mathscr P(X)$. $S$被称为$X$的一个\indexbf{划分}如果:
\begin{align*}
	\forall a\in S\forall b\in S(
		a=b \vee a\cap b=\varnothing) \wedge 
		\cup S=X.
\end{align*}
\end{definition}

\begin{definition}
令$\sim$是$X$上的等价关系. $X/\sim := \{[x]_\sim \mid x\in X \}$称为$X$的\indexbf{商集}.
\end{definition}

\begin{theorem}
令$\sim$是$X$上的等价关系. $X/\sim$是$X$的一个划分.
\end{theorem}

\begin{theorem}
令$S$为$X$的划分, 定义二元关系:
\begin{align*}
	\sim_S := \{(x,y) \in X^2 \mid \exists s\in S(
		x\in s\wedge y\in s)\}.
\end{align*}
那么, $\sim_S$是等价关系, $X/ \sim_S = S$. 若$X$上的等价关系$\sim $满足$X/\sim =S$, 则$\sim_S=\sim$.
\end{theorem}
\begin{proof}
$\cup S=X \to 
	\forall x\in X\exists s\in S(
		x\in s)$, 即$\sim_S$是自反的. 对称和传递性显然. 从而, $\sim_S$是等价关系. 

依商集和等价类的定义, $\forall s\in X/\sim_S \exists x\in X \forall t\big(
	t\in s \leftrightarrow
		\exists s'\in S(
			x\in s'\wedge t\in s')\big)$. 
这之后我遇到了困难.
\end{proof}

\section{序}
\begin{definition}
如果$X$上的二元关系$\leq$满足:
\begin{conditionlist}[label=\alph*)]
\item 自反 i.e.\ $\forall x\in X( x\leq x)$;
\item \indexbf{反对称} i.e.\ $\forall x\in X\forall y\in X(
	x\leq y\wedge y\leq x \to 
		x=y)$;
\item 传递 i.e.\ $\forall x\in X\forall y\in X\forall z\in X(
	x\leq y\wedge y\leq z \to x\leq z)$,
\end{conditionlist}
则称其为$X$上的\indexbf{偏序} (partial order) 或\indexbf{序}, 记$(X,\leq)$, 并称$X$是一个\indexbf{偏序集} (partially ordered set, appr.\ poset). 
如果它还是\indexbf{连接}的 i.e.\ $\forall x\in X\forall y\in X(
	x\leq y\vee y\leq x)$, 那么它是$X$上的\indexbf{线序}或\indexbf{全序} (total order), 此时也称$X$是一个\indexbf{线序集}.
\end{definition}

通常记$\geq := \leq^{-1}$, $< := \leq \cap \neq $, $> := <^{-1}$.


\begin{definition}
如果$X$上的关系$\indexfm[preceq]{\preceq}$只满足传递和自反, 称其为$X$上的\indexbf{拟序} (quasi-order)或\indexbf{预序} (preorder). $\succeq:=\preceq^{-1}$.
\end{definition}

\begin{theorem}
令$\preceq$是$X$上的拟序, 等价关系$\sim$可由$\sim=\preceq\cap\succeq$定义, 且商集$X/\sim$上的偏序关系$\leq$可定义为
\begin{align*}
	[x]\leq{} [y] \leftrightarrow x\preceq y.
\end{align*}
\end{theorem}

\begin{definition}
如果对于$a\in X$满足$\forall x\in X( \neg (a>x))$, 则称$a$为$X$的\indexbf{极小元}. 
如果对于$a\in X$满足$\forall x\in X (a\leq x)$, 则称$a$为$X$的\indexbf{最小元}. 
相反则有\indexbf{极大元}和\indexbf{最大元}.

令$X_0\subseteq X$. 如果$\exists a\in X\forall x\in X_0 (a\leq x)$, 则称$X_0$在$X$中有下界, $a$是$X_0$在$X$中的\indexbf{下界}. 
如果这样的下界$a$的集合有最大元$a_0$, 称其为\indexbf{下确界} (infimun, appr.\ inf), 记为$\inf X_0$. 同理有\indexbf{上界}和\indexbf{上确界} (supremun, appr.\ sup) $\sup X_0$.
\end{definition}

作为例子, 令$X\neq \varnothing$, $(\mathscr P(X), \subseteq)$是偏序集. 对于$\forall S\subseteq \mathscr P(X)$, $S$有上下确界, $\sup S = \cup S$, $\inf S = \cap S$.

\begin{definition}
如果$X$和其上的线序$(X,\leq)$满足任意$X_0\in \mathscr P(X)$, $X_0$都有最小元, 则称$\leq$为$X$上的\indexbf{良序}, $X$被称为\indexbf{良序集}.
\end{definition}

\chapter{实数}
\section{自然数}
根据Inf (axion \ref{Inf}), 这样的集合是存在的:
\begin{definition}
如果集合$X$满足:
\begin{align*}
	\varnothing \in X\wedge 
		\forall x(x\in X \to x^+\in X)
\end{align*}
则称其为\indexbf{归纳集}.
\end{definition}

容易知道$0:=\varnothing$属于任何归纳集, $1:=0^+$也属于任何归纳集, ..., 以此类推. 最小的归纳集被称为\indexbf{自然数集}, 它的严格定义如下:
\begin{definition}
\begin{align*}
	\indexfm[N]{\mathbb{N}} :=\{n\mid \forall X(X\text{ 是归纳集} \to n\in X)\}.
\end{align*}
$\mathbb{N}$是\indexbf{自然数集}, 其元素是\indexbf{自然数}.
\end{definition}

从定义上可以看出, $\mathbb N$是归纳集, 而且是任何归纳集的子集.

\begin{theorem} \indexbf{归纳原理} 令$\varphi(n)$是一个性质. 
如果
\begin{conditionlist}[label=\alph*)]
\item $\varphi(0)$成立;
\item $\forall n\in\mathbb N (\varphi(n)\to\varphi(n^+))$成立,
\end{conditionlist}
那么, $\forall n\in \mathbb N \,\varphi(n)$成立. 
\end{theorem}
\begin{proof}
构造集合$M = \{n\in\mathbb N\mid \varphi(n)\}$, 根据它是归纳集, 可知$\mathbb N \subseteq M$, 但$M$是根据Sep从$\mathbb N$中分离出来的, 得知$M=\mathbb N$.
\end{proof}

在$\mathbb N$上定义偏序关系$\leq = \in \cap =$, 而且可以证明, 这是一个良序.

\begin{theorem} \indexbf{第二归纳原理}
令$\varphi(n)$是一个性质.
\[
	\forall n\in \mathbb N (
		\forall k\in n\,\varphi(k) \to \varphi(n)) \to
		\forall n\in \mathbb N \,\varphi(n)
\]
\end{theorem}

\section{递归定理}
接下来我们要定义一些二元函数, 即我们熟悉的自然数集上的运算. 
我们可以递归地给出它们的定义, 但这种定义的合理性需要递归定理的辩护. 

\begin{definition}
以$n\in \mathbb N$或$\mathbb N$为定义域的函数称为\indexbf{序列}, 其中前者称为长度为$n$的\indexbf{有穷序列}, 后者称为\indexbf{无穷序列}, 通常分别记为$\langle a_k\mid k<n \rangle$或$\langle a_k \mid k\in \mathbb N\rangle$, 或简记为$\langle a_k\rangle_{k<n}$或$\langle a_k\rangle_{k\in \mathbb N}$.
值域通常记为$\{x_k\mid k<n\}$或$\{x_k\mid k\in \mathbb N\}$, 或简记为$\{ a_k\}_{k<n}$或$\{ a_k\}_{k\in \mathbb N}$.
特别地, 长度为$0$的序列称为\indexbf{空序列}.

若序列的到达域是$Y$, 则通常称为$Y$内的序列. 
$Y$内所有有穷序列的集合可记为$A^{<\mathbb N} := \bigcup_{n\in \mathbb N} A^n$.
\end{definition}

\begin{theorem} \indexbf{递归定理}
\begin{align*}
	\forall A\forall a\in A\forall g \in A^{A\times \mathbb N}\exists ! f\in A^\mathbb N\big(
		f(0)=a \wedge \forall n\in \mathbb N(
			f(n^+) = g(f(n),n))\big).
\end{align*}
\end{theorem}
这里的$g$扮演了一个 ``递推式'' 的角色.
\begin{proof}
首先, 我们需要证明$f$的\emph{存在性}. 

注意到$f$是$A$中的无穷序列, 我们考虑用满足条件的有穷序列去逼近它.
基于$a$和$g$的\indexbf{$m$-近似}定义为有穷序列$t\colon m^+\to A$满足
\begin{align*}
	t_0 = a\wedge \forall k\in m^+\big(
		t_{k^+} = g(t_k,k)\big).
\end{align*}
并记$
\mathcal F = \{ t\in \mathscr P(\mathbb N\times A)\mid t\text{是$m$-近似}\}$, 及$
f = \cup \mathcal F$. 接下来我们证明这个$f$是所寻找的函数.

首先证明它是函数. 
由theorem \ref{compatibility}, 知这当且仅当$\mathcal F$相容. 
令$t,u\in F$. 记$m=\dom t$, $n=\dom u$. 不妨设$m\leq n$.
$t,u$相容\emph{iff} $\forall k\in m (t_k = u_k)$. 
这由1) $t_0=u_0=a$; 2) $t_k =u_k \to t_{k^+} = g( t_k,k) = g(u_k,k) = u_{k^+}$ 的成立和归纳原理保证.

接着确定$f \in A^\mathbb N$. $\dom f \subseteq \mathbb N $和$\ran f \subseteq A$是显然的.
下证$\mathbb N \subseteq \dom f$. 注意到$\dom f = \cup \{ n\in \mathbb N\mid \text{存在$n$-近似}\}$.
接下来就是证明$\dom f$是归纳集, 即$\forall n\in \mathbb N$存在$n$-近似. 
$0$-近似由$\{(0,a)\}$给出; $k$-近似$t$存在时, 只需为之并上$\{(k^+, g(t_k, k))\}$即可得到$k^+$近似.

至于$f$满足$f(0)=a$与$\forall n\in \mathbb N(f(n^+) = g(f(n),n))$, 用其任意近似证明即可, 只需想到$f$与近似的相容性.

如此我们确定了$f$的存在性, 然后来证明它的\emph{唯一性}. 
假设有$h: \mathbb N \to A$也满足定理, 只需用归纳法证明$h(n)=f(n)$对任意$n\in \mathbb N$成立即可. 
\end{proof}

这样的$f$只能是一元函数, 定义运算需要带参数的版本:

\begin{theorem} \indexbf{带参数的递归定理}
\begin{align*}
&
	\forall A\forall P\forall a\in A^P
	\forall g \in A^{P\times A\times \mathbb N}
	\exists ! f\in A^{P\times \mathbb N}\big(
\\ &\qquad
		\forall p\in P\,f(p,0)=a(p) \wedge \forall n\in \mathbb N\forall p\in P(
			f(p,n^+) = g(p,f(n),n))\big).
\end{align*}

\end{theorem}
\textbf{注}: 如果固定$p$, 这个定理与递归定理几乎一致, 从而我们需要考虑以$p$为变元的函数作为递归定理中的到达域.
\begin{proof}
令$G\colon A^P\times \mathbb N \to A^P$; $(t,n) \mapsto h$, 其中$h$满足$\forall p\in P \big(
	h(p) = g(p,t(p),n)\big)$. 
考虑到$t$和$g$都是函数, 复合的$h$当然是函数, 而且唯一.

由递归定理, 有这样的函数$F \colon \mathbb N \to A^P$满足 
1) $\forall p\in P\big(
	F(0)=a\in A^P\big)$; 
2) $\forall n\in \mathbb N\big(
	F(n^+) = G(F(n), n)\big)$. 

可以验证$f(p,n) = F(n)(p)$即是我们想找的函数.
\end{proof}

有了带参数的递归定理, 自然数上的运算可以看作以下存在而且唯一的二元函数:

\begin{definition}
\indexbf{加法}$+\colon \mathbb N\times \mathbb N \to \mathbb N$, 满足
1) $\forall m\in \mathbb N \big( 
	+(m,0)=m\big)$;
2) $\forall m\in\mathbb N\forall n\in \mathbb N \big(
	+(m,n^+)=(+(m,n))^+\big)$. 通常记$+(m,n) =: m+n$.
\end{definition}

\begin{definition}
\indexbf{乘法}$\cdot \colon \mathbb N\times \mathbb N \to \mathbb N$, 满足
1) $\forall m\in \mathbb N \big( 
	\cdot(m,0)=m\big)$;
2) $\forall m\in\mathbb N\forall n\in \mathbb N \big(
	\cdot(m,n^+)=(\cdot(m,n))^+\big)$. 通常记$\cdot(m,n) $为$mn$或$m\cdot n$.
\end{definition}

我们熟悉的关于乘法和加法的性质都可以由归纳原理证出, 此处不再赘述.

\section{势}

\begin{definition}\label{equinumerosity}
两个集合$X$, $Y$\indexbf{等势} (equinumerous) 指的是$\exists f \in Y^X (f\text{是双射 })$, 记为$|X|=|Y|$.
\end{definition}

\begin{definition}
若存在$f \in Y^X$ s.t. $f$是单射, 则称$|X|\leq |Y|$. 
当$|X|\leq |Y|$而$|X|\neq |Y|$时, 我们就称$X$的势小于$Y$的势, 记为$|X| < |Y|$.
\end{definition}

\begin{theorem}
\begin{align*}
|X| \leq |Y| \leftrightarrow \exists Y_0\in \mathscr P(Y) \big(|X| = |Y_0| \big).
\end{align*}
\end{theorem}
\begin{proof}
对于$f\in Y^X$且$f$是单的, 取$Y_0 = \ran f$即可.
\end{proof}

下面我们将介绍Cantor-Bernstein-Schr\"oeder 定理, 它的证明当中最重要的一部分是以下引理的证明:
\begin{lemma}\label{lemma_CBS}
\begin{align*}
	A'\subseteq B \subseteq A \wedge |A'|=|A| 
		\to |B|=|A|.
\end{align*}
\end{lemma}
\textbf{注}: 我们已经知道有$A$到$B$的单射了, 怎么找到一个同时是满的呢? 
这要求这个双射$h$要把$A-B$和$B$映射到$B$的分划上, 可$h[A-B]$在$B$之中, 它在$h$下必须映射到$h[B]$里; 
同理$h\big\lbrack h[A-B]\big\rbrack$必须映射到$h[B]-h\big\lbrack h[A-B]\big\rbrack$中\ldots

也就是说: $A$和$A-B$在$h^n$即$h$的任意个复合中, 总是映射到一对不相交的集合$h^n[A]$, $h^n[A-B]$中 (否则不可能是单的), 而且$h^n[A-B]$在$h$的映射下, 又只能落到$h^n[B]$中. 
从而: $h^n[B]$不断在缩小, 给$h^n[A-B]$腾出空间. 
这是以下证明的重要思路.
\begin{proof}
令$h$见证了$A'$和$A$的等势, 即$h\in A'^A$且$h$是双射.
记:
\begin{align*}
	A_0 := A, && 
	B_0 := B,
\end{align*}
并定义序列:
\begin{align*}
	A_{n+1}=h[A_n], &&
	B_{n+1}=h[B_n].
\end{align*}

我们可以归纳地证出:
\begin{align*}
	\forall n\in \mathbb N(
		A_{n+1}\subseteq B_n \subseteq A_n),
\end{align*}
只需认识到$A_1\subseteq A' \subseteq B_0\subseteq A_0$, 
且对于任意$k\subseteq \mathbb N$只要$A_{k+1}\subseteq B_k \subseteq A_k$,
	则$h[A_{k+1}]\subseteq h[B_k]\subseteq  h[A_k]$.
	
定义$C_n = A_n - B_n$, 下验证$h[C_n] = C_{n+1}$: 
\begin{align*}
	h[C_n] = h[A_n - B_n] = h[A_n] - h[B_n] = A_{n+1} - B_{n+1} = C_{n+1}, 
\end{align*}
其中第二个等式的成立依赖于$h$是一个单射, 因为: $h[A_n] $中$h[B_n]$全部由$B_n$映射而来, 这是单性要求的; 从而$h[A_n]-h[B_n]$是$A_n-B_n$的像, 因为映射到自身的值域总是满的.

将这些所有的$C_n$并起来:
\begin{align*}
	C:=\bigcup^\infty_{n=0} C_n,
\end{align*}
从而
\begin{align*}
	h[C] = h\bigg\lbrack\bigcup^\infty_{n=0} C_n\bigg\rbrack = \bigcup^\infty_{n=1} C_n = C - C_0 = C- (A-B).
\end{align*}

我们可以看到$C$中的元素有这样的性质, 它含于某个$C_n$中, 并将被$h$映射到下一个$C_{n+1}$中; 而且要么它是$C_0=A-B$中的元素, 要么是$C_{n-1}$中某个元素的在$h$下的像. 这有些像Hilbert的无限旅馆. $h\upharpoonright C$到是从$C$到它的子集$h[C]$的满射, 而它的单性已由$h$自身的单性保证了. 

而$h$在$A-C$的限制却不一定是满的, 因为$h[A]=A'\subseteq B$本身就不一定是到$B$的满射. 因为
\begin{align*}
	A-C = (A - C_0) \cap (A-h[C]) \subseteq A-C_0 = A-(A-B) =B,
\end{align*}
所以$A-C$和$h[C]$构成了$B$的分划.

因而我们可以这样定义$A$到$B$的双射:
\begin{align*}
	i(x)=
	\begin{cases}
	h(x), & x\in C;\\
	x, & x\in A-C.
	\end{cases}
\end{align*}
\end{proof}



\begin{theorem}\label{Cantor-Bernstein-Schroeder theorem}
\indexbf{Cantor-Bernstein-Schr\"oeder 定理}

如果$|X|\leq |Y|$且$|Y|\leq |X|$, 则$|X|=|Y|$.

\end{theorem}
\begin{proof}
$|X|\leq |Y|$和$|Y|\leq |X|$分别蕴含了单射$f\in Y^X$和$g\in X^Y$的存在, 且有
\begin{align*}
	g\big\lbrack f[X]\big\rbrack \subseteq g[Y] \subseteq X, 
\end{align*}
和$|g[Y]|=|Y|$, $|g\big\lbrack f[X]\big\rbrack|=|f[X]|=|X|$.
由引理\ref{lemma_CBS}, 这意味着$|X|=|Y|$.
\end{proof}

\textbf{注}: 早先的Cantor-Bernstein-Schr\"oeder 定理利用了$\properclass{AC}$, 但这里的证明避免了它.

这个定理说明了$\leq$拥有传递性, 这意味着$\leq$是一个偏序. 而且, 在$\mathbb N$中, 它和先前定义的自然数的偏序$\leq$是等同的. 这暗示我们可以这样做: 记$|n| = n$, 只要$n\in \mathbb N$.

\backmatter
\nocite{*}
\printbibliography[heading=bibliography, title={参考文献}]

\indexprologue{这里列出了笔记中出现的重要符号.}
\printindex[symbol]

\printindex

\end{document}