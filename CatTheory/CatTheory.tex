% TeXplates/Mathematics.tex
% v0.5.0
% https://github.com/HoyanMok/TeXplates
\documentclass[openany, a5paper]{book} 
% \documentclass{ctexbook} 如果用中文
% \documentclass[10pt,a4paper]{ctexart}  字体大小和纸张大小,默认分别为10pt和letterpaper
% 五号 = 10.5pt,小四=12pt,四号=14pt
% 其他可选参量如twocolumn, 两行排版

\usepackage{../TeXplatesMathematics}
\addbibresource{CatTheory.bib} % 把这里改成实际的文件名

% 文章标题页信息:
\title{Category Theory}
\author{%
	Hoyan Mok\thanks{%
		hoyanmok@outlook.com
	}
}
\date{\today} % 自动生成日期
% \titlepic{\includegraphics{latex-project-logo.pdf}}

\begin{document}
\pagenumbering{Alph}
\maketitle % 打印标题
\frontmatter
\chapter{Preface}
Preface.

\tableofcontents

\mainmatter
\chapter{Categories}
\section{Categories}

\begin{definition}[Category]
	A \indexbf{category} $\mathcal C$ consists of three ingredients:
	\begin{enumerate}
		\item A \emph{class} $\obj(\mathcal C)$, called the \indexbf{objects};
		\item For any $A, B \in \obj(\mathcal C)$, a set of \indexbf{morphisms} $\Hom(A, B)$;
		\item A function $\Hom(A, B) \times \Hom(B, C) \to \Hom(A, C)$, called the \indexbf{composition}, for any $A, B, C \in \obj(\mathcal C)$, denoted as $(f, g) \mapsto gf$, 
	\end{enumerate}
	and they follow the following axioms:
	\begin{enumerate}[label=(\roman*)]
		\item If $(A, B) \neq (A', B')$, then $\Hom(A, B) \cap \Hom(A', B') = \varnothing$;
		\item \emph{Associativity}: the composition is associative, i.e. $h(gf) = (hg)f$;
		\item \emph{Identity}: For any $A \in \obj(\mathcal C)$, there is an identity morphism $\id_A \in \Hom(A, A)$, such that $f \id_A = f = \id_A f$, for any $B \in \obj(\mathcal C)$ and $f \in \Hom(A, B)$.
	\end{enumerate}
\end{definition}

A morphism can be shown by:
% https://tikzcd.yichuanshen.de/#N4Igdg9gJgpgziAXAbVABwnAlgFyxMJZABgBpiBdUkANwEMAbAVxiRAEEQBfU9TXfIRQAmclVqMWbAELdxMKAHN4RUADMAThAC2SMiBwQkARmoM6AIxgMACvzwE2DGGpwhq9Zq0Qg1crkA

\begin{center}\begin{tikzcd}
	A \arrow[rr, "f"] &  & B
\end{tikzcd}\end{center}

Examples of categories: $\Set$, $\Grp$, $\Ab$, $\Top$, $\Ord$, $\Ring$, $\Mod$, \ldots

If $\obj(\mathcal C)$ is a set, 
then $\mathcal C$ is called a \indexbf{small category}.

If $(X, \leq)$ is a preorder set, then $\forall x, y \in X$, 
\begin{equation}\label{eq: preorder morphism}
	\Hom(x, y) = 
	\begin{cases}
		\varnothing & x > y,
		\\
		\{(x, y)\} & x \leq y,
	\end{cases}
\end{equation}
and $(y, z) (x, y) = (x, z)$.
With this we can say that $X$ is a category.
The morphism $(x, y)$ is also denoted by $i^x_y$.

\begin{definition}[Isomorphism]
	Let $\mathcal C$ be a category and $A, B \in \obj(\mathcal C)$, $f \in \Hom(A, B)$.
	If $\exists g \in \Hom(B, A)$ s.t.\ $gf = \id_B$ and $fg = \id_A$, then $f$ is called an \indexbf{isomorphism} from $A$ to $B$. 
	$g$ is called the \indexbf{inverse} of $f$.
\end{definition}

\begin{definition}[Subcategory]
	We say $\mathcal S$ a \indexbf{subcategory} of $\mathcal C$, if
	\begin{enumerate}[label=(\roman*)]
		\item $\obj(\mathcal S) \subseteq \obj(\mathcal C)$;
		\item $\forall A, B \in \obj(\mathcal S)$, $\Hom_{\mathcal S}(A, B) \subseteq \Hom_{\mathcal C}(A, B)$;
		\item $\forall A, B, C \in \obj(\mathcal S)$, 
		\begin{center}
			% https://tikzcd.yichuanshen.de/#N4Igdg9gJgpgziAXAbVABwnAlgFyxMJZABgBpiBdUkANwEMAbAVxiRAEEQBfU9TXfIRQBGclVqMWbAELdeIDNjwEiAJjHV6zVohABhbuJhQA5vCKgAZgCcIAWyRkQOCElETtbS3Ku2HidxckdQ8pXRMfEBt7R2ogxBCcOiwGNkgwVmoAIxgwKCQAWgBmJy0wkBNvagY6HIYABX5lIRBrLBMACxxDLiA
		\begin{tikzcd}
			A \arrow[r, "f"] \arrow[rr, "gf"', bend right] & B \arrow[r, "g"] & C
		\end{tikzcd}
		\end{center}
		then $gf$ is the same in both $\Hom_{\mathcal S}(A, C)$ and $\Hom_{\mathcal C}(A, C)$,
		\item $\forall A \in \obj(\mathcal S)$, 
		$\id_A \in \Hom_{\mathcal S}(A, A)$ is the same in $\Hom_{\mathcal C}(A, A)$.
	\end{enumerate}
\end{definition}

\begin{definition}[Full subcategory]
	Let $\mathcal S$ be a subcategory of $\mathcal C$.
	If $\forall A, B \in \obj(\mathcal S)$, 
	$\Hom_{\mathcal S}(A, B) = \Hom_{\mathcal C}(A, B)$, 
	then $\mathcal S$ is called a \indexbf{full subcategory} of $\mathcal C$.
\end{definition}

\begin{definition}[Generated full subcategory]
	For any subclass $S \subseteq \obj(\mathcal C)$, 
	one can find a full subcategory $\mathcal S$ of $\mathcal C$ s.t.\ $\obj(\mathcal S) = S$, 
	which is called the full subcategory generated by $S$.
\end{definition}

$\Top_2$ is the full subcategory of $\Top$ that is generated by the class of all Hausdorff spaces.

\begin{definition}[Opposite category]
	Let $\mathcal C$ be a category.
	Then $\mathcal C^\text{op}$ is the category that:
	\begin{enumerate}
		\item $\obj(\mathcal C^\text{op}) = \obj(\mathcal C)$;
		\item $\forall A, B \in \obj(\mathcal C)$, 
		\begin{equation}
			\Hom_{\mathcal C}(A, B) = \Hom_{\mathcal C^\text{op}}(B, A).
		\end{equation}
	\end{enumerate}
\end{definition}

\section{Definitions of Different Categories}

\chapter{Functors}
\section{Functors}

\begin{definition}[Functor]
	Let $\mathcal C$ and $\mathcal D$ be categories.
	A \indexbf{functor} $F: \mathcal C \to \mathcal D$ is a function that satisfies the following axioms:
	\begin{enumerate}[label=(\roman*)]
		\item $\forall A \in \obj(\mathcal C)$, $F(A) \in \obj(\mathcal D)$;
		\item $\forall A, B \in \obj(\mathcal C)$, 
		$\forall f \in \Hom_{\mathcal C}(A, B)$, 
		 $F(f) \in \Hom_{\mathcal D}(F(A), F(B))$;
		\item $\forall A, B, C \in \obj(\mathcal C)$, 
		\begin{center}
			% https://tikzcd.yichuanshen.de/#N4Igdg9gJgpgziAXAbVABwnAlgFyxMJZABgBpiBdUkANwEMAbAVxiRAEEQBfU9TXfIRQBGclV
		\begin{tikzcd}
			A \arrow[r, "f"] \arrow[rr, "gf"', bend right] & B \arrow[r, "g"] & C
		\end{tikzcd}
		\end{center}
		then $F(gf) = F(g)  F(f)$.
		\item $\forall A \in \obj(\mathcal C)$, $F(\id_A) = \id_{F(A)}$.
	\end{enumerate}
\end{definition}

We can restate some definition using functors
\begin{theorem}[Subcategory, in language of functors]
	Let $\mathcal C$ and $\mathcal S$ be two categories, $\mathcal S \subseteq \mathcal C$.
	If the inclusion $I \colon \mathcal S \to \mathcal C$ is a functor, then $\mathcal S$ is a subcategory of $\mathcal C$.
\end{theorem}

The \indexbf{identity functor} from $\mathcal C$ to itself is $\indexmath[1C]{1_{\mathcal C}} \colon \mathcal C \to \mathcal C$ s.t.\ $\forall C, D \in \mathcal C$, $\forall f \in \Hom(C, D)$, 
\begin{equation}
	1_{\mathcal C}(C) = C,
	\quad
	1_{\mathcal C}(f) = f.
\end{equation}

\begin{theorem}
	Let $\mathcal C$ and $\mathcal D$ be two categories.
	$F \colon \mathcal C \to \mathcal D$ is a functor.
	$\forall A, B \in \obj(\mathcal C)$, if $f \in \Hom_{\mathcal C}(A, B)$ is an isomorphism, then $F(f)$ is an isomorphism.
\end{theorem}


\begin{definition}[Hom]
	Let $\mathcal C$ be a category and $A \in \obj(\mathcal C)$.
	The \indexbf{Hom functor} $F_A \colon \mathcal C \to \Set$ is defined as
	\begin{equation}
		\begin{aligned}
			F_A(B) &= \Hom(A, B),
		\\
			F_A(f) &\colon \Hom(A, B) \to \Hom(A, C); \;
			h \mapsto fh.
		\end{aligned}
	\end{equation}
\end{definition}

The Hom functor is also denoted by $\indexmath[Hom(A,)]{\Hom(A, \dummy)}$.
We call the $F_A(f) =: \Hom(A, f)$ the \indexbf{induced map}, and denote it by $\indexmath[f ast]{f_\ast}$
\begin{equation}
	f_* h = fh.
\end{equation}

\begin{definition}[Faithful functor]
	Let $\mathcal C$ and $\mathcal D$ be two categories.
	A \indexbf{faithful functor} $F \colon \mathcal C \to \mathcal D$ is a functor that satisfies $\forall A, B \in \obj(\mathcal C)$,
	\begin{equation}
		i \colon \Hom_{\mathcal C}(A, B) \to \Hom_{\mathcal D}(F(A), F(B)); \;
		f \mapsto F(f)
	\end{equation}
	is injective.
\end{definition}

\begin{definition}[Concrete category]
	Let $\mathcal C$ be a category.
	$\mathcal C$ is called a \indexbf{concrete category} if there exists a faithful functor $F \colon \mathcal C \to \Set$.
\end{definition}

\section{Contravariant Functors}

\begin{definition}[Contravariant functor]
	Let $\mathcal C$ and $\mathcal D$ be categories.
	A \indexbf{contravariant functor} $F: \mathcal C \to \mathcal D$ is a function that satisfies the following axioms:
	\begin{enumerate}[label=(\roman*)]
		\item $\forall A \in \obj(\mathcal C)$, $F(A) \in \obj(\mathcal D)$;
		\item $\forall A, B \in \obj(\mathcal C)$, 
		$\forall f \in \Hom_{\mathcal C}(A, B)$, 
		 $F(f) \in \Hom_{\mathcal D}(F(B), F(A))$;
		\item $\forall A, B, C \in \obj(\mathcal C)$, 
		\begin{center}
			% https://tikzcd.yichuanshen.de/#N4Igdg9gJgpgziAXAbVABwnAlgFyxMJZABgBpiBdUkANwEMAbAVxiRAEEQBfU9TXfIRQBGclV
		\begin{tikzcd}
			A \arrow[r, "f"] \arrow[rr, "gf"', bend right] & B \arrow[r, "g"] & C
		\end{tikzcd}
		\end{center}
		then $F(gf) = F(f)  F(g)$.
		\item $\forall A \in \obj(\mathcal C)$, $F(\id_A) = \id_{F(A)}$.
	\end{enumerate}
\end{definition}

To distinguish functors from contravariant functors, we sometimes call the functors \indexbf{covariant functors}.

$\dummy^\text{op} \colon \mathcal C \to \mathcal C^\text{op}$ is a contravariant functor.

\begin{definition}[Contravariant Hom]
	Let $\mathcal C$ be a category and $A \in \obj(\mathcal C)$.
	The \indexbf{contravariant Hom functor} $F_A \colon \mathcal C \to \Set$ is defined as
	\begin{equation}
		\begin{aligned}
			F_A(B) &= \Hom(B, A),
		\\
			F_A(f) &\colon \Hom(B, A) \to \Hom(C, A); \;
			h \mapsto hf.
		\end{aligned}
	\end{equation}
\end{definition}

The contravariant Hom functor is also denoted by $\indexmath[Hom(,A)]{\Hom(\dummy, A)}$.
We call the $F_A(f) =: \Hom(f, A)$ the \indexbf{induced map}, and denote it by $\indexmath[f ast]{f^\ast}$
\begin{equation}
	f^\ast h = hf.
\end{equation}


\section{Diagrams}
\begin{definition}[Diagram]
	A \indexbf{diagram} in a category $\mathcal C$ is a functor $D \colon \mathcal D \to \mathcal C$ where $\mathcal D$ is a small category.
\end{definition}

We have already seemed drawn diagrams like
\begin{equation}
	% https://tikzcd.yichuanshen.de/#N4Igdg9gJgpgziAXAbVABwnAlgFyxMJZABgBpiBdUkANwEMAbAVxiRAEEQBfU9TXfIRQBGclVqMWbAELdeIDNjwEio4ePrNWiEAGFu4mFADm8IqABmAJwgBbJGRA4ISURK1sLcyzfuJHzkgATNSaUjoAFiDUcBFYFjgOPD52rtSBiCHu4SDG3iDWqf7pLpnUAEYwYFBIALQALACcoZLaIBEA5AZcQA
\begin{tikzcd}
	A \arrow[r, "f"] \arrow[rd, "h"] \arrow[rd, "h'", bend right=49] & B \arrow[d, "g"] \\
																	 & C               
\end{tikzcd}
\end{equation}
where $A, B, C \in D(\obj(\mathcal D))$, and each arrow from one to another is a morphism in the image of morphism in $\mathcal D$ under $D$ e.g.\ $\exists D_A, D_B \in \obj(\mathcal D)$ s.t.\ 
\begin{equation}
	f \in D(\Hom_{\mathcal D}(D_A, D_B)) \subseteq \Hom_{\mathcal C}(A, B).
\end{equation}

\begin{definition}[Path]
	A \indexbf{path} in a category $\mathcal C$ is a functor $P \colon n + 1 \to \mathcal C$ where $n + 1$ is considered as a preorder with morphism defined in Eq.~\eqref{eq: preorder morphism}.  
\end{definition}

Conventionally we denote a path as:
\begin{equation}
	% https://tikzcd.yichuanshen.de/#N4Igdg9gJgpgziAXAbVABwnAlgFyxMJZABgBpiBdUkANwEMAbAVxiRAAUB9YkAX1PSZc+QigCM5KrUYs2XMXwEgM2PASIAmSdXrNWiDpw2LBqkUQDM26XrYAdOwGMoEHAn6nh6lABZru2QMuYDAAWjFeE2UhNVFkAFZ-GX1DQl4pGCgAc3giUAAzACcIAFskMhAcCCQJG0CQfO4ootKa6iqkLTqUxoUPBuKyxC6OxAt+lqGrSurEHwnBpD8ZpETutkaQ8Mj03iA
\begin{tikzcd}
	P_0 \arrow[r, "f_0"] & P_1 \arrow[r, "f_1"] & P_2 \arrow[r] & \cdots \arrow[r] & P_{n-1} \arrow[r, "f_{n-1}"] & P_n
	\end{tikzcd}
\end{equation}

A \indexbf{simple path} is a path such that $\forall i, j \in n + 1$, $P_i = P_j \to i = j$.

A diagram $D$ \indexbf{commutes} iff $A, B \in D(\obj \mathcal D)$, the compositions of morphisms in any two simple paths from $A$ to $B$ are the same.

\section{Natural transformations}
\begin{definition}[Natural transformation]
	Let $\mathcal C, \mathcal D$ be two categories and $F, G \colon \mathcal D \to \mathcal C$ be functors.
	A \indexbf{natural transformation} $\alpha \colon F \to G$ is \emph{one-parametre family of morphisms} in $\mathcal D$:
	\begin{equation}
		\alpha \colon \obj(\mathcal C) \to \{\Hom(F(A), G(A))\mid A \in \obj (\mathcal C)\};
		\;
		A \mapsto \alpha_A,
	\end{equation}
	such that $\forall A, B \in \obj(\mathcal C)$, $\forall f \in \Hom(A, B)$, the following diagram commutes:
	\begin{equation}
		% https://tikzcd.yichuanshen.de/#N4Igdg9gJgpgziAXAbVABwnAlgFyxMJZABgBpiBdUkANwEMAbAVxiRADEAKAQQEoQAvqXSZc+QigCM5KrUYs2AcR78hI7HgJEyk2fWatEHTgCFVwkBg3ii03dX0Kjys4NkwoAc3hFQAMwAnCABbJDIQHAgkACYHeUNjP3N-INDEcMikaTkDNgAdPMY0AAs6AH1uQQtAkJjqTMQAZjjcowKi0rKTKpTaxGyG5pynEGUktwEgA
		\begin{tikzcd}
			F(A) \arrow[d, "F(f)"] \arrow[r, "\alpha_A"] & G(A) \arrow[d, "G(f)"] \\
			F(B) \arrow[r, "\alpha_B"]                   & G(B)                  
		\end{tikzcd}
	\end{equation}
	or,
	\begin{equation}
		\alpha_B  F(f) = G(f) \alpha_A.
	\end{equation}
\end{definition}
%TODO: how to strictly express this

All natural transformations between two functors $F, G \colon \mathcal C \to \mathcal D$ is denoted as $\Nat(F, G)$. 
However, $\Nat(F, G)$ can only be considered as an object in our metalanguage, since it does not even have to be a class.

A \indexbf{natural isomorphism} is a natural transformation $\alpha \colon F \to G$ such that $\forall A \in \obj(\mathcal C)$, $\alpha_A$ is an isomorphism.

Natural transformations can compose, and for any functor, there exists an identity natural isomorphism.

You can define the contravariant version of natrual transformation too.

\begin{theorem}[Yoneda lemma]%
	\label{theorem: Yoneda lemma}
	Let $\mathcal C$ be a category and $F \colon \mathcal C \to \Set$ be a functor.
	Then, $\forall A \in \obj(\mathcal C)$, $\Nat(\Hom_{\mathcal C}(A, \dummy), G)$ is a set\footnote{Why?}%
	, and there exists a bijection
	\begin{equation}
		y \colon \Nat(\Hom_{\mathcal C}(A, \dummy), G) \to F(A)
	\end{equation}
	s.t.\ 
	\begin{equation}
		y(\tau) = \tau_A (1_A).
	\end{equation}
\end{theorem}
\begin{proof}
	$\tau \in \Nat(\Hom_{\mathcal C}(A, \dummy), G)$ means $\forall A, B \in \obj(\mathcal C)$, the diagram
	\begin{equation}
		\begin{tikzcd}
			{\Hom_{\mathcal C}(A, B)} \arrow[d, "\varphi_*"] \arrow[r, "\tau_B"] & F(B) \arrow[d, "F(\varphi)"] \\
			{\Hom_{\mathcal C}(A, C)} \arrow[r, "\tau_C"]                        & F(C)                          
		\end{tikzcd}
	\end{equation}
	commutes.
	Setting $B = A$ we have
	\begin{equation}
		\begin{tikzcd}
			{\Hom_{\mathcal C}(A, A)} \arrow[d, "\varphi_*"] \arrow[r, "\tau_A"] & F(A) \arrow[d, "F(\varphi)"] \\
			{\Hom_{\mathcal C}(A, C)} \arrow[r, "\tau_C"]                        & F(C)                          
		\end{tikzcd}
	\end{equation}
	which gives
	\begin{equation}
		F(\varphi) \tau_A(1_A) = \tau_C \varphi_*(1_A) = \tau_C(\varphi).
	\end{equation}

	\paragraph{Injectivity}
	Now assuming there exists another natural transformation $\sigma \colon \Hom_{\mathcal C}(A, \dummy) \to G$ such that $\sigma_A(1_A) = \tau_A(1_A)$, we have $\forall C \in \obj(\mathcal C)$,
	\begin{equation}
		\sigma_C(\varphi) = F(\varphi) \sigma_A(1_A) = F(\varphi) \tau_A(1_A) = \tau_C(\varphi),
	\end{equation}
	i.e.\ $y(\tau) = y(\sigma) \to \tau = \sigma$, or in plain words, $y$ is an injection.

	\paragraph{Surjectivity}
	$\forall a \in F(A)$, find a morphism $\tau_A \colon \Hom_{\mathcal C}(A, A) \to F(A)$ s.t.\ $\tau_A(1_A) = a$ (this is always possible e.g.\ we can set $\tau_A$ to be the constant function $\Hom_{\mathcal C}(A, A) \ni f \mapsto a$).
	Then, $\forall C \in \obj(\mathcal C)$, $\forall \varphi \in \Hom_{\mathcal C}(A, C)$, define morphism as $\tau_C(\varphi) = F(\varphi)(a)$.

	We have not yet proved that $\tau$ is natural, so we check if $\psi \in \Hom_{\mathcal C}(B, C)$, $\forall \vartheta \in \Hom_{\mathcal C}(A, B)$:
	\begin{equation}
		\tau_C \psi_* (\vartheta) = F(\psi_* \vartheta) (a)
		= F(\psi \vartheta ) (a) = F(\psi)  F(\vartheta) (a)
		= F(\psi) \tau_B (\vartheta).
	\end{equation}
\end{proof}

\appendix
\chapter{Appendix}

\backmatter
\nocite{*} % 这个表示列出所有没有在文中被引用的参考文献
\printbibliography[heading=bibliography, title={Bibliography}]

% \indexprologue{Here listed the important symbols used in the notes.}
% \printindex[symbol]

\printindex

\end{document}