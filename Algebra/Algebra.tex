% TeXplates/Mathematics.tex
% v0.1.2
% https://github.com/HoyanMok/TeXplates
\documentclass[openany]{ctexbook} 
% \documentclass{ctexbook} 如果用中文
% \documentclass[10pt,a4paper]{ctexart}  字体大小和纸张大小,默认分别为10pt和letterpaper
% 五号 = 10.5pt,小四=12pt,四号=14pt
% 其他可选参量如twocolumn, 两行排版

\usepackage{biblatex} %[style=gb7714-2015]{biblatex} 可以选择样式
\addbibresource{Algebra.bib} % 把这里改成实际的文件名

% 令参考资料能够加入目录中:
\defbibheading{bibliography}[\bibname]{% 
	% \addcontentsline{toc}{chapter}{参考文献}
	\chapter{#1}% 
	\markboth{#1}{#1}}

\usepackage{imakeidx} %索引
	\makeindex[intoc]
	\newcommand*{\indexbf}[1]{\textbf{#1}\index{#1}} % Index for definition
	\newcommand*{\indexfm}[2][ ]{#2\index{ @使用的符号!#1@$#2$}} % Used Symbol

% 将PATH换成绝对路径 (Windows) 或相对路径 (Mac OS或Linux)
% 使用「/」而不是「\」
\newcommand{\PATH}{PATH/}

% 对目录项等的修改
\usepackage{chngcntr}
	\counterwithout{section}{chapter} % So that the section won't reset when newing a chapter
\renewcommand{\thesection}{\textmd{\S}\arabic{section}}
\renewcommand{\thesubsection}{\arabic{section}.\arabic{subsection}}

% 引用的宏包:
% 宏包的使用, 可以在命令行运行texdoc <宏包名>获得文档
\usepackage{multicol} % 分栏 (全局分栏建议在文档类处设置)
\usepackage{amsmath} % AMS数学标准
	\makeatletter % '@' now normal "letter"
	\@addtoreset{equation}{section} % 每次换section就把equation清零
	\makeatother  % '@' is restored as "non-letter"
	\renewcommand\theequation{\oldstylenums{\arabic{section}}%
					-\oldstylenums{\arabic{equation}}} % 显示为section数-equation数
\usepackage{amssymb} % 数学符号
\usepackage{mathrsfs} % 花体
\usepackage{amsthm} %定义、证明、定理等
	\theoremstyle{plain}
		\newtheorem{axion}{Axion} %公理
		\newtheorem{theorem}{Theorem}[section] %定理
		\newtheorem{corollary}{Corollary} %推论
		\newtheorem{lemma}{Lemma} %引理
	\theoremstyle{definition}
		\newtheorem{definition}{Definition}[section] %定义
		\newtheorem{proposition}{Proposition} %命题
	\renewcommand{\proofname}{\textbf{Proof}}

\renewcommand{\thetheorem}{%
	\arabic{section}.\arabic{theorem}%
} % 公式编号不显示`\S`
\renewcommand{\thedefinition}{%
	\arabic{section}.\arabic{definition}%
} % 公式编号不显示`\S`

\usepackage{esint} % 积分
\usepackage{siunitx} % 标准SI数值和单位处理

\usepackage{tikz} % 绘图
\usepackage{float} % 浮动体 (供图片, 表格等) 扩展, 主要用于提供h模式
\usepackage{graphicx} % 插入图片
\usepackage{titlepic}
\usepackage[font=small, skip=5pt]{caption} % 缩小题注字体和题注与图片距离
\usepackage{subcaption} % 子图和子图的题注
\usepackage{svg} % svg位图
\usepackage{wrapfig} % 简单的图文绕排
\usepackage[inline]{enumitem} % 编号
\usepackage{geometry} % 调整页边距
% \geometry{left=1.6cm,right=1.6cm}
\usepackage{xcolor} % 颜色
\usepackage[colorlinks=true,bookmarks=true]{hyperref} % 引用, 交叉引用, 图表等的链接; 生成书签
\hypersetup{linkcolor=[rgb]{1,0.27,0},bookmarksopen = true}% 更多设置请查阅: texdoc hyperref


% 定义一些笔者常用的指令:
\newcommand{\me}{\mathrm{e}} % 自然对数的底
\newcommand{\mi}{\mathrm{i}} % 虚数单位
\newcommand{\dif}{\mathop{}\!\mathrm{d}} % 微分算子d
\newcommand*{\basis}[1]{\hat{\boldsymbol{#1}}} % 基底
\newcommand*{\bv}{\boldsymbol} % 向量加粗
\newcommand*{\id}{\mathrm{id}} % 单位映射

\newcommand*{\diff}[3][1]
{\if#11%
	\frac{\mathrm{d} #2}{\mathrm{d} #3}% 导数\diff{y}{x}
\else%
	\frac{\mathrm{d}^{#1} #2}{\mathrm{d} #3^{#1}}% n阶导数\diff[n]{y}{x}
\fi}
\newcommand*{\pdiff}[3][1]
{\if#11%
	\frac{\partial #2}{\partial #3}% 偏导数\pdiff{y}{x}
\else%
	\frac{\partial^{#1} #2}{\partial #3^{#1}}% n阶偏导数\pdiff[n]{y}{x}
\fi}
\newcommand{\emphbf}[1]{\emph{\textbf{#1}}}
% \indexbf 的定义见前imakeidx的引用下

% 笔者习惯的运算符:
\DeclareMathOperator{\tg}{tg}
\DeclareMathOperator{\ctg}{ctg}
\DeclareMathOperator{\arctg}{arctg}
\DeclareMathOperator{\sh}{sh}
\DeclareMathOperator{\ch}{ch}
\DeclareMathOperator{\dom}{dom}
\DeclareMathOperator{\ran}{ran}
\DeclareMathOperator{\interior}{int}
\DeclareMathOperator{\card}{card}
% \DeclareMathOperator*{\指令}{显示} 
% 带星号的版本会像\lim一样

% 一些符号:
\newcommand*{\GL}[2][n]{\mathrm{GL}_{#1}\left({#2}\right)}
\newcommand*{\SL}[2][n]{\mathrm{SL}_{#1}\left({#2}\right)}
\newcommand*{\Orth}[2][n]{\mathrm{O}_{#1}\left({#2}\right)}
\newcommand*{\SO}[2][n]{\mathrm{SO}_{#1}\left({#2}\right)}



% 文章标题页信息:
\title{Algebra}
\author{ Hoyan Mok\thanks{E-mail: victoriesmo@hotmail.com}
	}
\date{\today} % 自动生成日期
% \titlepic{\includegraphics{\PATH latex-project-logo.pdf}}

\begin{document}
\maketitle % 打印标题
\thispagestyle{empty}
\frontmatter

\addcontentsline{toc}{chapter}{Contents}
\tableofcontents

\mainmatter
\chapter{群. 环. 域}
\section{代数运算}
\begin{definition}[二元运算]
	集合的Cartesian平方到自身的映射$* \colon X^2 \to X$称为其上的一个\indexbf{二元运算}.
	通常我们记$*(a,b) := a * b$. 
	当$X$上定义了二元运算$*$后, 称$*$定义了$X$上的一种\indexbf{代数结构} $\indexfm[X ast]{(X,*)}$, 也称\indexbf{代数系统}. 
\end{definition}

当指代是明确的时候, 我们将混用集合及其代数结构.

作为习惯, 如果$\cdot, + \in X^{X^2}$, 我们记$ab := a \cdot b$并称其为$a$和$b$的\indexbf{积}, 称$a+b$为$a$和$b$的\indexbf{和}. 这些只是约定.

若$a* b = b* a$则称$*$或$(X,*)$是\indexbf{交换的}, 而若$(a* b)* c = a*(b* c)$则称$*$或$(X,*)$为\indexbf{结合的}. 

若$\exists e\in X$满足$\forall x\in A\big(
	e* x = x * e = x
\big)$, 则称其为$*$的一个\indexbf{单位元} (identity), 这时可把$(X,*)$记作$\indexfm[X ast e]{(X,*, e)}$. 可以证明一个代数结构最多只有一个单位元. 
乘法单位元通常记为$1$, 而加法单位元 (也叫\indexbf{零元}) 记为$0$.

\begin{definition}[半群和幺半群]
	若$*$是结合的, 称$(X,*)$是\indexbf{半群} (semigroup); 
	若$*$还有一个单位元, 则称$(X,*, e)$是\indexbf{幺半群} (monoid).
\end{definition}

倘若幺半群$(M, *, e)$是有限的 (即其元素有限), 称$\card M$为\indexbf{有限幺半群}的\indexbf{阶}.

半群中, 括号的位置是不重要的 (可用数学归纳法证明). 通常我们记$x_1x_2 \cdots x_n$为:
\begin{equation}
	\prod_{i=1}^1 x_i = x_1,\;\prod_{i=1}^{n+1} x_i = \left( \prod_{i=1}^n x_i  \right) x_n\,;
\end{equation}
同理$x_1+x_2+\cdots + x_n$为:
\begin{equation}
	\sum_{i=1}^1 x_i = x_1,\;\sum_{i=1}^{n+1} x_i = \left( \sum_{i=1}^n x_i  \right) + x_n\,.
\end{equation}
在半群不交换的场合, 指出递推式右端的顺序是重要的. 这种记法称为\indexbf{左正规}.

若$x := x_1 = x_2 = \cdots = x_n$, 记$\sum_{i=1}^n x_i = nx$, $\prod_{i=1}^n x_i = x^n$, 分别表示$x$的$n$倍和$x$的$n$次幂. 它们满足:
\begin{equation}
	nx+mx = (n+m)x, \; n(m x) = nm x, \qquad n,m\in \mathbb N_+\,;
\end{equation}
\begin{equation}
	x^n x^m = x^{n+m}, \; (x^m)^n = x^{nm}, \qquad n,m \in \mathbb N_+\,.
\end{equation}

在幺半群中, 还可以令$x^0 = 1$, $0x = 0$.

若半群$S$有子集$S'$, 使得$(S',*)$是半群, 那么称其为半群$(S,*)$的\indexbf{子半群}.
同理有幺半群\nolinebreak$M$的\indexbf{子幺半群}$M'$. 

若半群$(S,*, e)$的元素$a$满足$\exists a'\in S\big(
	a a' = a' a = e
\big)$, 那么称$a$为\indexbf{可逆的} (invertible), $a'$称为其\indexbf{逆元} (inverse element) 或\indexbf{逆} (inverse).
通常加法逆元记为$- a$, 乘法逆元记为$a^{-1}$, 且为可逆元素引入$n a$, $a^n$的概念, 其中$n \in \mathbb Z$. 当$n$为负数时, $na = -(-na)$, $a^n = (a^{-n})^{-1}$.


\section{群}

可逆幺半群$G$称为群, 即:
\begin{definition}[群]
	设有集合$G$. 若:
	\begin{enumerate}[label=G\arabic*)]\setcounter{enumi}{-1}
		\item 定义了二元运算$\cdot \colon G^2 \to G; (x,y) \mapsto xy$.
		\item 结合性: $\forall x,y,z\in G$, $(xy)z = x(yz)$.
		\item 单位元: $\exists e\in G \forall x\in G$, $xe = ex = x$.
		\item 可逆性: $\forall x\in G \exists x^{-1} \in G$, $x x^{-1} = x^{-1} x = e$.
	\end{enumerate}
	则称$(G, \cdot)$为\indexbf{群}.
\end{definition}

考虑$n$阶可逆实矩阵的集合$\indexfm[GL n R]{\GL{\mathbb R}}$和在其上定义的矩阵乘法组成的群$(\GL{\mathbb R}, \cdot )$称为$n$阶\indexbf{一般线性群}.

\appendix

\backmatter
\nocite{*} % 这个表示列出所有没有在文中被引用的参考文献
\printbibliography[heading=bibliography, title={参考文献}]


\printindex
\end{document}