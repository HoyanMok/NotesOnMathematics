% TeXplates/Mathematics.tex
% v0.1.6
% https://github.com/HoyanMok/TeXplates
\documentclass[openany]{ctexbook} 
% \documentclass{ctexbook} 如果用中文
% \documentclass[10pt,a4paper]{ctexart}  字体大小和纸张大小,默认分别为10pt和letterpaper
% 五号 = 10.5pt,小四=12pt,四号=14pt
% 其他可选参量如twocolumn, 两行排版

\usepackage{xpatch}

\ExplSyntaxOn
\xpatchcmd \fontspec_new_script:nn
	{ \__fontspec_warning:nxx }
	{ \__fontspec_info:nxx }
	{}{\fail}
\ExplSyntaxOff % 沉默字体警告

% 将PATH换成绝对路径 (Windows) 或相对路径 (Mac OS或Linux)
% 使用「/」而不是「\」
\newcommand{\PATH}{./}

\usepackage{biblatex} %[style=gb7714-2015]{biblatex} 可以选择样式
\addbibresource{Algebra.bib} % 把这里改成实际的文件名

% 令参考资料能够加入目录中:
\defbibheading{bibliography}[\bibname]{% 
	% \addcontentsline{toc}{chapter}{参考文献}
	\chapter{#1}% 
	\markboth{#1}{#1}}

\usepackage{imakeidx} %索引
\makeindex[intoc, title={索引}]
\makeindex[intoc, name=symbol, title={符号列表}]
\newcommand*{\indexbf}[1]{\emph{\textbf{#1}}\index{#1}} % Index for definition
\newcommand*{\indexfm}[2][\ ]{#2\index[symbol]{#1@$#2$}} % Used Symbol
% \indexfm[name for sort]{display} 


% 对目录项等的修改
\usepackage{chngcntr}
	\counterwithout{section}{chapter} % So that the section won't reset when newing a chapter
\renewcommand{\thesection}{\textmd{\S}\arabic{section}}
\renewcommand{\thesubsection}{\arabic{section}.\arabic{subsection}}

% 引用的宏包:
% 宏包的使用, 可以在命令行运行texdoc <宏包名>获得文档
\usepackage{multicol} % 分栏 (全局分栏建议在文档类处设置)
\usepackage{amsmath} % AMS数学标准
	\makeatletter % '@' now normal "letter"
	\@addtoreset{equation}{section} % 每次换section就把equation清零
	\makeatother  % '@' is restored as "non-letter"
	\renewcommand\theequation{\oldstylenums{\arabic{section}}%
					-\oldstylenums{\arabic{equation}}} % 显示为section数-equation数
\usepackage{amssymb} % 数学符号
\usepackage{mathrsfs} % 花体
\usepackage{amsthm} %定义、证明、定理等
	\theoremstyle{plain}
		\newtheorem{axion}{Axion} %公理
		\newtheorem{theorem}{Theorem}[section] %定理
		\newtheorem{corollary}{Corollary} %推论
		\newtheorem{lemma}{Lemma} %引理
	\theoremstyle{definition}
		\newtheorem{definition}{Definition}[section] %定义
		\newtheorem{proposition}{Proposition} %命题
	\renewcommand{\proofname}{\textbf{Proof}}

\renewcommand{\thetheorem}{%
	\arabic{section}.\arabic{theorem}%
} % 公式编号不显示`\S`
\renewcommand{\thedefinition}{%
	\arabic{section}.\arabic{definition}%
} % 公式编号不显示`\S`

\usepackage{esint} % 积分
\usepackage{siunitx} % 标准SI数值和单位处理

\usepackage{tikz} % 绘图
\usepackage{float} % 浮动体 (供图片, 表格等) 扩展, 主要用于提供h模式
\usepackage{graphicx} % 插入图片
\usepackage{titlepic}
\usepackage[font=small, skip=5pt]{caption} % 缩小题注字体和题注与图片距离
\usepackage{subcaption} % 子图和子图的题注
\usepackage{svg} % svg位图
\usepackage{wrapfig} % 简单的图文绕排
\usepackage[inline]{enumitem} % 编号
	% 新列表:
	\newlist{conditionlist}{enumerate}{2}
	\setlist[conditionlist,1]{topsep = 0pt, itemsep = 0pt, parsep = 0pt,%
		label=\arabic*), leftmargin=2\parindent}
	\setlist[conditionlist,2]{topsep = 0pt, itemsep = 0pt, parsep = 0pt,%
		label=\alph*), leftmargin=3\parindent}
\usepackage{geometry} % 调整页边距
% \geometry{left=1.6cm,right=1.6cm}
\usepackage{xcolor} % 颜色
\usepackage[colorlinks=true,bookmarks=true]{hyperref} % 引用, 交叉引用, 图表等的链接; 生成书签
\hypersetup{linkcolor=[rgb]{1,0.27,0},bookmarksopen = true}% 更多设置请查阅: texdoc hyperref


% 定义一些笔者常用的指令:
\newcommand{\me}{\mathrm{e}} % 自然对数的底
\newcommand{\mi}{\mathrm{i}} % 虚数单位
\newcommand{\dif}{\mathop{}\!\mathrm{d}} % 微分算子d
\newcommand*{\basis}[1]{\hat{\boldsymbol{#1}}} % 基底
\newcommand*{\bv}{\boldsymbol} % 向量加粗
\newcommand*{\id}{\mathrm{id}} % 单位映射
\newcommand*{\IFF}{\;\leftrightarrow\;} % 充要条件

\newcommand*{\diff}[3][1]
{\if#11%
	\frac{\mathrm{d} #2}{\mathrm{d} #3}% 导数\diff{y}{x}
\else%
	\frac{\mathrm{d}^{#1} #2}{\mathrm{d} #3^{#1}}% n阶导数\diff[n]{y}{x}
\fi}
\newcommand*{\pdiff}[3][1]
{\if#11%
	\frac{\partial #2}{\partial #3}% 偏导数\pdiff{y}{x}
\else%
	\frac{\partial^{#1} #2}{\partial #3^{#1}}% n阶偏导数\pdiff[n]{y}{x}
\fi}
\newcommand{\emphbf}[1]{\emph{\textbf{#1}}}
% \indexbf 的定义见前imakeidx的引用下

% 笔者习惯的运算符:
\DeclareMathOperator{\tg}{tg}
\DeclareMathOperator{\ctg}{ctg}
\DeclareMathOperator{\arctg}{arctg}
\DeclareMathOperator{\sh}{sh}
\DeclareMathOperator{\ch}{ch}
\DeclareMathOperator{\dom}{dom}
\DeclareMathOperator{\ran}{ran}
\DeclareMathOperator{\interior}{int}
\DeclareMathOperator{\card}{card}

\DeclareMathOperator{\Aut}{Aut}
\DeclareMathOperator{\Inn}{Inn}
\DeclareMathOperator{\characteristic}{char}
% \DeclareMathOperator*{\指令}{显示} 
% 带星号的版本会像\lim一样

% 一些符号:
\newcommand*{\GL}{\mathrm{SL}}
\newcommand*{\Orth}{\mathrm{O}}
\newcommand*{\SO}{\mathrm{SO}}
\newcommand*{\GF}{\mathrm{GF}}






% 文章标题页信息:
\title{Algebra}
\author{ Hoyan Mok\thanks{E-mail: victoriesmo@hotmail.com}
	}
\date{\today} % 自动生成日期
% \titlepic{\includegraphics{\PATH latex-project-logo.pdf}}

\begin{document}
\pagenumbering{Alph}
\maketitle % 打印标题
\thispagestyle{empty}
\frontmatter
\phantomsection
\addcontentsline{toc}{chapter}{目录}
\tableofcontents

\mainmatter
\part{线性代数}
\chapter{群. 环. 域}
\section{代数运算}
\begin{definition}[二元运算]
	集合的Cartesian平方到自身的映射$* \colon X^2 \to X$称为其上的一个\indexbf{二元运算}.
	通常我们记$*(a,b) := a * b$. 
	当$X$上定义了二元运算$*$后, 称$*$定义了$X$上的一种\indexbf{代数结构} $\indexfm[X ast]{(X,*)}$, 也称\indexbf{代数系统}. 
\end{definition}

当指代是明确的时候, 我们将混用集合及其代数结构.

作为习惯, 如果$\cdot, + \in X^{X^2}$, 我们记$ab := a \cdot b$并称其为$a$和$b$的\indexbf{积}, 称$a+b$为$a$和$b$的\indexbf{和}. 这些只是约定.

若$a* b = b* a$则称$*$或$(X,*)$是\indexbf{交换的}, 而若$(a* b)* c = a*(b* c)$则称$*$或$(X,*)$为\indexbf{结合的}. 

若$\exists e\in X$满足$\forall x\in A\big(
	e* x = x * e = x
\big)$, 则称其为$*$的一个\indexbf{单位元} (identity), 这时可把$(X,*)$记作$\indexfm[X ast e]{(X,*, e)}$. 可以证明一个代数结构最多只有一个单位元. 
乘法单位元通常记为$1$, 而加法单位元 (也叫\indexbf{零元}) 记为$0$.

\begin{definition}[半群和幺半群]
	若$*$是结合的, 称$(X,*)$是\indexbf{半群} (semigroup); 
	若$*$还有一个单位元, 则称$(X,*, e)$是\indexbf{幺半群} (monoid).
\end{definition}

倘若幺半群$(M, *, e)$是有限的 (即其元素有限), 称$\card M$为\indexbf{有限幺半群}的\indexbf{阶}.

作为重要的例子, \indexbf{置换幺半群} 定义为$(X^X, \circ, \id_X)$, 有幺半群结构的$X^X$通常记作$M(X)$.

半群中, 括号的位置是不重要的 (可用数学归纳法证明). 通常我们记$x_1x_2 \cdots x_n$为:
\begin{equation}
	\prod_{i=1}^1 x_i = x_1,\;\prod_{i=1}^{n+1} x_i = \left( \prod_{i=1}^n x_i  \right) x_n\,;
\end{equation}
同理$x_1+x_2+\cdots + x_n$为:
\begin{equation}
	\sum_{i=1}^1 x_i = x_1,\;\sum_{i=1}^{n+1} x_i = \left( \sum_{i=1}^n x_i  \right) + x_n\,.
\end{equation}
在半群不交换的场合, 指出递推式右端的顺序是重要的. 这种记法称为\indexbf{左正规}.

若$x := x_1 = x_2 = \cdots = x_n$, 记$\sum_{i=1}^n x_i = nx$, $\prod_{i=1}^n x_i = x^n$, 分别表示$x$的$n$倍和$x$的$n$次幂. 它们满足:
\begin{equation}
	nx+mx = (n+m)x, \; n(m x) = nm x, \qquad n,m\in \mathbb N_+\,;
\end{equation}
\begin{equation}\label{exoponentiation}
	x^n x^m = x^{n+m}, \; (x^m)^n = x^{nm}, \qquad n,m \in \mathbb N_+\,.
\end{equation}

在幺半群中, 还可以令$x^0 = 1$, $0x = 0$.

若半群$S$有子集$S'$, 使得$(S',*)$是半群, 那么称其为半群$(S,*)$的\indexbf{子半群}.
同理有幺半群\nolinebreak$M$的\indexbf{子幺半群}$M'$. 

若半群$(S,*, e)$的元素$a$满足$\exists a'\in S\big(
	a a' = a' a = e
\big)$, 那么称$a$为\indexbf{可逆的} (invertible), $a'$称为其\indexbf{逆元} (inverse element) 或\indexbf{逆} (inverse).
通常加法逆元记为$- a$, 乘法逆元记为$a^{-1}$, 且为可逆元素引入$n a$, $a^n$的概念, 其中$n \in \mathbb Z$. 当$n$为负数时, $na = -(-na)$, $a^n = (a^{-n})^{-1}$.

因为群未必是Abelian, 我们可以也用弱化的\indexbf{左可逆} $\exists y$ s.t.\ $y * x = 1$或\indexbf{右可逆}的概念.


\section{群}

可逆幺半群$G$称为群, 即:
\begin{definition}[群]
	设有集合$G$. 若:
	\begin{conditionlist}[label=G\arabic*)]\setcounter{enumi}{-1}
		\item 定义了二元运算$\mathord{\cdot} \colon G^2 \to G; (x,y) \mapsto xy$.
		\item 结合性: $\forall x,y,z\in G$, $(xy)z = x(yz)$.
		\item 单位元: $\exists e\in G \forall x\in G$, $xe = ex = x$.
		\item 可逆性: $\forall x\in G \exists x^{-1} \in G$, $x x^{-1} = x^{-1} x = e$.
	\end{conditionlist}
	则称$(G, \cdot)$为\indexbf{群}.
\end{definition}

交换群又叫做\indexbf{Abelian群}. 

作为重要的例子, 设 $X$ 是一个集合, $\indexfm[S X]{S(X)} = \{ f \in X^X \mid \text{$f$ 是双射}\}$. 
我们断言, $(S(X), \circ, \id_X)$ 是一个群, 称为\indexbf{变换群}或\indexbf{置换群}, 其中 $\circ$ 是函数的复合, $\id_X$是恒等变换. 
当它的阶数 $\card X = n$ 是有限的时候, 记$\indexfm[S n]{S_n} := S(X)$.

群也有子群的概念. 
设$(G, \cdot, e)$是一个群. 当一个集合$G' \subset G$满足:
\begin{conditionlist}[label=SG\arabic*)]
	\item $e \in G'$;
	\item $\forall x,y\in G'$, $xy \in G'$;
	\item $x \in G' \to x^{-1} \in G'$,
\end{conditionlist}
则称$(G', \cdot , e)$是一个$G$的\indexbf{子群}.
倘若还有$G' \neq G$则称其为一个\indexbf{真子群}\footnote{\cite{kostrikin1982introduction}等文献把\indexbf{平凡群}$\{e\}$也排在真子群的定义外.}.

\begin{theorem}
	非空的 $G'$ 是群 $(G, \cdot, 1)$ 的子群 $\IFF$ $\forall x,y \in G' (xy^{-1} \in G')$.
\end{theorem}
\begin{proof} 根据子群的定义, $to$ 是显然的, 下给出 $\leftarrow$ 的证明:
	\begin{conditionlist}[label=SG\arabic*)]
		\item $\forall x \in G' (x x^{-1} = 1 \in G)$;
		\item $\forall x,y\in G'$, $x1^{-1} {1y^{-1}}^{-1} = xy \in G'$;
		\item $\forall x \in G'$, $1x^{-1} = x^{-1} \in G'$.
	\end{conditionlist}
\end{proof}

这里将不加证明地给出:
\begin{lemma}\label{theorem:子群族的交}
	群 $G$ 的子群族 $\mathscr H = \{H \mid \text{$H$ 是 $G$ 的子群}\}$ 的交 $\cap \mathscr H$ 也是 $G$ 的子群.
\end{lemma}

设 $G$ 有子集 $S$ , 我们说群 $(G, \cdot, 1)$ 是由 $S$ 生成的, 意思是说 $G$ 没有包含 $S$ 的真子群. 记为$G = \indexfm[S]{%
	\langle S\rangle}$.
\begin{theorem}
	$\langle S \rangle = \left\{
		\prod^{n-1}_{i=0} s_i \middle|
			\forall i \in n(s_i \in S \vee s_i^{-1} \in S)
\right\}$.
\end{theorem}
\begin{proof}
	根据群的定义, 形如 $\prod^{n-1}_{i=0} s_i$ 的将构成一个群. 如果存在一个不能写成这种形式的元素, 那么它们将构成一个真子群, 这和 $\langle S\rangle$的定义相违背. 
	
\end{proof}

我们把半群的公式~\eqref{exoponentiation} 推广到整数次幂, 证明在此忽略了.
\begin{theorem}\label{group exoponentiation}
	$\forall g \in G$, $\forall n,m\in \mathbb Z$,
	\begin{equation}
		g^m g^n = g^{m+n}, \quad
		(g^m)^n = g^{mn}.
	\end{equation}
\end{theorem}

\begin{definition}[循环群]
	设$(G, \cdot , 1)$是一个乘法群, $\exists g_0 \in G$, 使得$\forall g \in G$, $\exists n \in \mathbb Z$, $a^n = g$, 那么我们称它是一个\indexbf{循环群}, $g_0$是一个\indexbf{生成元} (generator), 并记作$G = \indexfm[g 0]{\langle g_0 \rangle}$.
\end{definition}

对于群 $G$ 中任意元素 $g$, 我们称 $\card \langle g \rangle$ 为元 $g$ 的\indexbf{阶数}, 或称 $g$ 为 \indexbf{$n$ 阶元}. 而且它将满足:
\begin{theorem}
	任意群 $G$ 中若有 $q \in \mathbb Z$ 阶元 $g$, 则 $\langle g \rangle = \{e, g, \dots, g^{q-1}\}$, 且:
	\begin{equation}
		g^n = e \IFF n = kq, \qquad n \in \mathbb Z\,.
	\end{equation}
\end{theorem}
证明利用带余除法和定理~\ref{group exoponentiation}, 证明是显然的. 从该定理, 我们可以论断: 循环群都是Abelian群.

\begin{definition}[同构]
	两个群 $(G, *)$, $(G', \circ)$ 如若满足: $\exists f\colon G \to G'$ s.t.\ \begin{conditionlist}[label=\roman*)]
		\item $\forall a, b \in G$, $f(a * b) = f(a) \circ f(b)$; \label{item:保结构}
		\item $f$是双射,
	\end{conditionlist}
	则称 $f$ 是一个\indexbf{同构映射}或\indexbf{同构} (isomorphism), 并认为两个群是互相\indexbf{同构}的 (isomorphic), 记为$\indexfm[G simeq G prime]{G \simeq G'}$.
\end{definition}

同构关系的自反性, 传递性和对称性是平凡的.

\begin{theorem}
	设群 $(G, *, 1)$, $(G', \circ, 1')$ 被 $f$ 见证同构, 那么$f(1) = 1'$.
\end{theorem}
\begin{proof}
	$\forall g' \in G'$, 记 $g := f^{-1}(g')$, 那么 $f(g)\circ f(1) = f(g * 1) = g' = f(1 * g) = f(1) \circ f(g)$. 从而$f(1) = 1'$.
\end{proof}

\begin{theorem}
	设群 $(G, *, 1)$, $(G', \circ, 1')$ 被 $f$ 见证同构, 那么$\forall g \in G$, $f(g^{-1}) = f(g)^{-1}$.
\end{theorem}
\begin{proof}
	$f(g) \circ f(g^{-1}) = f(g * g^{-1}) = f(1) = 1' 
		= f(g^{-1} * g) = f(g^{-1})\circ f(g)$.
\end{proof}

\begin{theorem}
	$\card \langle g_0\rangle = \card \langle g'_0 \rangle 
		\to \langle g_0\rangle \simeq \langle g'_0 \rangle$.
\end{theorem}
\begin{proof}
	倘若$\card \langle g_0\rangle = \infty$, 那么$\nexists n \in \mathbb Z - \{0\}$, s.t.\ $g_0^n = e$; 这意味着, 存在这样的双射 $f\colon \mathbb Z \to \langle g_0\rangle$, 满足$f(n) = g_0^n$, 见证了 $(\mathbb Z, +, 0) \simeq (\langle g_0\rangle , *, e)$. 

	如果阶数是有限的, 只需令$f\colon g^k \to g'^k$, 其中$k= 0$, $1$, $\cdots$, $\card\langle g_0\rangle$.
\end{proof}

\begin{theorem}[\indexbf{Cayley定理}]
	设 $(G, *, e)$ 任意 $n$ 阶有限群. 
	$\exists H \subset S_0$ s.t.\ $(H, \circ,\id_X)$是$S_n$的子群且 $G \simeq H$. 
\end{theorem}
\begin{proof}
	取$H := \{L_g \mid g \in G\}$, 其中$L_g \colon G \to G; g' \mapsto gg'$可以证明是双射. 那么 $L\colon G \to H; g \mapsto L_g$ 见证了 $H \simeq G$.
\end{proof}

若 $\varphi \colon G \to G$ 见证了 $G \simeq G$ (如 $\id_G$), 那么称 $\varphi$ 是群 $G$ 的一个 \indexbf{自同构} (automorphism). 所有自同构组成的集合 $\Aut(G)$ 和其上的函数复合 $\circ$ 构成了 $S(G)$ 的一个子群, 称为$G$的\indexbf{自同构群}.

自同构群有一特殊的子群$\Inn(G) := \{I_a \colon g \mapsto a g a^{-1} \mid a \in G\}$,
称为\indexbf{内自同构群}. 

\begin{definition}[同态]
	设有群 $(G, *, e)$ 和 $(G', \circ, e')$, 映射 $f \colon G \to G'$ 若满足
	\begin{equation*}
		\forall a, b \in G, \quad 
			f(a * b) = f(a) \circ f(b),
	\end{equation*}
	则称其为群 $(G, *)$ 到群 $(G', \circ)$ 的一个\indexbf{同态} (homomorphism), 也叫\indexbf{态射} (morphism). 类似映射, 可定义\indexbf{单态射} (monomorphism), \indexbf{满态射} (epimorphism).

	集合 $\ker f := f^{-1}(\{e'\})$ 叫做同态 $f$ 的\indexbf{核} (kernel). 群到自身的同态映射称为\indexbf{自同态} (endomorphism).
\end{definition}

同态 $f$ 的核是$G$的子群, 而 $G$ 在同态下的像是 $G'$ 的子群.

\begin{theorem}
	如果同态的核是平凡群(即, $\ker f = \{e\}$), 那么这个同态是单的.
\end{theorem}
\begin{proof}
	如果$\exists g_1, g_2 \in G$, s.t.\ $f(g_1) = f(g_2)$, 
	那么
	\begin{equation*}
		f(g_1 * g_2^{-1}) 
		= f(g_1) \circ f(g_2^{-1}) 
		= f(g_1) \circ f(g_2)^{-1} \circ f(g_2) \circ f(g_2^{-1})
		= e' \circ f(e)
		= e'
	\end{equation*}
	从而$g_1 * g_2^{-1} \in \ker f$, 同理$g_2^{-1} * g_1 \in \ker f$, 即$g_1^{-1} = g_2^{-1}$ 或 $g_1 = g_2$, 即: $f$是单的.
\end{proof}

作为例子, 映射
\begin{equation*}
	f \colon G \to \Inn(G);\, g \mapsto I_g
\end{equation*}
满足同构的条件~\ref{item:保结构}, 因$f(a) \circ f(b) = I_{ab} = f(ab)$; 但它不一定是双射, 因而是一个同态.

\section{环}

\begin{definition}[环]
	集合$R$非空, 其上定义了加法 $+$ 和乘法 $\cdot$, 且满足:
	\begin{conditionlist}[label=R\arabic*)]
		\item $(R, +, 0)$ 是阿贝尔群;
		\item $(R, \cdot)$ 是半群; 
		\item 乘法对加法有\indexbf{分配律}: 
		\begin{equation*}
			(a + b) c = ac + bc, \qquad
			c (a + b) = ca + cb
		\end{equation*}
		对$\forall a,b,c \in R$成立.
	\end{conditionlist}
	那么, 我们称 $(R, +, \cdot)$ 是一个\indexbf{环} (ring)\footnote{%
		如果$(R,\cdot)$不结合, 通常称\indexbf{非结合环}.}.
	而且唤$(R, +)$作其加法群, 称$(R, \cdot)$为其乘法半群. 倘若 $(R, \cdot)$ 还有单位元 $1$, 那么我们称$(R, +, \cdot)$ 为有单位元的环.
\end{definition}

若环 $R$ 非空的子集 $L$ 满足 
\begin{equation*}
	\forall x, y \in L \big(
		x - y \in L \; \wedge \; xy \in L
	\big)\,,
\end{equation*}
则称 $L$ 是 $R$ 的一个\indexbf{子环}.

若环的乘法半群是交换的, 则称这个环是一个\indexbf{交换环}.

作为例子, $(\mathbb Z, +, \cdot)$ 是我们熟悉的\indexbf{整数环}, $n\mathbb Z := \{nk \mid k \in \mathbb Z\}$ 是它的一个子环 ($n \in \mathbb Z$). 
交换环 $R$ 上的所有 $n$ 阶方阵之集合 $M_n(R)$ 也是环.

\begin{definition}[同态]
	设 $R$ 和 $R'$ 是两个环, 有一个映射 $f$ 对加法群和乘法半群都是同态 (保持运算), 即:
	\begin{equation*}
		f(x)f(y) = f(xy),
		\quad
		f(x) + f(y) = f(x + y),
	\end{equation*}
	那么, 我们称其为 $R$ 到 $R'$ 的一个\indexbf{同态}或\indexbf{态射}, 集合 $\ker f := \{a \in R \mid f(a) = 0\}$ 称为同态的\indexbf{核}. 同态 $f$ 的核是 $R$ 的子环. 类似地我们也有\indexbf{单同态}, \indexbf{满同态}和\indexbf{同构}的概念. 两个环同构记为 $R \cong R'$.
\end{definition}

设 $(R,+ ,\cdot)$ 是环, $X$ 是一个集合, 在 $R^X$ 上定义加法和乘法:
\begin{equation*}
	f + g \colon x \mapsto f(x) + g(x);
	\qquad
	fg \colon x \mapsto f(x) g(x),
\end{equation*}
就得到了\indexbf{函数环} $(R^X, +, \cdot)$, 其零元是 $0_X \colon x \mapsto 0$. 如果 $R$ 有单位元 $1$,那么 $R^X$ 也有单位元 $1_X \colon x \mapsto 1$, $\forall x \in X $.

作为例子, 考虑到将$\indexfm[k n]{\lbrack k\rbrack _n} \in \mathbb Z / \equiv \bmod n $ 映射到 $n^\mathbb Z \ni \bmod n := \{(m, k) \in \mathbb Z \times m \mid n \equiv k \bmod n\}$ 的同构, 
\indexbf{模$n$的剩余类环}\index{剩余类环} $(\mathbb Z_n, +, \cdot)$ 即可看作函数环 $n^\mathbb Z$ 的一个交换子环, 其中$\indexfm[Z n]{\mathbb Z_n} := \{\lbrack k\rbrack_n \mid k \in n\}$. 
同构关系让我们也能用剩余类的代表元组成的集合 $n$ 代替剩余类本身进行运算, 这种情况下, $n$ 称为\indexbf{模 $n$ 的剩余类的导出集}, 我们能用加法表和乘法表给出它的代数结构.

\begin{definition}[整环]
	环 $R$ 中, $a \in R$, 如果 $\exists b \in R - \{0\}$ s.t.\ $ab = 0$, 则称 $a$ 为环 $R$ 的一个零因子; 类似则可定义\indexbf{右零因子}\footnote{\cite{kostrikin1982introduction}中把 $0$ 排除在外了.}. 
	左零因子和右零因子统称\indexbf{零因子}. 零元 $0$ 则称为\indexbf{平凡零因子}. 
	
	若非平凡的交换环 $R$ 带单位元$1 \neq 0$, 且没有非平凡零因子, 则称 $R$ 是一个\indexbf{整环} (entire ring 或 integral domain).
\end{definition}

也有将无非平凡左零因子的带单位的非平凡环称为 \indexbf{domain} 的.

\begin{theorem}[消去律] \label{theorem:消去律}
	设 $R$ 是带单位元 $1 \neq 0$ 的交换环.
	环 $R$ 是整环 $\IFF$ $\forall x,y,c \in R$, $cx = cy \wedge c\neq 0  \;\to\; x = y$.
\end{theorem}
\begin{proof}
	如果 $R$ 满足消去律, 那么 $ab = 0 = 0b = a0$ 将给出 $a = 0 \vee b = 0$的论断;
	如果 $R$ 是整环, 那么 $cx = cy$ 即 $c(x - y) = 0$ 将得出 $c = 0 \vee x = y $; 
	倘若 $c \neq 0$, 那么这就是消去律.
\end{proof}

有单位元的环 $R$ 中元素 $x$ 的可逆性往往指关于乘法的可逆性. 

\begin{theorem}
	设 $R$ 是带单位元 $1$ 的环, $U(R) := \{x \in R \mid \text{$x$ 可逆}\}$ 是一个乘法群.
\end{theorem}
\begin{proof}
	单位元 $1$ 当然可逆. 由定义可逆元素的逆也是可逆的. 如果 $x, y \in R$ 可逆, 那么 
	\begin{equation*}
		(xy)(y^{-1} x^{-1}) = x(y y^{-1})x^{-1} = xx^{-1} = 1 =y^{-1} x^{-1} x y = (xy)^{-1} (xy),
	\end{equation*}
	即 $xy$ 可逆.
\end{proof}

如果 $U(R) = R - \{0\}$, 那么我们称 $R$ 是一个\indexbf{除环} (division ring), 也称\indexbf{斜域}或\indexbf{反对称域} (skew field). 除环没有零因子.

\section{域}

交换除环 $F$ 称为\indexbf{域} (field). 群 $P^* = U(P)$ 称为域的乘法群. 如果 $y \neq 0$, 那么我们通常记 $x/y = \frac x y := xy^{-1}$. 

我们可类似环, 定义同构和自同构. 同态的意义不大, 因为如果 $F$ 到 $F'$ 的同态 $f$ 的核 $\ker f \neq \{0\}$, 那么 $\ker f = F$. 如果 $F'$ 是域 $F$ 的子环, 而且也是一个域, 则称其为 $F$ 的一个\indexbf{子域}, 反之称 $F$ 为 $F'$ 的一个\indexbf{扩域}.

类似群的生成, 包含 $F \cup\{a\}$ 的最小 $F$ 的扩域, 记为 $F(a)$. 如有理数域 $\mathbb Q$ 的扩域 $\mathbb Q(\sqrt 2)$.

\begin{theorem}\label{theorem:素剩余类环}
	有限剩余类环 $\mathbb Z_p$ 是域, 当且仅当 $p$ 是素数.
\end{theorem}
\begin{proof}
	记 $\mathbb Z_p$ 的元素为 $[0]$, $[1]$, \ldots, $[p -1]$.
	由素数的定义, $\forall [k]  \in \mathbb Z_p^* := \mathbb Z_p - \{[0]\}$,
	\begin{equation*}
		[k], [2k], \cdots, [(p-1) k]
	\end{equation*}
	都不为 $[0]$, 而且两两不等.
	进而, $\exists i \in \mathbb N_+$ s.t.\ $i < p \wedge [ik] = 1$.
	又 $\mathbb Z_p$ 是交换环, 可知这个 $[i] = [k]^{-1}$, 即 $\mathbb Z_p$ 的乘法组成一个群.
\end{proof}

出于 $\mathbb Z_p$ 的这个性质, 我们也记其为 $\mathbb F_p$ 或 $\GF(p)$. $p^n$元有限域 $\GF(p^n)$ 也是存在的.

\begin{corollary}[\indexbf{Fermat小定理}]
	设 $p$ 是素数, $a \in \mathbb N$ 且 $a \nmid p$. 
	\begin{equation*}
		a^{p-1} \equiv 1 \pmod p\,.
	\end{equation*}
\end{corollary}
\begin{proof}
	当 $[k] \in \mathbb Z_p^*$ 时, $I_{[k]} \colon \mathbb Z_p^* \to \mathbb Z_p^*;\, [n] \mapsto [kn]$ 如定理~\ref{theorem:素剩余类环} 是 $S(\mathbb Z_p^*)$ 的元素.
	从而:
	\begin{equation*}
		\left( \prod_{k=1}^{p-1} [k] \right) [a]^{p-1} = \prod_{k=1}^{p-1} [k]\,.  
	\end{equation*}
	因为域都是整环, 满足消去律~\ref{theorem:消去律}, 从而 $[a]^{p-1} = [1]$.
\end{proof}

\begin{definition}[素域]
	若域 $P$ 不含任何非平凡真子域, 则称其为\indexbf{素域} (prime field).
\end{definition}

\begin{lemma}
	$\mathbb Q$ 和 $\mathbb Z_p$ 是素域. 
\end{lemma}
\begin{proof}
	让集合 $\{0,1\}$ 对加法, 减法, 乘法和除法封闭, 我们将得到 $\mathbb Q$ 或 $\mathbb Z_p$ 的导出集 $p$, 取决于 $1$ 在加法群中的阶数.
\end{proof}


\begin{theorem}
	任意非平凡域 $F$ 必含且只含一个素子域 $P$, 而且它将同构于 $\mathbb Q$ 或 $\mathbb Z_p$, 其中 $p$ 是素数. 
\end{theorem}
\begin{proof}
	若有两个素子域, 它们的交必然也是 $F$ 的子域, 根据素域的定义, 这个交不可能是真子域, 从而这两个素域相等. 这就保证了, 如果存在这么一个素子域 $P$, 它一定是唯一的. 接下来我们研究它的存在性.

	定义 $\mathbb Z$ 到 $F$ 的同态 $f(n) = ne$, 其中 $e$ 是 $F$ 的单位元. 
	其核为 $\ker f = m \mathbb Z$, 其中 $m \in \mathbb N$.
	
	如果 $m = 0$, 那么 $ne \neq o$, 其中 $o$ 是 $F$ 的零元, 只要 $n \neq 0$. 考虑 $f$ 在 $\mathbb Q$ 上的扩张, 可以证明 $P := f(\mathbb Q) = \{ne \mid n \in \mathbb Z\}$ 即构成了与 $\mathbb Q$ 同构的素子域.

	如果 $m \neq 0$, 那么 $m =p$ 是素数. 
	如果 $m$ 不是素数, 假设它有两个 ($m$ 和 $1$ 以外的) 因数 $a, b$, $ab e = o$ 意味着 $a e = o$ 或 $b e = o$ (定理~\ref{theorem:消去律}), 将与 $\ker f = m\mathbb Z$ 矛盾.
	考虑 $f$ 在 $p$ (作为 $\mathbb Z_p$ 的导出集) 上的限制, $P := \{o, e, 2e, \cdots, (p-1)e\}$ 即构成了与 $\mathbb Z_p$ 同构的素子域.
\end{proof}

在刚才的证明中, 我们已经遭遇了:
\begin{definition}[特征]
	设域 $F$ 的单位元和零元分别是$e$, $o$. 若存在$p \in \mathbb N$ 使得 $pe = o$, 则称 $p$ 为域的\indexbf{特征} (characteristic), 记为$\indexfm[char F]{\characteristic(F)} = p$; 特别地, 定义$\characteristic(F) = 0$, 如果不存在这样的 $p$.
\end{definition}

\chapter{线性空间}
\section{线性空间}
\begin{definition}[线性空间]
	设 $\mathbb F$ 是一个域, $(V, +, \bv 0)$ 是一个Abelian群. 
	如果定义标量乘积运算: $\mathbb F \times V \to V ;\, (\lambda, \bv x) \mapsto \lambda \bv x$ 且满足:
	\begin{conditionlist}
		\item $1\bv x = \bv x$, $\forall \bv x \in V$ (\indexbf{酉性});
		\item $(\alpha \beta) \bv x = \alpha (\beta \bv x)$, $\forall \alpha, \beta \in \mathbb F$, $\forall \bv x \in V$;
		\item $(\alpha + \beta) \bv x = \alpha \bv x + \beta \bv x$, $\forall \alpha, \beta \in \mathbb F$, $\forall \bv x \in V$;
		\item $\lambda (\bv x + \bv y ) = \lambda \bv x + \lambda \bv y$,
	\end{conditionlist}
	那么, 我们称 $V$ 是 $\mathbb F$ 上的一个\indexbf{线性空间}, 或称\indexbf{向量空间}, 其元素称为\indexbf{向量}, 相对而言 $\mathbb F$ 的元素则被称为\indexbf{纯量乘积}.
\end{definition}

通常我们称 $(\bv x_i)_{i \in I}$ 为\indexbf{向量组},  $I$ 是指标集.

\begin{definition}[线性组合]
	设 $V$ 是 $\mathbb F$ 上的线性空间. 倘若$\forall i \in n$, $\lambda_i \in \mathbb F$, $\bv x_i \in V$, $n$ 是正整数, 那么
	\begin{equation*}
		\sum_{i \in n} \lambda_i \bv x_i
	\end{equation*}
	称为向量组 $(\bv x_i)_{i \in n}$ 的一个系数为 $(\lambda_i)_{i \in n}$ 的\indexbf{线性组合}, $i \in n$.
\end{definition}

可数向量甚至不可数个向量之和的研究, 将在泛函分析中得到更加细致的讨论.

\begin{definition}[线性包络]
	设 $V$ 是 $\mathbb F$ 上的线性空间, $(\bv x_i)_{i\in n}$ 是其中的一个向量组, $n$ 是正整数. 
	其\indexbf{线性包络} (linear span)定义为 
	\begin{equation*}
		\langle \bv x_i\rangle_{i \in n}
		= \left\{ 
			\sum_{i \in n} \lambda_i \bv x_i 
		\middle|
			(\lambda_i)_{i \in n} \in \mathbb F^n
		\right\}\,.
	\end{equation*}
	或者, 设 $M \subset V$, 那么其线性包络定义为
	\begin{equation*}
		\langle M \rangle 
		= \left\{ 
			\sum_{i \in n} \lambda_i \bv x_i
		\middle|
			n \in \mathbb N,\, \forall i \in n (\lambda_i \in \mathbb F\, \wedge \, \bv x_i \in M)  
		\right\}\,.
	\end{equation*}
\end{definition}

\begin{definition}
	设 $V'$ 是 $\mathbb F$ 上的线性空间 $V$ 的加法子群, 且对标量乘积封闭, i.e.\ $\forall \bv x \in V'$, $\forall \lambda \in \mathbb F$, $\lambda \bv x \in V'$, 那么, 我们称 $V'$ 是 $V$ 的一个 (线性) \indexbf{子空间}.
\end{definition}

显然 $\langle M \rangle$ 对 $\forall M \in 2^V$ 都是 $V$ 的子空间 (而且是包含 $M$ 的最小的那个), 从而我们也说这种情况下 $\langle M \rangle$ 是 $M$ \indexbf{张出} (span) 或\indexbf{生成}的线性空间.

\begin{definition}[线性相关]
	设 $V$ 是 $\mathbb F$ 上的线性空间, 其中有线性组 $(\bv x_i)_{i \in n}$. 
	若 $\exists (\alpha_i)_{i\in n} \in \mathbb F^n$ s.t.\ $\exists i \in n (\alpha_i \neq 0)$ 且
	\begin{equation*}
		\sum_{i \in n} \alpha_i \bv x_i = \bv 0\,,
	\end{equation*}
	那么称向量组 $(\bv x_i)_{i \in n}$ 是\indexbf{线性相关}的. 
	反之则称它们\indexbf{线性无关}或\indexbf{线性独立}.
\end{definition}

\begin{theorem}
	若 $(\bv x_i)_{i \in n}$ 是线性相关的, 则 $\exists i \in n$ s.t.\ 
	\begin{equation*}
		\exists (\beta_j)_{j \in n - \{i\}} \in 2^\mathbb F 
		\quad\text{s.t.\ }\quad 
			\bv x_i = \sum_{j \in n - \{i\}} \beta_j \bv x_j
		\,.
	\end{equation*}
\end{theorem}
\begin{proof}
	证明此定理只需取 $i$ 使得见证线性相关的线性组合中 $\bv x_i$ 的系数不为 $0$ 即可.
\end{proof}

\begin{definition}[维数]
	设 $V$ 是 $\mathbb F$ 上的线性空间. 若 $\exists n \in \mathbb N$, 满足
	\begin{equation*}
		n = \max \{r \mid \exists (x_i)_{i \in r} \text{\ s.t.\ 它们是线性独立的}) \}\,,
	\end{equation*}
	那么称 $n$ 是 $V$ 的\indexbf{维数}, 记为 $\dim V = n$, $V$ 是 \textbf{$n$ 维线性空间}. 倘若不存在这样的 $n$, 则 $V$ 是\indexbf{无穷维线性空间}. 
\end{definition}

特别地, $\dim \{\bv 0\} = 0$.

\begin{definition}[基底]
	设 $V$ 是 $\mathbb F$ 上的 $n$ 线性空间, $(\basis e_i)_{i \in n}$ 倘若线性无关, 则称其为 $V$ 的一组\indexbf{基底}. 特别地, 如果 $\dim V = 0$, 空集 $\varnothing$ 是它的一组基底.
\end{definition}

\begin{theorem}[唯一分解]
	设 $V$ 是 $\mathbb F$ 上的 $n$ 线性空间, $(\basis e_i)_{i \in n}$ 是其一组基底.
	那么 $\forall \bv v \in V$, $\exists ! (v_i)_{i \in n}$, s.t.\ 
	\begin{equation*}
		\bv v = \sum_{i \in n} v_i \basis e_i.
	\end{equation*}
\end{theorem}
\begin{proof}
	唯一性只需要假定有两组分解, 相减并利用基底的线性独立性即可证明. 
	下面只证存在性:
	根据维数的定义, $(\bv v, \basis e_0, \cdots, \basis e_{n - 1})$ 线性相关, 从而
	$\exists \alpha \in \mathbb F\, \exists (\alpha_i)_{i \in n} \in \mathbb F^n$ s.t.\ $(\alpha, \alpha_0, \cdots, \alpha_{n - 1})$ 不全为 $0$ 且
	\begin{equation*}
		\alpha \bv v + \sum_{i \in n} \alpha_i \basis e_i = \bv 0\,,
	\end{equation*}
	考虑到基底的线性独立性, $\alpha \neq 0$, 由域的可逆性, 我们得出了一组线性组合系数 $( - \alpha_i / \alpha)_{i \in n}$\,.
\end{proof}

根据这个定理, 我们断言线性空间 $V$ 的基底 $(\basis e_i)_{i \in n}$ 张出 $V$ 本身, 
i.e.\ $V = \langle\basis e_i\rangle_{i \in n}$. 

\begin{corollary}
	设 $V'$ 是 $V$ 的子空间. 如果 $V' \subsetneq V$, 那么 $\dim V' < \dim V$.
\end{corollary}

\begin{corollary}
	如果线性无关的向量组 $(\bv e_j)_{j \in n} $ 满足 $\forall j \in n$, $\bv e_j \in  \langle\bv f_i\rangle_{i \in m}$, 那么 $n \leq m$.
\end{corollary}

\begin{theorem}[\indexbf{Steintz 替换}]
	设 $V$ 是 $\mathbb F$ 上的 $n$ 线性空间, $(\basis e_i)_{i \in n}$ 是其一组基底.
	任意线性无关组 $(\basis f_i)_{i \in s}$, 都可从基底中取出 $(\basis e_{i_k})_{i_k \in n,\, k \in t}$ 使得
	\begin{equation*}
		(\basis f_0, \cdots, \basis f_{s-1}, \basis e_{i_0}, \cdots, \basis e_{i_{t-1}})
	\end{equation*}
	是 $V$ 的一组基底.
\end{theorem}
\begin{proof}
	取 $i_0$ 使得 $\basis e_{i_0} \notin \langle\basis f_i\rangle_{i \in s}$; 
	接着取 $i_{k+1}$ 使得 $\basis e_{i_{k+1}} \notin \langle \basis f_0, \cdots, \basis f_{s-1}, \basis e_{i_k} \rangle $, 直到不能进行下去, 剩下的基底全部都可由前面的向量组线性表出, 令此时 $k = t - 1$.
	从而: $V$ 中任何向量都可由基底 $(\basis e_i)_{i \in n}$ 表出, 从而也就可以由
	$
		(\basis f_0, \cdots, \basis f_{s-1}, \basis e_{i_0}, \cdots, \basis e_{i_{t-1}})
	$ 表出, 从而 $s + t \geq n$. 

	另一方面, 不难通过归纳得知, 
	$
		(\basis f_0, \cdots, \basis f_{s-1}, \basis e_{i_0}, \cdots, \basis e_{i_{t-1}})
	$ 是线性无关的, 由维数的定义, 我们断言 $t + s \leq n$. 
	即 $t + s = n$, 我们已然得到 $V$ 的一组基底了.
\end{proof}


	

\chapter{线性算子}

\chapter{内积空间}

\chapter{张量}


\appendix
\chapter{复数与多项式}
\section{复数}

\section{多项式}


\backmatter
\nocite{*} % 这个表示列出所有没有在文中被引用的参考文献
\printbibliography[heading=bibliography, title={参考文献}]

\indexprologue{这里列出了笔记中出现的重要符号.}
\printindex[symbol]


\printindex
\end{document}