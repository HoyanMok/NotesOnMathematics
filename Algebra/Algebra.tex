% TeXplates/Mathematics.tex
% v0.1.7
% https://github.com/HoyanMok/TeXplates
\documentclass[openany]{ctexbook} 
% \documentclass{ctexbook} 如果用中文
% \documentclass[10pt,a4paper]{ctexart}  字体大小和纸张大小,默认分别为10pt和letterpaper
% 五号 = 10.5pt,小四=12pt,四号=14pt
% 其他可选参量如twocolumn, 两行排版

\usepackage{xpatch}

\ExplSyntaxOn
\xpatchcmd \fontspec_new_script:nn
	{ \__fontspec_warning:nxx }
	{ \__fontspec_info:nxx }
	{}{\fail}
\ExplSyntaxOff % 沉默字体警告

% 将PATH换成绝对路径 (Windows) 或相对路径 (Mac OS或Linux)
% 使用「/」而不是「\」
\newcommand{\PATH}{./}

\usepackage{biblatex} %[style=gb7714-2015]{biblatex} 可以选择样式
\addbibresource{Algebra.bib} % 把这里改成实际的文件名

% 令参考资料能够加入目录中:
\defbibheading{bibliography}[\bibname]{% 
	% \addcontentsline{toc}{chapter}{参考文献}
	\chapter{#1}% 
	\markboth{#1}{#1}}

\usepackage[notbib, notindex]{tocbibind}

\usepackage{imakeidx} %索引
\makeindex[intoc, title={索引}]
\makeindex[intoc, name=symbol, title={符号列表}]
\newcommand*{\indexbf}[1]{\emph{\textbf{#1}}\index{#1}} % Index for definition
\newcommand*{\indexfm}[2][\ ]{#2\index[symbol]{#1@$#2$}} % Used Symbol
% \indexfm[name for sort]{display} 


% 对目录项等的修改
\usepackage{chngcntr}
	\counterwithout{section}{chapter} % So that the section won't reset when newing a chapter
\renewcommand{\thesection}{\textmd{\S}\arabic{section}}
\renewcommand{\thesubsection}{\arabic{section}.\arabic{subsection}}

% 引用的宏包:
% 宏包的使用, 可以在命令行运行texdoc <宏包名>获得文档
\usepackage{multicol} % 分栏 (全局分栏建议在文档类处设置)
\usepackage{amsmath} % AMS数学标准
	\makeatletter % '@' now normal "letter"
	\@addtoreset{equation}{section} % 每次换section就把equation清零
	\makeatother  % '@' is restored as "non-letter"
	\renewcommand\theequation{\oldstylenums{\arabic{section}}%
					-\oldstylenums{\arabic{equation}}} % 显示为section数-equation数
\usepackage{amssymb} % 数学符号
\usepackage{mathrsfs} % 花体
\usepackage{amsthm} %定义、证明、定理等
	\theoremstyle{plain}
		\newtheorem{axion}{Axion} %公理
		\newtheorem{theorem}{Theorem}[section] %定理
		\newtheorem{corollary}{Corollary} %推论
		\newtheorem{lemma}{Lemma} %引理
	\theoremstyle{definition}
		\newtheorem{definition}{Definition}[section] %定义
		\newtheorem{proposition}{Proposition} %命题
	\renewcommand{\proofname}{\textbf{Proof}}

\renewcommand{\thetheorem}{%
	\arabic{section}.\arabic{theorem}%
} % 公式编号不显示`\S`
\renewcommand{\thedefinition}{%
	\arabic{section}.\arabic{definition}%
} % 公式编号不显示`\S`

\usepackage{esint} % 积分
\usepackage{siunitx} % 标准SI数值和单位处理

\usepackage{tikz} % 绘图
\usepackage{float} % 浮动体 (供图片, 表格等) 扩展, 主要用于提供h模式
\usepackage{graphicx} % 插入图片
\usepackage{titlepic}
\usepackage[font=small, skip=5pt]{caption} % 缩小题注字体和题注与图片距离
\usepackage{subcaption} % 子图和子图的题注
\usepackage{svg} % svg位图
\usepackage{wrapfig} % 简单的图文绕排
\usepackage[inline]{enumitem} % 编号
	% 新列表:
	\newlist{conditionlist}{enumerate}{2}
	\setlist[conditionlist,1]{topsep = 0pt, itemsep = 0pt, parsep = 0pt,%
		label=\arabic*), leftmargin=2\parindent}
	\setlist[conditionlist,2]{topsep = 0pt, itemsep = 0pt, parsep = 0pt,%
		label=\alph*), leftmargin=3\parindent}
\usepackage{geometry} % 调整页边距
% \geometry{left=1.6cm,right=1.6cm}
\usepackage{xcolor} % 颜色
\usepackage[colorlinks=true,bookmarks=true]{hyperref} % 引用, 交叉引用, 图表等的链接; 生成书签
\hypersetup{linkcolor=[rgb]{1,0.27,0},bookmarksopen = true}% 更多设置请查阅: texdoc hyperref


% 定义一些笔者常用的指令:
\newcommand{\me}{\mathrm{e}} % 自然对数的底
\newcommand{\mi}{\mathrm{i}} % 虚数单位
\newcommand{\dif}{\mathop{}\!\mathrm{d}} % 微分算子d
\newcommand*{\basis}[1]{\hat{\boldsymbol{#1}}} % 基底
\newcommand*{\bv}{\boldsymbol} % 向量加粗
\newcommand*{\id}{\mathrm{id}} % 单位映射
\newcommand*{\IFF}{\;\leftrightarrow\;} % 充要条件

\newcommand*{\diff}[3][1]
{\if#11%
	\frac{\mathrm{d} #2}{\mathrm{d} #3}% 导数\diff{y}{x}
\else%
	\frac{\mathrm{d}^{#1} #2}{\mathrm{d} #3^{#1}}% n阶导数\diff[n]{y}{x}
\fi}
\newcommand*{\pdiff}[3][1]
{\if#11%
	\frac{\partial #2}{\partial #3}% 偏导数\pdiff{y}{x}
\else%
	\frac{\partial^{#1} #2}{\partial #3^{#1}}% n阶偏导数\pdiff[n]{y}{x}
\fi}

\newcommand*{\inbasis}[2]{\left.%
	{#1}\right|_{#2}
}

\newcommand{\emphbf}[1]{\emph{\textbf{#1}}}
% \indexbf 的定义见前imakeidx的引用下

% 笔者习惯的运算符:
\DeclareMathOperator{\tg}{tg}
\DeclareMathOperator{\ctg}{ctg}
\DeclareMathOperator{\arctg}{arctg}
\DeclareMathOperator{\sh}{sh}
\DeclareMathOperator{\ch}{ch}
\DeclareMathOperator{\dom}{dom}
\DeclareMathOperator{\ran}{ran}
\DeclareMathOperator{\interior}{int}
\DeclareMathOperator{\card}{card}
\DeclareMathOperator{\rank}{rank}

\DeclareMathOperator{\Aut}{Aut}
\DeclareMathOperator{\Inn}{Inn}
\DeclareMathOperator{\characteristic}{char}
\DeclareMathOperator{\codim}{codim}
\DeclareMathOperator{\Hom}{Hom}
\DeclareMathOperator{\End}{End}
\DeclareMathOperator{\diag}{diag}
\DeclareMathOperator{\tr}{tr}
% \DeclareMathOperator*{\指令}{显示} 
% 带星号的版本会像\lim一样

% 一些符号:
\newcommand*{\GL}{\mathrm{SL}}
\newcommand*{\Orth}{\mathrm{O}}
\newcommand*{\SO}{\mathrm{SO}}
\newcommand*{\GF}{\mathrm{GF}}
\newcommand*{\Cl}{\mathrm{Cl}}






% 文章标题页信息:
\title{Algebra}
\author{ Hoyan Mok\thanks{E-mail: victoriesmo@hotmail.com}
	}
\date{\today} % 自动生成日期
\titlepic{\includegraphics[width=8 cm]{\PATH Rubik's_cube.pdf}}

\begin{document}
\pagenumbering{Alph}
\maketitle % 打印标题
\thispagestyle{empty}
\frontmatter
\chapter{笔记说明}
本笔记是笔者学习线性代数时的教材, 主要参考资料是\cite{代数学引论第二卷}. 

笔记假定读者已经熟悉朴素集合论的术语与符号, 并已经学习了以矩阵和行列式运算为主的初级线性代数. 
但本笔记力求自足, 将矩阵与行列式运算, 置换和多项式等内容附在附录中, 以资读者在阅读正文时可以随时查阅.

笔记后附有符号列表和索引, 方便读者 (也是方便笔者自己) 查阅.

你可以在\url{https://github.com/HoyanMok/NotesOnMathematics/tree/master/Algebra}获得本笔记最新的PDF与\TeX{}源文档. 
封面的来源是\url{https://commons.wikimedia.org/wiki/File:Rubik%27s_cube.svg}.

\tableofcontents

\mainmatter
\part{线性代数}
\chapter{群. 环. 域}
\section{代数运算}
\begin{definition}[二元运算]
	集合的Cartesian平方到自身的映射$* \colon X^2 \to X$称为其上的一个\indexbf{二元运算}.
	通常我们记$*(a,b) := a * b$. 
	当$X$上定义了二元运算$*$后, 称$*$定义了$X$上的一种\indexbf{代数结构} $\indexfm[X ast]{(X,*)}$, 也称\indexbf{代数系统}. 
\end{definition}

当指代是明确的时候, 我们将混用集合及其代数结构.

作为习惯, 如果$\cdot, + \in X^{X^2}$, 我们记$ab := a \cdot b$并称其为$a$和$b$的\indexbf{积}, 称$a+b$为$a$和$b$的\indexbf{和}. 这些只是约定.

若$a* b = b* a$则称$*$或$(X,*)$是\indexbf{交换的}, 而若$(a* b)* c = a*(b* c)$则称$*$或$(X,*)$为\indexbf{结合的}. 

若$\exists e\in X$满足$\forall x\in A\big(
	e* x = x * e = x
\big)$, 则称其为$*$的一个\indexbf{单位元} (identity), 这时可把$(X,*)$记作$\indexfm[X ast e]{(X,*, e)}$. 可以证明一个代数结构最多只有一个单位元. 
乘法单位元通常记为$1$, 而加法单位元 (也叫\indexbf{零元}) 记为$0$.

\begin{definition}[半群和幺半群]
	若$*$是结合的, 称$(X,*)$是\indexbf{半群} (semigroup); 
	若$*$还有一个单位元, 则称$(X,*, e)$是\indexbf{幺半群} (monoid).
\end{definition}

倘若幺半群$(M, *, e)$是有限的 (即其元素有限), 称$\card M$为\indexbf{有限幺半群}的\indexbf{阶}.

作为重要的例子, \indexbf{置换幺半群} 定义为$(X^X, \circ, \id_X)$, 有幺半群结构的$X^X$通常记作$M(X)$.

半群中, 括号的位置是不重要的 (可用数学归纳法证明). 通常我们记$x_1x_2 \cdots x_n$为:
\begin{equation}
	\prod_{i=1}^1 x_i = x_1,\;\prod_{i=1}^{n+1} x_i = \left( \prod_{i=1}^n x_i  \right) x_n\,;
\end{equation}
同理$x_1+x_2+\cdots + x_n$为:
\begin{equation}
	\sum_{i=1}^1 x_i = x_1,\;\sum_{i=1}^{n+1} x_i = \left( \sum_{i=1}^n x_i  \right) + x_n\,.
\end{equation}
在半群不交换的场合, 指出递推式右端的顺序是重要的. 这种记法称为\indexbf{左正规}.

若$x := x_1 = x_2 = \cdots = x_n$, 记$\sum_{i=1}^n x_i = nx$, $\prod_{i=1}^n x_i = x^n$, 分别表示$x$的$n$倍和$x$的$n$次幂. 它们满足:
\begin{equation}
	nx+mx = (n+m)x, \; n(m x) = nm x, \qquad n,m\in \mathbb N_+\,;
\end{equation}
\begin{equation}\label{exoponentiation}
	x^n x^m = x^{n+m}, \; (x^m)^n = x^{nm}, \qquad n,m \in \mathbb N_+\,.
\end{equation}

在幺半群中, 还可以令$x^0 = 1$, $0x = 0$.

若半群$S$有子集$S'$, 使得$(S',*)$是半群, 那么称其为半群$(S,*)$的\indexbf{子半群}.
同理有幺半群\nolinebreak$M$的\indexbf{子幺半群}$M'$. 

若半群$(S,*, e)$的元素$a$满足$\exists a'\in S\big(
	a a' = a' a = e
\big)$, 那么称$a$为\indexbf{可逆的} (invertible), $a'$称为其\indexbf{逆元} (inverse element) 或\indexbf{逆} (inverse).
通常加法逆元记为$- a$, 乘法逆元记为$a^{-1}$, 且为可逆元素引入$n a$, $a^n$的概念, 其中$n \in \mathbb Z$. 当$n$为负数时, $na = -(-na)$, $a^n = (a^{-n})^{-1}$.

因为群未必是Abelian, 我们可以也用弱化的\indexbf{左可逆} $\exists y$ s.t.\ $y * x = 1$或\indexbf{右可逆}的概念.


\section{群}\label{section:群}
可逆幺半群$G$称为群, 即:
\begin{definition}[群]
	设有集合$G$. 若:
	\begin{conditionlist}[label=G\arabic*)]\setcounter{enumi}{-1}
		\item 定义了二元运算$\mathord{\cdot} \colon G^2 \to G; (x,y) \mapsto xy$.
		\item 结合性: $\forall x,y,z\in G$, $(xy)z = x(yz)$.
		\item 单位元: $\exists e\in G \forall x\in G$, $xe = ex = x$.
		\item 可逆性: $\forall x\in G \exists x^{-1} \in G$, $x x^{-1} = x^{-1} x = e$.
	\end{conditionlist}
	则称$(G, \cdot)$为\indexbf{群}.
\end{definition}

交换群又叫做\indexbf{Abelian群}. 

作为重要的例子, 设 $X$ 是一个集合, $\indexfm[S X]{S(X)} = \{ f \in X^X \mid \text{$f$ 是双射}\}$. 
我们断言, $(S(X), \circ, \id_X)$ 是一个群, 称为\indexbf{变换群}或\indexbf{置换群}, 其中 $\circ$ 是函数的复合, $\id_X$是恒等变换. 
当它的阶数 $\card X = n$ 是有限的时候, 记$\indexfm[S n]{S_n} := S(X)$.

群也有子群的概念. 
设$(G, \cdot, e)$是一个群. 当一个集合$G' \subset G$满足:
\begin{conditionlist}[label=SG\arabic*)]
	\item $e \in G'$;
	\item $\forall x,y\in G'$, $xy \in G'$;
	\item $x \in G' \to x^{-1} \in G'$,
\end{conditionlist}
则称$(G', \cdot , e)$是一个$G$的\indexbf{子群}.
倘若还有$G' \neq G$则称其为一个\indexbf{真子群}\footnote{\cite{kostrikin1982introduction}等文献把\indexbf{平凡群}$\{e\}$也排在真子群的定义外.}.

\begin{theorem}
	非空的 $G'$ 是群 $(G, \cdot, 1)$ 的子群 $\IFF$ $\forall x,y \in G' (xy^{-1} \in G')$.
\end{theorem}
\begin{proof} 
	根据子群的定义, $\to$ 是显然的, 下给出 $\gets$ 的证明:
	\begin{conditionlist}[label=SG\arabic*)]
		\item $\forall x \in G' (x x^{-1} = 1 \in G)$;
		\item $\forall x,y\in G'$, $x1^{-1} {1y^{-1}}^{-1} = xy \in G'$;
		\item $\forall x \in G'$, $1x^{-1} = x^{-1} \in G'$.
	\end{conditionlist}
\end{proof}

这里将不加证明地给出:
\begin{lemma}\label{theorem:子群族的交}
	群 $G$ 的子群族 $\mathscr H = \{H \mid \text{$H$ 是 $G$ 的子群}\}$ 的交 $\cap \mathscr H$ 也是 $G$ 的子群.
\end{lemma}

设 $G$ 有子集 $S$ , 我们说群 $(G, \cdot, 1)$ 是由 $S$ 生成的, 意思是说 $G$ 没有包含 $S$ 的真子群. 记为$G = \indexfm[S]{%
	\langle S\rangle}$.
\begin{theorem}
	$\langle S \rangle = \left\{
		\prod^{n-1}_{i=0} s_i \middle|
			\forall i \in n(s_i \in S \vee s_i^{-1} \in S)
\right\}$.
\end{theorem}
\begin{proof}
	根据群的定义, 形如 $\prod^{n-1}_{i=0} s_i$ 的将构成一个群. 如果存在一个不能写成这种形式的元素, 那么它们将构成一个真子群, 这和 $\langle S\rangle$的定义相违背. 
	
\end{proof}

我们把半群的公式~\eqref{exoponentiation} 推广到整数次幂, 证明在此忽略了.
\begin{theorem}\label{group exoponentiation}
	$\forall g \in G$, $\forall n,m\in \mathbb Z$,
	\begin{equation}
		g^m g^n = g^{m+n}, \quad
		(g^m)^n = g^{mn}.
	\end{equation}
\end{theorem}

\begin{definition}[循环群]
	设$(G, \cdot , 1)$是一个乘法群, $\exists g_0 \in G$, 使得$\forall g \in G$, $\exists n \in \mathbb Z$, $a^n = g$, 那么我们称它是一个\indexbf{循环群}, $g_0$是一个\indexbf{生成元} (generator), 并记作$G = \indexfm[g 0]{\langle g_0 \rangle}$.
\end{definition}

对于群 $G$ 中任意元素 $g$, 我们称 $\card \langle g \rangle$ 为元 $g$ 的\indexbf{阶数}, 或称 $g$ 为 \emphbf{$n$~阶元}%
	\index{n阶元@$n$~阶元}%
	. 而且它将满足:
\begin{theorem}
	任意群 $G$ 中若有 $q \in \mathbb Z$ 阶元 $g$, 则 $\langle g \rangle = \{e, g, \dots, g^{q-1}\}$, 且:
	\begin{equation}
		g^n = e \IFF n = kq, \qquad n \in \mathbb Z\,.
	\end{equation}
\end{theorem}
证明利用带余除法和定理~\ref{group exoponentiation}, 证明是显然的. 从该定理, 我们可以论断: 循环群都是Abelian群.

\begin{definition}[同构]
	两个群 $(G, *)$, $(G', \circ)$ 如若满足: $\exists f\colon G \to G'$ s.t.\ \begin{conditionlist}[label=\roman*)]
		\item $\forall a, b \in G$, $f(a * b) = f(a) \circ f(b)$; \label{item:保结构}
		\item $f$是双射,
	\end{conditionlist}
	则称 $f$ 是一个\indexbf{同构映射}或\indexbf{同构} (isomorphism), 并认为两个群是互相\indexbf{同构}的 (isomorphic), 记为$\indexfm[G simeq G prime]{G \simeq G'}$.
\end{definition}

同构关系的自反性, 传递性和对称性是平凡的.

\begin{theorem}
	设群 $(G, *, 1)$, $(G', \circ, 1')$ 被 $f$ 见证同构, 那么$f(1) = 1'$.
\end{theorem}
\begin{proof}
	$\forall g' \in G'$, 记 $g := f^{-1}(g')$, 那么 $f(g)\circ f(1) = f(g * 1) = g' = f(1 * g) = f(1) \circ f(g)$. 从而$f(1) = 1'$.
\end{proof}

\begin{theorem}
	设群 $(G, *, 1)$, $(G', \circ, 1')$ 被 $f$ 见证同构, 那么$\forall g \in G$, $f(g^{-1}) = f(g)^{-1}$.
\end{theorem}
\begin{proof}
	$f(g) \circ f(g^{-1}) = f(g * g^{-1}) = f(1) = 1' 
		= f(g^{-1} * g) = f(g^{-1})\circ f(g)$.
\end{proof}

\begin{theorem}
	\begin{equation*}
	\card \langle g_0\rangle = \card \langle g'_0 \rangle 
		\to \langle g_0\rangle \simeq \langle g'_0 \rangle\,.
	\end{equation*}
\end{theorem}
\begin{proof}
	倘若$\card \langle g_0\rangle = \infty$, 那么$\nexists n \in \mathbb Z - \{0\}$, s.t.\ $g_0^n = e$; 这意味着, 存在这样的双射 $f\colon \mathbb Z \to \langle g_0\rangle$, 满足$f(n) = g_0^n$, 见证了 $(\mathbb Z, +, 0) \simeq (\langle g_0\rangle , *, e)$. 

	如果阶数是有限的, 只需令$f\colon g^k \to g'^k$, 其中$k= 0$, $1$, $\cdots$, $\card\langle g_0\rangle$.
\end{proof}

\begin{theorem}[\indexbf{Cayley定理}]
	设 $(G, *, e)$ 任意 $n$ 阶有限群. 
	$\exists H \subset S_0$ s.t.\ $(H, \circ,\id_X)$是$S_n$的子群且 $G \simeq H$. 
\end{theorem}
\begin{proof}
	取$H := \{L_g \mid g \in G\}$, 其中$L_g \colon G \to G; g' \mapsto gg'$可以证明是双射. 那么 $L\colon G \to H; g \mapsto L_g$ 见证了 $H \simeq G$.
\end{proof}

若 $\varphi \colon G \to G$ 见证了 $G \simeq G$ (如 $\id_G$), 那么称 $\varphi$ 是群 $G$ 的一个 \indexbf{自同构} (automorphism). 所有自同构组成的集合 $\indexfm[Aut G]{\Aut(G)}$ 和其上的函数复合 $\circ$ 构成了 $S(G)$ 的一个子群, 称为$G$的\indexbf{自同构群}.

自同构群有一特殊的子群$\indexfm[Inn G]{\Inn(G)} := \{I_a \colon g \mapsto a g a^{-1} \mid a \in G\}$,
称为\indexbf{内自同构群} (inner isomorphism), 其元素称为\indexbf{共轭映射} (conjugation). 

\begin{definition}[共轭]
	设 $G$ 是一个群, $a, b \in G$. 如果 $\exists I_g \in \Inn(G)$, 使得 $I_g(a) = b$, 那么我们称 $a$ 和 $b$ 互为\indexbf{共轭} (conjugate).
\end{definition}

我们毫不费力地就能证明共轭关系是等价关系, 而且当 $G$ 是Abelian群的时候, 其任意元素的共轭都是其自身.

\begin{definition}[共轭类]
	设 $G$ 是一个群. 由共轭规定的等价类称为\indexbf{共轭类} (Conjugacy class), 记为 $\indexfm[Cl g]{\Cl(g)}$, $g$ 为其代表元. 
	称 $\card \{ \Cl(g) \mid g \in G\}$ 为 $G$ 的\indexbf{类数} (class number). 
	如果有一个函数 $f$ 满足 $g' \in \Cl(g) \,\to\, f(g) = g(g')$, 那么称 $f$ 是一个 \indexbf{类函数} (class function).
\end{definition}

\begin{definition}[正规子群]
	设 $G$ 是一个群, $N$ 是其子群. 倘若 $\forall I \in \Inn(G)$, $I(N) = N$, 即其在共轭映射下不变, 则称其为 $G$ 的一个\indexbf{正规子群} (normal subgroup), 记为 $\indexfm[N triangleleft G]{N \triangleleft G}$.
\end{definition}

可以看出Abelian群的所有子群都是正规子群. 
以下是正规子群的另一种定义方法:
\begin{theorem}
\begin{equation*}
	N \triangleleft G \IFF
		\forall g, h \in G\; (gh \in N \IFF hg \in N)\,.	
\end{equation*}
\end{theorem}
\begin{proof}
	只需注意到 $I_g(gh) = g^{-1} gh g = hg$.
\end{proof}


\begin{definition}[同态]
	设有群 $(G, *, e)$ 和 $(G', \circ, e')$, 映射 $f \colon G \to G'$ 若满足
	\begin{equation*}
		\forall a, b \in G, \quad 
			f(a * b) = f(a) \circ f(b),
	\end{equation*}
	则称其为群 $(G, *)$ 到群 $(G', \circ)$ 的一个\indexbf{同态} (homomorphism), 也叫\indexbf{态射} (morphism). 类似映射, 可定义\indexbf{单态射} (monomorphism), \indexbf{满态射} (epimorphism).

	集合 $\indexfm[ker f]{\ker f} := f^{-1}(\{e'\})$ 叫做同态 $f$ 的\indexbf{核} (kernel). 群到自身的同态映射称为\indexbf{自同态} (endomorphism).
\end{definition}

同态 $f$ 的核是 $G$ 的正规子群, 即$\ker f \triangleleft G$, 而 $G$ 在同态下的像是 $G'$ 的子群.

\begin{theorem}
	如果同态的核是平凡群(即, $\ker f = \{e\}$), 那么这个同态是单的.
\end{theorem}
\begin{proof}
	如果$\exists g_1, g_2 \in G$, s.t.\ $f(g_1) = f(g_2)$, 
	那么
	\begin{equation*}
		f(g_1 * g_2^{-1}) 
		= f(g_1) \circ f(g_2^{-1}) 
		= f(g_1) \circ f(g_2)^{-1} \circ f(g_2) \circ f(g_2^{-1})
		= e' \circ f(e)
		= e'
	\end{equation*}
	从而$g_1 * g_2^{-1} \in \ker f$, 同理$g_2^{-1} * g_1 \in \ker f$, 即$g_1^{-1} = g_2^{-1}$ 或 $g_1 = g_2$, 即: $f$是单的.
\end{proof}

作为例子, 映射
\begin{equation*}
	f \colon G \to \Inn(G);\, g \mapsto I_g
\end{equation*}
满足同构的条件~\ref{item:保结构}, 因$f(a) \circ f(b) = I_{ab} = f(ab)$; 但它不一定是双射, 因而是一个同态.

\begin{definition}[陪集]
	设 $(G, *, e)$ 是一个群, $S$ 是其子群, $g \in G$, 那么我们称 $\indexfm[g ast S]{g * S} := \{ g * s \mid s \in S\}$ 为 $S$ 在 $G$ 内的\indexbf{左陪集} (left coset); 
	同理 $\indexfm[S ast g]{S * g} := \{ s * g \mid s \in S\}$ 为 $S$ 在 $G$ 内的\indexbf{右陪集} (right coset). 
	这里我们称 $g$ 是一个代表元.
	如果 $g * S = S * g$, 则称其为\indexbf{陪集}.
\end{definition}

\begin{theorem}
	\begin{equation*}
		N \triangleleft G 
			\IFF \forall g \in G,\; g * N = N * g.
	\end{equation*}
\end{theorem}

\begin{definition}[商群]
	如果 $N \triangleleft G$, 那么我们记 $\indexfm[G N]{G / N} := \{ g * N \mid g \in G\}$, 称为 $G$ 对 $N$ 的\indexbf{商群}. 
	这个群的乘法定义为子群元素的积的集合: 
	\begin{equation*}
		(g * N) \cdot (g'* N)
		:= \{s * t \mid s \in g * N, \; t \in g; * N\} 
		= (g * g') * N
	\end{equation*}, 单位元是 $e * N = N$ 自身.
\end{definition}




\section{环}

\begin{definition}[环]
	集合$R$非空, 其上定义了加法 $+$ 和乘法 $\cdot$, 且满足:
	\begin{conditionlist}[label=R\arabic*)]
		\item $(R, +, 0)$ 是Abelian群;
		\item $(R, \cdot)$ 是半群; 
		\item 乘法对加法有\indexbf{分配律}: 
		\begin{equation*}
			(a + b) c = ac + bc, \qquad
			c (a + b) = ca + cb
		\end{equation*}
		对$\forall a,b,c \in R$成立.
	\end{conditionlist}
	那么, 我们称 $(R, +, \cdot)$ 是一个\indexbf{环} (ring)\footnote{%
		如果$(R,\cdot)$不结合, 通常称\indexbf{非结合环}.}.
	而且唤$(R, +)$作其加法群, 称$(R, \cdot)$为其乘法半群. 倘若 $(R, \cdot)$ 还有单位元 $1$, 那么我们称$(R, +, \cdot)$ 为有单位元的环.
\end{definition}

若环 $R$ 非空的子集 $L$ 满足 
\begin{equation*}
	\forall x, y \in L \big(
		x - y \in L \; \wedge \; xy \in L
	\big)\,,
\end{equation*}
则称 $L$ 是 $R$ 的一个\indexbf{子环}.

若环的乘法半群是交换的, 则称这个环是一个\indexbf{交换环}.

作为例子, $(\mathbb Z, +, \cdot)$ 是我们熟悉的\indexbf{整数环}, $n\mathbb Z := \{nk \mid k \in \mathbb Z\}$ 是它的一个子环 ($n \in \mathbb Z$). 
交换环 $R$ 上的所有 $n$ 阶方阵之集合 $M_n(R)$ 也是环.

\begin{definition}[同态]
	设 $R$ 和 $R'$ 是两个环, 有一个映射 $f$ 对加法群和乘法半群都是同态 (保持运算), 即:
	\begin{equation*}
		f(x)f(y) = f(xy),
		\quad
		f(x) + f(y) = f(x + y),
	\end{equation*}
	那么, 我们称其为 $R$ 到 $R'$ 的一个\indexbf{同态}或\indexbf{态射}, 集合 $\ker f := \{a \in R \mid f(a) = 0\}$ 称为同态的\indexbf{核}. 同态 $f$ 的核是 $R$ 的子环. 类似地我们也有\indexbf{单同态}, \indexbf{满同态}和\indexbf{同构}的概念. 两个环同构记为 $\indexfm[R cong R prime]{R \cong R'}$.
\end{definition}

设 $(R,+ ,\cdot)$ 是环, $X$ 是一个集合, 在 $R^X$ 上定义加法和乘法:
\begin{equation*}
	f + g \colon x \mapsto f(x) + g(x);
	\qquad
	fg \colon x \mapsto f(x) g(x),
\end{equation*}
就得到了\indexbf{函数环} $(R^X, +, \cdot)$, 其零元是 $0_X \colon x \mapsto 0$. 如果 $R$ 有单位元 $1$,那么 $R^X$ 也有单位元 $1_X \colon x \mapsto 1$, $\forall x \in X $.

作为例子, 考虑到将$\indexfm[k n]{\lbrack k\rbrack _n} \in \mathbb Z / \equiv \bmod n $ 映射到 $n^\mathbb Z \ni \bmod n := \{(m, k) \in \mathbb Z \times m \mid n \equiv k \bmod n\}$ 的同构, 
\indexbf{模$n$的剩余类环}\index{剩余类环} $(\mathbb Z_n, +, \cdot)$ 即可看作函数环 $n^\mathbb Z$ 的一个交换子环, 其中$\indexfm[Z n]{\mathbb Z_n} := \{\lbrack k\rbrack_n \mid k \in n\}$. 
同构关系让我们也能用剩余类的代表元组成的集合 $n$ 代替剩余类本身进行运算, 这种情况下, $n$ 称为\indexbf{模 $n$ 的剩余类的导出集}, 我们能用加法表和乘法表给出它的代数结构.

\begin{definition}[整环]
	环 $R$ 中, $a \in R$, 如果 $\exists b \in R - \{0\}$ s.t.\ $ab = 0$, 则称 $a$ 为环 $R$ 的一个零因子; 类似则可定义\indexbf{右零因子}\footnote{\cite{kostrikin1982introduction}中把 $0$ 排除在外了.}. 
	左零因子和右零因子统称\indexbf{零因子}. 零元 $0$ 则称为\indexbf{平凡零因子}. 
	
	若非平凡的交换环 $R$ 带单位元$1 \neq 0$, 且没有非平凡零因子, 则称 $R$ 是一个\indexbf{整环} (entire ring 或 integral domain).
\end{definition}

也有将无非平凡左零因子的带单位的非平凡环称为 \indexbf{domain} 的.

\begin{theorem}[消去律] \label{theorem:消去律}
	设 $R$ 是带单位元 $1 \neq 0$ 的交换环.
	环 $R$ 是整环 $\IFF$ $\forall x,y,c \in R$, $cx = cy \wedge c\neq 0  \;\to\; x = y$.
\end{theorem}
\begin{proof}
	如果 $R$ 满足消去律, 那么 $ab = 0 = 0b = a0$ 将给出 $a = 0 \vee b = 0$的论断;
	如果 $R$ 是整环, 那么 $cx = cy$ 即 $c(x - y) = 0$ 将得出 $c = 0 \vee x = y $; 
	倘若 $c \neq 0$, 那么这就是消去律.
\end{proof}

有单位元的环 $R$ 中元素 $x$ 的可逆性往往指关于乘法的可逆性. 

\begin{theorem}
	设 $R$ 是带单位元 $1$ 的环, $\indexfm[U R]{U(R)} := \{x \in R \mid \text{$x$ 可逆}\}$ 是一个乘法群.
\end{theorem}
\begin{proof}
	单位元 $1$ 当然可逆. 由定义可逆元素的逆也是可逆的. 如果 $x, y \in R$ 可逆, 那么 
	\begin{equation*}
		(xy)(y^{-1} x^{-1}) = x(y y^{-1})x^{-1} = xx^{-1} = 1 =y^{-1} x^{-1} x y = (xy)^{-1} (xy),
	\end{equation*}
	即 $xy$ 可逆.
\end{proof}

如果 $U(R) = R - \{0\}$, 那么我们称 $R$ 是一个\indexbf{除环} (division ring), 也称\indexbf{斜域}或\indexbf{反对称域} (skew field). 除环没有零因子.

\section{域}

交换除环 $F$ 称为\indexbf{域} (field)\footnote{%
	作为总结: 域上定义了加法和乘法, 加法是Abelian群, 乘法是Abelian幺半群, 而且零元以外的元素都关于乘法可逆, 最后, 乘法对加法有分配律. }. 
群 $P^* = U(P)$ 称为域的乘法群. 如果 $y \neq 0$, 那么我们通常记 $x/y = \frac x y := xy^{-1}$. 

我们可类似环, 定义同构和自同构. 同态的意义不大, 因为如果 $F$ 到 $F'$ 的同态 $f$ 的核 $\ker f \neq \{0\}$, 那么 $\ker f = F$. 如果 $F'$ 是域 $F$ 的子环, 而且也是一个域, 则称其为 $F$ 的一个\indexbf{子域}, 反之称 $F$ 为 $F'$ 的一个\indexbf{扩域}.

类似群的生成, 包含 $F \cup\{a\}$ 的最小 $F$ 的扩域, 记为 $F(a)$. 如有理数域 $\mathbb Q$ 的扩域 $\mathbb Q(\sqrt 2)$.

\begin{theorem}\label{theorem:素剩余类环}
	有限剩余类环 $\mathbb Z_p$ 是域, 当且仅当 $p$ 是素数.
\end{theorem}
\begin{proof}
	记 $\mathbb Z_p$ 的元素为 $[0]$, $[1]$, \ldots, $[p -1]$.
	由素数的定义, $\forall [k]  \in \indexfm[Z p ast]{\mathbb Z_p^*} := \mathbb Z_p - \{[0]\}$,
	\begin{equation*}
		[k], [2k], \cdots, [(p-1) k]
	\end{equation*}
	都不为 $[0]$, 而且两两不等.
	进而, $\exists i \in \mathbb N_+$ s.t.\ $i < p \wedge [ik] = 1$.
	又 $\mathbb Z_p$ 是交换环, 可知这个 $[i] = [k]^{-1}$, 即 $\mathbb Z_p$ 的乘法组成一个群.
\end{proof}

出于 $\mathbb Z_p$ 的这个性质, 我们也记其为 $\indexfm[F p]{\mathbb F_p}$ 或 $\GF(p)$. 值得一提的是, $p^n$ 元有限域 $\GF(p^n)$ 也是存在的.

\begin{corollary}[\indexbf{Fermat小定理}]
	设 $p$ 是素数, $a \in \mathbb N$ 且 $a \nmid p$. 
	\begin{equation*}
		a^{p-1} \equiv 1 \pmod p\,.
	\end{equation*}
\end{corollary}
\begin{proof}
	当 $[k] \in \mathbb Z_p^*$ 时, $I_{[k]} \colon \mathbb Z_p^* \to \mathbb Z_p^*;\, [n] \mapsto [kn]$ 如定理~\ref{theorem:素剩余类环} 是 $S(\mathbb Z_p^*)$ 的元素.
	从而:
	\begin{equation*}
		\left( \prod_{k=1}^{p-1} [k] \right) [a]^{p-1} = \prod_{k=1}^{p-1} [k]\,.  
	\end{equation*}
	因为域都是整环, 满足消去律~\ref{theorem:消去律}, 从而 $[a]^{p-1} = [1]$.
\end{proof}

\begin{definition}[素域]
	若域 $P$ 不含任何非平凡真子域, 则称其为\indexbf{素域} (prime field).
\end{definition}

\begin{lemma}
	$\mathbb Q$ 和 $\mathbb Z_p$ 是素域. 
\end{lemma}
\begin{proof}
	让集合 $\{0,1\}$ 对加法, 减法, 乘法和除法封闭, 我们将得到 $\mathbb Q$ 或 $\mathbb Z_p$ 的导出集 $p$, 取决于 $1$ 在加法群中的阶数.
\end{proof}


\begin{theorem}
	任意非平凡域 $F$ 必含且只含一个素子域 $P$, 而且它将同构于 $\mathbb Q$ 或 $\mathbb Z_p$, 其中 $p$ 是素数. 
\end{theorem}
\begin{proof}
	若有两个素子域, 它们的交必然也是 $F$ 的子域, 根据素域的定义, 这个交不可能是真子域, 从而这两个素域相等. 这就保证了, 如果存在这么一个素子域 $P$, 它一定是唯一的. 接下来我们研究它的存在性.

	定义 $\mathbb Z$ 到 $F$ 的同态 $f(n) = ne$, 其中 $e$ 是 $F$ 的单位元. 
	其核为 $\ker f = m \mathbb Z$, 其中 $m \in \mathbb N$.
	
	如果 $m = 0$, 那么 $ne \neq o$, 其中 $o$ 是 $F$ 的零元, 只要 $n \neq 0$. 考虑 $f$ 在 $\mathbb Q$ 上的扩张, 可以证明 $P := f(\mathbb Q) = \{ne \mid n \in \mathbb Z\}$ 即构成了与 $\mathbb Q$ 同构的素子域.

	如果 $m \neq 0$, 那么 $m =p$ 是素数. 
	如果 $m$ 不是素数, 假设它有两个 ($m$ 和 $1$ 以外的) 因数 $a, b$, $ab e = o$ 意味着 $a e = o$ 或 $b e = o$ (定理~\ref{theorem:消去律}), 将与 $\ker f = m\mathbb Z$ 矛盾.
	考虑 $f$ 在 $p$ (作为 $\mathbb Z_p$ 的导出集) 上的限制, $P := \{o, e, 2e, \cdots, (p-1)e\}$ 即构成了与 $\mathbb Z_p$ 同构的素子域.
\end{proof}

在刚才的证明中, 我们已经遭遇了:
\begin{definition}[特征]
	设域 $F$ 的单位元和零元分别是$e$, $o$. 若存在$p \in \mathbb N$ 使得 $pe = o$, 则称 $p$ 为域的\indexbf{特征} (characteristic), 记为$\indexfm[char F]{\characteristic(F)} = p$; 特别地, 定义$\characteristic(F) = 0$, 如果不存在这样的 $p$.
\end{definition}

\chapter{线性空间}
\section{线性空间}
\begin{definition}[线性空间]
	设 $\mathbb F$ 是一个域, $(V, +, \bv 0)$ 是一个Abelian群. 
	如果定义标量乘积运算: $\mathbb F \times V \to V ;\, (\lambda, \bv x) \mapsto \lambda \bv x$ 且满足:
	\begin{conditionlist}
		\item $1\bv x = \bv x$, $\forall \bv x \in V$ (\indexbf{酉性});
		\item $(\alpha \beta) \bv x = \alpha (\beta \bv x)$, $\forall \alpha, \beta \in \mathbb F$, $\forall \bv x \in V$;
		\item $(\alpha + \beta) \bv x = \alpha \bv x + \beta \bv x$, $\forall \alpha, \beta \in \mathbb F$, $\forall \bv x \in V$;
		\item $\lambda (\bv x + \bv y ) = \lambda \bv x + \lambda \bv y$,
	\end{conditionlist}
	那么, 我们称 $V$ 是 $\mathbb F$ 上的一个\indexbf{线性空间}, 或称\indexbf{向量空间}, 其元素称为\indexbf{向量}, 相对而言 $\mathbb F$ 的元素则被称为\indexbf{纯量}.
\end{definition}

通常我们称 $(\bv x_i)_{i \in I}$ 为\indexbf{向量组},  $I$ 是指标集.

\begin{definition}[线性组合]
	设 $V$ 是 $\mathbb F$ 上的线性空间. 倘若$\forall i \in n$, $\lambda_i \in \mathbb F$, $\bv x_i \in V$, $n$ 是正整数, 那么
	\begin{equation*}
		\sum_{i \in n} \lambda_i \bv x_i
	\end{equation*}
	称为向量组 $(\bv x_i)_{i \in n}$ 的一个系数为 $(\lambda_i)_{i \in n}$ 的\indexbf{线性组合}, $i \in n$.
\end{definition}

可数向量甚至不可数个向量之和的研究, 将在泛函分析中得到更加细致的讨论.

\begin{definition}[线性包络]
	设 $V$ 是 $\mathbb F$ 上的线性空间, $(\bv x_i)_{i\in n}$ 是其中的一个向量组, $n$ 是正整数. 
	其\indexbf{线性包络} (linear span)定义为 
	\begin{equation*}
		\langle \bv x_i\rangle_{i \in n}
		= \left\{ 
			\sum_{i \in n} \lambda_i \bv x_i 
		\middle|
			(\lambda_i)_{i \in n} \in \mathbb F^n
		\right\}\,.
	\end{equation*}
	或者, 设 $M \subset V$, 那么其线性包络定义为
	\begin{equation*}
		\langle M \rangle 
		= \left\{ 
			\sum_{i \in n} \lambda_i \bv x_i
		\middle|
			n \in \mathbb N,\, \forall i \in n (\lambda_i \in \mathbb F\, \wedge \, \bv x_i \in M)  
		\right\}\,.
	\end{equation*}
\end{definition}

\begin{definition}[子空间]
	设 $V'$ 是 $\mathbb F$ 上的线性空间 $V$ 的加法子群, 且对标量乘积封闭, i.e.\ $\forall \bv x \in V'$, $\forall \lambda \in \mathbb F$, $\lambda \bv x \in V'$, 那么, 我们称 $V'$ 是 $V$ 的一个 (线性) \indexbf{子空间}.
\end{definition}

显然 $\langle M \rangle$ 对 $\forall M \in 2^V$ 都是 $V$ 的子空间 (而且是包含 $M$ 的最小的那个), 从而我们也说这种情况下 $\langle M \rangle$ 是 $M$ \indexbf{张出} (span) 或\indexbf{生成}的线性空间.

\begin{definition}[线性相关]
	设 $V$ 是 $\mathbb F$ 上的线性空间, 其中有线性组 $(\bv x_i)_{i \in n}$. 
	若 $\exists (\alpha_i)_{i\in n} \in \mathbb F^n$ s.t.\ $\exists i \in n (\alpha_i \neq 0)$ 且
	\begin{equation*}
		\sum_{i \in n} \alpha_i \bv x_i = \bv 0\,,
	\end{equation*}
	那么称向量组 $(\bv x_i)_{i \in n}$ 是\indexbf{线性相关}的. 
	反之则称它们\indexbf{线性无关}或\indexbf{线性独立}.
\end{definition}

\begin{theorem}\label{theorem:线性相关IFF线性组合}
	向量组 $(\bv x_i)_{i \in n}$ 是线性相关的, 当且仅当 $\exists i \in n$ s.t.\ 
	\begin{equation*}
		\exists (\beta_j)_{j \in n - \{i\}} \in 2^\mathbb F 
		\quad\text{s.t.\ }\quad 
			\bv x_i = \sum_{j \in n - \{i\}} \beta_j \bv x_j
		\,.
	\end{equation*}
\end{theorem}
\begin{proof}
	证明此定理只需取 $i$ 使得见证线性相关的线性组合中 $\bv x_i$ 的系数不为 $0$ 即可.
\end{proof}

\begin{definition}[维数]
	设 $V$ 是 $\mathbb F$ 上的线性空间. 若 $\exists n \in \mathbb N$, 满足
	\begin{equation*}
		n = \max \{r \mid \exists (x_i)_{i \in r} \text{\ s.t.\ 它们是线性独立的}) \}\,,
	\end{equation*}
	那么称 $n$ 是 $V$ 的\indexbf{维数}, 记为 $\dim V = n$, $V$ 是 \textbf{$n$ 维线性空间}. 倘若不存在这样的 $n$, 则 $V$ 是\indexbf{无穷维线性空间}. 
\end{definition}

特别地, $\dim \{\bv 0\} = 0$.

\begin{definition}[基底]
	设 $V$ 是 $\mathbb F$ 上的 $n$ 线性空间, $(\basis e_i)_{i \in n}$ 倘若线性无关, 则称其为 $V$ 的一组\indexbf{基底}. 特别地, 如果 $\dim V = 0$, 空集 $\varnothing$ 是它的一组基底.
\end{definition}

因为基底的顺序并不重要, 有时我们也有基底向量的集合 $\{\basis e_i\}_{i \in n}$ 表示它.

\begin{theorem}[唯一分解]\label{theorem:唯一分解}
	设 $V$ 是 $\mathbb F$ 上的 $n$ 线性空间, $(\basis e_i)_{i \in n}$ 是其一组基底.
	那么 $\forall \bv v \in V$, $\exists ! (v_i)_{i \in n}$ (称为 $\bv v$ 在基底 $(\basis e_i)_{i \in n}$ 下的\indexbf{坐标}), s.t.\ 
	\begin{equation*}
		\bv v = \sum_{i \in n} v_i \basis e_i.
	\end{equation*}
\end{theorem}
\begin{proof}
	唯一性只需要假定有两组分解, 相减并利用基底的线性独立性即可证明. 
	下面只证存在性:
	根据维数的定义, $(\bv v, \basis e_0, \cdots, \basis e_{n - 1})$ 线性相关, 从而
	$\exists \alpha \in \mathbb F\, \exists (\alpha_i)_{i \in n} \in \mathbb F^n$ s.t.\ $(\alpha, \alpha_0, \cdots, \alpha_{n - 1})$ 不全为 $0$ 且
	\begin{equation*}
		\alpha \bv v + \sum_{i \in n} \alpha_i \basis e_i = \bv 0\,,
	\end{equation*}
	考虑到基底的线性独立性, $\alpha \neq 0$, 由域的可逆性, 我们得出了一组线性组合系数 $( - \alpha_i / \alpha)_{i \in n}$\,.
\end{proof}

根据这个定理, 我们断言线性空间 $V$ 的基底 $(\basis e_i)_{i \in n}$ 张出 $V$ 本身, 
i.e.\ $V = \langle\basis e_i\rangle_{i \in n}$. 

若 $\bv v$ 在基底 $\hat e = (\basis e_i)_{i \in n}$ 下的坐标为 $(v_i)_{i \in n}$, 记之为 $\inbasis{\bv v}{\hat e}$. 

\begin{corollary}
	设 $V'$ 是 $V$ 的子空间. 如果 $V' \subsetneq V$, 那么 $\dim V' < \dim V$.
\end{corollary}

\begin{corollary}\label{corollary:向量组线性表出的向量组秩不多于自身}
	如果线性无关的向量组 $(\bv e_j)_{j \in n} $ 满足 $\forall j \in n$, $\bv e_j \in  \langle\bv f_i\rangle_{i \in m}$, 那么 $n \leq m$.
\end{corollary}

我们称一个向量组中, 如果存在 $r$ 个线性无关的向量, 且所有 $r + 1$ 个向量都线性相关, 则我们称 $r$ 为向量组的\indexbf{秩} (rank), 而那 $r$ 个线性无关的向量是\indexbf{最大线性无关组}.
我们接下来证明这样的最大线性无关组总是存在, 而且其个数等于向量组张出的线性空间之维数:

\begin{theorem}\label{corollary:秩与维数}
	设 $ (\bv x_j)_{j \in m}$ 是线性空间 $V$ 的向量组.
	\begin{align*}
		\dim \langle \bv x_j\rangle_{j \in m} = r
		\IFF
		\exists \{\bv x_{j_k}\}_{k \in r} \in 2^{\{\bv x_j\}_{j \in m}} 
		&\left( 	
			\text{$(\bv x_{j_k})_{k \in r}$ 是最大线性无关组}
		\right)\,.
	\end{align*}
\end{theorem}
\begin{proof}
	由维数的定义, $r + 1$ 个线性无关的向量将不可能张出维数为 $r$ 的线性空间.
	倘若不存在 $r$ 个线性无关向量, 在 $\langle \bv x_j\rangle_{j \in m} $中取出一组基底共 $r$ 个线性无关的向量, 这是违背推论~\ref{corollary:向量组线性表出的向量组秩不多于自身} 的. 
	因而, 最大线性无关组总是存在, 而且其个数等于向量组张出的线性空间之维数.
\end{proof}

\begin{theorem}[\indexbf{Steintz 替换}]\label{theorem:Steintz}
	设 $V$ 是 $\mathbb F$ 上的 $n$ 线性空间, $(\basis e_i)_{i \in n}$ 是其一组基底.
	任意线性无关组 $(\basis f_i)_{i \in s}$, 都可从基底中取出 $(\basis e_{i_k})_{i_k \in n,\, k \in t}$ 使得
	\begin{equation*}
		(\basis f_0, \cdots, \basis f_{s-1}, \basis e_{i_0}, \cdots, \basis e_{i_{t-1}})
	\end{equation*}
	是 $V$ 的一组基底.
\end{theorem}
\begin{proof}
	取 $i_0$ 使得 $\basis e_{i_0} \notin \langle\basis f_i\rangle_{i \in s}$; 
	接着取 $i_{k+1}$ 使得 $\basis e_{i_{k+1}} \notin \langle \basis f_0, \cdots, \basis f_{s-1}, \basis e_{i_k} \rangle $, 直到不能进行下去, 剩下的基底全部都可由前面的向量组线性表出, 令此时 $k = t - 1$.
	从而: $V$ 中任何向量都可由基底 $(\basis e_i)_{i \in n}$ 表出, 从而也就可以由
	$
		(\basis f_0, \cdots, \basis f_{s-1}, \basis e_{i_0}, \cdots, \basis e_{i_{t-1}})
	$ 表出, 从而 $s + t \geq n$. 

	另一方面, 不难通过归纳得知, 
	$
		(\basis f_0, \cdots, \basis f_{s-1}, \basis e_{i_0}, \cdots, \basis e_{i_{t-1}})
	$ 是线性无关的, 由维数的定义, 我们断言 $t + s \leq n$. 
	即 $t + s = n$, 我们已然得到 $V$ 的一组基底了.
\end{proof}

设 $\mathbb F$ 上的 $n$ 维线性空间有两组基底 $\hat e = (\basis e_j)_{j \in n}$, $\hat f = (\basis f_i)_{i \in n}$, 考虑定理~\ref{theorem:唯一分解}, 我们写出:
\begin{equation}
	\basis f_i = \sum_{j \in n} a_{ji} \basis e_j\,,
	\qquad
	\forall i \in n\,.
\end{equation}
这里的 $a_{ji}$ 决定了矩阵
\begin{equation}\label{equation:转换矩阵}
	\bv A = (a_{ij})_{i,j \in n} =
	\begin{pmatrix}
		a_{00} & a_{01} & \cdots & a_{0, n-1} \\
		a_{10} & a_{11} & \cdots & a_{1, n-1} \\
		\vdots & \vdots & \ddots & \vdots     \\
		a_{n-1, 0} & a_{n-1, 1} & \cdots & a_{n-1, n-1}
	\end{pmatrix}\,.
\end{equation}
矩阵~\eqref{equation:转换矩阵} 被称为 $\hat e$ 到 $\hat f$ 的一个\indexbf{转换矩阵}. 值得注意的是下标的位置 (这与有限维向量空间的线性映射的矩阵差了一个转置, 见~\ref{section:线性映射}). 让我们引入矩阵和与积的概念%
\footnote{本笔记不想再重复了, 请参见任意一本初等线性代数教材, 或\cite{kostrikin1982introduction}. },
用 $\hat f$ 把 $\hat e$ 表出, 就可以得到转换矩阵的逆 $\bv A^{-1}$. 
这两个矩阵之间的关系是 $\bv A \bv A^{-1} = \bv A^{-1} \bv A = \bv I$.

设 $\bv v \in V$, 
\begin{equation*}
	\bv v = \sum_{i \in n} v_i \basis e_i
	= \sum_{i \in n} v'_i \basis f_i
	= \sum_{i \in n} v'_i \sum_{j \in n} a_{ji} \basis e_j
\end{equation*}

那么, 
\begin{equation*}
	\inbasis{\bv v}{\hat e} = \left( 
		\sum_{j \in n} a_{ij} v'_j 
 \right)_{i \in n}
\end{equation*}
或 $\inbasis{\bv v}{\hat e} = \bv A \inbasis{\bv v}{\hat f}$. 同理 $\inbasis{\bv v}{\hat f} = \bv A^{-1} \inbasis{\bv v}{\hat e}$. 

\begin{definition}[同构]
	如果 $\mathbb F$ 上的线性空间 $V$, $W$ 之间存在 $f \colon V \to W$ s.t.\ 
	\begin{conditionlist}
		\item $f$ 是双射;
		\item $\forall \alpha, \beta \in \mathbb F$, $\forall \bv u, \bv v \in V$, $f(\alpha \bv v + \beta \bv u) = \alpha f(\bv v) + \beta f(\bv u)$,
	\end{conditionlist}
	那么, 两个线性空间被认为是\indexbf{同构}的.
\end{definition}

我们指出同构关系具有等价关系的性质, 并且将基底映射到基底, 并保持维数, 这里不再一一验证了. 类似地, 我们建立线性空间\indexbf{同态}的概念, 即保持线性结构的映射, 双同态即是同构. 线性空间 $V$ 到 $U$ 的同态集记作 $\indexfm[L V U]{\mathcal L (V, U)}$.

\begin{theorem}
	所有 $\mathbb F$ 上的 $n$ 维线性空间都同构于 (坐标空间) $\mathbb F^n$.
\end{theorem}
\begin{proof}
	任取 $\mathbb F$ 上的 $n$ 维线性空间 $V$ 中的向量 $\bv v$ 和一组基底 $\hat e$, 向量$\bv v$ 到它的坐标 $\left. \bv v\right|_{\hat e} \in \mathbb F^n$ 都是一个同构.
\end{proof}

线性空间的交依然是线性空间, 但是它们的并却不一定. 

\begin{definition}[子空间的和]
	设 $\forall i \in m$, $U_i$ 都是 $V$ 的子空间, 定义\footnote{这里不用 $+$ 表示集合的并.}
	\begin{equation*}
		\sum_{i \in m} U_i := 
			\left\langle \bigcup_{i \in m} U_i\right\rangle =
			\left\{
					\sum_{i \in m} \bv u_i 
					\middle\vert
					(\bv u_i)_{i \in m} \in \prod_{i \in m} U_i
			\right\} 
	\end{equation*}
	为 $U$ 和 $W$ 的\indexbf{和}. 若 $\forall i \in m$, $U_i \cap \sum_{j \in m;\, j \neq i} U_i= \{\bv 0\}$, 那么记 $\bigoplus_{i \in m} U_i := \sum_{i \in m} U_i$, 称为\indexbf{直和}.
\end{definition}

\begin{theorem}[\indexbf{Grassmann 恒等式}]\label{Grassmann}
\begin{equation*}
	\dim (U + W) = \dim U + \dim W - \dim (U \cap W)\,.
\end{equation*}
\end{theorem}
\begin{proof}
	设 $\dim (U \cap W) = m$, 有基底 $\hat e = (\basis e_i)_{i \in m}$, $\dim U = k$, $\dim W = \ell$. 由定理, $\dim U$ 可取基底 $(\basis e_0, \cdots \basis e_{m-1};\; \basis f_0, \cdots, \basis f_{k - m - 1})$, $\dim V$ 可取基底 $(\basis e_0, \cdots \basis e_{m-1};\; \basis g_0, \cdots, \basis g_{\ell - m - 1})$, 那么
	\begin{equation*}
		U + W = \langle
			\basis e_0, \cdots \basis e_{m-1};\;
			\basis f_0, \cdots, \basis f_{k - m - 1};\;
			\basis g_0, \cdots, \basis g_{\ell - m - 1}
		\rangle\,.
	\end{equation*}

	接下来我们证明向量组
	\begin{equation*}
		\basis e_0, \cdots \basis e_{m-1};\;
		\basis f_0, \cdots, \basis f_{k - m - 1};\;
		\basis g_0, \cdots, \basis g_{\ell - m - 1}
	\end{equation*}
	线性独立. 若存在非平凡的线性组合:
	\begin{equation*}
		\sum_{s \in m} \varepsilon_s \basis e_s 
			+ \sum_{i \in k - m} \varphi_i \basis f_i
			+ \sum_{j \in \ell - m} \gamma_j \basis g_j
		= \bv 0\,,
	\end{equation*}
	但是前两项是 $U$ 中的元素, 第三项是 $W$ 中的元素, 这将说明它们都属于 $U \cap W$, 这意味着第三项可用 $\hat e$ 表出, 这是一个矛盾.
\end{proof}

\begin{corollary}
	$U = \sum_{i \in m} U_i$ 是直和, 当且仅当:
	\begin{equation*}
		\dim U = \sum_{i \in m} \dim U_i\,.
	\end{equation*}
\end{corollary}
\begin{proof}
	利用Grassmann恒等式~\ref{Grassmann} 和数学归纳法易证.
\end{proof}

\begin{theorem}[向量在直和上分解的唯一性]\label{theorem:向量在直和上分解的唯一性}
	设 $U = \sum_{i \in m} U_i$. 
	$U = \sum_{i \in m} U_i$ 还是直和, 当且仅当:
		\begin{equation*}
			\forall \bv u \in U
				\exists! (\bv u_i)_{i \in m} \in \prod_{i \in m} U_i
					\left( \bv u  = \sum_{i \in m} \bv u_i \right)\,.
		\end{equation*}
\end{theorem}
\begin{proof}
	$\gets$: 假设 $\exists \bv u \in U$ s.t.\ $\exists i, j \in m
		\big(i \neq j \,\wedge\, \bv u \in U_i \cap U_j\big)$, 那么 $\bv u_0$ 在其上的分解式不唯一: $\bv u_i$ 和 $\bv u_j$ 可以其中一个取 $\bv u$, 另一个取 $\bv 0$.
	
	$\to$: 这相当于证明 $\sum_{i \in m} \bv u_i = \bv 0 \to \forall i \in m \big( \bv u_i = \bv 0\big)$, 其逆否命题的成立, 只需要将非零项移至另一侧并用直和的定义即可验证.
\end{proof}

\begin{theorem}
	域 $\mathbb F$ 上的 $n$ 维线性空间 $V$ 的任意 $m$ 维线性子空间 $U$, 都能找到 $V$ 的线性子空间 $W$ 使得 $V = U \oplus W$ (称 $V$ 和 $W$ 是\indexbf{互补}的子空间). 
\end{theorem}
\begin{proof}
	证明用Steintz替换~\ref{theorem:Steintz} 即可.
\end{proof}

记 $\codim U = \dim V - \dim U$.

当 $L$ 是 $V$ 的一个子空间时, 我们记线性空间作为加法群的陪集 $\indexfm[x plus L]{\bv x + L} := \{\bv x + \bv y \mid \bv y \in L\}$, 并记其代表元为. 
考虑到线性空间作为加法群是Abelian群, 其所有子群 (子空间蕴含了加法子群) 都是正规子群, 从而:

\begin{definition}[商空间]
	域 $\mathbb F$ 上的线性空间 $V$ 有子空间 $L$, 记线性空间作为加法群的商群 $ V / L$, 并在 $\mathbb F \times V / L$ 上定义标量乘法:
	\begin{equation*}
		\alpha (\bv x + L) := \alpha \bv x + L \,,
	\end{equation*}
	那么称 $V / L$ 是一个\indexbf{商空间}. 不难验证商空间是一个线性空间.
\end{definition}

我们记商空间上的同余等价类:
\begin{equation*}
	\bv x \equiv \bv x' \mod L \IFF  \bv x - \bv x' \in L\,.
\end{equation*}

\begin{theorem}
	设 $V$ 的子空间 $U$ 和 $W$ 互余, 那么
	\begin{equation*}
		f \colon W \to V / U; \; \bv w \mapsto \bv w + U
	\end{equation*}
	见证了 $W$ 和 $V / U$ 的同构.
\end{theorem}
\begin{proof}
	映射 $f$ 对线性结构的保持是平凡的. 
	
	设 $\bv v + U \in V / U$.
	因为 $V \oplus U + W$, $\exists \bv u \in U$, $\exists \bv w \in W$ s.t.\ $\bv v = \bv u + \bv w$. 从而
	\begin{equation*}
		\bv v + U 
		= (\bv u + \bv w) + U
		= (\bv x + U) + (\bv w + U)
		= U + (\bv w + U)
		= \bv w + U
		= f(\bv w)\,, 
	\end{equation*}
	所以 $f$ 是满的. 满射 $f$ 的单性由
	\begin{equation*}
		\ker f = \{\bv w \in W \mid f(\bv w) = U\} 
		= \{\bv w \in W \mid \bv w \in U\}
		= W \cap U
		= \{\bv 0\} 
	\end{equation*}
	保证.
\end{proof}

\section{对偶空间}
\begin{definition}[线性型]
	设 $V$ 是一个域~$\mathbb F$ 上的线性空间.
	同态~$f \colon V \to \mathbb F$ 被称为 $V$ 上的一个\indexbf{线性型} (linear form). 
	在不同的情景, 它也可能被称作\indexbf{线性泛函} (linear functional), \indexbf{线性函数}等.
\end{definition}

作为 $n$ 维有限维空间的例子, 设有线性型 $\ell$, 它作用于 $\bv x \in V$ 时, 设基底为 $\hat e$, 那么:
\begin{equation*}
	\ell \colon \bv x \mapsto \inbasis{\bv \ell}{\hat e} \inbasis{\bv x}{\hat e}\,,
\end{equation*}
其中 $\inbasis{\bv \ell}{\hat e}$ 是 $1 \times n$ 的行向量. 坐标变换到 $\hat f$ 时, 设转换矩阵是 $\bv P$, 那么:
\begin{equation*}
	\inbasis{\bv \ell}{\hat e} \inbasis{\bv x}{\hat e} 
	= \inbasis{\bv \ell}{\hat e} \bv P \inbasis{\bv x}{\hat f}
	= \inbasis{\bv \ell}{\hat f} \inbasis{\bv x}{\hat f}\,, 
\end{equation*}
即:
\begin{equation}
	\inbasis{\bv \ell}{\hat f} = \bv P \inbasis{\bv \ell}{\hat e}\,.
\end{equation}

定义线性型的线性组合 $\alpha f + \beta g$ 为:
\begin{equation*}
	(\alpha f + \beta g)(\bv x) := \alpha f(\bv x) + \beta g(\bv x),
	\quad \forall \bv x \in V \forall \alpha, \beta \in \mathbb F\,.
\end{equation*}

如此我们注意到 $V$ 上所有的线性型构成了一个线性空间, 其中零元是 $0_V \colon \bv x \mapsto 0$.

\begin{definition}[对偶空间]
	线性空间 $V$ 上所有的线性型构成线性空间 $\indexfm[V ast]{V^*}$, 称为 $V$的\indexbf{对偶空间} (dual space), 线性组合和零元已定义如前. 通常对偶空间的元素可称为\indexbf{余向量} (covector), 或\indexbf{共变向量} (covariant vector, 与此同时, $V$ 的元素对应地称为\indexbf{反变向量}, contravariant vector).
\end{definition}

为区别两种向量, 有用 $x^i$ 表示反变向量而用 $\ell_i$ 表示共变向量, 并引入Einstein求和约定的, 见之后第\ref{chapter:张量}章.

我们继续以 $n$ 维线性空间为例子. 
设 $V$ 中有基底 $\hat e = (\basis e_i)_{i \in n}$, 取 $V^*$ 的基底 $\hat e^* := (\basis e^*_i)_{i \in n}$, 使得 $\basis e^*_i(\basis e_i) = \delta_{ij}$, 其中 $\delta_{ij}$ 是Kronecker符号, 当且仅当 $i = j$ 时取值为 $1$, 否则为 $0$.

不难证明它们是线性独立的, 而且能线性表示所有余向量. 这组基底称为\indexbf{对偶基底}. 
而且作为推论:
\begin{theorem}
设 $V$ 是有限维线性空间, 那么
\begin{equation*}
	\dim V^* = \dim V\,.
\end{equation*}
\end{theorem}

考虑到 $V^{**} := (V^*)^*$ 和 $V$ 的维数也当相同, 它们之间应该存在同构关系. 这个同构有一个自然的构造:

\begin{theorem}[自然同构]
	设 $V$ 是 $n$ 维线性空间, 映射 $\varepsilon \colon V \to V^{**}$ 定义如下:
	\begin{equation*}
		\bv x \mapsto \varepsilon_{\bv x}\;; 
		\quad
		\varepsilon_{\bv x} \colon V^* \to \mathbb F\,; 
			f \mapsto f(\bv x)\,.
	\end{equation*}
	
	映射 $\varepsilon$ 是一个同构.
\end{theorem}
\begin{proof}
	事实 $\varepsilon \in \mathcal L(V, V^{**})$ 的验证是枯燥的. 
	这里我们只证明它是个双射:

	选取 $V$ 的基底 $\hat e = (\basis e_i)_{i \in n}$, 就能立马得出结论 $\hat \varepsilon = (\varepsilon_{\basis e_i})_{i \in n}$ 是 $V^{**}$ 的基底. 
\end{proof}

这个同构被称为\indexbf{自然同构}, 这样得到的 $\hat e^* = (e^*_i)_{i \in n}$ 被称为 $\hat e$ 的\indexbf{对偶基底}.

\begin{lemma}\label{lemma:子空间的对偶}
	设 $L$ 是 $n$~维线性空间~$V$ 的子空间, $\hat f := (f_i)_{i \in n}$ 是对偶空间~$V^*$ 的一组基底. 
	倘若 $(\left. f_i \right|_L)_{i \in n}$ 表示基底各自在 $L$ 上的限制,
	那么 $L^* = \langle \left. f_i \right|_L \rangle_{i \in n}$.
\end{lemma}
\begin{proof}
	首先, 显然 $\langle \left. f_i \right|_L \rangle_{i \in n} \subseteq L^*$.
	设 $r := \dim L$, $\hat e := (\basis e_i)_{i \in r}$ 是 $L$ 的基底. 
	由定理~\ref{theorem:Steintz}, 将其扩充至 $V$ 的基底 $(\basis e_i)_{i \in n}$.

	$\forall f \in L^*$, 取线性型 $\tilde f := \sum_{i \in n} \beta_i f_i \in V^*$ 满足 $\forall i' \leq r$, $\tilde f(\basis e_{i'}) = 0$. 
	显然 $f = \left. \tilde f \right|_L = \sum_{i \in n} \beta_i \left. f_i \right|_L$.
\end{proof}


\begin{lemma}\label{lemma:线性相关则行列式为0}
	设线性空间~$V$ 中有线性相关的向量组~$(\bv x_j)_{j\in m}$, 而 $\forall i \in m$, $f_i \in V^*$. 那么:
	\begin{equation*}
		\det\left( 
			f_i (\bv x_j)
		 \right)_{i,j \in m} = 0\,.
	\end{equation*}
\end{lemma}
\begin{proof}
	根据定理~\ref{theorem:线性相关IFF线性组合}, $\exists j_0 \in m$ 使得 $\bv x_{j_0}$ 是其他 $(\bv x_j)_{j\in m;\; j \neq j_0}$ 的线性组合. 
	根据行列式的性质, 将$j_0$ 列减去其他各列 ($j \neq j_0$) 乘上线性组合的系数~$\lambda_j$, 不改变行列式的值, 但该列变成了
	\begin{equation*}
		f_i(\bv x_{j_0}) 
			- \sum_{j \in m;\; j \neq j_0} \alpha_j f_i(\bv x)
			= f_i\left(\bv x_{j_0} 
					- \sum_{j \in m;\; j \neq j_0} \alpha_j \bv x_j
				\right)
			= f_i(\bv 0)
			= 0\,.
	\end{equation*}
	
	这给出了 $
		\det\left( 
			f_i (\bv x_j)
		 \right)_{i,j \in m} = 0\,.
	$ 的证明.
\end{proof}

\begin{lemma}\label{lemma:线性无关IFF行列式非零}
	设 $V$ 是 $n$~维线性空间, 而 $\hat f := (f_i)_{i \in n}$ 是对偶空间~$V^*$ 的一组基底.
	向量组~$(\bv x_j)_{j \in n}$ 线性无关当且仅当
	\begin{equation*}
		\det(f_i(\bv x_j))_{i,j \in n} \neq 0\,.
	\end{equation*}
\end{lemma}
\begin{proof}
	由引理~\ref{lemma:线性相关则行列式为0}, 我们已经证明了行列式非零则线性无关. 
	反过来, 若线性无关, 取 $\hat e = \hat f^*$ 即 $\hat f$ 的对偶基底. 
	考虑到 $\hat x = (\bv x_j)_{j \in n}$ 也是一组基底, 那么存在转换矩阵 $\bv P$, 而且它的行列式恰是 $\det(f_i(\bv x_j))_{i,j \in n}$. 
	转换矩阵是可逆的, 它的行列式非零.
\end{proof}

\begin{theorem}\label{theorem:线性相关, 对偶空间基底的阶数}
	设 $V$ 是 $n$~维线性空间, 而 $\hat f := (f_i)_{i \in n}$ 是对偶空间~$V^*$ 的一组基底.
	那么 $V$ 的子空间~$\langle \bv x_j\rangle_{j \in m}$ 的维数 $r$ 等于
	\begin{equation*}
		(f_i(\bv x_j))_{i \in n, j \in m}
	\end{equation*}
	的最大非零子式的阶数.
\end{theorem}
\begin{proof}
	由引理~\ref{lemma:线性相关则行列式为0}, 阶数比 $r$ 大的子式必为 $0$, 我们只需证明有 $r$~阶非零子式.

	取 $(\bv x_j)_{j \in m}$ 中的一组线性无关组~$(\bv x_{j_k})_{k \in r}$, 再在 $\left. \hat f \right|_{\langle \bv x_j\rangle_{j \in m}}$ 中取出线性无关组~$(\bar f_{k})_{k \in r}
	:=
	\left(  
		\left. f_{i_k} \right|_{\langle \bv x_j\rangle_{j \in m}}
	\right)_{k \in r}$ (引理~\ref{lemma:子空间的对偶}), 引理~\ref{lemma:线性无关IFF行列式非零} 告诉我们
	\begin{equation*}
		\det(\bar f_i (\bv x_{j_k}))_{i, k \in r} \neq 0\,.
	\end{equation*}
\end{proof}

\begin{corollary}\label{corollary:线性相关与坐标阶数}
	设 $V$ 是 $n$~维线性空间, 有基底 $\hat e$, 向量组 $(\bv x_j)_{j \in m}$ 的维数等于矩阵 $(
			\inbasis{\bv x_j}{\hat e}
		)_{j \in m}$ 的最大非零子式的阶数.
	. 
\end{corollary}
\begin{proof}
	在定理~\ref{theorem:线性相关, 对偶空间基底的阶数} 中令 $\hat f = \hat e^*$ 即可.
\end{proof}

\section{多重线性型}
\begin{definition}[多重线性型]
	设 $V_0$, $V_1$, \ldots, $V_{p - 1}$, $U$ 是 $\mathbb F$ 上的线性空间. 若映射
	\begin{equation*}
		f \colon \prod_{i \in p} V_i  \to U
	\end{equation*}
	满足 $\forall i \in p$, 
	\begin{equation*}
		\forall (\bv a_j)_{j \in p;\;j \neq i} \in 
			\prod_{j \in p;\; j \neq i} V_j\,,
		\quad
		f_i \colon V_i \to U;\;\bv x \mapsto f(\bv a_0, \cdots, \bv a_{i - 1}, 
			\bv x, \bv a_{i + 1}, \cdots, \bv a_{p - 1}) \in \mathcal L (V_i, U)\,,
	\end{equation*}
	则称 $f$ 是 $V_0$, \ldots, $V_{p - 1}$ 上的\indexbf{多重线性型}, 或 \emphbf{$p$-线性型}%
		\index{p-线性型@$p$-线性型}%
	. 这些多重线性型的集合记为 $\indexfm[L V 0 dots V p minus 1 U]{%
		\mathcal L(V_0, \cdots, V_{p - 1};\; U)}$.
\end{definition}

如 $V_0 = V_1 = \cdots = V_{p - 1}$, 那么我们记 $V^p$ 上的多重线性型的集合为 $\indexfm[L p V U]{%
	\mathcal L_p(V;\; U)}$. 
当 $U = \mathbb F$ 时, 我们也可省略 $\mathbb F$ 不写.

\begin{definition}[对称与反对称]
	若 $V$, $U$ 是 $\mathbb F$ 上的线性空间, $f \in \mathcal L_p(V, U)$. 
	如果 $\forall \pi \in S_p$, $\forall (\bv x_i)_{i \in p} \in V^p$, 
	\begin{equation*}
		f\left( 
		\bv x_{\pi(i)} 
	 \right)_{i \in p} = f(\bv x_i)_{i \in p}\,,
	\end{equation*}
	那么我们称 $f$ 为\indexbf{对称的}.
	如 $\forall \pi \in S_p$, $\forall (\bv x_i)_{i \in p} \in V^p$,
	\begin{equation*}
		f\left( 
		\bv x_{\pi(i)} 
	 	\right)_{i \in p} = \varepsilon_\pi f(\bv x_i)_{i \in p}\,,
	\end{equation*}
	那么我们称 $f$ 为\indexbf{反对称的}.
\end{definition}

我们可以给出行列式的公理化构造, 它在实数上的计算方法我们已经在线性代数课程中非常熟悉了:

\begin{definition}[行列式]\label{definition:行列式的公理化构造}
	设 $\mathbb F$ 是一个域. 多重线性型 $\indexfm[det]{%
		\det} \in \mathcal L_n(\mathbb F)$ 若满足:
		\begin{conditionlist}
			\item $\det$ 是反对称的;
			\item $\det \bv I = 1$, 其中 $\bv I = (\delta_{ij})_{i, j \in n}$,
		\end{conditionlist}
		记方阵 $\bv X := (\bv x_i)_{i \in n}$, 则称 $\det \bv X$ 是 $\bv X$ 的\indexbf{行列式}.
\end{definition}


\chapter{线性算子}
\section{线性映射}
	\label{section:线性映射}

\begin{definition}[线性映射]
	设 $V$, $W$ 是域 $\mathbb F$ 上的线性空间. 
	如映射 $\mathscr A \in \mathcal L (V, W)$, 即 $\mathscr A$ 是 $V$ 到 $W$ 的一个同态, 那么我们称 $\mathscr A$ 是 $V$ 到 $W$ 的一个\indexbf{线性映射}, 并称其为\indexbf{线性的}. 特别地, 如果它还是自同态, 我们称其为\indexbf{线性变换}\footnote{也有将线性映射统称为线性变换的. }或\indexbf{线性算子}.
\end{definition}

\begin{theorem}
	设 $\mathscr A \in \mathcal L (V, W)$, 倘若 $(\bv v_i)_{i \in s} \in V^s$, 
	\begin{equation*}
		f \left( 
			\langle \bv v_i \rangle_{i \in s}
		\right) = \left\langle
			f(\bv v_i)
		\right\rangle_{i \in s}\,.
	\end{equation*} 
\end{theorem}

\begin{corollary}
	设 $\mathscr A \in \mathcal L (V, W)$, 而 $U$ 是 $V$ 的有限维子空间, 那么 $\dim f(U) \leq \dim U$.
\end{corollary}

我们将指出, 我们在这里所说的线性映射在某基底下可表为矩阵. 
设 $V$, $W$ 分别是 $m$, $n$ 维线性空间, 给定各自的基底 $\hat e$, $\hat f$, 那么我们可以用矩阵
\begin{equation}\label{equation:线性映射的矩阵表示}
	\inbasis{\mathscr A}{\hat e,\, \hat f} 
		:= \bv A = (a_{ij})_{i \in n,\, j \in m}
		= \left( 
			\inbasis{f(\basis e_j)}{\hat f}
		 \right)_{j \in m}
\end{equation}
来表示 $\mathscr A \in \mathcal L (V, W)$. 

\begin{theorem}
	由式~\eqref{equation:线性映射的矩阵表示} 决定的线性映射和 $\mathbb F$ 上的 $m \times n$ 矩阵是一一对应的, 且:
	\begin{equation}\label{equation:线性映射的复合与矩阵的积}
		\inbasis{(\mathscr B \circ \mathscr A)}{\hat e,\, \hat g} = \bv B \bv A\,,
	\end{equation}
	其中 $\mathscr A \colon V \to U$, $\mathscr B \colon U \to W$, $V$, $U$ 和 $W$ 分别有基底 $\hat e$, $\hat f$ 和 $\hat g$.
\end{theorem}
\begin{proof}
	由线性映射到矩阵的单性由式~\eqref{equation:线性映射的矩阵表示} 易证 (意思是, 只需假定有两个线性映射共用矩阵, 它们将由~\ref{theorem:唯一分解} 得出是同一个映射). 而满性只需验证由 $\inbasis{f(\bv v)}{\hat f} = \bv A \inbasis{\bv v}{\hat e}$ 决定的映射是 $\mathcal L(V, W)$ 的元素.

	同态的复合依然是同态是显然的 (可以进行枯燥的验证, 但没必要). 
	式~\eqref{equation:线性映射的复合与矩阵的积} 则由下式给出:
	\begin{equation*}
		\inbasis{(\mathscr B \circ \mathscr A)(\bv v)}{\hat g} 
		= \inbasis{\mathscr B \big(
			\mathscr A(\bv v)\big)}{\hat g}
		= \bv B \inbasis{\mathscr A(\bv v)}{\hat f}
		= \bv B \bv A \inbasis{\bv v}{\hat e}\,.
	\end{equation*}
\end{proof}


\begin{definition}[秩]
	设 $\mathscr A \in \mathcal L(V, W)$, 记 $\rank \mathscr A := \dim \mathscr A(V)$ 为线性映射~$\mathscr A$ 的\indexbf{秩}.
	同时我们称 $\dim \ker \mathscr A$ 为其\indexbf{亏数}或\indexbf{零化度} (nullity).
\end{definition}

\begin{theorem}
	若 $V$, $W$ 都是有限维向量空间. 任取它们分别的基底 $\hat e$, $\hat f$, 都有 $\rank \mathscr A = \rank \bv A$, 其中 $\bv A = \inbasis{\mathscr A}{\hat e,\, \hat f}$.
\end{theorem}
\begin{proof}
	由定义式~\eqref{equation:线性映射的矩阵表示}, 矩阵的列向量组将张出 $\mathscr A (V)$.
	由\ref{corollary:秩与维数}, 这就给出了我们的定理\footnote{%
	矩阵的秩的最大非零子式定义和列向量组定义等价已由推论~\ref{corollary:线性相关与坐标阶数}保证.}%
	.
\end{proof}

\begin{theorem}\label{theorem:核和像的维数}
	设 $V$ 是域 $\mathbb F$ 上的有限维线性空间, $W$ 是域 $\mathbb F$ 上的线性空间, $\mathscr A \in \mathcal L(V, W)$, 那么
	\begin{equation*}
		\dim \ker \mathscr A + \dim \mathscr A(V) = \dim V\,. 
	\end{equation*}
\end{theorem}
\begin{proof}
	记 $\dim V = n$, $\dim \mathscr A = r$, $\dim \ker \mathscr A = k$.

	取 $\ker \mathscr A$ 的一组基底 $(\basis e_i)_{i \in k}$ (显然 $k \leq n$), 并将它扩充为 $V$ 的基底 $\hat e$ (我们又用了Steintz替换原则~\ref{theorem:Steintz}).
	考虑到 $\mathscr A(V) = \langle \mathscr A(\basis e_i) \rangle_{i \in n}$.
	但 $\langle \mathscr A(\basis e_i) \rangle_{i \in k} = \{\bv 0\}$.
	利用 $\mathscr A$ 的线性, 我们给出 $\forall (\lambda_i)_{i \in n} \in \mathbb F^n$:
	\begin{equation*}
		\sum_{i \in n} \lambda_i \mathscr A(\basis e_i) 
			= \sum_{i \in n\,\wedge\, i \notin k} \lambda_i \mathscr A(\basis e_i) 
				+  \mathscr A\left( 
					\sum_{i \in k} \lambda_i \basis e_i \right)
			= \sum_{i \in n\,\wedge\, i \notin k} \lambda_i \mathscr A(\basis e_i) .
	\end{equation*}
	
	即是: $\big(\mathscr A(\basis e_i)\big)_{i \in n\,\wedge\, i \notin k}$ 将构成 $\mathscr A(V)$ 的一组基底. 从而:
	\begin{equation*}
		r + k = n\,.
	\end{equation*}
\end{proof}

在 $\mathcal L(V, W)$ 上定义加法和数乘, 可以验证它是一个线性空间. 

\section{线性算子代数}
域~$\mathbb F$ 上的线性空间~$V$ 的自同态 $\mathcal L(V, V)$ 可记作 $\indexfm[L V]{\mathcal L(V)}$ 或 $\indexfm[End]{\End(V)}$. 
如前已述, 它的元素唤作线性算子. 
给定 $n$~维线性空间~$V$ 的一组基底~$\hat e$ (同时作为定义域和到达域的基底), $\mathcal L(V)$ 的元素可用 $n$~阶方阵表示. 
其中恒等变换 $\id_V$ 对应的矩阵通常记作 $\bv I$, 即 $n$~阶单位阵. 零映射记为 $\indexfm[O]{\mathscr O} \colon \bv x \mapsto \bv 0$.

习惯上记 $\mathscr A \bv x := \mathscr A(\bv x)$, $\mathscr A \mathscr B := \mathscr A \circ \mathscr B$. 

\begin{definition}[逆算子]
	设 $\mathscr A, \mathscr B \in \mathcal L (V)$. 
	若
	\begin{equation*}
		\mathscr A \mathscr B = \mathscr B \mathscr A = \id_V\,,
	\end{equation*}
	则称它们互为\indexbf{逆算子}, 记 $\mathscr A = \mathscr B^{-1}$ 或 $\mathscr B = \mathscr A^{-1}$.
\end{definition}

\begin{theorem}
	设 $V$ 是有限维线性空间, $\mathscr A \in \mathcal L(V)$. 
	\begin{equation*}
		\exists \mathscr B \in \mathcal L(V) \,\big(
				\mathscr A = \mathscr A^{-1}\big) 
			\IFF
				\rank \mathscr A = \dim V 
			\IFF
				\ker \mathscr A = \{\bv 0\}\,.
	\end{equation*}
\end{theorem}
\begin{proof}
	利用定理~\ref{theorem:核和像的维数} 立刻就能证明.
\end{proof}

\begin{definition}[代数]
	如果一个环~$A$ 同时是域~$\mathbb F$ 上的线性空间, 而且数乘满足:
	\begin{equation*}
		\forall \lambda \in \mathbb F\,\forall \mathscr A, \mathscr B \in A\;
			\big(
				\lambda(\mathscr A \mathscr B) 
					= (\lambda \mathscr A) \mathscr B
						= \mathscr A (\lambda \mathscr B)\big)\,,
	\end{equation*}
	那么我们称 $A$ 是 $\mathbb F$ 上的一个\indexbf{代数} (algebra)\footnote{实际上这里是结合的, 有单位元的代数. 出于简单在本部分我们还是径直称其为代数. }.
	若 $A'$ 同时作为 $A$ 的子环和子空间, 那么 $A'$ 是 $A$ 的一个子代数.
\end{definition}

在这个意义上, $\mathcal L(V)$ 被称为\indexbf{线性算子代数}.

多项式环 $\mathbb F[X]$ 即是 $\mathbb F$ 上的无穷维代数的例子, 而它在 $X = \mathscr A \in \mathcal L(V)$ 时的取值, 即 $\mathbb F[\mathscr A]$ (我们记 $\mathscr A^0 := \id_V$), 可以验证是 $\mathcal L(V)$ 的子代数.

考虑到 $\mathbb F$ 是交换的, $\mathbb F[\mathscr A]$ 也是交换的. 

\begin{definition}[极小多项式]
	设 $V$ 是域~$\mathbb F$ 上的线性空间, $\mathscr A \in \mathcal L(V)$, $P(X) \in \mathbb F[X]$.
	如果 $P(\mathscr A) = \mathscr O$, 那么称多项式 $P(X)$ \indexbf{零化}线性算子 $\mathscr A$. 首项系数为 $1$ 的零化 $\mathscr A$ 的多项式称为其\indexbf{极小多项式}, 可记为 $\mu_{\mathscr A}(X)$.
\end{definition}

\begin{theorem}[极小多项式存在]
	设 $V$ 是域~$\mathbb F$ 上的 $n$~维线性空间. 
	$\forall \mathscr A \in \mathcal L(V)$, 都存在极小多项式 $\mu_\mathscr A (X)$, 且 $\deg \mu_{\mathscr A} (X) = \dim \mathbb F[\mathscr A]$.
\end{theorem}
\begin{proof}
	考虑到 $\mathbb F[\mathscr A]$ 是 $\mathcal L(V)$ 的子代数, $\dim \mathbb F[\mathscr A] \leq \dim \mathcal L(V)$. 

	如果次数 $< n^2 + 1$ 的非零多项式 $P(X)$ 都不能零化 $\mathscr A$, 我们即可说: 
	$\id_V, \mathscr A, \cdots, \mathscr A^{n^2 + 1}$ 的任意非平凡线性组合都不为 $0$, 我们在维数小于 $n^2$ 维的线性空间里找到了 $n^2 + 1$ 个线性无关的向量, 这显然是不可能的.
\end{proof}

\begin{theorem}[极小多项式唯一]
	设 $V$ 是域~$\mathbb F$ 上的线性空间, $\mathscr A \in \mathcal L(V)$. 
	若 $P(X) = X^n + \sum_{i \in n} p_i X^i$, $Q(X) = X^n + \sum_{i \in n} q_i X^i$ 都是 $\mathscr A$ 的极小多项式, 那么 $(p_i)_{i \in n} = (q_i)_{i \in n}$.
\end{theorem}
\begin{proof}
	因为 $\mathbb F[\mathscr A]$ 是一个代数, $P(\mathscr A) - Q(\mathscr A) = \sum_{i \in n} (p_i - q_i) \mathscr A^i = \mathscr O$. 
	我们得到了一个能零化 $\mathscr A$ 次数小于 $n$ 的多项式. 
	如果它不是零多项式, 记 $m = \deg \big( P(X) - Q(X)\big)$, 那么 $\frac{1}{p_m - q_m} \sum_{i \in n} (p_i - q_i) X^i = \mathscr O$ 将是一个首项为 $1$ 的零化 $\mathscr A$ 但其次数小于极小多项式的次数, 这是违背极小多项式的定义的.
\end{proof}

\begin{theorem}[可逆算子与极小多项式的常数项]
	设 $V$ 是域~$\mathbb F$ 上的 $n$~维线性空间, $\mathscr A \in \mathcal L(V)$, 其极小多项式是 $\mu_\mathscr A (X)$. 
	算子~$\mathscr A$ 可逆当且仅当 $\mu_\mathscr A$ 的常数项非零.
\end{theorem}
\begin{proof}
	如果极小多项式常数项为 $0$, 即 $\mu_\mathscr A(X) = \sum_{i \in n - \{0\}} p_i X^i$, 那么由线性空间的分配律与方幂的分解,
	\begin{equation*}
		\mathscr A 
		\left(
			\sum_{i \in n - \{0\}} p_i \mathscr A^{i - 1}
		\right) = \mathscr O\,.
	\end{equation*}

	由极小多项式的定义, $\sum_{i \in n - \{0\}} p_i \mathscr A^{i - 1} \neq \mathscr O$. 
	那么, $\exists \bv x \in V$, $\sum_{i \in n - \{0\}} p_i \mathscr A^{i - 1} \bv x \in \ker \mathscr A - \{\bv 0\}$, 这表明 $\rank \mathscr A < n$, 即不可逆. 

	如果极小多项式常数项不为 $0$, 
	\begin{equation*}
		\mathscr A \frac{1}{- p_0} 
		\left(
			\sum_{i \in n - \{0\}} p_i \mathscr A^{i - 1}
		\right) = \id_V\,,
	\end{equation*}
	给出了 $\mathscr A$ 的逆.
\end{proof}

\begin{theorem}[化零算子的多项式是极小多项式的倍式]
	设 $V$ 是域~$\mathbb F$ 上的线性空间.
	能零化 $\mathscr A \in \mathcal L (V)$ 的多项式 $P(X) \in \mathbb F[X]$ 一定是 $\mu_\mathscr A (X)$ 的倍式.
\end{theorem}
\begin{proof}
	作带余除法 $P(X) = Q(X) \mu_\mathscr A(X) + R(X)$, 其中 $\deg R(X) < \deg \mu_\mathscr A(X)$. 
	如果 $R(X) \neq 0$, 那么 $R(\mathscr A) = P(\mathscr A) - Q(\mathscr A) \mu_\mathscr A(X) = \mathscr O$ 说明 $R(X)$ 是次数比 $\mu_\mathscr A(X)$ 还小的能化零 $\mathscr A$ 的多项式, 这与极小多项式的定义是矛盾的.
\end{proof}

\begin{definition}[幂零算子]
	设 $V$ 是域~$\mathbb F$ 上的线性空间.
	线性算子~$\mathscr A \in \mathcal L(V)$ 如果满足 $\exists m \in \mathbb N_+$ 使得 $\mathscr A^m = \mathscr O$, 那么称其是一个\indexbf{幂零算子} (nilpotent operator). 
	数~$d := \min \{m \in \mathbb N_+ \mid \mathscr A^m = \mathscr O\}$ 则被称为幂零算子的\indexbf{幂零指数}.
\end{definition}

由域作为零环的性质, 我们很容易验证 (只需要显式设出极小多项式):

\begin{theorem}[幂零算子的极小多项式]
	设 $V$ 是域~$\mathbb F$ 上的线性空间. 若 $\mathscr A \in \mathcal L(V)$ 的幂零指数为 $d$, 那么其极小多项式就是 $X^d$.
\end{theorem}


通过枯燥的运算可以得出:
\begin{theorem}[线性算子在不同基底下的矩阵]
	设 $V$ 是域~$\mathbb F$ 上的 $n$~维线性空间. 若 $\bv A$ 和 $\bv A'$ 分别是 $\mathscr A \in \mathcal L(V)$ 在基底~$\hat e$ 和 $\hat e'$ 下的矩阵, 那么:
	\begin{equation*}
		\bv A' = \bv P^{-1} \bv A \bv P,\,
	\end{equation*}
	其中 $\bv P$ 是 $\hat e$ 到 $\hat e'$ 的转换矩阵.
\end{theorem}

也就是说: 相似矩阵是同一线性算子在不同基底下的坐标表示.
借相似关系, 行列式和迹的性质:

\begin{theorem}[不变量]
	设 $\bv A, \bv A' \in M_n(\mathbb F)$. 
	设 $V$ 是域~$\mathbb F$ 上的 $n$~维线性空间. 若 $\bv A$ 和 $\bv A'$ 分别是 $\mathscr A \in \mathcal L(V)$ 在基底~$\hat e$ 和 $\hat e'$ 下的矩阵, 那么:
	\begin{equation*}
		\det \bv A = \det \bv A'\,, \quad
		\tr \bv A = \tr \bv A'\,.
	\end{equation*}
\end{theorem}

如此我们可以径直称 $\det \mathscr A := \det \bv A$ 为线性算子的\indexbf{行列式} (determinent), 而不必指出基底;
而 $\tr \mathscr A := \tr \bv A$ 为线性算子的\indexbf{迹} (trace).

\section{不变子空间与特征向量}

\begin{definition}[正交等方算子组]
	设 $V$ 是线性空间, $\forall i \in m$, $\mathscr P_i \in \mathcal L(V)$. 
	倘若 $\{\mathscr P_i\}_{i \in m}$ 满足
	\begin{equation*}
		\forall i,j \in m, \quad
			\mathscr P_i \mathscr P_j = \delta_{ij} \mathscr P_i\,,
	\end{equation*}
	则称其为一个\indexbf{正交等方算子组}, 其中每一个算子都称为\indexbf{等方算子}或\indexbf{投影} (projection).
	它们对应的矩阵组 $\{\bv P_i\}_{i \in m}$ 则称为\indexbf{正交等方矩阵组}. 
	
	如果正交等方算子组~$\{\mathscr P_i\}_{i \in m}$ 还满足 $\sum_{i \in m} \mathscr P_i = \id_V$, 那么称它是一个\indexbf{完备正交组}.
\end{definition}

以下事实很容易得到验证:
一个正交等方算子组中的算子满足 $\mathscr P_i^2 = \mathscr P_i$, 
$\mathscr P_i \mathscr P_j = \mathscr O$ 若 $i \neq j$; 
而且如果 $i \neq j$, $\mathscr P_j(V) \subseteq \ker \mathscr P_i$.


\begin{theorem}[线性空间用正交完备算子组的像直和分解]
	设 $V$ 是线性空间, $\{\mathscr P_i\}_{i \in m}$ 是正交完备算子组.
	\begin{equation*}
		V = \bigoplus_{i \in m} \mathscr P_i(V)\,.
	\end{equation*}
\end{theorem}
\begin{proof}
	$\forall \bv x \in V$, 
	\begin{equation*}
		\bv x = \id_V(\bv x) = \sum_{i \in m} \mathscr P_i \bv x\,,
	\end{equation*}
	注意到 $\mathscr P_i \bv x \in \mathscr P_i(V)$, 从而 $V = \sum_{i \in m} \mathscr P_i(V)$.

	现在要证它是一个直和. $\forall i \in m\; \forall \bv x \in \mathscr P_i(V) \cap \left( \sum_{j \in m;\; j \neq i} \mathscr P_j(V)\right)$, 
	\begin{equation*}
		\exists \bv (x_j)_{j \in m} \in V^m  \;\text{s.t.\ }
			\bv x = \mathscr P_i \bv x_i = \sum_{j \in m;\; j \neq i} \mathscr P_j \bv x_j\,,
	\end{equation*}
	将上式两侧同时作用算子~$\mathscr P_i$ 即有
	\begin{equation*}
		\mathscr P_i \bv x = \mathscr P_i^2 \bv x_i = \mathscr P_i \bv x_i = \bv x = 
		\sum_{j \in m;\; j \neq i} \mathscr P_i \mathscr P_j (\bv x_j) = \bv 0\,.
	\end{equation*}
\end{proof}

\begin{definition}[不变子空间]
	设 $V$ 是线性空间, $U$ 是其子空间, $\mathscr A \in \mathcal L(V)$. 倘若 $\mathscr A(U) \subseteq U$, 我们称 $U$ 是线性算子~$\mathscr A$ 的一个\indexbf{不变子空间}.
\end{definition}

\chapter{内积空间}

\chapter{张量}
	\label{chapter:张量}


\appendix
\chapter{置换}
\section{置换群}

置换群~$S_n$ 的定义已在正文的~\ref{section:群} 中给出, 我们在此重复一遍: 
有限集~$n\in \mathbb N_+$ 上的\indexbf{置换群}~$\indexfm[S n]{S_n}$ 定义为 $n^n$ 中的双射的集合, 乘法定义为函数的复合, 单位元是 $\id_n$. 

不难证明 $\card S_n = \mathrm P^n_n = n!$.

设 $\pi \in S_n$. 
元素~$i,j \in n$ 如果满足 $\exists k \in \mathbb N$, $\pi^k(i) = j$, 那么我们称 $i$ 和 $j$ 是 \emphbf{$\pi$-等价的}%
	\index{pi-等价@$\pi$-等价}%
	.
不难证明这是等价关系, 而且把 $n$ 分成等价类 $\{n_k\}_{k \in p}$, $p \in \mathbb N_+$.
每个等价类~$n_k$ 称为置换~$\pi$ 的\indexbf{轨道}, 其元素个数~$\ell_k := \card n_k$ 称为\emphbf{轨道~$n_k$ 的长度}%
	\index{轨道长度}%
	.

为方便, 我们定义 $\pi_k$ 为:
\begin{equation*}
	\pi_k(i) = 
	\begin{cases}
		\pi(i) & i \in n_k \\
		\id_n & i \notin n_k
	\end{cases}\,,
\end{equation*}
我们得到了 $\pi = \prod_{k \in p} \pi_k$, 这是轨道间不相交的结论.

若置换~$\pi$ 至多只有一个轨道的长度大于 $1$ i.e.\ $
	\exists k_0 \in p \,\forall k \in p (k \neq k_0 \to \ell_k = 1)
$,  我们称这个置换为\indexbf{轮换}或\indexbf{循环}, 并径直称 $\ell_{k_0}$ 为这个轮换的\emphbf{长度}%
	\index{轮换长度}%
	.
轮换~$\pi$ 可记为 $(\pi^k (i))_{k \in \ell_{k_0}}$ 其中 $i \in n_k$. 不难验证 $i$ 在 $n_k$ 中的选择无关紧要. 我们记 $\id_n = (0)$. 
当 $\ell_{k_0} = 2 $ 时, 我们也唤轮换~$\pi$ 为对换.

我们称两个轮换\indexbf{不相交}, 如果它们的长度 $\leq 2$, 且最长轨道不相交.

以上的叙述可以总结为:
\begin{theorem}\label{theorem:置换的轮换分解}
	置换群~$S_n$ 中的每一个置换, 要么是 $\id_n$, 要么存在唯一的不相交长度~$\leq 2$ 的轮换的集合 $\{\pi_k\}_{k \in p}$, 使得 $\pi = \prod_{k \in p} \pi_k$.
\end{theorem}

\begin{theorem}\label{theorem:置换的对换分解}
	置换群~$S_n$ 中的每一个置换~$\pi$ 都可写为对换 $(\sigma_k)_{k \in q}$ 的乘积, 即
	$\pi = \prod_{k \in q} \sigma_k$%
		\footnote{注意, 这时不可对调 $\sigma_k$ 间的位置. }%
		.
	
	而且, 倘若存在 $(\sigma'_k)_{k \in q'}$ 也满足 $\pi = \prod_{k \in q'} \sigma'_k$, 那么 $q \equiv q' \pmod 2$.
\end{theorem}
\begin{proof}
	因为每个长为 $r$ 的轮换都可写成:
	\begin{equation*}
		(\pi^k(i))_{k \in r} = \prod_{k \in r} (i, \pi^{r - k}(i)),\,
	\end{equation*}
	则由定理~\ref{theorem:置换的轮换分解}, 每一个置换都可以写成对换的乘积.

	我们先证明, 若 $\id_n = \prod_{k \in q} \sigma_k$, 其中 $\forall k \in q$, $\sigma_k$ 是对换, 那么 $q \equiv 0 \pmod 2$.

	我们用递归的方法证明这点. 
	设 $\sigma_{q - 1} = (S, T)$, $S, T \in n$. 
	为方便, 我们记 $p := \max\{ k \mid \sigma_k = (S,t), t\in n\}$.
	令 $(\sigma'_k)_{k \in q} := (\sigma_k)_{k \in q}$, $\sigma'_p := (S, t)$.

	除非出现以下情况:
	\begin{conditionlist}[label=\alph*)]
		\item $p = 0$. \label{item:矛盾}
		\item $p \neq 0$ 但 $\sigma'_{p - 1} = (S, t)$. \label{item:消去}
	\end{conditionlist}
	否则, 不断重复下列过程:
	\begin{conditionlist}[label=\arabic*)]
		\item 如果 $\sigma_{p - 1} = (S, r)$, 其中 $r \neq t$:
			由于
			\begin{equation*}
				(S, r)(S, t) = (S, t, r) = (t, r, S) 
				= (t, S)(t, r) = (S, t)(t, r)\,,
			\end{equation*}
			那么重新令
			\begin{equation*}
				\sigma'_{p - 1} = (S, t),\; 
				\sigma'_p = (t, r)\;
				\text{其他不变,} 
			\end{equation*}
			将仍然满足 $\id_n = \prod_{k \in q} \sigma'_k$. 
			执行~\ref{item:每次循环结束}. 
		\item 如果 $\sigma_{p - 1} = (t, r)$:
			由于
			\begin{equation*}
				(t, r)(S, t) = (t, S, r) = (r, t, S) 
				= (r, S)(r, t) = (S, r)(r, t)\,,
			\end{equation*}
			那么重新令
			\begin{equation*}
				\sigma'_{p - 1} = (S, r),\; 
				\sigma'_p = (r, t)\;
				\text{其他不变,} 
			\end{equation*}
			将仍然满足 $\id_n = \prod_{k \in q} \sigma'_k$. 
			执行~\ref{item:每次循环结束}. 
		\item 如果 $\sigma_{p - 1} = (r, u)$, 其中 $\{r, u\} \cap \{S, t\} = \varnothing$:
			由于
			\begin{equation*}
				(r, u)(S, t) = (S, t)(r, u)\,,
			\end{equation*}
			那么重新令
			\begin{equation*}
				\sigma'_{p - 1} = (S, t),\; 
				\sigma'_p = (r, u)\;
				\text{其他不变,} 
			\end{equation*}
			将仍然满足 $\id_n = \prod_{k \in q} \sigma'_k$. 
			执行~\ref{item:每次循环结束}. 
		\item 重新令 $p := \max\{ k \mid \sigma'_k = (S,t), t\in n\}$ 以及 $\sigma'_p = (S, t)$. \label{item:每次循环结束}
	\end{conditionlist}
	直到满足~\ref{item:矛盾} 或~\ref{item:消去} 为止. 
	这个循环将总是能在有限次结束, 因为每次 $p$ 都减小了 $1$.

	当过程到结束时, 如果满足~\ref{item:矛盾}, 那么 $\prod_{k \in q} \sigma'_k (S)= (S, t) \prod_{k \in q, \, k \neq 0} \sigma'_k(S) = (S, t)(S) = t \neq S$, 与 $\id_n(S) = S$ 矛盾; 
	那么, 只可能是满足~\ref{item:消去}, 此时因 $(S, t)(S, t) = \id_n$, 将它们消去, 我们得到了 $\id_n$ 的 $q' = q - 2$ 个对换的分解. 
	
	重复这样的过程直到 $q' = 0$ 或 $q' = 1$ 为止, 而后者是不可能的, 因为 $\id_n$ 永远不可能等于对换. 所以: $q \equiv 0 \pmod 2$.

	最后, 我们断言, 任意置换和它的逆分解成的对换数目之和是偶数. 
	即, 考虑 $\pi$ 的两种分解, $\pi = \prod_{k \in q} \sigma_k = \prod_{k \in q'} \sigma'_k$, 
	那么 $\id_n = \pi \pi^{-1} = \prod_{k \in q} \sigma^{-1}_k \prod_{k \in q'} \sigma'_k = \prod_{k \in q} \sigma_k \prod_{k \in q'} \sigma'_k$, 由前 $q + q' \equiv 0 \pmod 2$.
\end{proof}

据此我们把置换群的元素分为\indexbf{奇置换} (分解得到奇数个对换) 和\indexbf{偶置换} (分解得到偶数个对换), 并引入置换的\indexbf{符号}或\indexbf{奇偶性}~$\indexfm[epsilon pi]{\varepsilon_\pi}$, 其值对于偶置换是 $1$, 奇置换是 $0$.

所有偶置换的集合 $\indexfm[A n]{A_n}$ 是 $S_n$ 的子群.
\chapter{矩阵和行列式}
以下只是一些定义的罗列, 与一些术语的规定, 矩阵与行列式的性质则散见于正文中.
如果读者感到陌生, 可参阅任意一本初等线性代数教材, 如\cite{kostrikin1982introduction}.

\section{矩阵}
\begin{definition}[矩阵]
	设 $\mathbb F$ 是一个域. 
	将 $\{a_{ij}\}_{i \in m,\, j \in n} \in 2^{\mathbb F}$ ($n, m \in \mathbb N_+$) 
	排成一个长方形的表: 
	\begin{equation}\label{equation:矩阵}
		\bv A := \indexfm[a ij i in m j in n]{(a_{ij})_{i \in m,\, j \in n}} = 
		\begin{pmatrix}
			a_{00} & a_{01} & \cdots & a_{0, n-1} \\
			a_{10} & a_{11} & \cdots & a_{1, n-1} \\
			\vdots & \vdots & \ddots & \vdots     \\
			a_{m-1, 0} & a_{m-1, 1} & \cdots & a_{m-1, n-1}
		\end{pmatrix}\,.
	\end{equation}

	式~\eqref{equation:矩阵} 定义的 $\bv A$ 被称为 $\mathbb F$ 上的 $m \times n$ 的\indexbf{矩阵}, $m \times n$ 被称为它的尺寸或大小, $\{a_{ij}\}_{i \in m,\, j \in n}$ 是它的\indexbf{元素}. 所有 $\mathbb F$ 上的 $m \times n$~矩阵的集合记为 $\indexfm[M m times n F]{M_{m \times n}(\mathbb F)}$. 
\end{definition}

元素全为 $0$ 的矩阵记为 $\indexfm[O]{\bv O}$, 有时为了强调它的尺寸, 将之写在右下角 i.e.\ $\indexfm[O m times n]{\bv O_{m \times n}}$.
 
通常, 我们称 $1 \times n$ 或 $n \times 1$ 的矩阵为 $n$~维\indexbf{向量}%
	\index{n维向量@$n$~维向量}%
	, 前者是\indexbf{行向量}, 后者是\indexbf{列向量}. 
列向量的集合也可记为 $\mathbb F^n$, 即认为它是 $\mathbb F$ 的 $n$~次 Cartesian~幂的元素.
但是, 当上下文明确时, 我们不特意在术语上区分行向量和列向量.

我们也常把矩阵写成列向量组的形式, 即
\begin{equation}\label{equation:矩阵的列向量组表示}
	\bv A := \indexfm[x j j in n]{(\bv x_j)_{j \in n}}, \qquad 
	\forall j \in n \big(
			\bv x_j \in \mathbb F_n \big)\,. 
\end{equation}

设矩阵~$\bv A$ 的尺寸为 $n \times n$, 我们称其为 $n$~维\indexbf{方阵}, 其集合记为 $\indexfm[M n F]{M_n(\mathbb F)}$.

\begin{definition}[对角矩阵]
	若方阵~$\bv A$ 的元素只有对角线上的元素非零 i.e.\ $ a_{ij} \neq 0  \to i = j$, 称其为\indexbf{对角矩阵}, 记为 $\indexfm[diag a ii i in n]{\diag(a_{ii})_{i \in n}}$. 特别地 $\indexfm[I]{\bv I} := \diag(1)_{i \in n}$ 称为 $n$~维\indexbf{单位阵}.
\end{definition}

\begin{definition}[转置]
	设 $\bv A = (a_{ij})_{i \in m,\, j \in n} \in M_{m \times n}(\mathbb F)$.
	我们称 $(a_{ji})_{j \in n,\, i \in m} \in M_{n \times n}(\mathbb F)$ 为矩阵~$\bv A$ 的\indexbf{转置}, 记为 $\indexfm[A T]{\bv A^\mathrm T}$.
\end{definition}

\begin{definition}[和]
	在 $M_{m \times n}$ 上定义\emphbf{和}%
		\index{矩阵的和}%
	:
	\begin{equation*}
		\bv A + \bv B 
			= (a_{ij})_{i \in m,\,j \in n} + (b_{ij})_{i \in m,\,j \in n}
			= (a_{ij} + b_{ij})_{i \in m,\,j \in n}\,.
	\end{equation*}
\end{definition}

不难验证, $(M_{m \times n}, +, \bv O_{m \times n})$ 构成了一个Abelian群.

\begin{definition}[积]
	在 $M_{m \times \ell}(\mathbb F)$ 和 $M_{\ell \times n}(\mathbb F)$ 间定义\emphbf{积}%
		\index{矩阵的积}%
	($\mathord{\cdot} \colon M_{m \times \ell}(\mathbb F) \times M_{\ell \times n}(\mathbb F) \to M_{m \times n}(\mathbb F)$):
	\begin{equation*}
		\bv A \bv B = (a_{ij})_{i \in m,\,j \in \ell}  (b_{ij})_{i \in \ell,\,j \in n}
			= \left( 
				\sum_{k \in \ell} a_{ik} b_{kj}
			 \right)_{i \in m,\,j \in n}\,.
	\end{equation*}
\end{definition}
	
由域的性质, 我们能验证矩阵的乘法运算是结合的, 而且满足对和的分配律.

\begin{definition}[逆]
	设方阵~$\bv A \in M_n(\mathbb F)$. 
	若 $\exists \bv B \in M_n(\mathbb F)$, s.t.\ $\bv B \bv A = \bv A \bv B = \bv I$ 则称其为 $\bv A$ 的\indexbf{逆}, 并记为 $\indexfm[A -1]{\bv A^{-1}}$, 同时称 $\bv A$ 是可逆的.
\end{definition}

\begin{definition}[相似]
	设 $\bv A, \bv A' \in M_n(\mathbb F)$. 
	如果 $\exists \bv B \in M_n(\mathbb F)$ s.t.\ $\bv B$ 可逆且有 $\bv A' = \bv B^{-1} \bv A \bv B$, 那么我们称 $\bv A$ 和 $\bv A'$ \indexbf{相似}, 记为 $\bv A \sim \bv A'$.
\end{definition}

不难看出, 相似关系是一个等价关系. 

\begin{definition}[迹]
	设 $\bv A = (a_{ij})_{i,j \in n} \in M_n(\mathbb F)$. 
	方阵~$\bv A$ 的迹定义为 $\indexfm[tr]{\tr} \bv A := \sum_{i \in n} a_{ii}$.
\end{definition}

\begin{theorem}[迹的交换性]
	设 $\bv A, \bv B \in M_n(\mathbb F)$. $\tr (\bv A \bv B) = \tr (\bv B \bv A)$.
\end{theorem}
\begin{proof}
	\begin{equation*}
		\bv A \bv B = \left(\sum_{k \in n} a_{ik} b_{kj} \right)\,, \quad 
		\bv B \bv A = \left(\sum_{k \in n} b_{ik} a_{kj} \right)\,.
	\end{equation*} 
	
	从而
	\begin{equation*}
		\tr (\bv A \bv B) 
		= \sum_{i \in n} \sum_{k \in n} a_{ik} b_{ki} 
		= \sum_{k \in n} \sum_{i \in n} b_{ki} a_{ik}
		= \tr (\bv B \bv A)\,.
	\end{equation*}
\end{proof}

\section{行列式}
行列式的公理化构造我们在定义~\ref{definition:行列式的公理化构造} 中已经给出了. 
我们这里做出一个不加解释的定义%
	\footnote{几何解释可见于\cite{kostrikin1982introduction}.}%
, 并不加证明地给出它的一些性质.

\begin{definition}[行列式]\label{definition:行列式}
	方阵 $\bv A \in M_n(\mathbb F)$ 的\indexbf{行列式} $\det \bv A := 
		\indexfm[a ij i j in n]{ \vert a_{ij} \vert _{i,j \in n} }$ 定义为:
	\begin{equation*}
		\det \bv A = \vert a_{ij} \vert _{i,j \in n}
			= \sum_{\pi \in S_n} \varepsilon_\pi \prod_{i \in n} a_{i, \pi(i)}
	\end{equation*}
\end{definition}

\begin{theorem}[行列式的反对称性]
	设 $\bv A = (\bv a_j)_{j \in n}$, 那么 $\forall \pi \in S_n$, $|\bv a_{\pi(j)}|_{j \in n} = \varepsilon_{\pi} \det \bv A$.
\end{theorem}

\begin{theorem}
	$\det \bv A = \det \bv A^{\mathrm T}$.
\end{theorem}

\begin{theorem}[行列式的线性~1]
	设 $\bv A = (\bv a_j)_{j \in n}$, $\bv A' = (\bv a'_j)_{j \in n}$, 
	其中 $j \neq j_0$ 时 $\bv a'_j = \bv a_j$; 
	但 $\bv a'_{j_0} = \lambda \bv a_{j_0}$, $\lambda \in \mathbb F$. 
	\begin{equation*}
		\det \bv A' = \lambda \det \bv A\,.
	\end{equation*}
\end{theorem}

\begin{theorem}[行列式的线性~2]
	设 $\bv A = (\bv a_j)_{j \in n}$, $\bv A' = (\bv a'_j)_{j \in n}$, 
	其中 $j \neq j_0$ 时 $\bv a'_j = \bv a_j$.
	\begin{equation*}
		\det \bv A + \det \bv A' = |\bv a''_j|_{j \in n}\,,
	\end{equation*}
	其中 $j \neq j_0$ 时 $\bv a''_j = \bv a_j$; $\bv a''_{j_0} = \bv a'_{j_0} + \bv a_{j_0}$.
\end{theorem}

设 $I, J \in \mathscr P(n)$, 而 $|I| = |J|$. 
记 $\indexfm[M IJ]{M_{IJ}} := |a_{k\ell}|_{k \in n - I;\, \ell \in n - J}$ 为 $\bv A$ 的\indexbf{子式} (minor), 而\indexbf{代数子式} (cofactor) 则是定义为 $\indexfm[A IJ]{A_{IJ}} := (- 1)^{\sum_{i \in I} i + \sum_{j \in J}j} M_{IJ}$.
如果 $|I| = |J| = 1$, 那么我们称 $\indexfm[M ij]{M_{ij}} := M_{\{i\}\{j\}}$ 为\indexbf{首子式} (first minor), 对应的代数子式也可记为 $\indexfm[A ij]{A_{ij}}$. 
那么, 以下的定理将给出一种计算行列式的递推方法:

\begin{theorem}[行列式按行 (列) 展开]
	$\forall k \in n$, 
	\begin{equation*}
		\det \bv A = \sum_{i \in n} a_{ik} A_{ik}\,;
		\quad
		\det \bv A = \sum_{j \in n} a_{kj} A_{kj}\,.
	\end{equation*}
\end{theorem}

以下的定理确保了定义~\ref{definition:行列式的公理化构造} 和定义~\ref{definition:行列式} 的一致性, 并且这是唯一的构造方式.

\begin{theorem}[行列式的唯一性]
	设 $\mathcal D \in \mathcal L_n(\mathbb F^n; \mathbb F)$ (即 $\mathcal D \colon M_n(\mathbb F) \to \mathbb F$ 而且是 $n$~重线性的). 
	倘若 $\mathcal D$ 还是反对称的, 即 $\forall \pi \in S_n$, $\mathcal D (\bv x_{\pi(j)})_{j \in n} = \varepsilon_\pi \mathcal D (\bv x_j)_{j \in n}$, 
	那么 $\forall \bv A \in M_n(\mathbb F)$, $\mathcal D (\bv A) = \mathcal D(\bv I) \det \bv A$.
\end{theorem}

它的证明需要利用线性和反对称性, 利用行变换将矩阵转换为对角的. 
取 $\mathcal D(\bv I) = 1$, 我们就得到了 $\mathcal D = \det$.

取 $\mathcal D (\bv I) = \det \bv B$ 推出的一个重要的性质是:
\begin{theorem}[行列式的积]
	$\forall \bv A, \bv B \in M_n(\mathbb F)$, 
	$\det (\bv A \bv B) = \det \bv A \det \bv B$.
\end{theorem}



\chapter{多项式}
\section{多项式环}
\begin{definition}[多项式环]
	设 $R$ 是一个交换环, $\langle X \rangle := \{ X^n \mid n \in \mathbb N\}$ 是 $X$ 生成的幺半群, 记 $I := X^0$. 
	若形如 $\indexfm[P X]{P(X)} := \sum_{i \in \mathbb N} p_i X^i$ 的形式 (称为\indexbf{多项式}, 其中只有有限个 $p_i$ 非零) 的集合%
		\footnote{$R^{< \mathbb N} := \bigcup_{n \in \mathbb N} R^n$}%
		:
	\begin{equation*}
		R[X] :=
		\left\{ 
			\sum_{i \in \mathbb N} p_i X^i
			\middle|
			(p_i)_{i \in \mathbb N} \in R^{< \mathbb N}
		\right\}
	\end{equation*}
	上定义了加法:
	\begin{equation*}
		P(X) + Q(X) 
			:= \sum_{i \in \mathbb N} p_i X^i + \sum_{i \in \mathbb N} q_i X^i 
				= \sum_{i \in \mathbb N} (p_i + q_i) X^i
	\end{equation*}
	和乘法:
	\begin{equation*}
		P(X)Q(X) 
			:= \left( \sum_{i \in \mathbb N} p_i X^i \right) 
				\left( \sum_{i \in \mathbb N} q_i X^i \right) 
				= \sum_{\ell \in \mathbb N} \sum_{i + j = \ell} p_i q_j X^\ell\,,  
	\end{equation*}
	那么, 我们称 $\indexfm[R X]{R[X]}$ 是 $R$ 上变元~$X$ 的\indexbf{多项式环}.
	在变元是明确的时候, 多项式~$P(X)$ 也简记为 $P$.

\end{definition}

记 $\indexfm[deg P X]{\deg P(X)} := \max \{n \mid p_n \neq 0\}$ 为多项式~$P(X)$ 的\indexbf{次数}.
而 $(p_n)_{n \in \deg P(X) + 1}$ 是多项式的\indexbf{系数}, 其中 $p_0$ 是\indexbf{常数项}, 而 $p_{\deg P(X)}$ 是\indexbf{最高次项系数}或\indexbf{首项系数}.

所有系数都为 $0$ 的多项式被称为\indexbf{零多项式}, 次数为 $1$ 的多项式被称为\indexbf{线性多项式}.

不难验证, $R[X]$ 的单位元和零元分别是 $R$ 的单位元和零元.

以下给出一些不难证明的定理, 如果读者感到困难, 请翻阅参考资料\cite{kostrikin1982introduction}:
\begin{theorem}\label{theorem:多项式的运算与次数}
	$\forall P(X), Q(X) \in A[A]$, 
	$\deg \big(P(X) + Q(X)\big) \leq \max \{\deg P(X), \deg Q(X)\}$,
	$\deg \big( P(X) Q(X) \big) \leq \deg P(X) + \deg Q(X)$.
\end{theorem}

\begin{theorem}
	如果 $A$ 是整环, $A[X]$ 也是整环%
		\footnote{考虑到 $A$ 是 $A[X]$ 的子环, 逆命题也成立.}
	.
\end{theorem}

\begin{theorem}[多项式环的泛性]
	设 $R$ 是一个交换环, $A$ 是 $R$ 的子环.
	$\forall t \in R$, $\exists ! \varPi_t \in \Hom (A[X], R)$, s.t.\ 
	$\varPi_t (X) = t \wedge\; \forall a \in A \big( \varPi_t(a) = a \big)$.
\end{theorem}
\begin{proof}
	不难验证
	\begin{equation*}
		\varPi_t \colon \sum_{n \in \mathbb N} p_n X^n
			\mapsto \sum_{n \in \mathbb N} p_n t^n  
	\end{equation*}
	即所求.
\end{proof}

我们把这样的 $\varPi_t (P) =: P(t)$ 称为 $P$ 在 $X = t$ 时的取值, 或者说用 $t$ 替换 $X$. 当两个多项式不相等时, 它们的值却可能相等. 

\begin{definition}[代数元和超越元]
	若 $t \in R$ 满足 $\exists P \in A[X]$ s.t.\ $\exists n \in \mathbb N (p_n \neq 0) \wedge P(t) = 0$, 那么 $t$ 是 $A$ 上的一个\indexbf{代数元}.
	若 $t \in R$ 满足 $\varPi_t$ 是单的, 那么我们称其为 $A$ 上的一个\indexbf{超越元}.
\end{definition}

对于 $A = \mathbb Q$, $R = \mathbb C$ 的情况, 代数元和超越元又被称为\indexbf{代数数}和\indexbf{超越数}.

类似于整数的带余除法, 我们可以建立整的多项式环上的带余除法理论.

\begin{theorem}[多项式的带余除法]\label{theorem:多项式的带余除法}
	设 $A$ 是整环, $F(X) \in A[X]$, 其首项系数在 $A$ 中可逆. 
	$\forall G(X) \in A[X]$, $\exists! Q(X), R(X) \in A[X]$ s.t.\ 
	\begin{equation*}
		F(X) = Q(X) G(X) + R(X)\; \wedge \deg R(X) < \deg G(X)\,.
	\end{equation*}
\end{theorem}
\begin{proof}
	设 $F(X) = \sum_{i \in n^+} f_i X^i$, $G(X) = \sum_{j \in m^+} g_j X^j$, 其中 $n = \deg F(X)$, $m = \deg G(X)$. \footnote{这里排除了零多项式, 事实上, 它的情况是非常简单的, 只需让 $Q(X) = R(X) = 0$ 即可.} 

	我们采取归纳法证明这样的 $Q(X)$, $R(X)$ 的存在性:
	$n = 0$ 时, 倘 $m > 0$, 则令 $Q(X) = 0$, $R(X) = F(X)$; 
	倘 $m = 0$, 则令 $R(X) = 0$, $Q(X) = f_0 g_0^{-1}$.

	若 $n > 0$, 倘 $m > n$, 则令 $Q(X) = 0$, $R(X) = F(X)$ 即可;
	倘 $m \leq n$, 记
	\begin{equation*}
		\bar F(X) := F(X) - f_0 g_0^{-1} X^{n - m} G(X)\,.
	\end{equation*}

	因 $\deg \bar F(X) < \deg F(X)$, 如若 $\bar F(X)$ 满足可找到 $\bar Q(X)$ 和 $R(X)$ s.t.\ $\bar F(X) = \bar Q(X) G(X) + R(X)$, 则令 $Q(X) = f_0 g_0^{-1} X^{n - m} + \bar Q(X)$. 
	这里的 $Q(X)$, $R(X)$ 即所要找的.

	综上, $\forall n \in \mathbb N$ 都成立.

	现在我们还需要证明唯一性.

	倘若 $F(X) = Q(X) G(X) + R(X) = Q'(X) G(X) + R'(X)$, 那么
	\begin{equation}\label{equation:商和余式唯一}
		[Q(X) - Q'(X)] G(X) = R(X) - R'(X)
	\end{equation}
	由定理~\ref{theorem:多项式的运算与次数}, 再考虑到 $A$ 是一个整环, 假设 $Q(X) \neq Q'(X)$, $R(X) \neq R'(X)$ 我们能得到
	\begin{equation*}
		\deg \big(R(X) - R'(X)\big) = \deg \big( Q(X) - Q'(X)\big) + \deg G(X)\,.
	\end{equation*}

	从而我们得到 $\max\{\deg R(X), \deg R'(X) \} \geq \deg(R(X) - R'(X)) \geq G(X)$, 这和 $\deg R(X) < \deg G(X)$ 矛盾. 从而, 要么 $Q(X) = Q'(X)$, 要么 $R(X) = R'(X)$, 而这两者因式~\ref{equation:商和余式唯一} 能互相推出.
\end{proof}

在整的多项式环中, 由首项可逆的 $G(X)$ 和 $F(X)$ 得到 $F(X) = Q(X) G(X) + R(X)$ 的运算称为\indexbf{多项式的带余除法} (polynomial long division). 
这里 $F(X)$ 被称为\indexbf{被除式} (dividend), $G(X)$ 称为\indexbf{除式} (divisor); 得到的 $Q(X)$ 称为\indexbf{商} (quotient) 而 $R(X)$ 称为\indexbf{余式} (remainder).

倘若余式 $R(X) = 0$, 则称 $G(X)$ \indexbf{整除} $F(X)$, 或 $F(X)$ 被 $G(X)$ 整除, 此时 $G(X)$ 被称为 $F(X)$ 的一个\indexbf{因式} (factor), 而 $F(X)$ 则是 $G(X)$ 的一个\indexbf{倍式} (multiple).

\section{多项式的根}
\backmatter
\nocite{*} % 这个表示列出所有没有在文中被引用的参考文献
\printbibliography[heading=bibliography, title={参考文献}]

\indexprologue{这里列出了笔记中出现的重要符号.}
\printindex[symbol]


\printindex
\end{document}