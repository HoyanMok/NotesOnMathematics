\documentclass{article}%默认英文, 中文请使用:{ctexart}
\usepackage{amsmath} %数学
	\makeatletter % '@' now normal "letter"
	\@addtoreset{equation}{section}
	\makeatother  % '@' is restored as "non-letter"
	\renewcommand\theequation{\oldstylenums{\thesection}%
					-\oldstylenums{\arabic{equation}}}
\usepackage{amsthm} %定义、证明、定理等
	\theoremstyle{plain}
		\newtheorem{axion}{Axion} %公理
		\newtheorem{theorem}{Theorem}[section] %定理
		\newtheorem{corollary}{Corollary} %推论
		\newtheorem{lemma}{Lemma} %引理
	\theoremstyle{definition}
		\newtheorem{definition}{Definition}[section] %定义
		\newtheorem{proposition}{Proposition} %命题
	\renewcommand{\proofname}{\textbf{Proof}}
\usepackage{amssymb} %mathematical symbols
\usepackage{mathrsfs} %allow you to use \mathscr to display captial letters in script style
\usepackage{makeidx} %Generate index
	\makeindex
\usepackage{enumerate} %编号
\usepackage{tikz}
\usepackage[colorlinks=true,bookmarks=true]{hyperref}%hyperlinks for reference, table of contents and indexes
\hypersetup{linkcolor=[rgb]{1,0.27,0},bookmarksopen = true}%More options read texdoc hyperref

% \usepackage[style=gb7714-2015]{biblatex}
% \addbibresource{<File>.bib}

\newcommand*{\indexbf}[1]{\textit{\textbf{#1}}\index{#1}}
\newcommand*{\indexfm}[2]{#1\index{  Symbols@Symbols used!#2@$#1$}}

\def\reflect#1{{\setbox0=\hbox{#1}\rlap{\kern0.5\wd0
  \special{x:gsave}\special{x:scale -1 1}}\box0 \special{x:grestore}}}
\def\XeTeX{\leavevmode
  \setbox0=\hbox{X\lower.5ex\hbox{\kern-.15em\reflect{E}}\kern-.1667em \TeX}%
  \dp0=0pt\ht0=0pt\box0 }

\newcommand*{\basis}[1]{\hat{\boldsymbol{#1}}}%基底
\newcommand*{\bv}{\boldsymbol}%向量加粗
\newcommand*{\pdiff}[2]{\frac{\partial #1}{\partial #2}}
\newcommand*{\diff}[2]{\frac{\mathrm{d} #1}{\mathrm{d} #2}}
\newcommand\dif{\mathop{}\!\mathrm{d}}
\DeclareMathOperator{\tg}{tg}
\DeclareMathOperator{\ctg}{ctg}
\DeclareMathOperator{\arctg}{arctg}
\DeclareMathOperator{\sh}{sh}
\DeclareMathOperator{\ch}{ch}
\DeclareMathOperator{\dom}{dom}
\DeclareMathOperator{\ran}{ran}
\DeclareMathOperator{\interior}{int}

\title{Point Set Topology}
\author{Hoyan Mok}
\begin{document}
\maketitle
\tableofcontents
\newpage
\section{Topological Spaces and Continuous Mappings}
\subsection{Metric Space}
\begin{definition}\label{metric_def}
function
\begin{align}\label{metric}
	\indexfm{d}{d}\colon X^2\to\mathbb{R}
\end{align}
$\forall x_1,x_2,x_2\in X$ satisfied: 
\begin{enumerate}[a)]
	\item	$d(x_1,x_2)=0\Leftrightarrow x_1=x_2$;
	\item	$d(x_1,x_2)=d(x_2,x_1)$ (symmetry);
	\item	$d(x_1,x_3)\leqslant d(x_1,x_2)+d(x_2,x_3)$ (Triangle inequality),
\end{enumerate}
is called a \indexbf{metric} or \indexbf{distance} in $X$. Such $X$ is said to be equiped with metric $d$, $(X,d)$ is called a \indexbf{metric space}.
\end{definition}

Some examples:
\begin{itemize}
\item $(\mathbb{R}^n;d_p)$, where $d_p(x_1,x_2)=\left(\sum^n_{i=1}\left|x^i_1-x^i_2\right|^p\right)^{1/p}$, while $d_\infty(x_1,x_2)=
\max\limits_{1\leqslant i\leqslant n}
\left|x^i_1-x^i_2\right|$.
\item Similarly we can define metric spaces as $(C[a,b];d_p)$ or $C_p[a,b]$. $d_p(f,g)=\left(
	\int^b_a\left|f-g\right|^p\,\mathrm{d}x
\right)^{\frac{1}{p}}$. $C_\infty$ is called a \textbf{Chebyshev metric}\index{Chebyshev metric}.
\item On class $\mathfrak{\tilde{R}}[a,b]$ over $\mathfrak{R}[a,b]$ similar metric can be defined. Functions are considered of one same class if they are equivalent expect on a set not larger than null set. 
\end{itemize}

\indexbf{Hilbert space} denoted by $(\indexfm{\mathbb{H}}{H};d)$ is defined as:

\begin{align}
	\mathbb{H}=\left\{ 
		x=(x_1,x_2,\cdots) \mid \forall i\in \mathbb{Z}_+\left( 
			\forall x_i\in \mathbb{R}\wedge \sum^\infty_{i=1} x_1^2 < \infty\right)\right\}
\end{align}
equiped with a metirc $d$:

\begin{align}
	d\colon \mathbb{H}^2\to \mathbb{R}; \,x,y\mapsto \sqrt {\sum^\infty_{i=1} (x_i-y_i)^2 }.
\end{align}

To justify this definition, we need to introduce a lemma:
\begin{lemma}\label{Schwarz_inequality}
\begin{align}\label{Schwarz}
	\forall n\in\mathbb{Z}\forall \bv u\in\mathbb{R}^n\forall \bv v\in\mathbb{R}^n\left(
		\sum^n_{i=1} u_iv_i \leq \sqrt{ \sum^n_{i=1} u_i^2}\sqrt{ \sum^n_{i=1} v_i^2}\right)
\end{align}
This is called \indexbf{Schwarz inequality}.
\end{lemma}
\begin{proof}
If $\forall i=1,\cdots,n ( v_i=0)$ the equivalence has already been satisfied, therefore the following considers the situation that $\exists i\in \{1,\cdots n\}( v_i\neq 0)$. $\forall \lambda \in\mathbb R$, the quadratic polynomial of $\lambda$
\begin{align*}
	\sum ^n_{i=1} (u_i+\lambda v_i)^2 = 
		\sum ^n_{i=1} u_i^2 +2\lambda \sum ^n_{i=1} u_iv_i + \lambda^2 \sum ^n_{i=1} v_i^2 \geq 0
\end{align*}
has at most one root. Hence $\Delta \leq 0$ will lead to the inequality \ref{Schwarz}.
\end{proof}

Apply this inequality to $\sum ^n_{i=1} ( u_i+v_i)^2 = 
	\sum ^n_{i=1} u_i^2 +\sum ^n_{i=1}  v_i^2 +2\sum ^n_{i=1}  v_i\sum ^n_{i=1} u_i$ we can get
\begin{align*}
	\sum ^n_{i=1} ( u_i+v_i)^2 \leq 
		\sum ^n_{i=1} u_i^2 +\sum ^n_{i=1}  v_i^2 +
			2\sqrt{ \sum^n_{i=1} u_i^2}\sqrt{ \sum^n_{i=1} v_i^2} = \left(
				\sqrt{ \sum^n_{i=1} u_i^2}+\sqrt{ \sum^n_{i=1} v_i^2} \right)^2 ,
\end{align*}
in which substitude $u_i,v_i$ by $x_i-y_i, x_i+y_i$ will result in triangle inequality. The inequality holds as the $n$ limits to $+\infty$.

\begin{definition}
Let $(X,d)$ be a metric space, the \indexbf{distant} between a non-empty set $\varnothing\neq A\in \mathscr P(X)$ and a point $x$ is defined as:
\begin{align*}
	d(A,x) = \inf \{d(x,y)\mid y\in A\},
\end{align*}
and we let $d(x,A) = d(A,x)$. 
Also, the \indexbf{distant} between two non-empty sets $A,B$ is defined as:
\begin{align*}
	d(A,B) = \inf \{d(x,y)\mid x\in A\wedge y\in B\}.
\end{align*}
\end{definition}

A metric space $(X,d)$ is called \indexbf{discrete} if 
\begin{align*}
	\forall x\in X\left(
		\exists \delta_x\in \mathbb{R}_+ \big(
			\forall y \in X(
				y\neq x\to d(x,y)>\delta_x)\big)\right) .
\end{align*}

\begin{lemma}\label{quadruple_inequality}
If $(X,d)$ is a metric space, then $\forall a,b,u,v $, $\left| d(a,b) - d(u,v) \right| \leq d(a,u)+d(b,v) $.
\end{lemma}
\begin{proof}
	Without loss of generality, we assume that $d(a,b) > d(u,v)$. According to the triangle inequality (see def. \ref{metric_def}), $d(a,b) \leq d(a,u)+d(u,v)+d(v,b)$, which is to prove.  
\end{proof}

\begin{definition}\label{delta_ball}
$\delta\in\mathbb{R}_+$, $a\in X$. Set
\[
	B(a;\delta)=\{x\in X \mid d(a,x)<\delta\}
\]
is then called a \indexbf{ball} with centre $a\in X$, and a radius of $\delta$, or a \indexbf{$\delta$-ball} of point $a$.
\end{definition}

\begin{definition}\label{open_set_metric}
An \indexbf{open set} $G\subset X$ in metric space $(X,d)$ satisfies: $\forall x\in G$, $\exists B(x;\delta)$, s.t. $B(x;\delta)\subset G$.
\end{definition}

\begin{definition}\label{closed_set_metric}
A set $F\subset X$ in metric space $(X,d)$ is said to be a \indexbf{closed set} if its complement $\complement_X(F)$ is open.
\end{definition}

It can be proved that $\varnothing$ and $X$ itself is both open and closed.

\begin{proposition}\label{open_set_pro}
\begin{enumerate}[a)]
\item An infinite union of open sets is an open set.
\item A finite intersection of open sets is an open set.
\item A finite union of closed sets is a closed set.
\item An infinite intersection of closed sets is a closed set.
\end{enumerate}
\end{proposition}
\begin{proof}
\begin{enumerate}[a)]
	\item If open sets $G_\alpha\subset X,\forall\alpha\in A$, $\forall a\in\bigcap\limits_{\alpha\in A}G_\alpha$, $\exists\alpha_0\in A$, $a\in G_{\alpha_0}$, $\exists B(a;\delta)\subset G_{\alpha_0}\subset \bigcap\limits_{\alpha\in A}G_\alpha$.
	\item Open sets $G_1\cup G_2\subset X$, $a\in G_1\cap G_2$, therefore $\exists\delta_1,\delta_2\in\mathbb{R}_+$, $B(a;\delta_1)\subset G_1,B(a;\delta_2)\subset G_2$, without loss of generality, let $\delta_1\geq\delta_2$, 那么$a\in B(a;\delta_1)\cap B(a;\delta_2)=B(a;\delta_2)\subset G_1\cap G_2$.
	\item Just consider $\complement_X
		\left(\bigcap_{\alpha\in A}F_\alpha\right)
		=\bigcup_{\alpha\in A}\complement_X(F_\alpha)$ and a).
	\item Similarly, $\complement_X\left(F_1\cup F_2\right)=\complement_X(F_1)\cap\complement_X(F_2)$.
	
\end{enumerate}
\end{proof}

\subsection{Topological Space}
\begin{definition}\label{topological_space}
We say $X$ is equiped with a \indexbf{topological space} or equiped with \indexbf{topology} if we assigned a $\indexfm{\mathscr{T}}{T}\subset 2^X$, which has got the following propoties:
\begin{enumerate}[a)]
	\item $\varnothing\in\mathscr{T}; X\in\mathscr{T}$.
	\item $\forall\alpha\in A(
		G_\alpha\in\mathscr{T})\to 
			\bigcup\limits_{\alpha\in A}G_\alpha\in\mathscr{T}$.
	\item $G_1\in\mathscr{T}\wedge G_2\in\mathscr{T}\to G_1\cap G_2\in \mathscr{T}$.
\end{enumerate}

Then we call $(X,\mathscr{T})$ a \indexbf{topological space}. Every $G\in \mathscr{T}$ is called an \indexbf{open set}. 
\end{definition}

\begin{definition}\label{induced_topology}
A topology $\mathscr{T}_d$ insisting of the open sets in a metric space $( X, d)$ is called a \indexbf{topology induced by metric} $d$.
\end{definition}

A trivial example of topological space is \indexbf{trivial topology}, which consists only of empty set and the space itself, i.e.\ $\mathscr{T}=\{\varnothing, X\}$. Another trivial example of topological space is \indexbf{discrete topology}, which consists of all the subsets of the space i.e.\ $\mathscr{T} = 2^X$.

A \indexbf{cofinite space} is a base set $X$ equiped with a topology $\mathscr{T}$ defined as follows:
\begin{align}\label{cofinite_space}
	\mathscr{T} = \{ U\in 2^X\mid U=\varnothing \vee \complement_X U \text{ is finite} \}
\end{align}
\begin{proposition}
The set $\mathscr{T}$ under definition \ref{cofinite_space} is a topology.
\end{proposition}
\begin{proof}
\begin{enumerate}[a)]
\item $\varnothing \in \mathscr{T}$, $X\in\mathscr T$.
\item $\forall i \in I \left(
	\left|\complement_X  A_i\right| \in \mathbb N\right) \to
		\forall i_0\in I\left(
			\left| \bigcap_{i\in I} \complement_X A_i\right| \leq 
				\left|\complement_X A_{i_0}\right|\right)$, therefore $\bigcup_{i\in I}  A_i \in \mathscr T$.
\item $\forall A\in \mathscr T \forall B \in \mathscr T(
	A\cap B=\varnothing\in \mathscr{T} \vee
		\complement_X (A\cap B) = \complement_X A\cup \complement_X B \text{ is finite} )$, 
	therefore $\forall A\in \mathscr T \forall B \in \mathscr T(
		A\cap B \in \mathscr T)$.
\end{enumerate}
\end{proof}

Similarly, \indexbf{countable complement space} can be defined.

Let $X$ be equiped with two topology $\mathscr T_1, \mathscr T_2$. $\mathscr T_1\cup \mathscr T_2$ is possibly not a topology of $X$. For example, $\mathscr T _1 = \{ (x,+\infty) \mid x\in\mathbb R\}\cup\{ \varnothing,\mathbb R\}$ and $\mathscr T_2 = \{ (-\infty, y) \mid y\in\mathbb R\}\cup\{ \varnothing,\mathbb R\}$ are both topologies of $\mathbb R$, but there union $T_1\cup T_2$ is not. 

\begin{theorem}
Let $X$ be equiped with two topology $\mathscr T_1, \mathscr T_2$. Their intersection $\mathscr T_1\cap \mathscr T_2$ is also a topology on $X$.
\end{theorem}
\begin{proof}
\begin{enumerate}[a)]
\item $\{\varnothing, X\} \subseteq \mathscr T_1\wedge 
	\{\varnothing, X\} \subseteq  \mathscr T_2 \to 
		\{\varnothing, X\}\subseteq \mathscr T_1\cap \mathscr T_2$.
\item $\forall\alpha\in A(
		G_\alpha\in\mathscr T_1\cap \mathscr T_2)\to 
			\bigcup\limits_{\alpha\in A}G_\alpha\in\mathscr{T}_1 \wedge \bigcup\limits_{\alpha\in A}G_\alpha\in\mathscr{T}_2$. 
\item $\forall G_1\in \mathscr T_1\cap \mathscr T_2 \forall G_2 \in \mathscr T_1\cap \mathscr T_2 \big(
		G_1\cap G_2 \in \mathscr T_1 \wedge G_1\cap G_2\in\mathscr T_2\big)$			
\end{enumerate}
\end{proof}

\begin{definition}\label{matrization}
Let $(X, \mathscr T)$ be a topological space. If there exists a metric $d\colon X^2\to \mathbb R$ s.t.\ $(X, \mathscr T)$ is induced by $d$ then call $(X, \mathscr T)$ a \indexbf{metrizable space}, $(X, d)$ is its \indexbf{metrization}.
\end{definition}

\subsection{Neighbourhood}
\begin{definition}\label{neighbourhood}
Let $(X,\mathscr T)$ be a topological space. A set $U(x)$ is said to be a \indexbf{neighbourhood} of a point $x\in X$ if $\exists G\in \mathscr T(G\subseteq U(x)\wedge x\in G)$. 
If $U(x)\in \mathscr T$, it is called an \indexbf{open neighbourhood}.
Subset class $\{U(x)\subseteq X\mid U(x) \text{ is a neighbourhood of } x\}$ is called the \indexbf{neighbourhood system} of point $x$, denoted by $\mathscr U_x$
\end{definition}

\begin{theorem}\label{open_iff_neighbourhood}
Let $(X,\mathscr T)$ be a topological space, $U$ is a subset of $X$. 
$U$ is an open set \emph{iff} $\forall x\in U$, $U$ is a neighbourhood of $x$.
\end{theorem}
\begin{proof}
The necessity is trivial. 
$\forall x\in U\exists V(x)$ s.t.\ $V(x)$, being a subset of $U$, is an open neighbourhood of $x$. 
By definition of topology, $\bigcup\limits_{ x\in U}V(x) \in \mathscr T$. 
$\forall x\in U( x\in \bigcup\limits_{ x\in U}V(x) ) \to U\subseteq V$, 
while $\forall x\in U (V( x)\subseteq U)\to \bigcup\limits_{ x\in U}V(x) \subseteq U$, 
therefore $U=\bigcup\limits_{ x\in U}V(x) \in \mathscr T$.
\end{proof}

\begin{theorem}
Let $(X,\mathscr T)$ be a topological space, $\mathscr U_x$ is a neighbourhood system of point $x\in X$.
\begin{align*}
	\forall U\in \mathscr U_x\forall V\in \mathscr U_x( U\cap V\in \mathscr U_x)
\end{align*}
\end{theorem}
\begin{proof}
$\forall U\in \mathscr U_x\forall V\in \mathscr U_x\exists U_0\in \mathscr T\exists V_0\in \mathscr T(U_0\subseteq U\wedge V_0\subseteq V\wedge x\in U_0\cap V_0 )$, By definition of topology, $\mathscr T\ni U_0\cap V_0\subseteq U\cap V$. 
\end{proof}

In history topologies were once built on neighbourhood systems. The following theorem shows the way.
 
\begin{theorem}\label{topology_built_on_neighbourhood}
Let $X$ be a set and $\forall x \in X$ a collection of subsets $\mathscr U_x \in \mathscr P(X)$ is appointed, satisfying:
\begin{enumerate}[(1)]
\item $\forall x\in X\big(
	\mathscr U_x \neq \varnothing \wedge
		\forall U\in \mathscr U_x (x\in U)\big)$; 
\item $\forall x\in X\forall U\in\mathscr U_x\forall V\in\mathscr U_x (
	U\cap V\in \mathscr U_x)$;
\item $\forall x\in X\forall U\in\mathscr U_x\forall V\in \mathscr P(X)( 
	U\subseteq V \to V\in \mathscr U_x)$;
\item $\forall x\in X\forall U\in\mathscr U_x\exists V\in\mathscr U_x \big(
	V\subseteq U \wedge \forall y\in V(V\in \mathscr U_y)\big)$,
\end{enumerate}
then there exists only one topology $\mathscr T$ on $X$ s.t.\ $\forall x\in X$, $\mathscr U_x$ is the neighbourhood system of $x$ in $(X,\mathscr T)$.
\end{theorem}
\begin{proof}
Let $\mathscr T = \{G\in \mathscr P(X) \mid \forall x\in G( G\in \mathscr U_x)\}$.
\begin{enumerate}[a)]
\item Obviously $\varnothing \in \mathscr T$. 
	Since the condition (1) and the condition (3) in theorem \ref{topology_built_on_neighbourhood}, $X\in \mathscr T$.
\item Let $A, B\in \mathscr T$. Consider the condition (2) in theorem \ref{topology_built_on_neighbourhood} applied to $x\in A\cap B$.
\item Let $\forall i\in I( G_i \in \mathscr T)$. 
$\forall x\in \bigcup_{i\in I} G_i$, there must exists a $i\in I$ s.t.\ $x\in G_i$ and $G_i \in \mathscr U_x$. 
Since the condition (3) in theorem \ref{topology_built_on_neighbourhood}, $G_i \subseteq \bigcup_{i\in I} G_i$ has implied $\bigcup_{i\in I} G_i \in \mathscr U_x$.
\end{enumerate}
These tells that $\mathscr T$ is a topology on $X$.

The condition (4) in theorem \ref{topology_built_on_neighbourhood} tells that there always exists a $G\subset U$ for all $x\in X$ and $U\in \mathscr U_x$ s.t.\ $G\in \mathscr T$. Therefore $\mathscr U_x$ must be a subset of the neighbourhood system of $x$. 

For all neighbourhood $U$ of $x\in X$, there must be a open neighbourhood subset $G\subseteq U$, which is also a member of $\mathscr U_x$. Since the condition (3) in theorem \ref{topology_built_on_neighbourhood}, $U\in \mathscr U_x$. Therefore the neighbourhood system of $x$ must be a subset of $\mathscr U_x$. 

Therefore, $\mathscr U_x$ is the neighbourhood system of $x$.

Now prove the uniqueness. Let there be another topology $\mathscr T '$.
Since theorem \ref{open_iff_neighbourhood}, 
$\forall U\big(
	G\in \mathscr T' \leftrightarrow 
		\forall x\in G(
			G\in \mathscr U_x)\big)$. 
Therefore $\mathscr T' = \mathscr T$.
\end{proof}

\subsection{Continuous Mappings}
\begin{definition}\label{continuousness}
A mapping $f\colon X\to Y$, where $X$,$Y$ is respectively equiped with topology $\mathscr{T}_X$,$\mathscr{T}_Y$, is said to be \indexbf{continuous} at $x_0\in X$ (let $y_0 = f( x_0 ) \in Y$), if $\forall U( y_0 )$, $\exists U( x_0 )$ s.t. $f( U(x_0) )\subset U( y_0 )$. It is \indexbf{continuous} in $X$ if it is continuous at each point $x\in X$. 
\end{definition}

The set of continuous mappings from $X$ into $Y$ can be denoted by $C(X,Y)$ or $C(X)$ when $Y$ is clear. 

It can be easily proved that an identify function $e_X\colon X\to X$ where $X$ is equiped with a topology $\mathscr T$ is a continuous function.

\begin{theorem}\label{criterion_continuity} (\indexbf{criterion of continuity})

Let $(X, \mathscr T)$, $(Y; \mathscr S)$ be two topological space. A mapping $f \colon X \to Y$ is continuous \emph{iff}
\begin{align*}
	\forall V\in \mathscr S\big(
		\exists U \in \mathscr T(
			U=f^{-1}(V) ) \big).
\end{align*}
\end{theorem}
\begin{proof}
$\rightarrow$: It is obvious if $f  ^{-1}(G_Y) = \varnothing$. 
If $f  ^{-1}(G_Y) \neq \varnothing$ and $\forall x_0 \in f  ^{-1}(G_Y)$, 
since $f \in C(X,Y)$, for $G_Y \in \mathscr S$, $\exists U( x_0) $ s.t $f ( U( x_0)) \subset G_Y$. 
Also notice that $f ( U( x_0)) \subset G_Y \Rightarrow U( x_0) \subset f  ^{-1}(G_Y)$, therefore $f  ^{-1}(G_Y) $ is open (Theorem \ref{open_iff_neighbourhood}).

$\leftarrow$: $\forall x_0 \in X$, let $y_0 = f ( x_0)$, $f ^{-1} ( U( y_0)) \in \mathscr{T}$ if $U(y_0) \in \mathscr S\cap \mathscr U_{y_0}$.
Notice that $x_0 \in f ^{-1} ( U( y_0))$, $ f ^{-1} ( U( y_0))$ is a neighbourhood of $x_0$, therefore $f \in C( X, Y) $.
\end{proof}

\begin{theorem}
	Let $(X, \mathscr T_X)$, $(Y, \mathscr T_Y)$, $(Z, \mathscr T_Z)$ be topological spaces.
	If $f\colon X\to Y$ and $g\colon Y\to Z$ are both continuous, $g\circ f\colon X\to Z$ is also continuous. 
\end{theorem}
\begin{proof}
\begin{align*}
	\forall W\in \mathscr T_Z\left(
		g^{-1} (W)\in \mathscr T_Y\right) \rightarrow 
			\forall W\in \mathscr T_Z\left(
				f^{-1}\big( 
					g^{-1} (W)\big)\right)
\end{align*}
Since $f^{-1}\big(  g^{-1} (W)\big) = ( g\circ f)^{-1} (W)$, the theorem has been proved.
\end{proof}

\begin{definition}\label{homeomorphism}
$(X,\mathscr{T}_X)$ and $(Y,\mathscr{T}_Y)$ are both topological spaces. A bijective mapping $f \colon X\to Y$ is a \indexbf{homeomorphism} if $f \in C( X, Y) \wedge f ^{ -1} \in C( Y, X)$. 
\end{definition}
\begin{definition}\label{homeomorphic}
Two topological spaces $(X,\mathscr{T}_X)$ and $(Y, \mathscr{T}_Y)$ are said to be \indexbf{homeomorphic} if there exists a homeomorphism $f \colon X\to Y$.
\end{definition}

Homeomorphic topological spaces are identical with respect to their topological propoties since the theorem~\ref{criterion_continuity} has shown that their open sets correspond to each other. In fact homeomorphic relations are equivalent relations.

\subsection{Closure}
\begin{definition}\label{accumulation_point}
Let $X$ be a topological space and $A$ be a subset of $X$. Let $x\in X$.
If $\forall U\in \mathscr U_x \big(
	U\cap (A-\{x\}) \neq \varnothing\big)$, 
then $x$ is called a \indexbf{accumulation point}, \indexbf{cluster point} or \indexbf{limit point} of $A$. 
The set $A' := \{x\in X\mid x\text{ is a accumulation point of } A\}$ is called the \indexbf{derived set} of $A$.
A point $a\in A$ is called a \indexbf{isolated point} of $A$ if $a\notin A'$.
\end{definition}

\begin{theorem}\label{propoties_derived}
Let $X$ be a topological space and $A$, $B$ be subsets of $X$. 
1) $A\subseteq B \to A'\subseteq B'$; 
2) $(A\cup B)' = A'\cup B'$;
3) $(A')' \subseteq A\cup A'$.
\end{theorem}
\begin{proof}
\begin{enumerate}[1)]
\item When $A\subseteq B$, $U\cap (A-\{x\}) \subseteq U\cap (B-\{x\})$.
\item $(A\cup B)' = \{x\in X\mid \forall U\in \mathscr U_x \big(
	U\cap (A\cup B-\{x\}) \neq \varnothing\big)\}$. 
Also $U\cap (A\cup B - \{x\}) = U\cap (X-\{x\})\cap (A\cup B) = (U\cap A-\{x\})\cup (U\cap B-\{x\})$.
\item If $x\notin A\cup A'$, then $\exists G\in \mathscr U_x\cap \mathscr T \big(
	G\cap (A-\{x\}) = G\cap A = \varnothing\big)$. 
$\forall y \in G$, $G$ itself is a neighbourhood of $y$ that $G\cap (A-\{y\}) = G\cap A =\varnothing$, therefore $y\notin A'$. 
This means that $G$ is a neighbourhood of $x$ that $G\cap (A'-\{x\}) = G\cap A' =\varnothing$, i.e.\ $x\notin (A')'$.
\end{enumerate}
\end{proof}

\begin{definition}\label{closed_set}
Let $(X,\mathscr T)$ be a topological space and $F$ be a subset of $X$. $F$ is said to be \indexbf{closed} \emph{iff} $\complement_X (F) \in \mathscr T$. The collection all closed sets is denoted by $\mathscr F$.
\end{definition}

\begin{theorem}\label{closed_iff_accumulation}
Let $(X,\mathscr T)$ be a topological space and $F$ be a subset of $X$. $F$ is closed \emph{iff} $ F'  \subseteq F$.
\end{theorem}
\begin{proof}
$\to$: 
If $x\notin F$ then $x\in \complement_X (F)$, which is open in $(X,\mathscr T)$. 
Then $\complement_X (F)$ is a neighbourhood that $\complement_X (F) \cap (F-\{x\}) = \complement_X (F) \cap F =\varnothing$, i.e.\ $x\notin F'$.

$\leftarrow$:
$\forall x\notin F(x\notin F')$, then there exists an open neighbourhood $U$ of $x$ that $U\cap F = \varnothing$, then $\complement_X (F)$ is always a neighbourhood of its elements, since theorem \ref{open_iff_neighbourhood}, $\complement_X (F) \in \mathscr T$.
\end{proof}

\begin{definition}\label{closure}
Let $(X,\mathscr T)$ be a topological space and $A$ be a subset of $X$. Set $\overline A := A\cup A'$ is called a \indexbf{closure} of $A$.
\end{definition}

\begin{theorem}\label{closure_neighbourhood}
Let $(X,\mathscr T)$ be a topological space and $A$ be a subset of $X$. Let $x\in X$.
\[
	x\in \overline A \leftrightarrow
		\forall U\in\mathscr U(x)(
			U\cap A\neq \varnothing).
\]
\end{theorem}
\begin{proof}
$\to$: If $x\in A$ then $\{x\}\subseteq U\cap A$, 
else if $x\in A'$ then $U\cap A \supset (U-\{x\})\cap A \neq \varnothing$.

$\leftarrow$: 
If $\exists U\in\mathscr U(x)\big(
	(U-\{x\})\cap A= \varnothing\big)\wedge x\notin A$, 
then there exists a $U\in \mathscr U(x)$ s.t.\ $U\cap A=\varnothing$. 
\end{proof}


\begin{theorem}\label{closed_iff_closure}
Let $(X,\mathscr T)$ be a topological space and $A$ be a subset of $X$. 
$A$ is closed in $(X,\mathscr T)$ \emph{iff} $A=\overline A$.
\end{theorem}
\begin{proof}
Since theorem \ref{closed_iff_accumulation}, $A$ is closed iff $A'\subseteq A$, which iff $A = A\cup A' = \overline A$.
\end{proof}

\begin{corollary}\label{closure_closed}
Let $(X,\mathscr T)$ be a topological space and $A$ be a subset of $X$. 
$\overline A$ is always closed.
\end{corollary}
\begin{proof}
Since (3) of theorem \ref{propoties_derived}, $\overline{\overline A} = \overline A$.
\end{proof}

\begin{lemma}\label{subset_closure}
Let $(X,\mathscr T)$ be a topological space and $A$, $B$ be subsets of $X$.
$A\subseteq B \to \overline A\subseteq \overline B$.
\end{lemma}
\begin{proof}
$A\subseteq B \to A'\subseteq B'$ ((1) of theorem \ref{propoties_derived}), so $A\cup A'\subseteq B\cup B'$, i.e.\ $\overline A\subseteq \overline B$.
\end{proof}

We can say that the closure of a set is the smallest closed set containing it, as long as we prove the following theorem:

\begin{theorem}
Let $(X,\mathscr T)$ be a topological space and $A$ be a subset of $X$.
\begin{align*}
	\overline A = \bigcap_{F\in \mathscr F \wedge A\subseteq F} F.
\end{align*}
\end{theorem}
\begin{proof}
Since $\overline A$ itself is closed (corollary \ref{closure_closed}), $\bigcap_{F\in \mathscr F \wedge A\subseteq F} F \subseteq \overline A$. 
On the other hand, $\bigcap_{F\in \mathscr F \wedge A\subseteq F} F$ is closed, so $\overline{\bigcap_{F\in \mathscr F \wedge A\subseteq F} F} = \bigcap_{F\in \mathscr F \wedge A\subseteq F} F$.
Therefore, $A\subseteq \bigcap_{F\in \mathscr F \wedge A\subseteq F} F \to \overline A \subseteq \bigcap_{F\in \mathscr F \wedge A\subseteq F} F$ (Lemma \ref{subset_closure}). 
\end{proof}

\begin{theorem}
Let $(X,d)$ be a metric space and $A$ be a non-empty subset of $X$.
\begin{enumerate}[1)]
\item $\forall x\in X$, $x\in A' \leftrightarrow d(x,A-\{x\}) = 0$.
\item $\forall x\in X$, $x\in \overline A \leftrightarrow d(x,A)=0$.
\end{enumerate}
\end{theorem}
\begin{proof}
\begin{enumerate}[1)]
\item We have $x\in A' $ \emph{iff} $\forall \varepsilon\in \mathbb R_+\big(
	B (x,\varepsilon) \cap (A-\{x\}) \neq \varnothing\big)$, 
which is established \emph{iff} $\forall \varepsilon \in \mathbb R_+\exists y\in A-\{x\}\big(
	d(x,y) <\varepsilon\big)$.
\item We only need to substitude $A-\{x\}$ with $A$ in 1).
\end{enumerate}
\end{proof}

\begin{theorem}
Let $(X, \mathscr T)$ and $(Y,\mathscr S)$ be two topological spaces, and $f\colon X\to Y$. Note the collections of closed sets in $X$ and $Y$ by $\mathscr F_X$, $\mathscr F_Y$.
The statements below are equivalent:
\begin{enumerate}[(1)]
\item $f\in C(X,Y)$.
\item $\forall F\in\mathscr F_Y \big(
	f^{-1} (F) \in \mathscr F_X\big)$.
\item $\forall A\in \mathscr P(X) \big(
	f(\overline A) \subseteq \overline{ f(A)}\big)$.
\item $\forall B\in \mathscr P(Y) \big(
	\overline{f^{-1} (B)}\subseteq f^{-1} (\overline B) \big)$.
\end{enumerate} 
\end{theorem}
\begin{proof}
(1) $\to$ (2): Only to notice that $f^{-1} (\complement_Y F ) =\complement_X f^{-1} (F)$. 

(2) $\to$ (3): $\forall A \in \mathscr P(X)$ we have $f(A) \subseteq \overline{f(A)}$, so $A \subseteq f^{-1} \big(
	\overline{f(A)}\big)$. 
By (2) we know that $f^{-1} \big(\overline{f(A)}\big)$ is closed, therefore 
$\overline A \subseteq \overline{f^{-1} \big(\overline{f(A)}\big)} 
	= f^{-1} \big(\overline{f(A)}\big)$, so $f(\overline A) \subseteq \overline {f(A)}$.
	
(3) $\to$ (4): By (3) we know that $\forall B\in \mathscr P(Y)$, 
$f\big(\overline{f^{-1}(B)}\big) \subseteq \overline{f(f^{-1}(B))}$. 
Also $f(f^{-1}(B)) \subseteq B$ (equality satisfied when $f$ is surjective), 
then $f\big(\overline{f^{-1}(B)}\big) \subseteq \overline B$, 
so $\overline{f^{-1}(B)} \subseteq f^{-1} (\overline B)$.

(4) $\to$ (1): $\forall G\in \mathscr S$, $\complement_Y G \in \mathscr F_Y$, so by (3), we have
\begin{align*}
	\overline {\complement_X f^{-1}(G)} =\overline {f^{-1}\big(\complement_Y G\big)} 
		\subseteq  f^{-1}\big(\overline{\complement_Y G}\big) = f^{-1}\big(\complement_Y G\big)
			=\complement_X f^{-1}(G).
\end{align*}
However by the definition of closure $\complement_X f^{-1}(G) \subseteq \overline {\complement_X f^{-1}(G)}$.
Therefore $\complement_X f^{-1}(G) = \overline {\complement_X f^{-1}(G)}$, which means $\complement_X f^{-1}(G)$ is closed (theorem \ref{closed_iff_closure}), i.e.\ $f^{-1}(G)$ is open. 
\end{proof}

\subsection{Interior Points and Boundary}
\begin{definition}
Let $(X,\mathscr T)$ be a topological space and $A$ be a subset of $X$.
We call $x$ an \indexbf{interior point} of $A$ if $A$ is a neighbourhood of $x$.
We call $x$ an \indexbf{exterior point} of $A$ if $\complement_X(A)$ is a neighbourhood of $x$.
The sets of all interior points of $A$ is the \indexbf{interior} of $A$, noted by $\interior A$.
\end{definition}

\begin{theorem}\label{interior_and_closure}
Let $(X,\mathscr T)$ be a topological space and $A$ be a subset of $X$. 
$\interior A=\complement_X\big(
	\overline{
		\complement_X(A)}\big)$, 
$\overline A = \complement_X\big(
	\interior \complement_X(A)\big)$.
\end{theorem}
\begin{proof}
$x\in \interior A$ implies an open set $G$ which is a subset of $A$ and $x\in G$.
The complement of $G$ is closed, thus its closure is $\complement_X(G)$ itself. 
$\complement_X(A)\subseteq \complement_X(G) \rightarrow 
	\overline{\complement_X(A)}\subseteq \complement_X(G)$ (Lemma \ref{subset_closure}), 
therefore $x\notin \overline{\complement_X(A)}$, which is the first equation to prove.

To prove the second one only need to replace the $A$ with $\complement_X(A)$ in the first equation.
\end{proof}

\begin{theorem}
Let $(X,\mathscr T)$ be a topological space and $G$ be a subset of $X$. 
\[
	G\in \mathscr T \leftrightarrow
		\interior G=G
\]
\end{theorem}
\begin{proof}
$G$ is open \emph{iff} the complement of $G$ is closed. And $\complement_X (G) = \overline{\complement_X (G)}$.
The complement of the both size of this equation and Theorem \ref{interior_and_closure} give the proof of the theorem.
\end{proof}

With the propoties of closure and Theorem \ref{interior_and_closure} the folowing statements should be easy to prove:
\begin{theorem}
Let $(X,\mathscr T)$ be a topological space and $A$, $B$ be subsets of $X$. 
$\interior (A\cap B) = \interior A \cap \interior B$, $\interior(\interior A) =\interior A$,
\[
	\interior A = \bigcup\limits_{G\in\mathscr T \wedge G\subseteq A} G.
\]
\end{theorem}

Therefore we can say that the interior of $A$ is the largest open set contianed in $A$.

\begin{definition}
Let $(X,\mathscr T)$ be a topological space and $A$ be a subset of $X$. 
A point $x$ is said to be a \indexbf{boundary point} of $A$ if 
$\forall U\in \mathscr U(x)\big(
	U\cap A\neq \varnothing \wedge U\cap \complement_X(A)\neq \varnothing\big)$.
The set of all boundary points of $A$ is called the \indexbf{boundary} of $A$, noted by $\partial A$.
\end{definition}

\begin{theorem}\label{boundary_propoties}
Let $(X,\mathscr T)$ be a topological space and $A$ be a subset of $X$. 
(1) $\partial A = \overline A \cap \overline{\complement_X(A)}$;
(2) $\interior A = \overline A - \partial A$;
(3) $\overline A = \interior A\cup \partial A$.
\end{theorem}
\begin{proof}
(1): Apply Theorem \ref{closure_neighbourhood} to both $A$ and $\complement_X(A)$.

(2): $\interior A \cup \partial A=
	A\cup(\overline A\cap \overline{\complement_X (A)})=
		\overline A\cap \big(\interior A\cup \complement_X (\interior A)\big) = \overline A$.
		
(3) $\overline A -\partial A= 
	\overline A - (\overline A\cap \overline{\complement_X (A)}) = 
		\overline A \cap \complement_X\big(
			\overline{
				\complement_X(A)}\big) = 
			\overline A\cap \interior A = 
				\interior A$.
\end{proof}

\subsection{Basis}
\begin{definition}
Let $(X,\mathscr T)$ be a topological space and $\mathscr B$ be a subset of $\mathscr T$.
If $\forall G\in \mathscr T 
	\exists \mathscr B_G\in \mathscr P(\mathscr B) \big(
		G=\cup \mathscr B_G\big)$, 
then we call $\mathscr B$ a \indexbf{basis} or a \indexbf{base} of the topology $\mathscr T$.
\end{definition}

\begin{theorem}
Let $(X,\mathscr T)$ be a topological space and $\mathscr B$ be a subset of $\mathscr T$.
$\mathscr B$ is a basis of $\mathscr T$ \emph{iff} 
$\forall x\in X
	\forall U\in \mathscr U_x
		\exists B\in \mathscr B(
			x\in B\wedge B\subseteq U)$.
\end{theorem}
\begin{proof}
$\to$:
$U\in \mathscr U_x$ implies an open set $G\in \mathscr P(U)$ that contians $x$, which is the union of elements in $\mathscr B$.
Therefore there exists a $B\in \mathscr B$, which is the subset of $U$ and it contians $x$.

$\leftarrow$: 
$\forall G\in \mathscr T$, it is a neighbourhood of all the points in $G$.
For all $x$ in $G$ assign a $B_x\in \mathscr B$ which contians $x$ and is the subset of $G$, so that $G$ is the Union of all the $B_x$.
\end{proof}

\begin{theorem}\label{intersection_basis}
Let $\mathscr B$ be a basis of a topological space $(X,\mathscr T)$.
\begin{align*}
	\forall B_1\in \mathscr B\,B_2\in\mathscr B_2\,
		\forall x\in B_1\cap B_2 \,
			\exists B\in \mathscr B\,(x\in B\subseteq B_1\cap B_2).
\end{align*}
\end{theorem}
\begin{proof}
By definition of topology and basis, 
$B_1$ and $B_2$ are both open and their intersection $B_1\cap B_2$ is open as well.
Then there exists a collections of sets in $\mathscr B$ whose union is $B_1\cap B_2$, 
there must be at least a set $B$ that contians $x$.
\end{proof}

The topology $\mathscr T$ on $X$ is determinded if the basis $\mathscr B$ is given,
that is, if the union of $\mathscr B$ is $X$ and it satisfies Theorem \ref{intersection_basis}, 
then $\mathscr T$, which is defined by the collection of the unions of $B$s in $\mathscr B$,
is the \emph{only} topology on $X$ such that $\mathscr B$ is a basis of it. 

For example, \indexbf{lower limit topology} $\mathscr T_\ell$ on $\mathbb R$ is defined by giving a basis:
\begin{align*}
	\mathscr B_\ell = \{
		[a,b)\mid a,b\in \mathbb R \wedge a<b\},
\end{align*}
and $(\mathbb R,\mathscr T_\ell)$ is called the \indexbf{lower limit topological space}, the \indexbf{Sorgenfrey line} or the \indexbf{arrow}, denoted by $\mathbb R_\ell$.

It is obvious that $\mathscr T \varsubsetneqq\mathscr T_\ell$, where $\mathscr T$ is the standard topology on $\mathbb R$.

\begin{definition}\label{subbasis}
Let $(X,\mathscr T)$ be a topological space and $\mathscr S$ be a subset of $\mathscr T$.
If the collection $\mathscr B$ of the finite intersections of the non-empty sets in $\mathscr S$ is a basis of $\mathscr T$, i.e.\ 
\begin{align*}
	\mathscr B = \left\{
		\bigcap\limits_{i=1}^n S_i \mid
			S_i \in \mathscr S,\;i=1,\cdots,n,\;n\in\mathbb N_+\right\},
\end{align*}
then we call $\mathscr S$ a \indexbf{subbasis} or a \indexbf{subbase} of $\mathscr T$.
\end{definition}


\section{Subspace}
\printindex
% \printbibliography
\end{document}