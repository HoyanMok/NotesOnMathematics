% TeXplates/Mathematics.tex
% v0.2.2
% https://github.com/HoyanMok/TeXplates
\documentclass[openany]{book} 
% \documentclass{ctexbook} 如果用中文
% \documentclass[10pt,a4paper]{ctexart}  字体大小和纸张大小,默认分别为10pt和letterpaper
% 五号 = 10.5pt,小四=12pt,四号=14pt
% 其他可选参量如twocolumn, 两行排版
\usepackage{../TeXplatesMathematics}
\usepackage{xeCJK}
\usepackage{xr-hyper}
\externaldocument[PST-]{PointSetTopology}

\addbibresource{PointSetTopology.bib} % 把这里改成实际的文件名

\newcommand*{\rel}{\mathbin{\mathrm{rel}}}

\title{Algebraic Topology}
\author{Hoyan Mok}
\begin{document}
\pagenumbering{Alph}
\maketitle
\frontmatter

\tableofcontents
\mainmatter{}

\chapter{Homotopy and Fundamental Group}
\section{Homotopy}
\begin{definition}[Homotopy]%
    \label{def: Homotopy}
    $f, g \in C(X, Y)$.
    If $\exists H \in C(X \times [0, 1], Y)$ s.t.\ $H(x, 0) = f(x)$, $H(x, 1) = g(x)$, then we say $f$ and $g$ are \indexbf{homotopic}, denoted by $\indexmath[f simeq g]{f \simeq g} \colon X \to Y$ or just $X \to Y$.
    $H$ is called a \indexbf{homotopy} between $f$ and $g$, denoted by $\indexmath[H colon f simeq g]{H \colon f \simeq g}$ or $\indexmath[f simeq_H g]{f \simeq_H g}$.
\end{definition}

For $t \in [0, 1]$, $h_t \colon X \to Y ;x \mapsto H(x, t)$ is called a \emphbf{$t$-slice}\index{t-slice@$t$-slice}.

If $f$ is homotopic to a constant mapping, we say that $f$ is \indexbf{null-homotopic}.

A \indexbf{linear homotopy} is a homotopy between two functions to $Y \subseteq \mathbb R^n$ that change linearly, i.e.\ 
\begin{equation*}
    H(x, t) = (1 - t) f(x) + t g(x).
\end{equation*}


\begin{theorem}[Maps to convex set are homotopic]%
    \label{theorem: Maps to convex set are homotopic}
    $f, g \in C(X, Y)$. If $Y$ is a convex set in $\mathbb R^n$, 
    then $f \simeq g$.
\end{theorem}
\begin{proof}
    Consider linear homotopy.
\end{proof}

\begin{theorem}%
    \label{theorem: Homotopic relation is an equivalence relation}
    Homotopic relation is an equivalence relation.
\end{theorem}
\begin{proof}
    \emph{reflexity}.
    $f \simeq f$, just take $H(x, t) = f(x)$ for any $t$ (Such homotopy is called a \indexbf{constant homotopy}).

    \emph{Symmetry}. 
    $f \simeq g$ then $g \simeq f$. Just take $\bar H(x, t) = H(x, 1 - t)$ (Here $\indexmath[H bar]{\bar H}$ is called the inverse of $H$).

    \emph{Transivity}. 
    $f \simeq g \wedge g \simeq h \to f \simeq h$. Let
    \begin{equation*}
        H_1 H_2 (x, 2t) = \begin{cases}
            H_1(x, 2t) & t \in [0, 1/2], \\
            H_2(x, 2t - 1) & t \in [1/2, 1].
        \end{cases}
    \end{equation*}
    We can see that $H_1 H_2$ is also a homotopy (see Theorem~\ref{PST-theorem: composition of path} in Point Set Topology)
\end{proof}

Hence, we can define \indexbf{homotopy classes} on $C(X, Y)$, denoted by $\indexmath[X Y]{[X, Y]}$.

As you might expect after reading the proof of Theorem~\ref{theorem: Homotopic relation is an equivalence relation}, the homotopies between mappings within a homotopy class form a group.

\begin{theorem}[Composition of homotopies]%
    \label{theorem: Composition of homotopies}
    $f_1 \simeq f_2 \colon X \to Y$, $g_1 \simeq g_2 \colon Y \to Z$, then $g_1 \circ f_1 \simeq g_2 \circ f_2 \colon X \to Z$.
\end{theorem}
\begin{proof}[\textbf{Proof i}]
    Let $F \colon f_1 \simeq f_2$, $G \colon g_1 \simeq g_2$. 
    Define:
    \begin{equation*}
        \bv F \colon X \times [0, 1] \to Y \times [0, 1]; 
        (x, t) \mapsto (F(x, t), t). 
    \end{equation*}

    It can be verified tht $G \circ \bv F \colon g_1 \circ f_1 \simeq g_2 \circ g_2 \colon X \to Z$.
\end{proof}
\begin{proof}[\textbf{Proof ii}]
    Let $F \colon f_1 \simeq f_2$, $G \colon g_1 \simeq g_2$. 

    We can verify that $H_1 \colon (x, t) \mapsto g_1 \circ F(x, t)$ is a homotopy between $g_1 \circ f_1$ and $g_1 \circ f_2$;
    Similarly $H_2 \colon g_1 \circ f_2 \simeq g_2 \circ f_2$ can be defined.

    Now consider $H = H_1 H_2$, or in detailed,
    \begin{equation*}
        H(x, t) = \begin{cases}
            g_1 \circ F(x, 2t) & (x, t) \in X \times [0, 1/2] \\
            G(f_2(x), 2t - 1). & (x, t) \in X \times [1/2, 1]
        \end{cases}
    \end{equation*}
\end{proof}

\begin{lemma}[Identity map in convex space is null-homotopic]
    $X \subset \mathbb R^n$ is a convex space.
    $\forall x_0 \in X$,
    $\id_X \simeq (x \mapsto x_0)$.
\end{lemma}
\begin{proof}
    The linear homotopy can be constructed as:
    \begin{equation*}
        H_{x_0} (x, t) = tx + (1 - t) x_0.
    \end{equation*}
\end{proof}

\begin{theorem}[Continuous mappings from a convex set are null-homotopic]%
    \label{theorem: Continuous mappings from a convex set are null-homotopic}
    $X \subseteq \mathbb R^n$ is a convex set.
    $\forall f \in C(X, Y)$, $f$ is null-homotopic.
\end{theorem}
\begin{proof}
    Let $H_{x_0} (x, t) = tx + (1 - t) x_0$.
    Then, any $f \colon X \to Y$ can be written as $f = f \circ \id_X$, hence $f \simeq f \circ H_{x_0}(x, 1) = (x \mapsto f(x_0))$, which means $f$ is null-homotopic.
\end{proof}

\begin{theorem}[Constant mappings to a path-connected space belong to one homotopy class]%
    \label{theorem: Constant mappings to a path-connected space belong to one homotopy class}
    If $Y$ is a path-connected space, $y_0 \in Y$,
    then $[X, Y] = [x \mapsto y_0]$ (i.e.\ homotopy class of constant mapping to $\{y_0\}$)
\end{theorem}
\begin{proof}
    Let $f_1(x) = y_1$, $f_2(x) = y_2$ be two constant mappings, $a$ is a path from $y_1$ to $y_2$.
    Then the homotopy between $f_1$ and $f_2$ can be defined as:
    \begin{equation*}
        H(x, t) = a(t).
    \end{equation*}
\end{proof}



\begin{definition}[Homotopy relative to a set]
    Let $A \subseteq X$, $H \colon f \simeq g$.
    If $\forall a \in A$, $\forall t \in [0, 1]$, $f(a) = g(a) = H(a, t)$, we say that $f$ and $g$ are \indexbf{homotopic relative to $A$}, denoted by $\indexmath[H f simeq g rel A]{H \colon f \simeq g \rel A}$.
\end{definition}

We can have parallel results as Theorem~\ref{theorem: Homotopic relation is an equivalence relation} and Theorem~\ref{theorem: Composition of homotopies}:

\begin{theorem}%
    \label{theorem: Relatively homotopic relation is an equivalence relation}
    Given $A \subseteq X$, $\simeq \rel A$ is an equivalence relation in $C(X, Y)$.
\end{theorem}

\begin{theorem}[Composition of relative homotopies]%
    \label{theorem: Composition of relative homotopies}
    $f_1 \simeq f_2 \colon X \to Y \rel A$, $g_1 \simeq g_2 \colon Y \to Z \rel B$, and $f_1(A) \subset B$, then $g_1 \circ f_1 \simeq g_2 \circ f_2 \colon X \to Z$.
\end{theorem}

\begin{definition}[Fixed-endpoint Homotopy]%
    \label{def: Fixed-endpoint Homotopy}
    Let $a$, $b$ be two paths in $X$. 
    If $a \simeq b \rel \{0, 1\}$, we say that $a$ and $b$ are \indexbf{fixed-endpoint homotopic}\index{fixed-endpoint homotopy}.
    The paths in $X$ modulus fixed-point homotopy is denoted by $\indexmath[X]{[X]}$, called the \indexbf{path classes}.
    The path class which $a$ belongs to is denoted by $\indexmath[a]{\langle a \rangle}$.
\end{definition}

\section{Fundamental Group}

\begin{theorem}
    Let $a, b, c, d$ be four paths in $X$.
    \begin{align*}
        a \simeq b \rel \{0, 1\} &\IFF \bar a \simeq \bar b \rel \{0, 1\}, \\
        a \simeq b \rel \{0, 1\} \wedge c \simeq d \rel \{0, 1\} \wedge a(1) = c(0) &\then ac \simeq bd \rel \{0, 1\}.
    \end{align*}
\end{theorem}

\begin{definition}[Inverse and product of path classes]
    $\alpha, \beta \in [X]$, $a \in \alpha$, $b \in \beta$.
    $b(0) = a(1)$.
    We define $\alpha^{-1} := \langle \bar a \rangle$ to be the \indexbf{inverse} of the path class~$\alpha$, and $\alpha \beta := \langle ab \rangle$ to be the \indexbf{product} of the two path classes~$\alpha$ and $\beta$.
\end{definition}

While the product of paths does not obey associativity, we have:
\begin{theorem}[Associativity of product of path classes]%
    \label{theorem: Associativity of product of path classes}
    $\alpha, \beta, \gamma \in [X]$. $(\alpha \beta) \gamma = \alpha (\beta \gamma)$ (if they are productible).
\end{theorem}
\begin{proof}
    Consider $\forall a \in \alpha$, $\forall b \in \beta$, $\forall c \in \gamma$.

    Let $\tilde a(t) = t/3$, $\tilde b(t) = t/3 + 1/3$, $\tilde c(t) = t/3 + 2/3$. 
    $\tilde a, \tilde b$ and $\tilde c$ are three paths in $[0, 1]$, and $\tilde a (\tilde b \tilde c) \simeq (\tilde a \tilde b) \tilde c \rel \{0, 1\}$, since $[0, 1]$ is convex, therefore there is a linear homotopy between the two product paths.

    Now Let $f \colon [0, 1] \to X$ be
    \begin{equation*}
        f(t) = \begin{cases}
            a(3t), & t \in [0, 1/3]; \\
            b(3t - 1), & t \in [1/3, 2/3]; \\
            c(3t - 2), & t \in [2/3, 1].
        \end{cases}
    \end{equation*}

    $a(bc) = f \circ \tilde a (\tilde b \tilde c) \simeq f \circ (\tilde a \tilde b) \tilde c = (ab)c \rel \{0, 1\}$, by Theorem~\ref{theorem: Composition of homotopies}.
\end{proof}

\begin{theorem}[Identity-like properties of point path]%
    \label{theorem: Identity-like properties of point path}
    $\alpha \in [X]$. 
    Let the initial and the terminal point of $\alpha$ be $x_0$ and $x_1$.
    \begin{enumerate*}[label=\emph{(\roman*)}] % chktex 36
        \item $\alpha^{-1} \alpha = \langle t \mapsto x_1 \rangle$, $\alpha \alpha^{-1} = \langle t \mapsto x_0 \rangle$;
        \item $\alpha \langle t \mapsto x_0\rangle = \alpha = \langle t \mapsto x_1\rangle\alpha$.
    \end{enumerate*}
\end{theorem}
\begin{proof}
    Note that $\id_{[0, 1]}$ is a path in the convex set $[0, 1]$.
\end{proof}

For now path classes are not closed under production.

\begin{definition}[Fundamental group]%
    \label{def: Fundamental group}
    $x_0 \in X$. 
    The path classes of loops at $x_0$ (paths that have both endpoints at $x_0$), equiped with production, is the \indexbf{fundamental group} of $X$ at $x_0$, denoted by $\indexmath[pi1 X x0]{\pi_1(X, x_0)}$.
\end{definition}

\begin{definition}[Homomorphism induced by continuous function]
    $f \in C(X, Y)$, $x_0 \in X$.
    We define $f_\pi \colon [X] \to [Y]$ as $f_\pi (\langle a \rangle) = \langle f \circ a \rangle$, where $a$ is a path in $X$. 
    
    The limitation of $\indexmath[f pi]{f_\pi}$ on $\pi_1(X, x_0)$ is said to be a \indexbf{homomorphism induced by $f$}.
\end{definition}

For simplicity, we would write such homomorphism by $f_\pi$ (without explicitly referring limitation).

\begin{theorem}[Isomorphism induced by homeomrphism]%
    \label{theorem: Isomorphism induced by homeomrphism}
    Let $f$ be a homeomorphism from $X$ to $Y$, then $\forall x_0 \in X$, $f_\pi$ is an isomorphism from $\pi_1(X, x_0)$ to $\pi_1(Y, f(x_0))$.
\end{theorem}
\begin{proof}
    $f^{-1} \circ f = \id_X \then (f^{-1})_\pi \circ f_\pi = \id_{\pi_1(X, x_0)}$, $f \circ f^{-1} = \id_Y \then f_\pi \circ (f^{-1})_\pi = \id_{\pi_1(Y, f(x_0))}$, therefore $(f^{-1})_\pi$ is the inverse of $f_\pi$.
    An invertible homomorphism is an isomorphism. 
\end{proof}

\begin{theorem}[Fundamental groups of path connected space at different points are isomorphic]%
    \label{theorem: Fundamental groups of path connected space at different points are isomorphic}
    $X$ is path connected, $x_1, x_2 \in X$.
    $\pi_1(X, x_1) \cong \pi_1(X, x_2)$.
\end{theorem}
\begin{proof}
    $\langle a \rangle \in \pi_1(X, x_1)$, $\langle b \rangle \in \pi_1(X, x_2)$, $\langle c \rangle$ is a path class with initial point~$x_1$ and terminal point~$x_2$.

    It can be verify that
    \begin{equation*}
        g \colon \pi_1(X, x_1) \to \pi_2 (X, x_2); \langle a \rangle \mapsto \langle \bar c a c \rangle
    \end{equation*}
    is a homomorphism. Same as $g'(\langle b \rangle) = c b \bar c$.

    \begin{equation*}
        g \circ g'(\langle b \rangle) = \langle \bar c c b \bar c c\rangle = \id_{\pi_1(X, x_2)};
        \quad
        g' \circ g(\langle a \rangle) = \langle c \bar c a c \bar c\rangle = \id_{\pi_1(X, x_2)},
    \end{equation*}
    therefore $g$ is an isomorphism.
\end{proof}

With Theorem~\ref{theorem: Fundamental groups of path connected space at different points are isomorphic}, we can write the fundamental group of a path-connected space $X$ by $\indexmath[pi1 X]{\pi_1(X)}$.

For different path-connected branches, a topological space can have different fundamental groups, while they are isomorphic within one branch.

\begin{definition}[Simply connected]%
    \label{def: simply connected}
    If the fundamental group of a path connected space $X$ is trivial i.e.\ $\pi_1(X) \cong \{1\}$, we say that $X$ is \indexbf{simply connected}.
\end{definition}

\begin{theorem}[Convex set is simply connected]%
    \label{theorem: Convex set is simply connected}
    If $X \subset \mathbb R^n$ is convex, then $X$ is simply connected.
\end{theorem}
\begin{proof}
    $x_0 \in X$, $a \in C([0, 1], X)$ s.t.\ $a(0) = a(1) = x_0$.
    $H_{a, x_0}(s, t) = (1 - t) a(s) + t x_0$. 
\end{proof}

Now we can calculate the fundamental group of $S^n$.

\begin{definition}[Lift]
    Let $X, Y, Z$ be three topological spaces, and $f \in C(X, Z)$, $p \in C(Y, Z)$. 
    If $\indexmath[f]{\tilde f} \in C(X, Y)$, s.t.\ $f = p \circ \tilde f $, we say that $\tilde f$ is a \indexbf{lift} of $f$. 
\end{definition}

In some case, given $f$ and $p$, $\tilde f$ might do not exist.

\begin{lemma}[Lift of path]
    $a \in C([0, 1], S^1)$, $p \colon \mathbb R \to S^1; x \mapsto \me^{2\pi x \mi}$. Let $t_0 \in \mathbb R$ s.t.\ $p(x_0) = a(0)$.
    There exists a unique lift~$\tilde a \in C(\mathbb R, S^1)$ of $a$ s.t.\ $\tilde a(0) = x_0$.
\end{lemma}
\begin{proof}
    \emph{Existence}.
    The collection of open sets that the images under $a$ do not cover $S^1$, $\{(\alpha_i, \beta_i) \cap [0, 1] \mid a_i, b_i \in \mathbb R^I\wedge S^1 \subsetneq a((\alpha_i, \beta_i))\}$, is a cover of $S^1$ by the definition of continuity.
    Since $S^1$ is compact, there exists a finite subcover $\{(\alpha_i, \beta_i) \cap [0, 1] \mid a_i, b_i \in \mathbb R^n\wedge S^1 \subsetneq a((\alpha_i, \beta_i))\}$, where $n \in \mathbb N$.
    By dividing these open intevals into closed intevals that has no inner points intersecting, we can get $\varOmega = \{I_k := [t_i, t_{i+1}] \mid k \in m\}$ (This can be done by sorting $\alpha_i$ and $\beta_i$).

    The mapping~$p$ is locally homeomorphic i.e.\ there exists $[x_i, x_i'] \subset \mathbb R$ s.t.\ $p_i := p|_{[x_i, x_i']} \colon [x_i, x_i'] \to a(I_i)$ is a homeomorphism (and $p_i(x_i) = a(t_i)$), therefore $\tilde a_i := p^{-1}_i \circ a$ is a lift of $a_i := a|_{I_i}$.

    Since $p_0(t_0) = a(t_0)$, $p_{i+1}(t_i) = p_i(t_i)$, we can define piecewisely the lift of $a$ by $\tilde a = \cup \{\tilde a_i \mid i \in m\}$.

    \emph{Uniqueness}.
    Let $\tilde a'$ be another lift of $a$, $p(\tilde a'(t) - \tilde a(t)) = p\circ \tilde a'(t) / p\circ \tilde a(t) = a(t) / a(t) = 1$, therefore $\tilde a'(t) - \tilde a(t) \in \mathbb Z$. 
    Since $[0, 1]$ is connected, the image of $t \mapsto \tilde a'(t) - \tilde a(t)$ must be connected, which is possible only if it is constant.
    $\tilde a'(0) = \tilde a(0) = x_0$, therefore $\tilde a = \tilde a'$.
\end{proof}

\backmatter{}
\nocite{*} % 这个表示列出所有没有在文中被引用的参考文献
\printbibliography[heading=bibliography, title={bibliography}]

\indexprologue{Here listed the important symbols used in these notes}
\printindex[symbol]

\printindex
% \printbibliography
\end{document}