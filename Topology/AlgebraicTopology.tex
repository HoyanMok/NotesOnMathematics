% TeXplates/Mathematics.tex
% v0.2.2b
% https://github.com/HoyanMok/TeXplates
\documentclass[openany, oneside, a5paper]{book} 
% \documentclass{ctexbook} 如果用中文
% \documentclass[10pt,a4paper]{ctexart}  字体大小和纸张大小,默认分别为10pt和letterpaper
% 五号 = 10.5pt,小四=12pt,四号=14pt
% 其他可选参量如twocolumn, 两行排版
\usepackage{../TeXplatesMathematics}
\usepackage{xeCJK}
\usepackage{xr-hyper}
\externaldocument[PST-]{PointSetTopology}

\addbibresource{AlgebraicTopology.bib} % 把这里改成实际的文件名

\newcommand*{\rel}{\mathbin{\mathrm{rel}}}
\newcommand*{\pclass}[1]{\langle{#1}\rangle}    % path class

\title{Algebraic Topology}
\author{Hoyan Mok}
\begin{document}
\pagenumbering{Alph}
\maketitle
\frontmatter

\tableofcontents
\mainmatter{}

\chapter{Homotopy and Fundamental Group}
\section{Homotopy}
\begin{definition}[Homotopy]%
    \label{def: Homotopy}
    $f, g \in C(X, Y)$.
    If $\exists H \in C(X \times [0, 1], Y)$ s.t.\ $H(x, 0) = f(x)$, $H(x, 1) = g(x)$, then we say $f$ and $g$ are \indexbf{homotopic}, denoted by $\indexmath[f simeq g]{f \simeq g} \colon X \to Y$ or just $X \to Y$.
    $H$ is called a \indexbf{homotopy} between $f$ and $g$, denoted by $\indexmath[H colon f simeq g]{H \colon f \simeq g}$ or $\indexmath[f simeq_H g]{f \simeq_H g}$.
\end{definition}

For $t \in [0, 1]$, $h_t \colon X \to Y ;x \mapsto H(x, t)$ is called a \emphbf{$t$-slice}\index{t-slice@$t$-slice}.

If $f$ is homotopic to a constant mapping, we say that $f$ is \indexbf{null-homotopic}.

A \indexbf{linear homotopy} is a homotopy between two functions to $Y \subseteq \mathbb R^n$ that change linearly, i.e.\ 
\begin{equation*}
    H(x, t) = (1 - t) f(x) + t g(x).
\end{equation*}


\begin{theorem}[Maps to convex set are homotopic]%
    \label{theorem: Maps to convex set are homotopic}
    $f, g \in C(X, Y)$. If $Y$ is a convex set in $\mathbb R^n$, 
    then $f \simeq g$.
\end{theorem}
\begin{proof}
    Consider linear homotopy.
\end{proof}

\begin{theorem}%
    \label{theorem: Homotopic relation is an equivalence relation}
    Homotopic relation is an equivalence relation.
\end{theorem}
\begin{proof}
    \emph{reflexity}.
    $f \simeq f$, just take $H(x, t) = f(x)$ for any $t$ (Such homotopy is called a \indexbf{constant homotopy}).

    \emph{Symmetry}. 
    $f \simeq g$ then $g \simeq f$. Just take $\bar H(x, t) = H(x, 1 - t)$ (Here $\indexmath[H bar]{\bar H}$ is called the inverse of $H$).

    \emph{Transivity}. 
    $f \simeq g \wedge g \simeq h \to f \simeq h$. Let
    \begin{equation*}
        H_1 H_2 (x, 2t) = \begin{cases}
            H_1(x, 2t) & t \in [0, 1/2], \\
            H_2(x, 2t - 1) & t \in [1/2, 1].
        \end{cases}
    \end{equation*}
    We can see that $H_1 H_2$ is also a homotopy (see Theorem~\ref{PST-theorem: composition of path} in Point Set Topology)
\end{proof}

Hence, we can define \indexbf{homotopy classes} on $C(X, Y)$, denoted by $\indexmath[X Y]{[X, Y]}$.

As you might expect after reading the proof of Theorem~\ref{theorem: Homotopic relation is an equivalence relation}, the homotopies between mappings within a homotopy class form a group.

\begin{theorem}[Composition of homotopies]%
    \label{theorem: Composition of homotopies}
    $f_1 \simeq f_2 \colon X \to Y$, $g_1 \simeq g_2 \colon Y \to Z$, then $g_1 \circ f_1 \simeq g_2 \circ f_2 \colon X \to Z$.
\end{theorem}
\begin{proof}[\textbf{Proof i}]
    Let $F \colon f_1 \simeq f_2$, $G \colon g_1 \simeq g_2$. 
    Define:
    \begin{equation*}
        \bv F \colon X \times [0, 1] \to Y \times [0, 1]; 
        (x, t) \mapsto (F(x, t), t). 
    \end{equation*}

    It can be verified tht $G \circ \bv F \colon g_1 \circ f_1 \simeq g_2 \circ g_2 \colon X \to Z$.
\end{proof}
\begin{proof}[\textbf{Proof ii}]
    Let $F \colon f_1 \simeq f_2$, $G \colon g_1 \simeq g_2$. 

    We can verify that $H_1 \colon (x, t) \mapsto g_1 \circ F(x, t)$ is a homotopy between $g_1 \circ f_1$ and $g_1 \circ f_2$;
    Similarly $H_2 \colon g_1 \circ f_2 \simeq g_2 \circ f_2$ can be defined.

    Now consider $H = H_1 H_2$, or in detailed,
    \begin{equation*}
        H(x, t) = \begin{cases}
            g_1 \circ F(x, 2t) & (x, t) \in X \times [0, 1/2] \\
            G(f_2(x), 2t - 1). & (x, t) \in X \times [1/2, 1]
        \end{cases}
    \end{equation*}
\end{proof}

\begin{lemma}[Identity map in convex space is null-homotopic]
    $X \subset \mathbb R^n$ is a convex space.
    $\forall x_0 \in X$,
    $\id_X \simeq (x \mapsto x_0)$.
\end{lemma}
\begin{proof}
    The linear homotopy can be constructed as:
    \begin{equation*}
        H_{x_0} (x, t) = tx + (1 - t) x_0.
    \end{equation*}
\end{proof}

\begin{theorem}[Continuous mappings from a convex set are null-homotopic]%
    \label{theorem: Continuous mappings from a convex set are null-homotopic}
    $X \subseteq \mathbb R^n$ is a convex set.
    $\forall f \in C(X, Y)$, $f$ is null-homotopic.
\end{theorem}
\begin{proof}
    Let $H_{x_0} (x, t) = tx + (1 - t) x_0$.
    Then, any $f \colon X \to Y$ can be written as $f = f \circ \id_X$, hence $f \simeq f \circ H_{x_0}(x, 1) = (x \mapsto f(x_0))$, which means $f$ is null-homotopic.
\end{proof}

\begin{theorem}[Constant mappings to a path-connected space belong to one homotopy class]%
    \label{theorem: Constant mappings to a path-connected space belong to one homotopy class}
    If $Y$ is a path-connected space, $y_0 \in Y$,
    then $[X, Y] = [x \mapsto y_0]$ (i.e.\ homotopy class of constant mapping to $\{y_0\}$)
\end{theorem}
\begin{proof}
    Let $f_1(x) = y_1$, $f_2(x) = y_2$ be two constant mappings, $a$ is a path from $y_1$ to $y_2$.
    Then the homotopy between $f_1$ and $f_2$ can be defined as:
    \begin{equation*}
        H(x, t) = a(t).
    \end{equation*}
\end{proof}



\begin{definition}[Homotopy relative to a set]
    Let $A \subseteq X$, $H \colon f \simeq g$.
    If $\forall a \in A$, $\forall t \in [0, 1]$, $f(a) = g(a) = H(a, t)$, we say that $f$ and $g$ are \indexbf{homotopic relative to $A$}, denoted by $\indexmath[H f simeq g rel A]{H \colon f \simeq g \rel A}$.
\end{definition}

We can have parallel results as Theorem~\ref{theorem: Homotopic relation is an equivalence relation} and Theorem~\ref{theorem: Composition of homotopies}:

\begin{theorem}%
    \label{theorem: Relatively homotopic relation is an equivalence relation}
    Given $A \subseteq X$, $\simeq \rel A$ is an equivalence relation in $C(X, Y)$.
\end{theorem}

\begin{theorem}[Composition of relative homotopies]%
    \label{theorem: Composition of relative homotopies}
    $f_1 \simeq f_2 \colon X \to Y \rel A$, $g_1 \simeq g_2 \colon Y \to Z \rel B$, and $f_1(A) \subset B$, then $g_1 \circ f_1 \simeq g_2 \circ f_2 \colon X \to Z$.
\end{theorem}

\begin{definition}[Fixed-endpoint Homotopy]%
    \label{def: Fixed-endpoint Homotopy}
    Let $a$, $b$ be two paths in $X$. 
    If $a \simeq b \rel \{0, 1\}$, we say that $a$ and $b$ are \indexbf{fixed-endpoint homotopic}\index{fixed-endpoint homotopy}.
    The paths in $X$ modulus fixed-point homotopy is denoted by $\indexmath[X]{[X]}$, called the \indexbf{path classes}.
    The path class which $a$ belongs to is denoted by $\indexmath[a]{\pclass{a}}$.
\end{definition}

\section{Fundamental Group}

Fundamental group of a topological space at a point is the path classes at this point.
We need to introduce the multiplicative structure of path classes.

\begin{theorem}
    Let $a, b, c, d$ be four paths in $X$.
    \begin{align*}
        a \simeq b \rel \{0, 1\} &\IFF \bar a \simeq \bar b \rel \{0, 1\}, \\
        a \simeq b \rel \{0, 1\} \wedge c \simeq d \rel \{0, 1\} \wedge a(1) = c(0) &\then ac \simeq bd \rel \{0, 1\}.
    \end{align*}
\end{theorem}

\begin{definition}[Inverse and product of path classes]
    $\alpha, \beta \in [X]$, $a \in \alpha$, $b \in \beta$.
    $b(0) = a(1)$.
    We define $\alpha^{-1} := \pclass{\bar a}$ to be the \indexbf{inverse} of the path class~$\alpha$, and $\alpha \beta := \langle ab \rangle$ to be the \indexbf{product} of the two path classes~$\alpha$ and $\beta$.
\end{definition}

While the product of paths does not obey associativity, we have:
\begin{theorem}[Associativity of product of path classes]%
    \label{theorem: Associativity of product of path classes}
    $\alpha, \beta, \gamma \in [X]$. $(\alpha \beta) \gamma = \alpha (\beta \gamma)$ (if they are productible).
\end{theorem}
\begin{proof}
    Consider $\forall a \in \alpha$, $\forall b \in \beta$, $\forall c \in \gamma$.

    Let
    \begin{align*}
        \tilde a(t) &= t/3, \\
        \tilde b(t) &= t/3 + 1/3, \\
        \tilde c(t) &= t/3 + 2/3.
    \end{align*}

    $\tilde a, \tilde b$ and $\tilde c$ are three paths in $[0, 1]$, and $\tilde a (\tilde b \tilde c) \simeq (\tilde a \tilde b) \tilde c \rel \{0, 1\}$, since $[0, 1]$ is convex, therefore there is a linear homotopy between the two product paths.

    Now Let $f \colon [0, 1] \to X$ be
    \begin{equation*}
        f(t) = \begin{cases}
            a(3t), & t \in [0, 1/3]; \\
            b(3t - 1), & t \in [1/3, 2/3]; \\
            c(3t - 2), & t \in [2/3, 1].
        \end{cases}
    \end{equation*}

    $a(bc) = f \circ \tilde a (\tilde b \tilde c) \simeq f \circ (\tilde a \tilde b) \tilde c = (ab)c \rel \{0, 1\}$, by Theorem~\ref{theorem: Composition of homotopies}.
\end{proof}

\begin{theorem}[Identity-like properties of point path]%
    \label{theorem: Identity-like properties of point path}
    $\alpha \in [X]$. 
    Let the initial and the terminal point of $\alpha$ be $x_0$ and $x_1$.
    \begin{enumerate*}[label=\emph{(\roman*)}] % chktex 36
        \item $\alpha^{-1} \alpha = \pclass{t \mapsto x_1}$, $\alpha \alpha^{-1} = \pclass{t \mapsto x_0}$;
        \item $\alpha \pclass{t \mapsto x_0} = \alpha = \pclass{t \mapsto x_1}\alpha$.
    \end{enumerate*}
\end{theorem}
\begin{proof}
    Note that $\id_{[0, 1]}$ is a path in the convex set $[0, 1]$.
\end{proof}

For now path classes are not closed under production.

\begin{definition}[Fundamental group]%
    \label{def: Fundamental group}
    $x_0 \in X$. 
    The path classes of loops at $x_0$ (paths that have both endpoints at $x_0$), equiped with production, is the \indexbf{fundamental group} of $X$ at $x_0$, denoted by $\indexmath[pi1 X x0]{\pi_1(X, x_0)}$.
\end{definition}

\begin{definition}[Homomorphism induced by continuous function]
    $f \in C(X, Y)$, $x_0 \in X$.
    We define 
    \begin{equation*}
        f_\pi \colon [X] \to [Y],
        \quad
        \pclass{a} \mapsto \langle f \circ a \rangle
    \end{equation*}
    where $a$ is a path in $X$. 
    
    The limitation of $\indexmath[f pi]{f_\pi}$ on $\pi_1(X, x_0)$ is said to be a \indexbf{homomorphism induced by $f$}.
\end{definition}

For simplicity, we would write such homomorphism by $f_\pi$ (without explicitly referring limitation).

\begin{theorem}[Isomorphism induced by homeomrphism]%
    \label{theorem: Isomorphism induced by homeomrphism}
    Let $f$ be a homeomorphism from $X$ to $Y$, then $\forall x_0 \in X$, $f_\pi$ is an isomorphism from $\pi_1(X, x_0)$ to $\pi_1(Y, f(x_0))$.
\end{theorem}
\begin{proof}
    \begin{align*}
        f^{-1} \circ f &= \id_X 
        \then 
        (f^{-1})_\pi \circ f_\pi = \id_{\pi_1(X, x_0)}; \\
        f \circ f^{-1} &= \id_Y 
        \then 
        f_\pi \circ (f^{-1})_\pi = \id_{\pi_1(Y, f(x_0))},
    \end{align*}
    therefore $(f^{-1})_\pi$ is the inverse of $f_\pi$.
    An invertible homomorphism is an isomorphism. 
\end{proof}

\begin{theorem}[Fundamental group of product space]%
    \label{theorem: Fundamental group of product space}
    $x_0 \in X$, $y_0 \in Y$.
    \begin{equation*}
        \pi_1(X \times Y, (x_0, y_0)) \cong \pi_1(X, x_0) \times \pi_1(Y, y_0).
    \end{equation*}
\end{theorem}
\begin{proof}
    Let $j_X \in C(X \times Y, X)$ and $j_Y \in C(X \times Y, Y)$ be projections ($j_X(x, y) = x$, $j_Y(x, y) = y$), and define a homomorphism
    \begin{align*}
        \varphi\colon && &\pi_1(X \times Y, (x_0, y_0)) &&\to &\pi_1(X, x_0) \times \pi_1(Y, y_0); \\
        &&&\gamma &&\mapsto &((j_X)_\pi(\gamma), (j_Y)_\pi(\gamma)).
    \end{align*}

    $\varphi$ is a \emph{monomorphism}.
    Let $\pclass{c} \in \ker \varphi$ i.e.\
    \begin{equation*}
        \varphi(\pclass{c}) = (\pclass{t \mapsto x_0}, \pclass{t \mapsto y_0}).
    \end{equation*}
    Let
    \begin{equation*}
       H_X \colon j_X \circ c \simeq  t \mapsto x_0 \rel \{0\},
       \quad
       H_Y \colon j_Y \circ c \simeq t \mapsto y_0\rel \{0\}.
    \end{equation*}

    The homotopy between $c$ and $t \mapsto (x_0, y_0)$ is defined as
    \begin{equation*}
        F \colon [0, 1]^2 \to X \times Y;
        (t, s) \mapsto (H_X(t, s), H_Y(t, s)).
    \end{equation*}

    $\varphi$ is an \emph{epimorphism}.
    $\forall \pclass{a} \in \pi_1(X, x_0)$ and $\forall \pclass{b} \in \pi_1(Y, y_0)$.
    $c\colon t \mapsto (a(t), b(t)) \in C([0, 1], X \times Y)$.
    $\pclass{c} \in \varphi^{-1}(\{(\pclass{a}, \pclass{b})\})$.
\end{proof}

\begin{theorem}[Fundamental groups of path connected space at different points are isomorphic]%
    \label{theorem: Fundamental groups of path connected space at different points are isomorphic}
    $X$ is path connected, $x_1, x_2 \in X$.
    $\pi_1(X, x_1) \cong \pi_1(X, x_2)$.
\end{theorem}
\begin{proof}
    $\pclass{a} \in \pi_1(X, x_1)$, $\pclass{b} \in \pi_1(X, x_2)$, $\pclass{c}$ is a path class with initial point~$x_1$ and terminal point~$x_2$.

    It can be verified that
    \begin{equation}\label{homomorphism between fundamental groups at two points in one space}
        g_c \colon \pi_1(X, x_1) \to \pi_2 (X, x_2); \pclass{a} \mapsto \langle \bar c a c \rangle
    \end{equation}
    is a homomorphism. Same as $g_{\bar c} (\pclass{b}) = c b \bar c$.

    \begin{align*}
        g_c \circ g_{\bar c} (\pclass{b}) &= \langle \bar c c b \bar c c\rangle = \id_{\pi_1(X, x_2)};
        \\
        g_{\bar c} \circ g_c (\pclass{a}) &= \langle c \bar c a c \bar c\rangle = \id_{\pi_1(X, x_2)},
    \end{align*}
    therefore $g_c$ is an isomorphism.
\end{proof}

With Theorem~\ref{theorem: Fundamental groups of path connected space at different points are isomorphic}, we can write the fundamental group of a path-connected space $X$ by $\indexmath[pi1 X]{\pi_1(X)}$.

For different path-connected branches, a topological space can have different fundamental groups, while they are isomorphic within one branch.

\begin{theorem}%
    \label{theorem: relations between two induced homomorphisms}
    Let $X$ and $Y$ be two topological spaces, $f_1 \simeq f_2 \colon X \to Y$, $x_0 \in X$, $f_1(x_0) = y_1$, $f_2(x_0) = y_2$; 
    If $\exists c \in C([0, 1], Y)$ s.t.\ $c(0) = y_1$, $c(1) = y_2$, and $g_c$ is defined as in Eq.~\eqref{homomorphism between fundamental groups at two points in one space},
    then $g_c \circ f_{1, \pi} = f_{2, \pi}$.
\end{theorem}

\begin{equation*}
    % https://tikzcd.yichuanshen.de/#N4Igdg9gJgpgziAXAbVABwnAlgFyxMJZABgBoBGAXVJADcBDAGwFcYkQAdDtLAfXIAUADVIACAB69iAShABfUuky58hFACZS66nSat2XHvwEBNMQE9e62QqXY8BIpuI6GLNok7c+gs6MvkNjowUADm8ESgAGYAThAAtkhkIDgQSOQ0bvqeUbzAmqKGWHLyiiCxCUiaKWmIGbru7KG8AMal0XGJiMmpVZl6HuV5GYXeJXKUckA
\begin{tikzcd}
    &  & {\pi_1(Y, y_1)} \arrow[dd, "g_c"] \\
{\pi_1(X, x_0)} \arrow[rrd, "{f_{2, \pi}}"] \arrow[rru, "{f_{1, \pi}}"] &  &                                   \\
    &  & {\pi_1(Y, y_2)}                  
\end{tikzcd}
\end{equation*}
\begin{proof}
    %TODO: finish this proof
\end{proof}

\begin{definition}[Simply connected]%
    \label{def: simply connected}
    If the fundamental group of a path connected space $X$ is trivial i.e.\ $\pi_1(X) \cong \{1\}$, we say that $X$ is \indexbf{simply connected}.
\end{definition}

\begin{theorem}[Convex set is simply connected]%
    \label{theorem: Convex set is simply connected}
    If $X \subset \mathbb R^n$ is convex, then $X$ is simply connected.
\end{theorem}
\begin{proof}
    $x_0 \in X$, $a \in C([0, 1], X)$ s.t.\ $a(0) = a(1) = x_0$.
    $H_{a, x_0}(s, t) = (1 - t) a(s) + t x_0$. 
\end{proof}

\section{Examples of Fundamental Groups}

\subsection{\texorpdfstring{$S^1$}{S1}}


\begin{definition}[Lift]
    Let $X, Y, Z$ be three topological spaces, and $f \in C(X, Z)$, $p \in C(Y, Z)$. 
    If $\indexmath[f]{\tilde f} \in C(X, Y)$, s.t.\ $f = p \circ \tilde f $, we say that $\tilde f$ is a \indexbf{lift} of $f$. 
\end{definition}

\begin{equation*}
    % https://tikzcd.yichuanshen.de/#N4Igdg9gJgpgziAXAbVABwnAlgFyxMJZARgBoAGAXVJADcBDAGwFcYkQANEAX1PU1z5CKcqWLU6TVuwCaPPiAzY8BIgCYxEhizaIQALR4SYUAObwioAGYAnCAFskokDghINkneyvzrdx4hkLm6IHtrSemi+ILYOTjSuSEHhuiAAOml4jLAABD40jPQARjCMAAoCKsIgNlimABY4IDRw9VhWTYjk3JTcQA
\begin{tikzcd}
    & X \arrow[rd, "f"] \arrow[ld, "\tilde f"'] &   \\
Y \arrow[rr, "p"] &                                           & Z
\end{tikzcd}
\end{equation*}

In some case, given $f$ and $p$, $\tilde f$ might do not exist.

\begin{lemma}[Lift of path]
    $a \in C([0, 1], S^1)$, $p \colon \mathbb R \to S^1; x \mapsto \me^{2\pi x \mi}$. Let $t_0 \in \mathbb R$ s.t.\ $p(x_0) = a(0)$.
    There exists a unique lift~$\tilde a \in C([0, 1], \mathbb R)$ of $a$ s.t.\ $\tilde a(0) = x_0$.
\end{lemma}

\begin{equation*}
    % https://tikzcd.yichuanshen.de/#N4Igdg9gJgpgziAXAbVABwnAlgFyxMJZARgBoAGAXVJADcBDAGwFcYkRlzSACYykAL6l0mXPkIoufGgxZtEIADqKAtvRwALAEZbuAJUHCQGbHgJEATKWl0mrdgGUAesUHUQMKAHN4RUADMAJwgVJC4QHAgkK1s5dnpDAODQxDIIqMQY2XsFNESQIJCwmkikNOz5JUU8RlhuBJpGei0YRgAFUTMJEECsLw0cNwEgA
\begin{tikzcd}
    & {[0, 1]} \arrow[rd, "a"] \arrow[ld, "\tilde a"'] &     \\
\mathbb R \arrow[rr, "p"] &                                                  & S^1
\end{tikzcd}
\end{equation*}

\begin{proof}
    \emph{Existence}.
    The collection of open sets that the images under $a$ do not cover $S^1$, $\{(\alpha_i, \beta_i) \cap [0, 1] \mid a_i, b_i \in \mathbb R^I\wedge S^1 \subsetneq a((\alpha_i, \beta_i))\}$, is a cover of $[0, 1]$ by the definition of continuity.
    Since $[0, 1]$ is compact, there exists a finite subcover $\{(\alpha_i, \beta_i) \cap [0, 1] \mid a_i, b_i \in \mathbb R^n\wedge S^1 \subsetneq a((\alpha_i, \beta_i))\}$, where $n \in \mathbb N$.
    By dividing these open intevals into closed intevals that has no inner points intersecting, we can get $\varOmega = \{I_k := [t_i, t_{i+1}] \mid k \in m\}$ (This can be done by sorting $\alpha_i$ and $\beta_i$).

    The mapping~$p$ is locally homeomorphic i.e.\ there exists $[x_i, x_i'] \subset \mathbb R$ s.t.\ $p_i := p|_{[x_i, x_i']} \colon [x_i, x_i'] \to a(I_i)$ is a homeomorphism (and $p_i(x_i) = a(t_i)$), therefore $\tilde a_i := p^{-1}_i \circ a$ is a lift of $a_i := a|_{I_i}$.

    Since $p_0(t_0) = a(t_0)$, $p_{i+1}(t_i) = p_i(t_i)$, we can define piecewisely the lift of $a$ by $\tilde a = \cup \{\tilde a_i \mid i \in m\}$.

    \emph{Uniqueness}.
    Let $\tilde a'$ be another lift of $a$, $p(\tilde a'(t) - \tilde a(t)) = p\circ \tilde a'(t) / p\circ \tilde a(t) = a(t) / a(t) = 1$, therefore $\tilde a'(t) - \tilde a(t) \in \mathbb Z$. 
    Since $[0, 1]$ is connected, the image of $t \mapsto \tilde a'(t) - \tilde a(t)$ must be connected, which is possible only if it is constant.
    $\tilde a'(0) = \tilde a(0) = x_0$, therefore $\tilde a = \tilde a'$.
\end{proof}

Notice that we have the freedom to set $\tilde a(0) \in \mathbb Z$ (the lift is unique after setting that), what really matter is the difference $\tilde a(1) - \tilde a(0)$.
One can proof that $\indexmath[q a]{q(a)} := \tilde a(1) - \tilde a(0)$ does not depend on the chose of $\tilde a(0) \in \mathbb Z$. 
We call $q(a)$ the \indexbf{loop number} of path~$a$.

\begin{lemma}[Two loops that are never antipodal have the same loop number]%
    \label{lemma: Two loops that are never antipodal have the same loop number}
    Let $a$, $b$ be two loops at $z_0$ in $S^1$.
    If $\forall t \in [0, 1]$, $a(t) \neq -b(t)$, then $q(a) = q(b)$.
\end{lemma}
\begin{proof}
    Choose $\tilde a(0) = \tilde b(0) = 0$ (if not so, just translate the lift by an integer). 
    In this case, $q(a) = \tilde a(1)$, $q(b) = \tilde b(1)$.

    If $q(a) \neq q(b)$, without loss of generality, $q(a) > q(b)$, then $f := t \mapsto \tilde a(t) - \tilde b(t)$ is a continuous function from a compact space $[0, 1]$ to $\mathbb R$, therefore by the connectedness of $[0, 1]$, $\exists t_0 \in [0, 1]$ s.t.\ $f(t_0) = 1/2 \in [0, q(a) - q(b)]$, when
    \begin{equation*}
        p \circ \tilde a(t_0) + p \circ \tilde b(t_0) = \me^{2\pi\mi (\tilde b(t_0) + 1/2)} + \me^{2\pi\mi \tilde b(t_0)} = 0.
    \end{equation*}
\end{proof}

\begin{lemma}[Same loop number iff homotopic ralative to endpoint]%
    \label{lemma: Same loop number iff homotopic ralative to endpoint}
    Let $a$, $b$ be two loops at $z_0$ in $S^1$.
    $a \simeq b \rel \{0\}$ iff $q(a) = q(b)$.
\end{lemma}
\begin{proof}
    $\to$: Let $H \colon a \simeq b \rel \{0\}$, $h_s = t \mapsto H(t, s)$, $f_t = s \mapsto H(t, s)$.

    $\forall (t, s) \in [0, 1]^2$, $U := \{H(t, s) \me^{\mi \theta} \mid \theta \in (-\pi, \pi)\} \in \mathscr U_{S^1}(H(t, s))$. 
    Since $f_t \in C([0, 1], S^1)$, $\exists V(s) \in \mathscr U_{[0, 1]}(s)$ s.t.\ $H(V(s)) \subset U$.
    Which means, $\forall t \in [0, 1]$, $\forall s_1, s2 \in V(s)$, $f_t(s_1) \neq - f_t(s_2)$ or $h_{s_1}(t) \neq - h_{s_2}(t)$. 
    By Lemma~\ref{lemma: Two loops that are never antipodal have the same loop number}, $q(h_s) = q(h_{s'})$.

    $\varOmega = \{V(s) \mid s \in [0, 1]\}$ is an open cover of the compact space $[0, 1]$, therefore has a finite subcover $\varOmega = \{V_i \in \varOmega \mid i \in n\}$. 
    In each $V(s_i)$, $h_s$ has the same loop numbers.

    We therefore have $q(a) = q(h_0) = q(h_1) = q(b)$.

    $\gets$: $H \colon [0, 1]^2 \to S^1; (t, s) \mapsto p((1 - s)\tilde a(t) - s \tilde b(t))$.
\end{proof}

\begin{theorem}\label{theorem: fundamental group of S^1 is Z}
    $\pi_1(S^1) \cong \mathbb Z$.
\end{theorem}
\begin{proof}
    $z_0 \in S^1$.
    Let $Q \colon \pi_1(S^1, z_0) \to \mathbb Z; \pclass{a} \mapsto q(a)$.

    $\forall \pclass{a}, \pclass{b} \in \pi_1(S^1, z_0)$, choose $\tilde a(1) = \tilde b(0)$,
    \begin{align*}
        Q(\pclass{a} \pclass{b}) &= Q(\langle a b \rangle) 
        = q(ab) \\
        &= \tilde b(1) - \tilde a(0) = \tilde b(1) - \tilde b(0) + \tilde a(1) - \tilde a(0) = q(a) + q(b) \\
        &= Q(\pclass{a}) + Q(\pclass{b}),
    \end{align*}
    which means $Q$ is a homomorphism. 

    By Lemma~\ref{lemma: Same loop number iff homotopic ralative to endpoint}, $Q$ is a monomorphism. 
    $\forall n \in \mathbb Z$, $Q(\langle t \mapsto \me^{2\pi n t\mi}\rangle) = n$, therefore $Q$ is also an epimorphism.
\end{proof}

\subsection{\texorpdfstring{$S^n$, $n>2$}{Sn, n > 2}}

The situation for $S^n$ is much simpler:
\begin{theorem}\label{theorem: S^n for n >= 2 is simply connected}
    $\forall n \in \mathbb N$, if $n \geq 2$, then $S^n$ is simply connected.
\end{theorem}
\begin{proof}
    Let $x_0 \in S^n$, and $a$ be a loop at $x_0$ in $S^n$.
    $x \in S^n$ and $x \neq x_0$.
    Embed $S^n$ into $\mathbb R^{n+1}$ and let $B(x; \delta)$ be a $(n+1)$-D ball with radius $\delta$ around $x$ that $x_0 \notin B(x; \delta)$.

    $a^{-1}(B(x; \delta) \cap S^n)$ is a collection of open, disjoint intervals in $[0, 1]$, which can be considered as an open cover of $a^{-1}(\{x\})$, which is compact.
    Let the finite subcover of $a^{-1}(\{x\})$ be $\{(\alpha_i, \beta_i) \cap [0, 1] \mid \alpha_i, \beta_i \in \mathbb R, i \in m\}$, where $m \in \mathbb N$.

    
    Let $P \colon \mathbb R^{n+1} \times \mathbb R^{n+1} \times \mathbb R \to \mathbb R^{n+1}$ be:
    \begin{equation*}
        P(y, y_0, r) = \frac{y - y_0}{\|y - y_0\|} r + y_0,
    \end{equation*}
    which means project $y$ to the sphere with radius $r$ around $y_0$.

    Now we define the loop $b$ that go as:
    \begin{equation*}
        b(t) = \begin{cases}
            a(t),
                & t \notin (\alpha_i, \beta_i), \forall i \in m; \\
            P[P(a(t), x, \delta), 0, 1] , 
                & t \in (\alpha_i, \beta_i) - a^{-1}(\{x\}), \exists i \in m; \\
            \lim\limits_{\substack{t' \to t\\t' \in (\alpha_i, \beta_i) - a^{-1}(\{x\})}} b(t') , 
                & t \in a^{-1}(\{x\}) \cap (\alpha_i, \beta_i), \exists i \in m,
        \end{cases}
    \end{equation*}
    and the homotopy between $a$ and $b$ can be written as
    \begin{equation*}
        H \colon [0, 1]^2 \to S^n; (t, s) \mapsto P[(1 - s)a(t) + sb(t), 0, 1].
    \end{equation*}

    Since $b$ is a loop in $S^n - \{x\}$, while $S^n - \{x\} \cong \mathbb R^n$ (by stereographic projection), which is simply connected, we know that $b$ is homotopic to $t \mapsto x_0$ i.e.\ null-homotopic.
\end{proof}

By Theorem~\ref{theorem: Fundamental group of product space}, the fundamenal group of $T^2 := S^1 \times S^1$ is $\pi_1(T^2) \cong \mathbb Z \times \mathbb Z$, which is not isomorphic to $S^2$, therefore $T^2 \not\cong S^2$.

\section{Homotopy Types}

\begin{definition}[Homotopy type]
    If $\exists f \in C(X, Y)$, $\exists g \in C(Y, X)$ s.t.\ 
    \begin{align*}
        g \circ f \simeq \id_X,
        &&
        f \circ g \simeq \id_Y,
    \end{align*}
    then we say $X$ and $Y$ are \indexbf{homotopy equivalent}, or they are of the same \indexbf{homotopy type}, denoted by $\indexmath[X simeq Y]{X \simeq Y}$.
    $f$ is called a \indexbf{homotopy map} or a \indexbf{homotopy equivalence} from $X$ to $Y$, and $g$ is called a \indexbf{homotopy inverse} of $f$.
\end{definition}

An inverse of a homotopy map is not unique.

Some examples of spaces having same homotopy types:
\begin{itemize}
    \item $\mathbb R \simeq \mathbb R^n$ ($n \in \mathbb N_+$).
    \item $X \times [0, 1] \simeq X$.
\end{itemize}

\begin{theorem}\label{theorem: homotopy map induces isomorphism}
    If $X \simeq Y$ and $f$ is a homotopy map from $X$ to $Y$, $f(x_0) = y_0$, then $f_\pi$ is an isomorphism from $\pi_1(X, x_0)$ to $\pi_1(Y, y_0)$.
\end{theorem}
\begin{proof}
    %TODO: finish this
\end{proof}


Spaces with the simplest homotopy type are contractable spaces.

\begin{definition}[Contractable space]%
    \label{def: contractable space}
    If $X \simeq \{x\}$, we call $X$ a \indexbf{contractable space}.
\end{definition}



\section{Retractability}

\begin{definition}[Retractability]%
    \label{definition: retractability}
    $A \subset X$, $i \colon A \to X$ is an \indexbf{inclusion} from $A$ to $X$, meaning $\forall a \in A$, $i(a) = a$. 
    If $\exists r \in C(X, A)$ s.t.\ $r \circ i = \id_A$, then $A$ is called a \indexbf{retract} of $X$, $r$ is a \indexbf{retraction}, and $X$ is said to be \indexbf{retractable}.
\end{definition}

\begin{definition}[Deformation retractability]%
    \label{definition: deformation retractability}
    $A \subset X$, $i \colon A \to X$ is an inclusion. 
    If $\exists r \in C(X, A)$ s.t.\ $r \circ i = \id_A \wedge H \colon i \circ r \simeq \id_X$, then $A$ is called a \indexbf{deformation retract} of $X$, $H$ is a \indexbf{deformation retraction} of $X$, and $X$ is said to be \indexbf{deformation retractable}.
\end{definition}

\begin{theorem}[Spaces are homotopically equivalent to their deformation retracts]
    If $X \simeq Y$ and $X$ is a deformation retract of $Y$, then $X \simeq Y$.
    And, the retraction $r \colon Y \to X$ and the inclusion $i \colon X \to Y$ are homotopy inverse to each other.
\end{theorem}

\begin{theorem}[Contractable space can be deformatively retracted to all its points]
    If $X$ is a contractable space, $\forall x \in X$, $\{x\}$ is a deformation retract of $X$.
\end{theorem}


\begin{definition}[Strong deformation retractability]%
    \label{definition: strong deformation retractability}
    $A \subset X$, $i \colon A \to X$ is an inclusion. 
    If $\exists r \in C(X, A)$ s.t.\ $r \circ i = \id_A \wedge H \colon i \circ r \simeq \id_X \rel A$, then $A$ is called a \indexbf{strong deformation retract} of $X$, $H$ is a \indexbf{strong deformation retraction} of $X$, and $X$ is said to be \indexbf{strongly deformation retractable}.
\end{definition}

Some examples:
\begin{itemize}
    \item $X \times [0, 1]$ has strong deformation retracts $X \times \{t\}$ for each $t \in [0, 1]$.
    \item $S^{n-1}$ is a strong deformation retract of $\mathbb R^n \backslash \{0\}$. 
    \item Topological cone $CX = X \times [0, 1] / X \times \{1\}$ has a strong deformation retract at the tip of the cone i.e.\ $X \times \{1\}$.
    \item M\"obius belt can be strong-deformatively retracted to the circle which is the centre line of the belt.
\end{itemize}

\chapter{Van-Kampen Theorem}
\section{Free Abelian Group and Finitely Generated Group}
In this section, we only talk about Abelian groups, and their multiplications are called ``addition'', i.e.\ $(G, +)$.


\begin{definition}[Free Abelian group]%
    \label{definition: free abelian group}
    Let $(F, +)$ be an Abelian group.
    If $\exists A \subset F$ s.t.\ $\forall f \in F$, $\exists! n_f \colon A \to \mathbb Z$ s.t.\ 
    \begin{equation*}
        f = \sum_{a \in A} n_f(a) a,
        \quad 
        \card \{a \in A \mid n_f(a) \neq 0\} \in \mathbb N,
    \end{equation*}
    then we call $F$ a \indexbf{free Abelian group}, $A$ is a \indexbf{basis} of $F$.
\end{definition}

In plain words, all elements in $F$ can be uniquely decided by finite integer-linear combinations of the elements in $A$.
Notice that $A$ can be infinite.

Typical free Abelian groups are integer vectors groups $\mathbb Z^n$ ($n \in \mathbb N_+$), while $Z$

\begin{theorem}[Homomorphism induced by any function of basis to a group]%
    \label{theorem: homomorphism induced by any function of A to a group}
    Let $F$ be a free Abelian group, $A$ be a basis of $F$, $G$ is another Abelian group.
    $\forall f \colon A \to G$, $\exists! \varphi \in \Hom(F, G)$ s.t.\ $\forall a \in A$, $\varphi(a) = f(a)$.
\end{theorem}
\begin{proof}
    If $x = \sum_{i \in m} n_i a_i \in F$ ($n_i \in \mathbb Z$, $m \in \mathbb N$, $a_i \in A$), then
    \begin{equation*}
        \varphi(x) = \sum_{i \in m} n_i f(a_i).
    \end{equation*}
\end{proof}

\begin{definition}[Finitely generated Abelian group]%
    \label{definition: finitely generated Abelian group}
    $(F, +)$ is an Abelian group.
    If $\exists A \subset F$ s.t.\ $\card A \in \mathbb Z$ and $\forall f \in F$, $\exists n_f \colon A \to \mathbb Z$ s.t.\ 
    \begin{equation*}
        f = \sum_{a \in A} n_f(a) a,
    \end{equation*}
    then $F$ is called a \indexbf{finitely generated Abelian group}, $A$ \emphbf{generates} $F$\index{generate}.
    $A$ is a \indexbf{generating set} of $F$.
\end{definition}



\begin{theorem}[Finitely generated iff quotient of a free Abelian group]
    $F$ is an Abelian group.
    $F$ is finitely generated $\IFF$ there exists a free Abelian group $H$, $\exists j \colon H \to F$ s.t.\ $j$ is an epimorphism.
\end{theorem}

\begin{definition}[Direct sum of Abelian group]
    Let $H_i$ ($i \in n$) be Abelian subgroups of $H$. 
    If $\forall h \in H$, $\exists! h_i$ ($i \in n$) s.t.\
    \begin{equation*}
        h = \sum_{i \in n} h_i,
    \end{equation*}
    then we say that $H$ is a \indexbf{direct sum} of $H_i$ ($i \in n$), denoted as:
    \begin{equation*}
        H = \bigoplus_{i \in n} H_i.
    \end{equation*}
\end{definition}

If $H_i$ are not subgroup of $H$ and $H \cong \bigoplus_{i \in n} H_i$, $H$ is also called a direct sum of $H_i$ ($i \in n$).
In order to avoid confusion, this is called an outer direct sum.

The following theorem is very useful to construct a direct sum:

\begin{theorem}
    $H_1$, $H_2$, $H$ are Abelian groups. 
    If $H = H_1 + H_2$ (this is the Abelian group version of $H = H_1 H_2$) and $H_1 \cap H_2 = \{0\}$, then $H = H_1 \oplus H_2$.
\end{theorem}

\begin{theorem}
    Let $j \colon H \to F$ be an epimorphism, $F$ is a free Abelian group.
    \begin{equation*}
        H \cong \ker j \oplus F.
    \end{equation*}
\end{theorem}

We define some concepts that are very familar in vector spaces:
\begin{definition}[Independence and basis]
    Let $H$ be an Abelian group, $A$ is a subset of $H$.
    If $\forall n \colon A \to \mathbb Z$, 
    \begin{equation*}
        \sum_{a \in A} n(a) a = 0
        \then
        \forall a \in A,\;
        n(a) = 0,
    \end{equation*}
    then $A$ is an \indexbf{independent} set.
    And if $A$ generates $H$, we call it a \indexbf{basis} of $H$.
\end{definition}

\begin{theorem}
    Let $H$ be an Abelian group, and there exists a basis of $H$.
    All bases of $H$ have same cardinality.
\end{theorem}

\section{Free Product of Groups}

\section{Van-Kampen Theorem}

\chapter{Covering Space}
\section{Covering space}

\begin{definition}[Even cover]
    Let $p \colon E \to B$ be a continuous surjective, $\mathscr T_E$ is the topology of $E$, $U$ be an open subset of $B$, $I$ be an index set.
    If $\exists \langle V_i\rangle_{i \in I} \in \mathscr T_E^I$ s.t.\ 
    \begin{equation*}
        p^{-1}(U) = \coprod_{i \in I} V_i,
    \end{equation*}
    and $p|_{U_i} \colon U_i \to U$ is a homeomorphism from $U_i$ to $U$,
    then $U$ is said to be evenly covered by $p$. 
    Each $U_i$ is called a \indexbf{sheet} or a \indexbf{slice}.
\end{definition}

\begin{definition}[Covering space]
    Let $p \colon E \to B$ be a continuous surjective.
    If $\forall b \in B$, $\exists U \in \mathscr U(b)$ (\indexbf{evenly covered neighbourhood}) s.t.\ 
    $U$ is evenly covered by $p$,
    then we call $\indexmath[E p]{(E, p)}$ a \indexbf{covering space}, $p$ is a \indexbf{covering map}, $B$ is the \indexbf{base space}. 
\end{definition}

Many authors (\cite{基础拓扑学讲义}) impose path connectivity and local path connectivity onto $E$ and $B$.

\begin{theorem}
    Let $(E, p)$ be a covering space.
    $p$ is an open mapping.
\end{theorem}
\begin{proof}
    Let $G$ be an open set in $E$.
    $\forall b \in p(G)$, $\exists U \in \mathscr U(b)$ s.t.\ $U$ is evenly covered by $p$.

    Choose $e \in p^{-1}(b)$, which should be contained in a sheet $V \subset p^{-1}(U)$.
    $V \cap G$ is open, and since $p|_V$ is a homoemorphism, $p(V \cap G) \subset U \cap p(G) \subset p(G)$ is an open set contained in $G$ which $b$ belongs to.

    Therefore, $p(G)$ is open.
\end{proof}

\begin{theorem}[Restriction of a covering map]
    Let $(E, p)$ be a covering space onto $B$, $B_0 \subset B$, $E_0 = p^{-1}(B_0)$.
    $(E_0, p |_{E_0})$ is a covering space onto $B_0$.
\end{theorem}

\begin{theorem}[Product of covering maps]
    Let $(E, p)$ and $(F, q)$ be covering spaces onto $B$ and $C$.
    The \indexbf{product} of $(E, p)$ and $(F, q)$ ($E$ and $F$, or $p$ and $q$)
    \begin{align*}
        p \times q \colon
         E \times F &\to B \times C;
         \\
         (e, f) &\mapsto (p(e), q(f)),
    \end{align*}
    is also a covering map (onto $B \times C$).
\end{theorem}



\backmatter{}
\nocite{*} % 这个表示列出所有没有在文中被引用的参考文献
\printbibliography[heading=bibliography, title={bibliography}]

\indexprologue{Here listed the important symbols used in these notes}
\printindex[symbol]

\printindex
% \printbibliography
\end{document}