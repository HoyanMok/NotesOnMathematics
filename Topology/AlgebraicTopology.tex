% TeXplates/Mathematics.tex
% v0.2.1
% https://github.com/HoyanMok/TeXplates
\documentclass[openany]{book} 
% \documentclass{ctexbook} 如果用中文
% \documentclass[10pt,a4paper]{ctexart}  字体大小和纸张大小,默认分别为10pt和letterpaper
% 五号 = 10.5pt,小四=12pt,四号=14pt
% 其他可选参量如twocolumn, 两行排版
\usepackage{../TeXplatesMathematics}
\usepackage{xeCJK}
\usepackage{xr-hyper}
\externaldocument[PST-]{PointSetTopology}

\addbibresource{PointSetTopology.bib} % 把这里改成实际的文件名

\DeclareMathOperator{\rel}{rel}

\title{Algebraic Topology}
\author{Hoyan Mok}
\begin{document}
\maketitle
\frontmatter

\tableofcontents
\mainmatter{}

\chapter{Homotopy and Fundamental Group}
\section{Homotopy}
\begin{definition}[Homotopy]%
    \label{def: Homotopy}
    $f, g \in C(X, Y)$.
    If $\exists H \in C(X \times [0, 1], Y)$ s.t.\ $H(x, 0) = f(x)$, $H(x, 1) = g(x)$, then we say $f$ and $g$ are \indexbf{homotopic}, denoted by $\indexmath[f simeq g]{f \simeq g} \colon X \to Y$ or just $X \to Y$.
    $H$ is called a \indexbf{homotopy} between $f$ and $g$, denoted by $\indexmath[H colon f simeq g]{H \colon f \simeq g}$ or $\indexmath[f simeq_H g]{f \simeq_H g}$.
\end{definition}

For $t \in [0, 1]$, $h_t \colon X \to Y ;x \mapsto H(x, t)$ is called a \emphbf{$t$-slice}\index{t-slice@$t$-slice}.

If $f$ is homotopic to a constant mapping, we say that $f$ is \indexbf{null-homotopic}.

A \indexbf{linear homotopy} is a homotopy between two functions to $Y \subseteq \mathbb R^n$ that change linearly, i.e.\ 
\begin{equation*}
    H(x, t) = (1 - t) f(x) + t g(x).
\end{equation*}


\begin{theorem}[Maps to convex set are homotopic]%
    \label{theorem: Maps to convex set are homotopic}
    $f, g \in C(X, Y)$. If $Y$ is a convex set in $\mathbb R^n$, 
    then $f \simeq g$.
\end{theorem}
\begin{proof}
    Consider linear homotopy.
\end{proof}

\begin{theorem}%
    \label{theorem: Homotopic relation is an equivalence relation}
    Homotopic relation is an equivalence relation.
\end{theorem}
\begin{proof}
    \emph{reflexity}.
    $f \simeq f$, just take $H(x, t) = f(x)$ for any $t$ (Such homotopy is called a \indexbf{constant homotopy}).

    \emph{Symmetry}. 
    $f \simeq g$ then $g \simeq f$. Just take $\bar H(x, t) = H(x, 1 - t)$ (Here $\indexmath[H bar]{\bar H}$ is called the inverse of $H$).

    \emph{Transivity}. 
    $f \simeq g \wedge g \simeq h \to f \simeq h$. Let
    \begin{equation*}
        H_1 H_2 (x, 2t) = \begin{cases}
            H_1(x, 2t) & t \in [0, 1/2], \\
            H_2(x, 2t - 1) & t \in [1/2, 1].
        \end{cases}
    \end{equation*}
    We can see that $H_1 H_2$ is also a homotopy (see Theorem~\ref{PST-theorem: composition of path} in Point Set Topology)
\end{proof}

Hence, we can define \indexbf{homotopy classes} on $C(X, Y)$, denoted by $\indexmath[X Y]{[X, Y]}$.

As you might expect after reading the proof of Theorem~\ref{theorem: Homotopic relation is an equivalence relation}, the homotopies between mappings within a homotopy class form a group.

\begin{theorem}
    $f_1 \simeq f_2 \colon X \to Y$, $g_1 \simeq g_2 \colon Y \to Z$, then $g_1 \circ f_1 \simeq g_2 \circ g_2 \colon X \to Z$.
\end{theorem}
\begin{proof}
    % The \indexbf{mapping cylinder} of a mapping $f \colon X \to Y$ is defined as
    % \begin{equation*}
    %     \indexmath[Mf]{\mathrm Mf} 
    % \end{equation*}

    Let $F \colon f_1 \simeq f_2$, $G \colon g_1 \simeq g_2$. 
    Define:
    \begin{equation*}
        \bv F \colon X \times [0, 1] \to Y \times [0, 1]; x \mapsto (F(x, t), t). 
    \end{equation*}

    It can be verified tht $G \circ \bv F \colon g_1 \circ f_1 \simeq g_2 \circ g_2 \colon X \to Z$.
\end{proof}

\begin{theorem}[All mappings to a path-connected space are null-homotopic]
    If $Y$ is path-connected, $y_0 \in Y$, 
    then $[X, Y] = [x \mapsto y_0]$ (i.e.\ homotopy class of constant mapping to $\{y_0\}$)
\end{theorem}
\begin{proof}
    
\end{proof}

\begin{definition}[Homotopy relative to a set]
    Let $A \subseteq X$, $H \colon f \simeq g$.
    If $\forall a \in A$, $\forall t \in [0, 1]$, $f(a) = g(a) = H(a, t)$, we say that $f$ and $g$ are \indexbf{homotopic relative to $A$}, denoted by $\indexmath[H f simeq g rel A]{H \colon f \simeq g \rel A}$.
\end{definition}


\backmatter{}
\nocite{*} % 这个表示列出所有没有在文中被引用的参考文献
\printbibliography[heading=bibliography, title={bibliography}]

\indexprologue{Here listed the important symbols used in this notes}
\printindex[symbol]

\printindex
% \printbibliography
\end{document}